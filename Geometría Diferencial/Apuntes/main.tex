\documentclass{article}
\usepackage{hyperref}
\usepackage{Style}

\nocite{*} % Comentar si quiero citar
%\addbibresource{bibliografia.bib} % Quitar el comentado si quiero usar bibliografia

\begin{document}

\begin{minipage}{2.5cm}
    \includegraphics[width=2cm]{imagen_puc.jpg}
\end{minipage}
\begin{minipage}{14cm}
    {\sc Pontificia Universidad Católica de Chile\\
    Facultad de Matemáticas\\
    Departamento de Matemática\\
    Profesor: Pedro Gaspar -- Estudiante: Benjamín Mateluna}
\end{minipage}
\vspace{1ex}

{\centerline{\bf Geometría Diferencial - MAT2860}
\centerline{\bf Apuntes}}
\centerline{\bf 06 de Marzo de 2025}

\newpage
\tableofcontents

\newpage
\section{Introducción}
\subsection{Evaluaciones}
Habrán tres interrogaciones (I1, I2, I3) cada una vale un $25\%$ y un examen (EX) que vale un 
$25\%$. Las fechas son 14 de abril, 19 de Mayo, 16 de Junio y 3 de Julio respectivamente.

\newpage
\section{Curvas en \texorpdfstring{$\R^{n}$}{}}
\subsection{Curvas parametrizadas}
\noindent Consideramos $\R^{n}:=\{v=(v_{1},\cdots,v_{n}):v_{i}\in\R\}$. Un espacio vectorial sobre 
$\R$ de dimensión n, con el producto escalar dado por
\begin{equation*}
    \ip{v}{w}=\ds\sum_{i=1}^{n}v_{i}w_{i}\hspace{4mm}\text{con }v,w\in\R^{n}
\end{equation*}

\begin{dfn}
    Una curva parametrizada en $\R^{n}$ es una función continua $\alpha:I\subseteq\R\to\R^{n}$ con 
    $I$ un intervalo abierto. Escribimos $\alpha(t)=(\alpha_{1}(t),\cdots,\alpha_{n}(t))$.
\end{dfn}

\noindent Diremos que $\alpha$ es diferenciable si sus funciones coordenadas 
$\alpha_{i}\in\mathcal{C}^{\infty}$. En tal caso, el vector 
$\alpha'(t)=(\alpha_{1}'(t),\cdots,\alpha_{n}'(t))$ se llama vector tangente a la curva $\alpha$ en
$t\in I$

\begin{dfn}
    La traza de una curva parametrizada $\alpha:I\subseteq\R\to\R^{n}$ es $\alpha(I)=im(\alpha)$.
\end{dfn}

\noindent\textbf{Ejemplos}
\begin{enumerate}
    \item Si $p,v\in\R^{n}$ con $v\neq0$, la curva parametrizada $\alpha(t)=tv+p$ con $t\in\R$ 
    que describe una recta que pasa por $p=\alpha(0)$ con vector tangente $\alpha'(t)=v$.
    \item Sea $\beta:\R\to\R^{3}$ dada por $b(t):=t^{3}\cdot\overrightarrow{e}_{1}$ es una curva
    parametrizada diferenciable con $\beta'(t)=3t^{2}\cdot\overrightarrow{e}_{1}$.
    \item Sea $p\in\R^{2}$ y $r>0$ consideramos $\alpha(t)=(rcos(t),rsen(t))+p$, una curva
    parametrizada diferenciable cuya traza es $\alpha(\R)=\{(x,y)\in\R^{2}:\abs{(x,y)}{p}=r\}$
    \item Sean $a,b\in\R\setminus{\{0\}}$. La curva parametrizada $\alpha:\R\to\R^{3}$ dada por
    $\alpha(t)=(acos(t),asen(t),bt)$ con $t\in\R$ se llama una helice circular. 
    Además $\alpha'(t)=(-asen(t),acos(t),b)$.
    \item Sea $\alpha:\R\to\R^{2}$ dada por $\alpha(t)=(t^{3}-4t,t^{2}-4)$ es una curva 
    parametrizada diferenciable con $\alpha(-2)=\alpha(2)=0$, pero $\alpha'(-2)\neq\alpha(2)$.
\end{enumerate}

\subsection{Longitud y Parametro de Arco}
\noindent Sea $\alpha:I\subseteq\R\to\R^{n}$ una curva parametrizada, consideremos $[a,b]\subseteq I$.
Buscamos medir la longitud de $\alpha([a,b])$. Una estrategia, dada una partición $P:=\{a=t_{0}
<t_{1}<\cdots<t_{n}=b\}$ de $[a,b]$ calculamos

\begin{equation*}
    \ds\sum_{i=1}^{k}\abs{\alpha(t_{i})-\alpha(t_{i-1})}=:L_{a}^{b}(\alpha,P)
\end{equation*}

\noindent esta suma corresponde a la longitud de una curva poligonal que pasa por los puntos 
$\alpha(t_{i})$. Si $Q\supseteq P$ es otra partición de $[s,b]$, entonces 
$L_{a}^{b}(\alpha,Q)\geq L_{a}^{b}(\alpha,P)$.

\begin{dfn}
    La longitud de una curva parametrizada $\alpha$ sobre $[a,b]\subseteq I$ es
    \begin{equation*}
        L_{a}^{b}(\alpha)=sup\{L_{a}^{b}(\alpha,P):P\text{ es partición de }[a,b]\}.
    \end{equation*}
\end{dfn}

\noindent Si $\alpha$ es diferenciable sobre $[a,b]$ y hacemos $\abs{P}=max\{t_{i}-t_{i-1}\}$ muy
pequeña, esperariamos que $\abs{\alpha(t_{i})-\alpha(t_{i-1})}\approx
\abs{\alpha'(\overline{t_{i}})}(t_{i}-t_{i-1})$.

\begin{prop}
    Si $\alpha:I\subseteq\R\to\R^{n}$ es una curva parametrizada diferenciable sobre $[a,b]
    \subseteq I$, entonces 
    \begin{equation*}
        L_{a}^{b}(\alpha)=\int_{a}^{b}\abs{\alpha'(t)}dt
    \end{equation*}
    (Para la demostración revisar Montiel-Ros, página 5)
\end{prop}

\begin{cor}
    Tenemos que $\abs{\alpha(a)-\alpha(b)}\leq L_{a}^{b}(\alpha)$.
\end{cor}

\begin{cor}
    Si $F:\R^{n}\to\R^{n}$ cumple $\abs{DF(p)v}=\abs{v}$ para todo $p,v\in\R^{n}$, entonces
    $L_{a}^{b}(F\circ\alpha)=L_{a}^{b}(\alpha)$.
\end{cor}

\noindent De hecho, $F\circ\alpha: I\to\R^{n}$ es una curva parametrizada diferenciable, con

\begin{equation*}
    \abs{(F\circ\alpha)'(t)}=\abs{DF(\alpha(t))\alpha'(t)}=\abs{\alpha'(t)}
\end{equation*}

\noindent para todo $t\in I$, basta con integrar sobre $[a,b]$. Si $p_{0}\in\R^{n}$ y 
$A:\R^{n}\to\R^{n}$ es una transformación lineal ortogonal , esto es, $\ip{Au}{Av}=\ip{u}{v}$ 
para todo $u,v\in\R^{n}$, entonces $F:\R^{n}\to\R^{n}$ dada por $F(p)=Ap+p_{0}$ cumple

\begin{equation*}
    DF(p)v=\dv{}{t}F(p+tv)\big|_{t=0}=\dv{}{t}(A(p+tv)+p_{0})\big|_{t=0}=Av
\end{equation*}

\noindent Por lo tanto $\abs{DF(p)v}=\abs{Av}=\abs{v}$.

\begin{cor}
    Si $h:J\subseteq\R\to I\subseteq\R$ es un difeomorfismo y $\alpha: I\to\R$ es una curva 
    parametrizada diferenciable, entonces
    \begin{equation*}
        L_{a}^{b}(\alpha\circ h)=L_{c}^{d}(\alpha)
    \end{equation*}
    donde $h([a,b])=[c,d]$ para todo $[a,b]\subseteq J$.
\end{cor}

\noindent Por regla de la cadena tenemos que $(\alpha\circ h')(t)=h'(t)\alpha'(h(t))$. La curva 
$\alpha\circ h$ tiene la misma traza que $\alpha$, en efecto $(\alpha\circ h)(J)=\alpha(h(J))=
\alpha(I)$. Decimos que $\alpha\circ h$ es una reparametrización de la curva $alpha$.

\begin{dem}
    Como $h$ y $h^{-1}$ son diferenciables, se tiene que $h'(t)\neq0$ para todo $t\in J$. Veamos
    que
    \begin{equation*}
        1=\dv{}{t}(t)=(h^{-1}\circ h)'(t)=(h^{-1})'(h(t))h'(t)
    \end{equation*}
    Luego como J es un intervalo y $h'$ es continua, tenemos que $h'<0$ o $h>0$.
    \begin{itemize}
        
        \item Si $h'<0$, entonces $h(a)=c$, $h(b)=d$,
        \begin{equation*}
            \int_{a}^{b}\abs{(\alpha\circ h)'(t)}dt=\int_{a}^{b}\abs{\alpha'(h(t))}\abs{h'(t)}dt=
            \int_{c}^{d}\abs{\alpha'(s)}ds=L_{c}^{d}(\alpha)
        \end{equation*}

        \item Si $h'>0$, entonces $h(b)=c$, $h(a)=d$,
        \begin{equation*}
            \int_{a}^{b}\abs{(\alpha\circ h)'(t)}dt=\int_{a}^{b}\abs{\alpha'(h(t))}\abs{h'(t)}dt=
            \int_{d}^{c}-\abs{\alpha'(s)}ds=\int_{c}^{d}\abs{\alpha'(s)}ds=L_{c}^{d}(\alpha)
        \end{equation*}

    \end{itemize}
\end{dem}

\begin{dfn}
    Se dice que una curva parametrizada diferenciable $\alpha: I\subseteq\R\to\R^{n}$ es regular si
    $\alpha'(t)\neq0$ para todo $t\in I$. Si además $\abs{\alpha'(t)}=1$ para todo $t\in I$ se dice
    que $\alpha$ esta parametrizada por el arco.
\end{dfn}

\noindent Una curva $\alpha$ parametrizada por el arco tienen las siguientes propiedades

\begin{itemize}
    \item $\alpha'(t)$ es ortogonal a $\alpha''(t)$ para todo $t\in I$, en efecto
    
    \begin{equation*}
        0=\dv{}{t}(\abs{\alpha'(t)}^{2})=\dv{}{t}(\ip{\alpha'(t)}{\alpha'(t)})=2\ip{\alpha'(t)}
        {\alpha''(t)}
    \end{equation*}

    \item Se tiene que $L_{a}^{b}(\alpha)=\int_{a}^{b}\abs{\alpha'(t)}dt=b-a$.
\end{itemize}

\begin{teo}
    Si $\alpha: I\to\R^{n}$ es una curva parametrizada diferenciable regular, entonces $\alpha$
    admite una parametrización por arco. Concretamente , si $t_{0}\in I$ y definimos $s: I\to\R$ 
    por
    \begin{equation*}
        s(t):=\int_{t_{0}}^{t}\abs{\alpha'(t)}dt
    \end{equation*}
    entonces $s$ es un difeomorfismo sobre $J\subseteq\R$ y $\alpha\circ s^{-1}:J\to\R^{n}$ esta
    parametrizada por el arco.
\end{teo}

\begin{dem}
    Por TFC, sabemos que $s$ es diferenciable, mas aun, $s'(t)=\abs{\alpha'(t)}$ para todo 
    $t\in I$. Luego, $s'>0$, es decir, $s$ es creciente y $s(I)=J$ es un intervalo abierto. Además,
    por teorema de la función inversa, vemos que

    \begin{equation*}
        (s^{-1})'(r)=\frac{1}{s'(s^{-1}(r))}=\frac{1}{\abs{\alpha'(s^{-1}(r))}}\hspace{4mm}
        \forall r\in J
    \end{equation*}

    \noindent Por lo tanto $\abs{(\alpha\circ s^{-1})'(r)}=1$ para todo $r\in J$, luego 
    $\alpha\circ s^{-1}$ esta parametrizada por el arco.
\end{dem}

\noindent\textbf{Ejemplos}
\begin{enumerate}
    \item Sea $\alpha(t)=tv+p_{0}$ con $p_{0},v\in\R^{n}$ y $v\neq0$. Como $\alpha'(t)=v$, tenemos
    \begin{equation*}
        s(t)=\int_{0}^{t}\abs{v}dx=t\abs{v}
    \end{equation*}
    entonces $\alpha\circ s^{-1}(x)=x\cdot\frac{v}{\abs{v}}+p_{0}$ es una parametrización por el
    arco de $\alpha$.

    \item Consideremos $\alpha(t)=(rcost,rsent)+p_{0}$ con $p_{0}\in\R^{2}$ y $r>0$. Como 
    $\alpha'(t)=(-rsent,rcost)$ entonces $\abs{\alpha'(t)}=r$, tenemos que
    \begin{equation*}
        s(t)=\int_{0}^{t}rdx=rt
    \end{equation*}
    y $(\alpha\circ s^{-1})(x)=(rcos(\frac{x}{r}),rsen(\frac{x}{r}))+p_{0}$ es una curva 
    parametrizada por el arco para $\alpha$.

    \item Definimos $\alpha(t)=(acost,asent,bt)$ con $a,b\in\R\setminus\{0\}$. Como $\abs
    {\alpha'(t)}=\sqrt{a^{2}+b^{2}}$ una curva parametrizada por el arco es
    \begin{equation*}
        (\alpha\circ s^{-1})(x)=\left(acos\left(\frac{x}{\sqrt{a^{2}+b^{2}}}\right),
        asen\left(\frac{x}{\sqrt{a^{2}+b^{2}}}\right),\frac{bt}{\sqrt{a^{2}+b^{2}}}\right)
    \end{equation*}
\end{enumerate}

\subsection{Curvatura de un Curva Regular (Teoría Local de Curvas)}

\noindent\textbf{Notación:} Notamos por $\mathcal{J}$ a la función $\mathcal{J}:\R^{2}\to\R^{2}$
dada por $\mathcal{J}(x,y)=(-y,x)$ que cumple lo siguientes

\begin{itemize}
    \item $\mathcal{J}$ es una transformación lineal ortogonal.
    \item $\ip{u,\mathcal{J}u}=0$ y $\mathcal{J}(\mathcal{J}u)=-u$ para todo $u\in\R^{2}$.
    \item Si $\abs{u}=1$, entonces $\{u,\mathcal{J}u\}$ es una base ortonormal positiva de 
    $\R^{2}$.
    \item Si $A:\R^{2}\to\R^{2}$ es una transformación lineal ortogonal, entonces 
    $\mathcal{J}A=det(A)A\mathcal{J}$.
\end{itemize}

\noindent Nuestro objetivo es asociar a una curva parametrizada regular $\alpha: I\to\R^{n}$ una cantidad 
geometrica, para ello queremos definir una función $K(=K_{\alpha}): I\to\R$ tal que
\begin{enumerate}
    \item $K$ es invariante bajo movimientos rigidos.
    \item $K$ es invariante por parametrizaciones.
    \item $K\equiv0$ si y solo si $\alpha$ corresponde a un segmento de recta.
\end{enumerate}

\noindent Si tenemos $\alpha: I\to\R^{2}$ una curva parametrizada por el arco, definimos por $T: I\to\R^{2}$ 
dada por $T(s):=\alpha'(s)$ y $N: I\to\R^{2}$ como $N(s):=\mathcal{J}T(s)$. Recordemos que 
$\{T(s),N(s)\}$ es una base ortonormal en $\R^{n}$ para cada $s\in I$ (Tiedro de Frenet).
\vspace{4mm}

\noindent Notemos que $N(s)\perp T(s)$ y $T'(s)\perp T(s)$, luego, existe un $k(s)\in\R$ tal que 
$T'(s)=K(s)N(s)$. La función $K_{\alpha}=K: I\to\R$ se llama la curva de $\alpha$. Tomando el 
producto con $N(s)$,
\begin{equation*}
    K(s)=\ip{K(s)N(s)}{N(s)}
\end{equation*}
Por lo tanto $K(s)=\ip{T'(s)}{N(s)}$. Por otro lado, observemos que
\begin{equation*}
    N'(s)=\dv{}{s}\left(\J T(s)\right)=\J\dv{}{s}(T(s))=\J(K(s)N(s))=\J(K(s)\J T(s))=-K(s)T(s)
\end{equation*}

\begin{prop}
    Para una curva parametrizada por el arco $\alpha: I\to\R^{2}$ vale que $T'=KN$ y $N'=-KT$.
\end{prop}

\noindent\textbf{Ejemplos:}
\begin{enumerate}
    \item Una recta parametrizada por el arco $\alpha(s):=s\cdot \frac{v}{\abs{v}}+p_{0}$ con
    $v\in\R^{2}\setminus\{0\}$, tenemos que
    \begin{equation*}
        T(s)=\frac{v}{\abs{v}}\text{ , }N(s)=\frac{\J v}{\abs{v}}=\frac{\J v}{\abs{\J v}}
        \text{ y }K(s)=0\hspace{4mm}\forall s\in\R
    \end{equation*}

    \item Si $\alpha: I\to\R^{2}$ esta parametrizada y $K\equiv0$, entonces $T'(s)=0$ para todo
    $s\in I$, es decir, $\alpha''(s)=0$ para todo $s\in I$. Integrando dos veces concluimos que
    cada coordenada de $\alpha$ es una función lineal, luego $\alpha$ es un segmento de recta.

    \item Sea 
    $\alpha(s):=\left(rcos\left(\frac{s}{r}\right),rsen\left(\frac{s}{r}\right)\right)+p_{0}$,
    entonces
    \begin{equation*}
        T(s)=\left(-sen\left(\frac{s}{r}\right),cos\left(\frac{s}{r}\right)\right)\text{ y }
        N(s)=\left(-cos\left(\frac{s}{r}\right),-sen\left(\frac{s}{r}\right)\right)
    \end{equation*}
    Notemos que
    \begin{equation*}
        T'(s)=\left(-\frac{1}{r}cos\left(\frac{s}{r}\right),-\frac{1}{r}sen\left(
        \frac{s}{r}\right)\right)=\frac{1}{r}N(s)
    \end{equation*}
    Por lo tanto $K(s)=\frac{1}{r}\ip{N(s)}{N(s)}=\frac{1}{r}$.
\end{enumerate}

\noindent Consideremos ahora una curva regular $\beta:\widetilde{I}\to\R^{2}$ y una reparametrización
$\alpha=\beta\circ h: I\to\R^{2}$ parametrizada por el arco, donde $h: I\to\widetilde{I}$ es un
difeomorfismo con $h'>0$. Con esto
\begin{equation*}
    \abs{\beta'(t)}=\abs{(\beta\circ h\circ h^{-1})'(t)}=\abs{(\alpha\circ h^{-1})'(t)}=
    (h^{-1})'(t)
\end{equation*}
Así, definimos el diedro de Frenet de la curva $\alpha$ por
\begin{equation*}
    T_{\beta}(t):=\frac{\beta'(t)}{\abs{\beta'(t)}}=
    \frac{(\alpha\circ h^{-1})'(t)}{\abs{(\alpha\circ h^{-1})'(t)}}=
    \frac{\alpha'(h^{-1}(t))h^{-1}(t)}{\abs{\alpha'(h^{-1}(t))h^{-1}(t)}}=
    T_{\alpha}(h^{-1}(t))
\end{equation*}
Por otro lado
\begin{equation*}
    N_{\beta}=\J T_{\beta}(t)=\J T_{\alpha}(h^{-1}(t))=N_{\alpha}(h^{-1}(t))
\end{equation*}
y definimos la curvatura de la curva $\beta$ por
\begin{equation*}
    K_{\beta}(t):=K_{\alpha}(h^{-1}(t))\text{ , }t\in\widetilde{I}
\end{equation*}
Como $\beta'(t)=\abs{\beta'(t)}T_{\alpha}(h^{-1}(t))$ se tiene que
\begin{equation*}
    \beta''=(\abs{\beta'})'T_{\alpha}\circ h^{-1}+\abs{\beta'}^{2}(T'_{\alpha}\circ h^{-1})
\end{equation*}
y además $N_{\alpha}\circ h^{-1}=\J T_{\beta}=\frac{\J\beta'}{\abs{\beta'}}$ se sigue que
\begin{equation*}
    \frac{\ip{\beta''}{\J\beta'}}{\abs{\beta'}}=
    \ip{(\abs{\beta'})'T_{\alpha}\circ h^{-1}+\abs{\beta'}^{2}(T'_{\alpha}\circ h^{-1})}
    {N_{\alpha}\circ h^{-1}}=\abs{\beta'}^{2}\ip{T'_{\alpha}\circ h^{-1}}{N_{\alpha}\circ h^{-1}}
    =\abs{\beta'}^{2}K_{\alpha}\circ h^{-1}
\end{equation*}
Concluimos que $K_{\beta}=\frac{\ip{\beta''}{\J\beta'}}{\abs{\beta'}^{3}}$.

\begin{prop}
    Sea $\alpha: I\to\R^{2}$ una curva regular, entonces
    \begin{enumerate}
        \item Si $\phi:\widetilde{I}\to I$ es un difeomorfismo entonces $K_{\alpha\circ\phi}=
        sgn(\phi')K_{\alpha}\circ\phi$.
        
        \item Si $F:\R^{2}\to\R^{2}$ es un movimiento rigido, entonces $K_{F\circ\alpha}=
        (detDF)K_{\alpha}$.
    \end{enumerate}
\end{prop}

\begin{dem} Sea $\alpha: I\to\R^{2}$ una curva regular
    \begin{enumerate}
        \item Como $(\alpha\circ\phi)'(t)=\phi'(t)\alpha'(\phi(t))$, se sigue que 
        $\abs{(\alpha\circ\phi)'(t)}=\abs{\phi'(t)}\abs{\alpha'(\phi(t))}$, escrito de otro modo
        
        \begin{equation*}
            \abs{(\alpha\circ\phi)}=sgn(\phi')\cdot\phi'\abs{\alpha'\circ\phi}
        \end{equation*}
        Luego
        \begin{align*}
            K_{\alpha\circ\phi} &= \frac{\ip{(\alpha\circ\phi)''}{\J(\alpha\circ\phi)'}}
            {\abs{(\alpha\circ\phi)'}^{3}}=
            \frac{\ip{\phi''(\alpha'\circ\phi)+(\phi')^{2}\alpha''\circ\phi}
            {\phi'\J(\alpha'\circ\phi)}}{sgn(\phi')(\phi')^{3}\abs{\alpha'\circ\phi}^{3}} \\
            &= \frac{(\phi')^{3}\ip{\alpha''\circ\phi}{\J\alpha'\circ\phi}}
            {(\phi')^{3}\abs{\alpha'\circ\phi}^{3}}sgn(\phi')=sgn(\phi')K_{\alpha}\circ\phi
        \end{align*}
        
        \item Sabemos que $F(p)=Ap+p_{0}$, entonces $DF=A$. Luego,
        \begin{align*}
            \ip{(F\circ\alpha)''}{\J(F\circ\alpha)'} &= \ip{(DF(\alpha)\alpha')'}
            {\J(DF(\alpha)\alpha')}=\ip{(A\alpha')'}{\J(A\alpha')} \\
            &= \ip{A\alpha''}{(detA)A\J\alpha'}=detA\ip{\alpha''}{\J\alpha'}
        \end{align*}
        Además $\abs{(F\circ\alpha)'}=\abs{A\alpha'}=\abs{\alpha'}$. Juntando lo anterior vemos que
        \begin{equation*}
            K_{F\circ\alpha}=\frac{\ip{(F\circ\alpha)''}{\J(F\circ\alpha)'}}
            {\abs{(F\circ\alpha)'}^{3}} = detA\cdot K_{\alpha}
        \end{equation*}
    \end{enumerate}
\end{dem}

\begin{prop}
    Sea $\alpha: I\to\R^{2}$ una curva parametrizada por el arco. Supongamos que existe una función
    diferenciable $\theta: I\to\R$ tal que $T(s)=(cos(\theta(s)),sen(\theta(s)))$. Entonces 
    $K_{\alpha}=\dv{\theta}{s}$.
\end{prop}

\begin{dem}
    Recordemos que
    \begin{equation*}
        K_{\alpha}=\ip{T'_{\alpha}}{\J T_{\alpha}}=
        \ip{\left(-\dv{\theta}{s}sen\theta,\dv{\theta}{s}cos\theta\right)}
        {(-sen\theta,cos\theta)}=\dv{\theta}{s}\abs{(-sen\theta,cos\theta)}^{2}=\dv{\theta}{s}
    \end{equation*}
\end{dem}

\begin{teo}
    Sea $K: I\to\R$ una función diferenciable, entonces existe una unica curva parametrizada por 
    el arco $\alpha: I\to\R$, salvo por movimientos rigidos, tal que $K_{\alpha}=K$.
\end{teo}

\subsection{Teoría Local de Curvas en el Espacio}

\begin{dfn}
    Sea $\alpha: I\to\R^{3}$ parametrizada por el arco. La curvatura de $\alpha$ en $s\in I$ es
    \begin{equation*}
        K_{\alpha}:=\abs{T'(s)}
    \end{equation*}
\end{dfn}

\noindent\textbf{Observación:} Para curvas en $\R^{3}$, $K_{\alpha}\geq0$. Además, 
$K_{\alpha}\equiv0$ si y solo si $\alpha$ es un segmento de recta.

\begin{dfn}
    Sea $\alpha: I\to\R^{3}$ parametrizada por el arco, tal que $K_{\alpha}>0$. Definimos
    \begin{equation*}
        N(s):=\frac{T'(s)}{\abs{T'(s)}}
    \end{equation*}
\end{dfn}

\noindent\textbf{Observación:} Como $T(s)\perp T'(s)$, pues $\abs{T}=1$, está definición se condice
con el caso en $\R^{2}$, además de manera directa, obtenemos que $K_{\alpha}N(s)=T(s)$.

\begin{dfn}
    Sea $\alpha: I\to\R^{3}$ parametrizada por el arco. Definimos el vector binormal de $\alpha$
    en $s\in I$ por
    \begin{equation*}
        B(s)=T(s)\times N(s)
    \end{equation*}
\end{dfn}

\noindent\textbf{Observación:} Por definición del producto cruz el conjunto $\{T,N,B\}$ es una base
ortonormal positiva de $\R^{3}$ para todo $s\in I$ llamada el tiedro de Frenet de $\alpha$ en 
$s\in I$.
\vspace{4mm}

\noindent Notemos que $B'(s)=\dv{}{s}(T(s)\times N(s))=T'(s)\times N(s)+T(s)\times N'(s)=
T(s)\times N'(s)$. Además, $\abs{B}=\abs{T}\abs{N}=1$ y por lo tanto $B'\perp B$, por otro lado
$\ip{B'}{T}=\ip{T\times N'}{T}=0$, osea $B'\perp T$. Por lo tanto, existe $\tau(s)\in I$ tal quiero
\begin{equation*}
    B'(s)=\tau(s)N(s)
\end{equation*}
Se dice que $\tau(s)=:\tau_{\alpha}(s)$ es la torsión de $\alpha$ en $s\in I$. Finalmente, como
$N'\perp N$, tenemos que
\begin{equation*}
    N'(s)=aT(s)+bB(s)
\end{equation*}
donde

\begin{align*}
    a\ip{T}{T} &= \ip{N'}{T}=\ip{N'}{T}+\ip{N}{T'}-\ip{N}{T'} \\
    &= \dv{}{s}\ip{N}{T}-\ip{N}{T'}=-\ip{N}{KN}=-K
\end{align*}
y similarmente obtenemos que $b=\ip{N'}{B}=-\tau(s)$.

\begin{prop}
    Ecuaciones de Frenet-Serret
    \begin{itemize}
        \item $T'(s)=K(s)N(s)$
        \item $N'(s)=-K(s)T(s)-\tau(s) B(s)$
        \item $B'(s)=\tau(s) N(s)$
    \end{itemize}
\end{prop}

\noindent\textbf{Ejemplos:}
\begin{enumerate}
    \item Sea $\alpha: I\to\R^{3}$ una curva parametrizada por el arco. Supongamos que 
    $\alpha(I)\subseteq P$ con $P$ un plano. Podemos describir el plano con la ecuación
    $\ip{x-p_{0}}{u}=0$, donde $p_{0},u\in\R^{3}$ con $u$ unitario y perpendicular al plano. 
    Entonces $\ip{\alpha(s)-p_{0}}{u}=0$ para todo $s\in I$, derivando vemos que
    \begin{equation*}
        \ip{\alpha'(s)}{u}=\ip{T(s)}{u}=0\hspace{4mm}\forall s\in I
    \end{equation*}
    En ese caso, $K(s)$ es el valor absoluto de la curvatura de $\alpha$ como una curva plana.
    Supongamos que $K(s)>0$ para todo $s\in I$. Entonces
    \begin{equation*}
        0=\dv{}{s}=\ip{T}{u}=\ip{T'}{u}=K(s)\ip{N}{u}
    \end{equation*}
    lo que implica que $N\perp u$ para todo $s\in I$. Luego, $B(s)=\pm u$ para todo $s\in I$, se
    sigue que $\tau(s)=\ip{B'}{N}=0$.

    \item Supongamos que $\alpha$ es una curva parametrizada por el arco tal que 
    $\tau_{\alpha}\equiv0$, entonces $B'=\tau\cdot N=0$ para todo $s\in I$ y por lo tanto $B=u$,
    con $u\in\R^{3}$ y $\abs{u}=1$, así $T\times N=u$ para todo $s\in I$.
    \vspace{4mm}

    \noindent Ahora, usando las ecuaciones de frenet vemos que $T\perp u$ y $N\perp u$ para todo
    $s\in I$ y concluimos que
    \begin{equation*}
        \ip{\alpha(s)-\alpha(s_{0})}{u}=\ip{\int_{s_{0}}^{s}T(x)dx}{u}=
        \int_{s_{0}}^{s}\ip{T(x)}{u}dx=0\hspace{4mm}\forall s\in I
    \end{equation*}
\end{enumerate}

\begin{prop}
    Sea $\alpha: I\to\R^{3}$ una curva parametrizada por el arco, $p_{0}\in\R^{3}$, 
    $A:\R^{3}\to\R^{3}$ lineal, ortogonal y positiva. Sea $F:\R^{3}\to\R^{3}$ con 
    $F(p)=Ap+p_{0}$. Entonces
    \begin{align*}
        & K_{F\circ\alpha}=K_{\alpha}\hspace{4mm}\text{,}\hspace{4mm}\tau_{F\circ\alpha}=
        \tau_{\alpha} \\
        & T_{F\circ\alpha}=AT_{\alpha}\hspace{4mm}\text{,}\hspace{4mm}
        N_{F\circ\alpha}=AN_{\alpha}\hspace{4mm}\text{,}\hspace{4mm}
        B_{F\circ\alpha}=AB_{\alpha}
    \end{align*}
\end{prop}

\noindent Podemos extender las definiciones de curvatura, torsión y del tiedro de frenet para
curvas regulares $\beta: I\to\R^{3}$ por
\begin{equation*}
    K_{\beta}(t):=K_{\alpha}(h^{-1}(t))
\end{equation*}
donde $\alpha=\beta\circ h$ es una parametrización por el arco, con $h$ difeomorfismo, $h'>0$ y
$K_{\beta}>0$. Se cumple lo siguiente
\begin{itemize}
    \item $T_{\beta}(t)=T_{\alpha}(h^{-1}(t))$
    \item $N_{\beta}(t)=N_{\alpha}(h^{-1}(t))$
    \item $B_{\beta}(t)=B_{\alpha}(h^{-1}(t))$
    \item $\tau_{\beta}(t)=\tau_{\alpha}(h^{-1}(t))$
\end{itemize}

\begin{prop}
    Sea $\beta: I\to\R^{3}$ una curva regular, entonces
    \begin{enumerate}
        \item $K_{\beta}=\dfrac{\abs{\beta'\times\beta''}}{\abs{\beta'}^{3}}$
        \item $\tau_{\beta}=\dfrac{-det(\beta',\beta'',\beta''')}{\abs{\beta'\times\beta''}^{2}}=
        -\dfrac{\ip{\beta'}{\beta''\times\beta'''}}{\abs{\beta'\times\beta''}^{2}}$
        \item $T_{\beta}=\dfrac{\beta'}{\abs{\beta'}}$
        \item $B_{\beta}=\dfrac{\beta'\times\beta''}{\abs{\beta'\times\beta''}}$
        \item $N_{\beta}=\dfrac{\abs{\beta'}^{2}\beta''-\ip{\beta'}{\beta''}\beta'}
        {\abs{\abs{\beta'}^{2}\beta''-\ip{\beta'}{\beta''}\beta'}}$
    \end{enumerate}
\end{prop}

\begin{teo}
    (Teorema Fundamental de las curvas en el Espacio)
    
    Sea $K,\tau: I\subseteq\R\to\R$ funciones diferenciables con $K(s)>0$ para todo $s\in I$.
    Entonces existe $\alpha:I\to\R^{3}$ parametrizada por el arco tal que
    \begin{equation*}
        K_{\alpha}=K\hspace{4mm}\text{y}\hspace{4mm}\tau_{\alpha}=\tau
    \end{equation*}
    Además, si $\beta:I\to\R^{3}$ es parametrizada por el arco tal que $K_{\beta}=K$ y 
    $\tau_{\beta}=\tau$. Entonces existe un movimiento rigido $F:\R^{3}\to\R^{3}$ tal que
    $F\circ\beta=\alpha$.
\end{teo}

\begin{dem}
    El sistema
    \begin{equation*}
        (FS):\begin{pmatrix}
            T \\ N \\ B
        \end{pmatrix}'
        =A
        \begin{pmatrix}
            T \\ N \\ B
        \end{pmatrix}
        \hspace{4mm}\text{y}\hspace{4mm}A(s)=
        \begin{pmatrix}
            0 & K(s) & 0 \\
            -K(s) & 0 & -\tau(s) \\
            0 & \tau(s) & 0
        \end{pmatrix}
    \end{equation*}
    tiene solución única para cada condición inicial $T(s_{0})=T_{0}$, $N(s_{0})=N_{0}$ y 
    $B(s_{0})=B_{0}$. Sea $\{T_{0},N_{0},B_{0}\}$ una base ortonormal de $\R^{3}$. Consideremos
    \begin{equation*}
        M(s)=
        \begin{pmatrix}
            \ip{T}{T} & \ip{T}{N} & \ip{T}{B} \\
            \ip{N}{T} & \ip{N}{N} & \ip{N}{B} \\
            \ip{B}{T} & \ip{B}{N} & \ip{B}{B}
        \end{pmatrix}
        =
        \begin{pmatrix}
            T & N & B
        \end{pmatrix}^{T}\cdot
        \begin{pmatrix}
            T & N & B
        \end{pmatrix}
    \end{equation*}
    Por otro lado
    \begin{align*}
        M'(s) &=
        \begin{pmatrix}
            T & N & B
        \end{pmatrix}'^{T}\cdot
        \begin{pmatrix}
            T & N & B
        \end{pmatrix}+
        \begin{pmatrix}
            T & N & B
        \end{pmatrix}^{T}\cdot
        \begin{pmatrix}
            T & N & B
        \end{pmatrix}' \\
        &= A
        \begin{pmatrix}
            T & N & B
        \end{pmatrix}^{T}\cdot
        \begin{pmatrix}
            T & N & B
        \end{pmatrix}+
        \begin{pmatrix}
            T & N & B
        \end{pmatrix}^{T}
        \begin{pmatrix}
            T & N & B
        \end{pmatrix}A^{T} \\
        &= AM-MA
    \end{align*}
    La matriz $M_{0}(s)=I_{3}$ con $s\in I$ es una solución con $M_{0}(s_{0})=I_{3}=M(s_{0})$ 
    (pues $T_{0},N_{0},B_{0}$ son ortonormales). Por unicidad de la solución $M(s)\equiv I_{3}$
    para todo $s\in I$.
    \vspace{4mm}

    \noindent La matriz $(T\hspace{1mm}N\hspace{1mm}B)$ tiene determinante $1$ o $-1$. Sin embargo,
    el determinante es una función continua, luego es constante, pues su dominio es $I$, un conexo.
    Como vale $1$ en $s=s_{0}$ pues $T_{0},N_{0},B_{0}$ es positiva, vale $1$ sobre $I$.
    \vspace{4mm}

    \noindent Definimos $\alpha:I\to\R^{3}$ por
    \begin{equation*}
        \alpha(x)=\int_{s_{0}}^{s}T(x)dx
    \end{equation*}
    Por TFC, $\alpha'(s)=T(s)$ unitario, luego $\alpha$ es una curva parametrizada por el arco.
    Además
    \begin{equation*}
        K_{\alpha}(s)=\abs{T'(s)}=\abs{K(s)N(s)}=K(s)\abs{N(s)}=K(s)\hspace{4mm}\forall s\in I
    \end{equation*}
    De ahí,
    \begin{equation*}
        N_{\alpha}(s)=\frac{T'_{\alpha}(s)}{\abs{T'_{\alpha}(s)}}=\frac{T'(s)}{\abs{T'(s)}}=
        \frac{K(s)N(s)}{\abs{K(s)N(s)}}=N(s)
    \end{equation*}
    y $B_{\alpha}(s)=T_{\alpha}(s)\times N_{\alpha}(s)=T(s)\times N(s)=B(s)$, pues $T(s),N(s),B(s)$
    es base ortonormal positiva. Por tanto,
    \begin{equation*}
        \tau_{\alpha}=\ip{B'_{\alpha}(s)}{N_{\alpha}(s)}=\ip{B'(s)}{N(s)}=\ip(\tau N,N)=\tau(s)
    \end{equation*}
    Sea $A:\R^{3}\to\R^{3}$ ortogonal tal que
    \begin{align*}
        & AT_{\beta}(s_{0})=T_{\alpha}(s_{0}) \\
        & AN_{\beta}(s_{0})=N_{\alpha}(s_{0}) \\
        & AB_{\beta}(s_{0})=B_{\alpha}(s_{0})
    \end{align*}
    y $p_{0}=\alpha(s_{0})-A\beta(s_{0})$. Luego, $F:\R^{3}\to\R^{3}$ con $F(p)=Ap+p_{0}$. Defina
    $\gamma=F\circ\beta:I\to\R^{3}$. Queremos ver que $\gamma\equiv\alpha$. Como $F$ es movimiento
    rigido $\alpha$ y $\gamma$ tienen curvatura $K$ y torsión $\tau$ y tiedro
    \begin{align*}
        & T_{\gamma}=T_{F\circ\beta}=AT_{\beta} \\
        & N_{\gamma}=N_{F\circ\beta}=AN_{\beta} \\
        & B_{\gamma}=B_{F\circ\beta}=AB_{\beta}
    \end{align*}
    Luego
    $f(s)=\abs{T_{\gamma}(s)-T_{\alpha}(s)}^{2}+
    \abs{N_{\gamma}(s)-N_{\alpha}(s)}^{2}+\abs{B_{\gamma}(s)-B_{\alpha}(s)}^{2}$
    vale $0$ en $s=s_{0}$. Por otro lado
    \begin{equation*}
        f'(s)=2\ip{T_{\gamma}-T_{\alpha}}{T'_{\gamma}-T'_{\alpha}}
        +2\ip{N_{\gamma}-N_{\alpha}}{N'_{\gamma}-N'_{\alpha}}
        +2\ip{B_{\gamma}-B_{\alpha}}{B'_{\gamma}-B'_{\alpha}}=0\hspace{4mm}\forall s\in I
    \end{equation*}
    por lo tanto $f$ es constante y por lo mencionado $f\equiv0$. De este modo, 
    $\gamma'=T_{\gamma}\equiv T_{\alpha}=\alpha'$. Pero 
    \begin{equation*}
        \gamma(s_{0})=F(\beta{s_{0}})=A\beta(s_{0})+p_{0}=\alpha(s_{0})
    \end{equation*}
\end{dem}

\newpage
\section{Superficies Regulares}
\begin{dfn}
    Sea $\Sigma\subseteq\R^{3}$, decimos que $\Sigma$ es una superficie regular si para todo
    $p\in\Sigma$ existe un abierto $V\subseteq\R^{3}$ con $p\in V$ y una función diferenciable
    \begin{equation*}
        \varphi:\mathcal{V}\subseteq\R^{2}\to\R^{3}
    \end{equation*}
    tal que
    \begin{itemize}
        \item $\varphi(\mathcal{V})=V\cap\Sigma$
        \item $\varphi$ es homeomorfismo de $\mathcal{V}$ sobre $V\cap\Sigma$
        \item $D\varphi(q):\R^{3}\to\R^{3}$ es inyectiva, es decir, si $\varphi=\varphi(u,v)$, 
        entonces
        \begin{align*}
            & D\varphi(q)\cdot e_{1}=\dv{}{t}\varphi(q+te_{1})\big|_{t=0}=\varphi_{u}(q) \\
            & D\varphi(q)\cdot e_{2}=\dv{}{t}\varphi(q+te_{2})\big|_{t=0}=\varphi_{v}(q)
        \end{align*}
        son linealmente independientes, en otras palabras 
        $\varphi_{u}(q)\times\varphi_{v}(q)\neq0$.
    \end{itemize}
\end{dfn}

%\printbibliography % Quitar el comentado si quiero usar bibliografia

\end{document}
