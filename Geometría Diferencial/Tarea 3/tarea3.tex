\documentclass{article}
\usepackage{hyperref}
\usepackage{Style}

\nocite{*} % Comentar si quiero citar
%\addbibresource{bibliografia.bib} % Quitar el comentado si quiero usar bibliografia

\begin{document}

\begin{minipage}{2.5cm}
    \includegraphics[width=2cm]{imagen_puc.jpg}
\end{minipage}
\begin{minipage}{14cm}
    {\sc Pontificia Universidad Católica de Chile\\
    Facultad de Matemáticas\\
    Departamento de Matemática\\
    Profesor: Pedro Gaspar -- Estudiante: Benjamín Mateluna}
\end{minipage}
\vspace{1ex}

{\centerline{\bf Geomtría Diferencial - MAT2860}
\centerline{\bf Tarea I3}}
\centerline{\bf 26 de junio de 2025}

\section*{Problema 1}
Probaremos un resultado previo sobre transporte paralelo y curvas geodesicas en la esfera.
\begin{lema}
    Sea $\alpha:I\to\mathbb{S}^{2}$ una curva de la forma $\alpha(t)=usen(t)+vcos(t)$ donde $u,v$ 
    son ortonormales, entonces $P_{\alpha,t_{0},t_{1}}(w)=w+\ip{w}{\alpha'(t_{0})}(\alpha'(t_{1})
    -\alpha'(t_{0}))$.
\end{lema}
\begin{dem}
    Empezamos observando que $\alpha$ es una curva geodésica. Notemos que 
    $\ip{\alpha'(t)}{\alpha(t)}=0$ para todo $t\in I$ y además $\alpha''=-\alpha$. Sea 
    $w\in T_{p}\mathbb{S}^{2}$ donde $p=\alpha(t_{0})$ y sea $Y:I\to\mathbb{S}^{2}$ el único campo 
    paralelo a lo largo de $\alpha$ tal que $Y(t_{0})=w$, como $Y$ en particular es campo tangente 
    se tiene que
    \begin{equation*}
        Y(t)\in T_{\alpha(t)}\mathbb{S}^{2}=(\alpha(t))^{\perp}
    \end{equation*}
    es decir, $\ip{Y(t)}{\alpha(t)}=0$ para todo $t\in I$, lo que implica que $\ip{Y}{\alpha'}
    =-\ip{Y'}{\alpha}$. Notemos que
    \begin{equation*}
        0=\nabla_{\alpha'}Y=Y'-\ip{Y'}{\alpha}\alpha
    \end{equation*}
    tomando producto interno con $\alpha'$ vemos que
    \begin{equation*}
        \ip{Y'}{\alpha'}=\ip{Y'}{\alpha}\ip{\alpha}{\alpha'}=0
    \end{equation*}
    de este modo
    \begin{equation*}
        \dv{}{t}(\ip{Y}{\alpha'})=\ip{Y'}{\alpha'}+\ip{Y}{\alpha''}=-\ip{Y}{\alpha}=0
    \end{equation*}
    luego la función $\ip{Y}{\alpha'}$ es constante, evaluando en $t_{0}$ vemos que 
    $\ip{Y'}{\alpha}=-\ip{Y}{\alpha'}\equiv-\ip{w}{\alpha'(t_{0})}$. Tenemos que
    \begin{equation*}
        Y'(t)=\ip{w}{\alpha'(t_{0})}\alpha''(t)
    \end{equation*}
    integrando a ambos lados vemos que
    \begin{equation*}
        Y(t)-Y(t_{0})=Y(t)-w=\ip{w}{\alpha'(t_{0})}(\alpha'(t)-\alpha'(t_{0}))
    \end{equation*}
    lo que concluye la demostración.
\end{dem}

\begin{enumerate}
    \item Notemos que
    \begin{equation*}
        \gamma_{1}(s)=\begin{pmatrix}
            0 \\ 0 \\ 1
        \end{pmatrix}cos(s)+\begin{pmatrix}
            1 \\ 0 \\ 0
        \end{pmatrix}sen(s)\hspace{4mm}\text{y}\hspace{4mm}\gamma_{2}(s)=\begin{pmatrix}
            0 \\ 0 \\ 1
        \end{pmatrix}cos(s)+\begin{pmatrix}
            cos(\theta) \\ sen(\theta) \\ 0
        \end{pmatrix}sen(s)
    \end{equation*}
    además, los vectores que describen cada curva son ortonormales. Por otro lado se tiene que
    \begin{equation*}
        \gamma_{1}'(s)=(cos(s),0,sen(s))\hspace{4mm}\text{y}\hspace{4mm}\gamma_{2}'(s)
        =(cos(\theta)cos(s),sen(\theta)cos(s),-sen(s))
    \end{equation*}
    Dado $w_{0}\in T_{\gamma_{1}(0)}\mathbb{S}^{2}=T_{\gamma_{2}(0)}\mathbb{S}^{2}$ entonces 
    $w_{0}=(x_{0},y_{0},0)$, así, por el lema tenemos lo siguiente
    \begin{align*}
        w_{1}=P_{\gamma_{1},0,\pi}(w_{0}) &= w_{0}+\ip{w_{0}}{\gamma_{1}'(0)}
        (\gamma_{1}'(\pi)-\gamma_{1}'(0)) \\[2mm]
        &= w_{0}+\ip{w_{0}}{
            \begin{pmatrix}
                1 \\ 0 \\ 0
            \end{pmatrix}
        }\left(\begin{pmatrix}
            -1 \\ 0 \\ 0
        \end{pmatrix}-\begin{pmatrix}
            1 \\ 0 \\ 0
        \end{pmatrix}\right)=w_{0}+x_{0}\begin{pmatrix}
            -2 \\ 0 \\ 0
        \end{pmatrix}=\begin{pmatrix}
            -x_{0} \\ y_{0} \\ 0
        \end{pmatrix}
    \end{align*}
    por otro lado
    \begin{align*}
        w_{2}=P_{\gamma_{2},0,\pi}(w_{0}) &= w_{0}+\ip{w_{0}}{\gamma_{2}'(0)}
        (\gamma_{2}'(\pi)-\gamma_{2}'(0)) \\[2mm]
        &= w_{0}+\ip{w_{0}}{
            \begin{pmatrix}
                cos(\theta) \\ sen(\theta) \\ 0
            \end{pmatrix}
        }\left(\begin{pmatrix}
            -cos(\theta) \\ -sen(\theta) \\ 0
        \end{pmatrix}-\begin{pmatrix}
            cos(\theta) \\ sen(\theta) \\ 0
        \end{pmatrix}\right)\\
        &= \begin{pmatrix}
            x_{0}-2x_{0}cos^{2}(\theta)+y_{0}cos(\theta)sen(\theta) \\
            0
        \end{pmatrix}
    \end{align*}

    \item 
\end{enumerate}

\section*{Problema 2}
\noindent El toro, puede ser parametrizado por $X:(0,2\pi)\times(0,2\pi)\to\R^{3}$ dada por
\begin{equation*}
    X(u,v)=((R+rcos(u))cos(v),(R+rcos(u))sen(v),rsen(u))
\end{equation*}
con derivadas parciales
\begin{align*}
    X_{u}(u,v) &= (-rsen(u)cos(v),-rsen(u)sen(v),rcos(u)) \\
    X_{v}(u,v) &= (-(R+rcos(u))sen(v),(R+rcos(u))cos(v),0)
\end{align*}
entonces el campo normal definido por la parametrización es
\begin{equation*}
    (X_{u}\times X_{v})(u,v)=(-(R+rcos(u))rcos(u)cos(v),-(R+rcos(u))rcos(u)sen(v),
    -(R+rcos(u))rsen(v))
\end{equation*}
y además $\abs{X_{u}\times X_{v}}=r(R+rcos(u))$. Sea $u_{0}\in[0,2\pi]$ tal que $rsen(u_{0})=c$, 
que existe pues $c\in[-r,r]$. Consideramos $\eta:=(R+rcos(u_{0}))$, definimos la curva p.p.a 
$\alpha:(0,2\pi\eta)\to\R^{3}$ dada por
\begin{equation*}
    \alpha(t):=X\left(u_{0},\frac{t}{\eta}\right)=\left(\eta cos\left(\frac{t}{\eta}\right),
    \eta sen\left(\frac{t}{\eta}\right),rsen(u_{0})\right)
\end{equation*}
entonces, suponiendo que la parametrización es positiva
\begin{equation*}
    (N\circ\alpha)(t)=(N^{x}\circ X^{-1}\circ\alpha)(t)=N^{x}\left(u_{0},\frac{t}{\eta}\right)
    =\left(cos(u_{0})cos\left(\frac{t}{\eta}\right),cos(u_{0})sen\left(\frac{t}{\eta}\right),
    sen(u_{0})\right)
\end{equation*}
Por otro lado vemos que
\begin{equation*}
    \alpha'=\left(-sen\left(\frac{t}{\eta}\right),cos\left(\frac{t}{\eta}\right),0\right)
    \hspace{4mm}\text{y entonces}\hspace{4mm}
    \alpha''(t)=-\frac{1}{\eta}\left(cos\left(\frac{t}{\eta}\right),
    sen\left(\frac{t}{\eta}\right),0\right)
\end{equation*}
Nuestro objetivo ahora es determinar $\nabla_{\alpha'}\alpha'(t)$, para ello observemos que
\begin{equation*}
    \ip{\alpha''}{N\circ\alpha}=\frac{1}{\eta}\begin{pmatrix}
        cos(t/\eta) \\ sen(t/\eta) \\ 0
    \end{pmatrix}\cdot\begin{pmatrix}
        cos(u_{0})cos(t/\eta) \\ cos(u_{0})sen(t/\eta) \\ sen(u_{0})
    \end{pmatrix}=\frac{cos(u_{0})}{\eta}
\end{equation*}
de este modo
\begin{align*}
    \nabla_{\alpha'}\alpha' &= \alpha''-\ip{\alpha''}{N\circ\alpha}N\circ\alpha
    =-\frac{1}{\eta}\begin{pmatrix}
        cos(t/\eta) \\ sen(t/\eta) \\ 0
    \end{pmatrix}+\frac{cos(u_{0})}{\eta}\begin{pmatrix}
        cos(u_{0})cos(t/\eta) \\ cos(u_{0})sen(t/\eta) \\ sen(u_{0})
    \end{pmatrix} \\[2mm]
    &= \frac{1}{\eta}\begin{pmatrix}
        cos(t/\eta)(cos^{2}(u_{0})-1) \\ sen(t/\eta)(cos^{2}(u_{0})-1) \\ cos(u_{0})sen(u_{0})
    \end{pmatrix}
\end{align*}
adicionalmente se tiene que
\begin{equation*}
    N\circ\alpha\times\alpha'=\begin{pmatrix}
        cos(t/\eta)sen(u_{0}) \\ sen(t/\eta)sen(u_{0}) \\ -cos(u_{0})
    \end{pmatrix}
\end{equation*}
por último, se tiene que
\begin{align*}
    K_{g} &= [\nabla_{\alpha'}\alpha']=\frac{1}{\eta}\begin{pmatrix}
        cos(t/\eta)(cos^{2}(u_{0})-1) \\ sen(t/\eta)(cos^{2}(u_{0})-1) \\ cos(u_{0})sen(u_{0})
    \end{pmatrix}\cdot\begin{pmatrix}
        cos(t/\eta)sen(u_{0}) \\ sen(t/\eta)sen(u_{0}) \\ -cos(u_{0})
    \end{pmatrix} \\[2mm]
    &= \frac{1}{\eta}\left((cos^{2}(u_{0})-1)sen(u_{0})-cos^{2}(u_{0})sen(u_{0})\right)
    =\frac{-sen(u_{0})}{R+rcos(u_{0})}
\end{align*}

\section*{Problema 3}

\section*{Problema 4}
\begin{enumerate}
    \item Notemos que
    \begin{align*}
        \omega_{p}(v) &= \ip{v}{\frac{Jp}{\abs{p}^{2}}}=\frac{1}{\abs{p}^{2}}\ip{v}{
            \begin{pmatrix}
                -y \\
                x
            \end{pmatrix}
            }=\frac{1}{\abs{p}^{2}}(-v_{1}y+v_{2}x) \\[2mm]
        &= \frac{1}{\abs{p}^{2}}(-dx(v)y+dy(v)x)=\frac{-y}{x^{2}+y^{2}}dx(v)
        +\frac{x}{x^{2}+y^{2}}dy(v)
    \end{align*}
    conlcuimos que
    \begin{equation*}
        \omega_{p}=\frac{-y}{x^{2}+y^{2}}dx+\frac{x}{x^{2}+y^{2}}dy
    \end{equation*}
    además, para $p\in\R^{3}\setminus\{0\}$ las funciones
    \begin{equation*}
        \omega_{1}:=\frac{-y}{x^{2}+y^{2}}\hspace{4mm}\text{y}
        \hspace{4mm}\omega_{2}:=\frac{x}{x^{2}+y^{2}}
    \end{equation*}
    son diferenciables, y por lo tanto la 1-forma $\omega_{p}$ es diferenciable.
    
    \item Veamos la siguiente expresión
    \begin{align*}
        d\omega &= \left(\pdv{\omega_{1}}{x}dx+\pdv{\omega_{1}}{y}dy\right)\land dx
        +\left(\pdv{\omega_{2}}{x}dx+\pdv{\omega_{2}}{y}dy\right)\land dy
        =\left(\pdv{\omega_{2}}{x}dx-\pdv{\omega_{1}}{y}dy\right)dx\land dy \\[2mm]
        &= \left(\frac{y^{2}-x^{2}}{(x^{2}+y^{2})^{2}}
        -\frac{y^{2}-x^{2}}{(x^{2}+y^{2})^{2}}\right)dx\land dy=0
    \end{align*}
    
    \item Para finalizar, tenemos que
    \begin{align*}
        \int_{0}^{1}\omega_{\gamma(t)}(\gamma'(t))\hspace{1mm}dt &= \int_{0}^{1}
        -\frac{\gamma_{2}(t)}{\gamma_{1}^{2}(t)+\gamma_{2}^{2}(t)}\cdot\gamma_{1}'(t)
        +\frac{\gamma_{1}(t)}{\gamma_{1}^{2}(t)+\gamma_{2}^{2}(t)}\cdot\gamma_{2}'(t)
        =\int_{0}^{1}F(\gamma(t))\cdot\gamma'(t)\hspace{1mm}dt \\[2mm]
        &= \int_{\gamma}F(x,y)\hspace{1mm}ds
    \end{align*}
    es decir, la integral corresponde a la integral de línea del campo vectorial 
    $F(x,y)=(\omega_{1},\omega_{2})(x,y)$ sobre la curva $\gamma$.
\end{enumerate}

\section*{Problema 5}
\begin{enumerate}
    \item Veamos que $\omega_{ij}$ es una 1-forma, sea $p\in W$, sean $u,v\in\R^{3}$ y 
    $\alpha\in\R$, entonces
    \begin{align*}
        (\omega_{ij})_{p}(u+\alpha v) &= \ip{DE_{i}(p)(u+\alpha v)}{E_{j}(p)}
        =\ip{DE_{i}(p)u+\alpha DE_{i}(p)v}{E_{j}(p)} \\[2mm]
        &= \ip{DE_{i}(p)u}{E_{j}(p)}+\alpha\ip{DE_{i}(p)v}{E_{j}(p)}
        =(\omega_{ij})_{p}(v)+\alpha(\omega_{ij})_{p}(v)
    \end{align*}
    Por otro lado, notemos que
    \begin{align*}
        (\omega_{ij})_{p}(v) &= \ip{DE_{i}(p)v}{E_{j}(p)}=\ip{\sum_{i=1}^{3}\pdv{E_{i}}
        {x_{k}}(p)v_{k}}{E_{j}(p)}=\sum_{i=1}^{3}\ip{\pdv{E_{i}}{x_{k}}}{E_{j}(p)}v_{k} \\[2mm]
        &= \sum_{i=1}^{3}\ip{\pdv{E_{i}}{x_{k}}}{E_{j}}(p)(dx_{k})_{p}(v)
    \end{align*}
    es decir,
    \begin{equation*}
        \omega_{ij}=\ip{\pdv{E_{i}}{x}}{E_{j}}dx+\ip{\pdv{E_{i}}{y}}{E_{j}}dy
        +\ip{\pdv{E_{i}}{z}}{E_{j}}dz
    \end{equation*}
    como las funciones $E_{i}$ y el producto interno son diferenciables, concluimos que 
    $\omega_{ij}$ es una 1-forma diferenciable. Queda chequear que $\omega_{ij}=-\omega_{ji}$, en
    efecto, sea $p\in W$ y $v\in\R^{3}$, entonces
    \begin{equation*}
        0=D\ip{E_{i}}{E_{j}}(p)v=\ip{DE_{i}(p)v}{E_{j}(p)}+\ip{E_{i}}{DE_{j}(p)v}
        =\omega_{ij}+\omega_{ji}
    \end{equation*}
    lo que prueba lo pedido.
    
    \item Sea $\{e_{j}\}_{j=1}^{3}$ la base canónica de $\R^{3}$. En primer lugar, tenemos que
    \begin{equation*}
        E_{i}=\sum_{j=1}^{3}\beta_{ij}e_{j}
        \hspace{4mm}\text{donde }\beta_{ij}(p)=e_{j}^{*}(E_{i}(p))
    \end{equation*}
    además, se tiene que $D\beta_{ij}=d\beta_{ij}$, en efecto, dado $p\in W$ y $v\in\R^{3}$
    \begin{equation*}
        (d\beta_{ij})_{p}(v)=\pdv{\beta_{ij}}{x}(p)v_{1}+\pdv{\beta_{ij}}{y}(p)v_{2}
        +\pdv{\beta_{ij}}{z}(p)v_{3}=D\beta_{ij}(p)v
    \end{equation*}
    lo que implica que
    \begin{equation*}
        DE_{i}=D\left(\sum_{j=1}^{3}\beta_{ij}e_{j}\right)=\sum_{j=1}^{3}D\beta_{ij}e_{j}
        =\sum_{j=1}^{3}d\beta_{ij}e_{j}
    \end{equation*}
    Por otro lado, como $(E_{j}(p))_{j=1}^{3}$ es base ortonormal de $\R^{3}$, dado $p\in W$ y 
    $v\in\R^{3}$ se obtiene que
    \begin{equation*}
        DE_{i}(p)v=\sum_{k=1}^{3}\ip{DE_{i}(p)v}{E_{k}(p)}E_{k}(p)
        =\sum_{k=1}^{3}(\omega_{ik})_{p}(v)E_{k}(p)
        \hspace{4mm}\text{es decir, }DE_{i}=\sum_{k=1}^{3}\omega_{ik}E_{k}
    \end{equation*}
    de este modo,
    \begin{equation*}
        DE_{i}=\sum_{k=1}^{3}\omega_{ik}E_{k}
        =\sum_{k=1}^{3}\omega_{ik}\left(\sum_{j=1}^{3}\beta_{kj}e_{j}\right)
        =\sum_{j=1}^{3}\left(\sum_{k=1}^{3}\omega_{ik}\beta_{kj}\right)e_{j}
    \end{equation*}
    comparando las dos expresiones para $DE_{i}$ y usando que $\{e_{j}\}_{j=1}^{3}$ es base, 
    concluimos que
    \begin{equation*}
        d\beta_{ij}=\sum_{k=1}^{3}\omega_{ik}\beta_{kj}
    \end{equation*}
    Además, por la expresión para $E_{i}$, vemos que
    \begin{equation*}
        \omega_{i}=\sum_{j=1}^{3}\beta_{ij}dx_{j}
    \end{equation*}
    derivando se sigue que
    \begin{align*}
        d\omega_{i} &= d\left(\sum_{j=1}^{3}\beta_{ij}dx_{j}\right)
        =\sum_{j=1}^{3}d\beta_{ij}\land dx_{j}+\beta_{ij}d(dx_{j})
        =\sum_{j=1}^{3}d\beta_{ij}\land dx_{j} \\[2mm]
        &= \sum_{j=1}^{3}\left(\sum_{k=1}^{3}\omega_{ik}\beta_{kj}\right)\land dx_{j}
        =\sum_{k=1}^{3}\omega_{ik}\land\left(\sum_{j=1}^{3}\beta_{kj}dx_{j}\right)
        =\sum_{k=1}^{3}\omega_{ik}\land\omega_{k}=\sum_{k=1}^{3}\omega_{k}\land\omega_{ki}
    \end{align*}

    \item Notemos lo siguiente
    \begin{equation*}
        0=d(d\beta_{ij})=\sum_{k=1}^{3}d(\omega_{ik}\beta_{kj})
        =\sum_{k=1}^{3}d\beta_{kj}\land\omega_{ik}+\sum_{k=1}^{3}d\omega_{ik}\beta_{kj}
    \end{equation*}
    entonces
    \begin{equation*}
        \sum_{k=1}^{3}d\omega_{ik}\beta_{kj}=\sum_{k=1}^{3}\omega_{ik}\land d\beta_{kj}
        =\sum_{k=1}^{3}\omega_{ik}\land\sum_{s=1}^{3}\omega_{ks}\beta_{sj}
    \end{equation*}
    (entender mejor idea de las matrices)
\end{enumerate}

%\printbibliography % Quitar el comentado si quiero usar bibliografia

\end{document}
