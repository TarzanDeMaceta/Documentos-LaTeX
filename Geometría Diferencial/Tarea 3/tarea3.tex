\documentclass{article}
\usepackage{hyperref}
\usepackage{Style}

\nocite{*} % Comentar si quiero citar
%\addbibresource{bibliografia.bib} % Quitar el comentado si quiero usar bibliografia

\begin{document}

\begin{minipage}{2.5cm}
    \includegraphics[width=2cm]{imagen_puc.jpg}
\end{minipage}
\begin{minipage}{14cm}
    {\sc Pontificia Universidad Católica de Chile\\
    Facultad de Matemáticas\\
    Departamento de Matemática\\
    Profesor: Pedro Gaspar -- Estudiante: Benjamín Mateluna}
\end{minipage}
\vspace{1ex}

{\centerline{\bf Geomtría Diferencial - MAT2860}
\centerline{\bf Tarea I3}}
\centerline{\bf 26 de junio de 2025}

\section*{Problema 1}
Probaremos un resultado previo sobre transporte paralelo y curvas geodesicas en la esfera.
\begin{lema}
    Sea $\alpha:I\to\mathbb{S}^{2}$ una curva de la forma $\alpha(t)=usen(t)+vcos(t)$ donde $u,v$ 
    son ortonormales, entonces $P_{\alpha,t_{0},t_{1}}(w)=w+\ip{w}{\alpha'(t_{0})}(\alpha'(t_{1})
    -\alpha'(t_{0}))$.
\end{lema}
\begin{dem}
    Empezamos observando que $\alpha$ es una curva geodésica. Notemos que 
    $\ip{\alpha'(t)}{\alpha(t)}=0$ para todo $t\in I$ y además $\alpha''=-\alpha$. Sea 
    $w\in T_{p}\mathbb{S}^{2}$ donde $p=\alpha(t_{0})$ y sea $Y:I\to\mathbb{S}^{2}$ el único campo 
    paralelo a lo largo de $\alpha$ tal que $Y(t_{0})=w$, como $Y$ en particular es campo tangente 
    se tiene que
    \begin{equation*}
        Y(t)\in T_{\alpha(t)}\mathbb{S}^{2}=(\alpha(t))^{\perp}
    \end{equation*}
    es decir, $\ip{Y(t)}{\alpha(t)}=0$ para todo $t\in I$, lo que implica que $\ip{Y}{\alpha'}
    =-\ip{Y'}{\alpha}$. Notemos que
    \begin{equation*}
        0=\nabla_{\alpha'}Y=Y'-\ip{Y'}{\alpha}\alpha
    \end{equation*}
    tomando producto interno con $\alpha'$ vemos que
    \begin{equation*}
        \ip{Y'}{\alpha'}=\ip{Y'}{\alpha}\ip{\alpha}{\alpha'}=0
    \end{equation*}
    de este modo
    \begin{equation*}
        \dv{}{t}(\ip{Y}{\alpha'})=\ip{Y'}{\alpha'}+\ip{Y}{\alpha''}=-\ip{Y}{\alpha}=0
    \end{equation*}
    luego la función $\ip{Y}{\alpha'}$ es constante, evaluando en $t_{0}$ vemos que 
    $\ip{Y'}{\alpha}=-\ip{Y}{\alpha'}\equiv-\ip{w}{\alpha'(t_{0})}$. Tenemos que
    \begin{equation*}
        Y'(t)=\ip{w}{\alpha'(t_{0})}\alpha''(t)
    \end{equation*}
    integrando a ambos lados vemos que
    \begin{equation*}
        Y(t)-Y(t_{0})=Y(t)-w=\ip{w}{\alpha'(t_{0})}(\alpha'(t)-\alpha'(t_{0}))
    \end{equation*}
    lo que concluye la demostración.
\end{dem}

\begin{enumerate}
    \item Notemos que
    \begin{equation*}
        \gamma_{1}(s)=\begin{pmatrix}
            0 \\ 0 \\ 1
        \end{pmatrix}cos(s)+\begin{pmatrix}
            1 \\ 0 \\ 0
        \end{pmatrix}sen(s)\hspace{4mm}\text{y}\hspace{4mm}\gamma_{2}(s)=\begin{pmatrix}
            0 \\ 0 \\ 1
        \end{pmatrix}cos(s)+\begin{pmatrix}
            cos(\theta) \\ sen(\theta) \\ 0
        \end{pmatrix}sen(s)
    \end{equation*}
    además, los vectores que describen cada curva son ortonormales. Por otro lado se tiene que
    \begin{equation*}
        \gamma_{1}'(s)=(cos(s),0,sen(s))\hspace{4mm}\text{y}\hspace{4mm}\gamma_{2}'(s)
        =(cos(\theta)cos(s),sen(\theta)cos(s),-sen(s))
    \end{equation*}
    Así, por el lema tenemos lo siguiente
    \begin{align*}
        P_{\gamma_{1},0,\pi}(w_{0}) &= w_{0}+\ip{w_{0}}{\gamma_{1}'(0)}
        (\gamma_{1}'(\pi)-\gamma_{1}'(0)) \\[2mm]
        &= w_{0}+\ip{w_{0}}{
            \begin{pmatrix}
                1 \\ 0 \\ 0
            \end{pmatrix}
        }\left(\begin{pmatrix}
            -1 \\ 0 \\ 0
        \end{pmatrix}-\begin{pmatrix}
            1 \\ 0 \\ 0
        \end{pmatrix}\right)=w_{0}+w_{01}\begin{pmatrix}
            -2 \\ 0 \\ 0
        \end{pmatrix}=w_{1}
    \end{align*}
    por otro lado
    \begin{align*}
        P_{\gamma_{2},0,\pi}(w_{0}) &= w_{0}+\ip{w_{0}}{\gamma_{2}'(0)}
        (\gamma_{2}'(\pi)-\gamma_{2}'(0)) \\[2mm]
        &= w_{0}+\ip{w_{0}}{
            \begin{pmatrix}
                cos(\theta) \\ sen(\theta) \\ 0
            \end{pmatrix}
        }\left(\begin{pmatrix}
            -cos(\theta) \\ -sen(\theta) \\ 0
        \end{pmatrix}-\begin{pmatrix}
            cos(\theta) \\ sen(\theta) \\ 0
        \end{pmatrix}\right)=w_{2}
    \end{align*}

    \item 
\end{enumerate}

\section*{Problema 2}
\noindent El toro, puede ser parametrizado por $X:(0,2\pi)\times(0,2\pi)\to\R^{3}$ dada por
\begin{equation*}
    X(u,v)=((R+rcos(u))cos(v),(R+rcos(u))sen(v),rsen(u))
\end{equation*}
Si intersectamos con el plano $z=1$, entonces $rsen(u)=c$ y por lo tanto
\begin{equation*}
    rcos(u)=\pm\sqrt{r^{2}-c^{2}}:=c'
\end{equation*}
que esta bien definida pues $c\in[-r,r]$. Bajo esta restricción definimos
\begin{equation*}
    \eta:=R+rcos(u)=R+c'>0
\end{equation*}
Consideramos la curva $\alpha:(0,2\pi\eta)\to\R^{3}$ parametrizada por el arco dada por
\begin{equation*}
    \alpha(t)=\left(\eta cos\left(\frac{t}{\eta}\right),\eta sen\left(\frac{t}{\eta}\right),
    c\right)
\end{equation*}
Buscamos calcular su curvatura geodésica, en primer lugar debemos calcular la aplicación de gauss
de la superficie, para ello vemos lo siguiente
\begin{align*}
    X_{u}(u,v) &= (-rsen(u)cos(v),-rsen(u)sen(v),rcos(u)) \\
    X_{v}(u,v) &= (-(R+rcos(u))sen(v),(R+rcos(u))cos(v),0)
\end{align*}
entonces
\begin{equation*}
    (X_{u}\times X_{v})(u,v)=(-(R+rcos(u))rcos(u)cos(v),-(R+rcos(u))rcos(u)sen(v),
    -(R+rcos(u))rsen(v))
\end{equation*}
y además $\abs{(X_{u}\times X_{v})(u,v)}=(R+rcos(u))r$. Usando que $u$ es tal que $rsen(u)=c$, se
sigue que
\begin{equation*}
    N(\alpha(t))=-\frac{1}{r}\left(c'cos\left(\frac{t}{\eta}\right),
    c'sen\left(\frac{t}{\eta}\right),c\right)
\end{equation*}
Por otro lado vemos que
\begin{equation*}
    \alpha'(t)=\left(-sen\left(\frac{t}{\eta}\right),cos\left(\frac{t}{\eta}\right),0\right)
    \hspace{4mm}\text{y entonces}\hspace{4mm}
    \alpha''(t)=-\frac{1}{\eta}\left(cos\left(\frac{t}{\eta}\right),
    sen\left(\frac{t}{\eta}\right),0\right)
\end{equation*}
Nuestro objetivo ahora es determinar $\nabla_{\alpha'}\alpha'(t)$, para ello observemos que
\begin{equation*}
    \ip{\alpha''}{N\circ\alpha}(t)=\frac{c'}{r\eta}
\end{equation*}
de este modo
\begin{align*}
    \nabla_{\alpha'}\alpha'(t) &= \alpha''(t)-\ip{\alpha''}{N\circ\alpha}(t)N(\alpha(t)) \\[2mm]
    &= -\frac{1}{\eta}\left(cos\left(\frac{t}{\eta}\right),
    sen\left(\frac{t}{\eta}\right),0\right)+\frac{c'}{r^{2}\eta}
    \left(c'cos\left(\frac{t}{\eta}\right),c'sen\left(\frac{t}{\eta}\right),c\right) \\[2mm]
    &= \left(\frac{-c^{2}}{r^{2}\eta}cos\left(\frac{t}{\eta}\right),
    \frac{-c^{2}}{r^{2}\eta}sen\left(\frac{t}{\eta}\right),\frac{cc'}{r^{2}\eta}\right)
\end{align*}
donde usamos que
\begin{equation*}
    \frac{c'^{2}}{r^{2}\eta}-\frac{1}{\eta}=\frac{r^{2}-c^{2}-r^{2}}{r^{2}\eta}
    =\frac{-c^{2}}{r^{2}\eta}
\end{equation*}
adicionalmente se tiene que
\begin{equation*}
    N(\alpha(t))\times\alpha'(t)=-\frac{1}{r}\left(-ccos\left(\frac{t}{\eta}\right),
    csen\left(\frac{t}{\eta}\right),c\right)
\end{equation*}
por último, se tiene que
\begin{equation*}
    K_{g}=[\nabla_{\alpha'}\alpha']=-\frac{1}{r}\cdot\frac{-c}{r^{2}\eta}
    \left(-c^{2}cos^{2}\left(\frac{t}{\eta}\right)+c^{2}sen^{2}\left(\frac{t}{\eta}\right)-c'^{2}
    \right)
    =-\frac{c}{r^{3}\eta}\left(c^{2}cos\left(\frac{2t}{\eta}\right)+c'^{2}\right)
\end{equation*}

\section*{Problema 3}

\section*{Problema 4}
\begin{enumerate}
    \item Notemos que
    \begin{align*}
        \omega_{p}(v) &= \ip{v}{\frac{Jp}{\abs{p}^{2}}}=\frac{1}{\abs{p}^{2}}\ip{v}{
            \begin{pmatrix}
                -y \\
                x
            \end{pmatrix}
            }=\frac{1}{\abs{p}^{2}}(-v_{1}y+v_{2}x) \\[2mm]
        &= \frac{1}{\abs{p}^{2}}(-dx(v)y+dy(v)x)=\frac{-y}{x^{2}+y^{2}}dx(v)
        +\frac{x}{x^{2}+y^{2}}dy(v)
    \end{align*}
    conlcuimos que
    \begin{equation*}
        \omega_{p}=\frac{-y}{x^{2}+y^{2}}dx+\frac{x}{x^{2}+y^{2}}dy
    \end{equation*}
    además, para $p\in\R^{3}\setminus\{0\}$ las funciones
    \begin{equation*}
        \omega_{1}:=\frac{-y}{x^{2}+y^{2}}\hspace{4mm}\text{y}
        \hspace{4mm}\omega_{2}:=\frac{x}{x^{2}+y^{2}}
    \end{equation*}
    son diferenciables, y por lo tanto la 1-forma $\omega_{p}$ es diferenciable.
    
    \item Veamos la siguiente expresión
    \begin{align*}
        d\omega &= \left(\pdv{\omega_{1}}{x}dx+\pdv{\omega_{1}}{y}dy\right)\land dx
        +\left(\pdv{\omega_{2}}{x}dx+\pdv{\omega_{2}}{y}dy\right)\land dy
        =\left(\pdv{\omega_{2}}{x}dx-\pdv{\omega_{1}}{y}dy\right)dx\land dy \\[2mm]
        &= \left(\frac{y^{2}-x^{2}}{(x^{2}+y^{2})^{2}}
        -\frac{y^{2}-x^{2}}{(x^{2}+y^{2})^{2}}\right)dx\land dy=0
    \end{align*}
    
    \item Para finalizar, tenemos que
    \begin{align*}
        \int_{0}^{1}\omega_{\gamma(t)}(\gamma'(t))\hspace{1mm}dt &= \int_{0}^{1}
        -\frac{\gamma_{2}(t)}{\gamma_{1}^{2}(t)+\gamma_{2}^{2}(t)}\cdot\gamma_{1}'(t)
        +\frac{\gamma_{1}(t)}{\gamma_{1}^{2}(t)+\gamma_{2}^{2}(t)}\cdot\gamma_{2}'(t)
        =\int_{0}^{1}F(\gamma(t))\cdot\gamma'(t)\hspace{1mm}dt \\[2mm]
        &= \int_{\gamma}F(x,y)\hspace{1mm}ds
    \end{align*}
    es decir, la integral corresponde a la integral de línea del campo vectorial 
    $F(x,y)=(\omega_{1},\omega_{2})(x,y)$ sobre la curva $\gamma$.
\end{enumerate}

\section*{Problema 5}
\begin{enumerate}
    \item Se tiene lo siguiente
    \begin{align*}
        (\omega_{ij})_{p}(v) &= \ip{DE_{i}(p)v}{E_{j}(p)}
        =\ip{\sum_{k=1}^{3}\pdv{E_{i}}{x_{k}}(p)v_{k}}{E_{j}(p)}
        =\sum_{k=1}^{3}\ip{\pdv{E_{i}}{x_{k}}}{E_{j}(p)}v_{k} \\[2mm]
        &= \sum_{k=1}^{3}\ip{\pdv{E_{i}}{x_{k}}}{E_{j}}(p)(dx_{k})_{p}(v)
        =\left(\sum_{k=1}^{3}\ip{\pdv{E_{i}}{x_{k}}}{E_{j}}(p)(dx_{k})_{p}\right)(v)
    \end{align*}
    es decir,
    \begin{equation*}
        \omega_{ij}=\ip{\pdv{E_{i}}{x}}{E_{j}}dx+\ip{\pdv{E_{i}}{y}}{E_{j}}dy
        +\ip{\pdv{E_{i}}{z}}{E_{j}}dz
    \end{equation*}
    
    como las funciones $E_{i}$ y el producto interno son diferenciables, concluimos que 
    $\omega_{ij}$ es una 1-forma diferenciable.

    \item Se sigue que
    \begin{align*}
        (\omega_{i})_{p}(v)=\ip{v}{E_{i}(p)}=\sum_{k=1}^{3}(E_{i})_{k}(p)v_{k}
        =\sum_{k=1}^{3}(E_{i})_{k}(p)(dx_{k})(v)
    \end{align*}
    entonces
    \begin{equation*}
        \omega_{i}=(E_{i})_{1}dx+(E_{i})_{2}dy+(E_{i})_{3}dz
    \end{equation*}
    
    por la misma razón que antes, la 1-forma $\omega_{i}$ es diferenciable. Por otro lado ...

    \item 
\end{enumerate}

%\printbibliography % Quitar el comentado si quiero usar bibliografia

\end{document}
