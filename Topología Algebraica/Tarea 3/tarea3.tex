\documentclass{article}
\usepackage{hyperref}
\usepackage{Style}

\nocite{*} % Comentar si quiero citar
%\addbibresource{bibliografia.bib} % Quitar el comentado si quiero usar bibliografia

\begin{document}

\begin{minipage}{2.5cm}
    \includegraphics[width=2cm]{imagen_puc.jpg}
\end{minipage}
\begin{minipage}{14cm}
    {\sc Pontificia Universidad Católica de Chile\\
    Facultad de Matemáticas\\
    Departamento de Matemática\\
    Profesor: Mauricio Bustamante -- Estudiante: Benjamín Mateluna}
\end{minipage}
\vspace{1ex}

{\centerline{\bf Topología Álgebraica - MAT2850}
\centerline{\bf Tarea 4}}
\centerline{\bf 07 de noviembre de 2025}

\section*{Problema 1}
\begin{enumerate}
    \item En primer lugar veamos que $\sbullet$ esta bien definida, es decir, dados $\alpha,\beta$
    caminos basados en $x_{0}$ tales que $[\alpha]=[\beta]$, entonces
    $\widehat{x}\hspace{1mm}{\sbullet}\hspace{1mm}[\alpha]
    =\widehat{x}\hspace{1mm}{\sbullet}\hspace{1mm}[\beta]$. En efecto, por levantamiento de 
    homotopías relativas, se sigue que $\widehat{\alpha}(1)=\widehat{\beta}(1)$, lo que implica 
    que
    \begin{equation*}
        \widehat{x}\hspace{1mm}{\sbullet}\hspace{1mm}[\alpha]=\widehat{\alpha}(1)
        =\widehat{\beta}(1)=\widehat{x}\hspace{1mm}{\sbullet}\hspace{1mm}[\beta]
    \end{equation*}

    Además, como $p\widehat{\alpha}=\alpha$, se tiene que $\widehat{\alpha}(1)\in p^{-1}(x_{0})$. 
    Veamos que $\sbullet$ induce una acción por la derecha de $\pi_{1}(X,x_{0})$ en 
    $p^{-1}(x_{0})$. Sea $\widehat{x}\in p^{-1}(x_{0})$, notemos que el lazo $ct_{\widehat{x}}$ 
    levanta al lazo constante $ct_{x_{0}}$, luego $\widehat{x}\hspace{1mm}{\sbullet}
    \hspace{1mm}[ct_{x_{0}}]=ct_{\widehat{x}}(1)=\widehat{x}$.

    \vspace{1mm}
    Sean $\alpha,\beta$ lazos basados en $x_{0}$ y sea $\widehat{x}\in p^{-1}(x_{0})$. Afirmamos 
    que $[p(\widehat{\alpha}^{-1}*\widehat{\alpha*\beta})]=[\beta]$ en $\pi_{1}(X,x_{0})$. En 
    primer lugar, observamos que
    \begin{equation*}
        p(\widehat{\alpha}^{-1}*\widehat{\alpha*\beta})(0)=\alpha(1)=x_{0}
        \hhtext{y}
        p(\widehat{\alpha}^{-1}*\widehat{\alpha*\beta})(1)=\beta(1)=x_{0}
    \end{equation*}
    por lo que la expresión tiene sentido. Por otro lado, notemos que 
    $p(\widehat{\alpha}^{-1}*\widehat{\alpha*\beta})=p(\widehat{\alpha}^{-1})*\alpha*\beta$ y
    adicionalmente tenemos que
    \begin{equation*}
        [\alpha]*[p\widehat{\alpha^{-1}}]=[ct_{x_{0}}]=p_{*}[ct_{\widehat{x}}]
        =p_{*}[\widehat{\alpha}*\widehat{\alpha}^{-1}]=[p(\widehat{\alpha}*\widehat{\alpha}^{-1})]
        =[p\widehat{\alpha}*p\widehat{\alpha}^{-1}]=[p\widehat{\alpha}]*[p\widehat{\alpha}^{-1}]
        =[\alpha]*[p\widehat{\alpha}^{-1}]
    \end{equation*}
    y por lo tanto $p\widehat{\alpha}^{-1}\sim\alpha^{-1}$, lo que prueba la afirmación. Así,
    se tiene lo siguiente
    \begin{equation*}
        (\widehat{x}\hspace{1mm}{\sbullet}\hspace{1mm}[\alpha])
        \hspace{1mm}{\sbullet}\hspace{1mm}[\beta]
        =\widehat{\alpha}(1)\hspace{1mm}{\sbullet}\hspace{1mm}[\beta]
        =\widehat{\alpha}(1)\hspace{1mm}{\sbullet}\hspace{1mm}[
            p(\widehat{\alpha}^{-1}*\widehat{\alpha*\beta})
        ]=(\widehat{\alpha*\beta})(1)=\widehat{x}\hspace{1mm}{\sbullet}\hspace{1mm}[\alpha*\beta]
    \end{equation*}
    Notar que $(\widehat{\alpha}^{-1}*\widehat{\alpha*\beta})(0)=\widehat{\alpha}(1)$.
    
    \item Supongamos que $\widehat{X}$ es arcoconexo y sean $\widehat{x}_{1},\widehat{x}_{2}\in 
    p^{-1}(x_{0})$, existe $\widehat{\gamma}:[0,1]\to\widehat{X}$ continua tal que 
    $\widehat{\gamma}(0)=\widehat{x}_{1}$ y $\widehat{\gamma}(1)=\widehat{x}_{2}$. Definimos 
    $\gamma=p\widehat{\gamma}$ que es un lazo basado en $x_{0}$, entonces
    \begin{equation*}
        \widehat{x}_{1}\hspace{1mm}{\sbullet}\hspace{1mm}[\gamma]
        =\widehat{\gamma}(1)=\widehat{x}_{2}
    \end{equation*}

    Por otro lado, supongamos que $\sbullet$ es transitiva. Sean $\widehat{x},\widehat{y}
    \in\widehat{X}$, tenemos dos casos, el primero es si $\widehat{x},\widehat{y}
    \in p^{-1}(x_{0})$ para algún $x_{0}\in X$, entonces, existe un lazo $\gamma$ basado en 
    $x_{0}$ tal que $\widehat{\gamma}(0)=\widehat{x}$ y
    \begin{equation*}
        \widehat{\gamma}(1)=\widehat{x}\hspace{1mm}{\sbullet}\hspace{1mm}[\gamma]=\widehat{y}
    \end{equation*}
    por lo tanto, $\widehat{\gamma}$ es el camino buscado. El segundo caso es cuando $\widehat{x}
    \in p^{-1}(x)$ e $\widehat{y}\in p^{-1}(y)$ con $x\neq y$. Como $X$ es arcoconexo, existe un 
    camino $\gamma$ de modo que $\gamma(0)=x$ y $\gamma(1)=y$. Por lema del levantamiento existe 
    $\widehat{\gamma}$ un levantamiento de $\gamma$ tal que $\widehat{\gamma}(0)=\widehat{x}$ y 
    $\widehat{\gamma}(1)=\widehat{y}'\in p^{-1}(y)$ y por el caso anterior concluimos.
    
    \item Debemos probar que dado $\widehat{x}\in\widehat{X}$ se tiene que
    \begin{equation*}
        p_{*}(\pi_{1}(\widehat{X},\widehat{x}))=S_{\widehat{x}}
    \end{equation*}
    donde $S_{\widehat{x}}$ es el estabilizador de $\widehat{x}$. Sea $[\alpha]
    \in S_{\widehat{x}}$, entonces $\widehat{x}=\widehat{x}\hspace{1mm}{\sbullet}
    \hspace{1mm}[\alpha]=\widehat{\alpha}(1)$, como $p\widehat{\alpha}=\alpha$, concluimos que 
    $[\alpha]\in p_{*}(\pi_{1}(\widehat{X},\widehat{x}))$. Sea $[\alpha]
    \in p_{*}(\pi_{1}(\widehat{X},\widehat{x}))$, por ende existe un lazo basado en $\widehat{x}$, 
    digamos $\widehat{\alpha}$, tal que $[p\widehat{\alpha}]=[\alpha]$, entonces
    \begin{equation*}
        \widehat{x}\hspace{1mm}{\sbullet}\hspace{1mm}[\alpha]
        =\widehat{x}\hspace{1mm}{\sbullet}\hspace{1mm}[p\widehat{\alpha}]=\widehat{\alpha}(1)
        =\widehat{x}
    \end{equation*}
    
    \item Usando la parte anterior y orbita establizador, resulta que
    \begin{equation*}
        \abs{\frac{\pi_{1}(X,x_{0})}{p_{*}(\pi_{1}(\widehat{X},\widehat{x}))}}
        =\abs{O_{\widehat{x}}}=\abs{p^{-1}(x_{0})}
    \end{equation*}
    donde en la última igualdad usamos que la acción es transitiva, o equivalentemente, que 
    $\widehat{X}$ es arcoconexo.
\end{enumerate}

\section*{Problema 2}

\noindent El problema planteado es similar al problema 1.IV, pero faltan condiciones.

\section*{Problema 3}

\noindent Para $(x,y)\in[0,1]^{2}$, definimos las funciones $f_{0}(x,y)=(x,2y)$ cuando 
$y\leq\frac{1}{2}$ y $f_{1}(x,y)=(1-x,2y-1)$ cuando $y\geq\frac{1}{2}$, claramente cada una es 
continua en su respectivo dominio, luego, las funciones $\phi_{i}:=\pi_{K}\circ f_{i}$ son 
continuas. Definimos $\phi:[0,1]^{2}\to K$ por
\begin{equation*}
    \phi(x,y)=\begin{cases}
        \phi_{0}(x,y)=[x,2y] &\quad\text{ si }0\leq y\leq\frac{1}{2} \\
        \phi_{1}(x,y)=[1-x,2y-1] &\quad\text{ si }\frac{1}{2}\leq y \leq1
    \end{cases}
\end{equation*}
Veamos que este bien definida, sea $t\in[0,1]$, entonces $\phi_{0}(t,\frac{1}{2})=[t,1]$ y
$\phi_{1}(t,\frac{1}{2})=[1-t,0]$, así, por lema del pegado, la función $\phi$ es continua. 
Tenemos el diagrama

\vspace{2mm}
\centerline{
    \xymatrix{
        [0,1]^{2} \ar[d]^{\pi_{\T^{2}}} \ar[rd]^{\phi} \\
        \T^{2} \ar@{-->}[r] & K
    }
}
\vspace{2mm}
\noindent Nos gustaría que $\phi$ sea constante en las fibras de $\pi_{\T^{2}}$. Sea $t\in[0,1]$,
entonces $\phi(t,0)=[t,0]$ y $\phi(t,1)=[1-t,1]$, es decir, $\phi(t,0)=\phi(t,1)$. Por otro lado,
si $t\leq\frac{1}{2}$ vemos que $\phi(0,t)=[0,2t]$ y $\phi(1,t)=[1,2t]$, en otras palabras, 
$\phi(0,t)=\phi(1,t)$ y de manera análoga se tiene que $\phi(0,t)=\phi(1,t)$ cuando 
$t\geq\frac{1}{2}$. Sea $p:\T^{2}\to K$ la función inducida por $\phi$, lo anterior se resume en 
que el siguiente diagrama conmuta

\vspace{2mm}
\centerline{
    \xymatrix{
        [0,1]\times[0,\frac{1}{2}] \ar[r]^-{f_{0}} \ar[rd]^-{\phi_{0}} \ar[d]^{\pi_{\T^{2}}}
        & [0,1]^{2} \ar[d]^-{\pi_{K}}
        & [0,1]\times[\frac{1}{2},1] \ar[l]_-{f_{1}} \ar[ld]_-{\phi_{1}} \ar[d]^{\pi_{\T^{2}}} \\
        \T^{2} \ar[r]^-{p} & K & \T^{2} \ar[l]_-{p}
    }
}
\vspace{2mm}
\noindent Afirmamos que $(\T^{2},p)$ es un espacio cubriente regular de $K$. Bajo lo anterior, 
concluimos que
\begin{equation*}
    \Z^{2}\cong\pi_{1}(\T^{2},x_{0})\cong p_{*}(\pi_{1}(\T^{2},x_{0}))\unlhd\pi_{1}(K,y_{0})
\end{equation*}
con $y_{0}=p(x_{0})$. Lo que finaliza el ejercicio.

\vspace{1mm}
\noindent Para probar la afirmación, demostramos los siguientes puntos:
\begin{enumerate}
    \item La tupla $(\T^{2},p)$ es un espacio cubriente de $K$. Usaremos fuertemente el lema que 
    nos dice que una función continua y sobreyectiva es cociente si y solo si la imagen de un 
    abierto saturado es un abierto. Esto fue probado en el curso de Topología. Una definición de 
    conjunto saturado es que se puede escribir como unión de fibras y una propiedad importante es
    $q^{-1}(q(U))=U$ para toda función $q$ sobreyectiva y todo conjunto saturado.

    \vspace{1mm}
    \noindent Consideremos el abierto $(0,1)^{2}\subseteq[0,1]^{2}$ que es saturado bajo 
    $\pi_{K}$, por que las relaciones se encuentran en $\partial[0,1]^{2}$. Entonces 
    $U=\pi_{K}((0,1)^{2})$ es abierto en $K$. Por otro lado, los abiertos
    \begin{equation*}
        f_{0}^{-1}((0,1)^{2})=(0,1)\times(0,1/2)=U_{0}
        \hhtext{y}
        f_{1}^{-1}((0,1)^{2})=(0,1)\times(1/2,1)=U_{1}
    \end{equation*}
    son saturados bajo $\pi_{\T^{2}}$.
    
    \vspace{1mm}
    \noindent Sean $V_{0}=\pi_{\T^{2}}(U_{0})$ y $V_{1}=\pi_{\T{{2}}}(U_{1})$, afirmamos que 
    $p^{-1}(U)=V_{0}\sqcup V_{1}$. Notemos, por conmutatividad de los diagramas, que
    \begin{align*}
        p^{-1}(U) &= \pi_{\T^{2}}(\phi^{-1}(U))=\pi_{\T^{2}}(\phi_{0}^{-1}(U)
        \cup\phi_{1}^{-1}(U))
        =\pi_{\T^{2}}(f_{0}^{-1}(\pi_{K}^{-1}(U))\cup f_{1}^{-1}(\pi_{K}^{-1}(U))) \\[2mm]
        &= \pi_{\T^{2}}(f_{0}^{-1}((0,1)^{2})\cup f_{1}^{-1}((0,1)^{2}))
        =\pi_{\T^{2}}(U_{0}\cup U_{1})=V_{0}\cup V_{1}
    \end{align*}
    La unión es disjunta por que si $[x,y]\in V_{0}\cap V_{1}$, entonces 
    $(x,y)\in U_{0}\cap U_{1}$, lo que es imposible.
    
    \vspace{1mm}
    Un conjunto cerrado en $(0,1)^{2}$ es 
    acotado y por lo tanto compacto, entonces la función $\pi_{K}\big|_{(0,1)^{2}}:(0,1)^{2}\to U$ 
    es cociente. Como la función $f_{i}:U_{i}\to(0,1)^{2}$ es homeomorfismo con inversa constante 
    en las fibras de $\pi_{K}\big|_{(0,1)^{2}}$, se sigue que $p:V_{i}\to U$ es homeomorfismo. La 
    discusión anterior muestra que se cumple la condición de cubriente para todo punto 
    $[x,y]\in U$, es decir, $(x,y)\in(0,1)^{2}$.

    \newpage
    \vspace{1mm}
    El conjunto $W=([0,\frac{1}{4})\cup(\frac{3}{4},1])\times(0,1)$ es un abierto saturado en 
    $[0,1]^{2}$ bajo $\pi_{K}$ por que $(0,t),(1,t)\in W$ para todo $t\in(0,1)$, entonces 
    $U=\pi_{K}(W)$ es abierto en $K$. Luego, los abiertos
    \begin{equation*}
        V_{0}=\pi_{\T^{2}}(([0,1/4)\cup(3/4,1])\times(0,1/2))
        \hhtext{y}
        V_{1}=\pi_{\T^{2}}(([0,1/4)\cup(3/4,1])\times(1/2,1))
    \end{equation*}
    cumplen que $p^{-1}(U)=V_{0}\sqcup V_{1}$ con $p:V_{i}\to U$ homeomorfismo. El argumento es 
    idéntico al anterior.

    Tomamos el abierto saturado $W=[0,1]\times([0,\frac{1}{4})\cup(\frac{3}{4},1])$, para el 
    abierto $U=\pi_{K}(W)$ definimos los abiertos
    \begin{equation*}
        V_{0}=\pi_{\T^{2}}([0,1]\times(3/8,5/8))
        \hhtext{y}
        V_{1}=\pi_{\T^{2}}([0,1]\times([0,1/8)\cup(7/8,1]))
    \end{equation*}
    de manera similar, se obtiene que $p^{-1}(U)=V_{0}\sqcup V_{1}$. Sin embargo, no es tan claro
    que $p:V_{i}\to U$ es homeomorfismo. Definimos la función $g:W\to V_{0}$ 
    dada por
    \begin{equation*}
        g(x,y)=\begin{cases}
            [1-x,\frac{1}{2}(y+1)] &\quad\text{ si }0\leq y<\frac{1}{4} \\
            [x,\frac{1}{2}y] &\quad\text{ si }\frac{3}{4}<y\leq1
        \end{cases}
    \end{equation*}
    que es continua por lema del pegado. Un argumento idéntico al utilizado previamente nos dice 
    que $g$ es constante en las fibras de $\pi_{K}:W\to U$ . Sea $\varphi:U\to V_{0}$ la función
    inducida por $g$, veamos que $\varphi$ es la inversa de $p:V_{0}\to U$. Claramente $\varphi(U)
    \subseteq V_{0}$, por otro lado, dado $[x,y]$ con $y\leq\frac{1}{2}$ implica
    \begin{equation*}
        p\circ\varphi([x,y])=p([1-x,1/2(y+1)])=[x,y]
        \hhtext{y}
        \varphi\circ p([x,y])=\varphi([x,2y])=[x,y]
    \end{equation*}
    de manera similar resulta que $p\circ\varphi=id_{U}$ y 
    $\varphi\circ p=id_{V_{0}}$ para $y\geq\frac{1}{2}$, concluimos que $p:V_{0}\to U$ es 
    homeomorfismo. Análogamente obtenemos que $p:V_{1}\to U$ es homeomorfismo.

    \item El cubrimiento es regular, esto es equivalente a probar que el grupo 
    $\triangle(\T^{2},p)$ actúa transitivamente en $p^{-1}(x_{0})$ para algún $x_{0}\in K$.
    Definimos la función $\psi:[0,1]^{2}\to\T^{2}$ por
    \begin{equation*}
        \psi(x,y)=\begin{cases}
            [1-x,y+\frac{1}{2}] &\quad\text{ si }0\leq y\leq\frac{1}{2} \\
            [1-x,y-\frac{1}{2}] &\quad\text{ si }\frac{1}{2}\leq y \leq1
        \end{cases}
    \end{equation*}
    Veamos que esta bien definida, en efecto, dado $t\in[0,1]$ vemos que 
    $\psi(t,\frac{1}{2})=[1-t,0]$ y $\psi(t,\frac{1}{2})=[1-t,1]$. Por la misma razón que antes, 
    la función $\psi$ es continua. Tenemos el diagrama

    \vspace{2mm}
    \centerline{
        \xymatrix{
            [0,1]^{2} \ar[d]^{\pi_{\T^{2}}} \ar[rd]^{\psi} \\
            \T^{2} \ar@{-->}[r] & \T^{2}
        }
    }
    \vspace{2mm}
    \noindent Sea $t\in[0,1]$, luego $\psi(t,0)=[1-t,\frac{1}{2}]$ y 
    $\psi(t,1)=[1-t,\frac{1}{2}]$, lo que implica que $\psi(t,0)=\psi(t,1)$. Si $t\leq\frac{1}{2}$
    notamos que $\psi(0,t)=[1,t+\frac{1}{2}]$ y $\psi(1,t)=[0,t+\frac{1}{2}]$, es decir, 
    $\psi(0,t)=\psi(1,t)$, del mismo modo se tiene el resultado cuando $t\geq\frac{1}{2}$. 
    
    \vspace{1mm}
    Por lo tanto, la función $\psi$ desciende a una función continua $h:\T^{2}\to\T^{2}$ que 
    verifica $h^{2}=id$. Afirmamos que $h\in\triangle(\T^{2},p)$, en efecto, para 
    $y\leq\frac{1}{2}$ se sigue que
    \begin{equation*}
        ph([x,y])=p([1-x,y+1/2])=[x,2y]=p([x,y])
    \end{equation*}
    similarmente se tiene para $t\geq\frac{1}{2}$. Consideramos el punto $x_{0}=[0,0]\in K$. 
    Luego,
    \begin{equation*}
        p^{-1}(x_{0})=\left\{\left[0,0\right],\left[1,\frac{1}{2}\right]\right\}\subseteq\T^{2}
    \end{equation*}
    Notar que $h([0,0])=[1,\frac{1}{2}]$ y $h([1,\frac{1}{2}])=[0,0]$, Por ende, 
    $\triangle(\T^{2},p)$ actúa transitivamente en $p^{-1}(x_{0})$.
\end{enumerate}

%\printbibliography % Quitar el comentado si quiero usar bibliografia

\end{document}