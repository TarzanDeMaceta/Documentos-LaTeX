\documentclass{article}
\usepackage{hyperref}
\usepackage{Style}

\nocite{*} % Comentar si quiero citar
%\addbibresource{bibliografia.bib} % Quitar el comentado si quiero usar bibliografia

\begin{document}

\begin{minipage}{2.5cm}
    \includegraphics[width=2cm]{imagen_puc.jpg}
\end{minipage}
\begin{minipage}{14cm}
    {\sc Pontificia Universidad Católica de Chile\\
    Facultad de Matemáticas\\
    Departamento de Matemática\\
    Profesor: Mauricio Bustamante -- Estudiante: Benjamín Mateluna}
\end{minipage}
\vspace{1ex}

{\centerline{\bf Topología Álgebraica - MAT2850}
\centerline{\bf Tarea 4}}
\centerline{\bf 07 de noviembre de 2025}

\section*{Problema 1}
\begin{enumerate}
    \item En primer lugar veamos que $\sbullet$ esta bien definida, es decir, dados $\alpha,\beta$
    caminos basados en $x_{0}$ tales que $[\alpha]=[\beta]$, entonces
    $\widehat{x}\hspace{1mm}{\sbullet}\hspace{1mm}[\alpha]
    =\widehat{x}\hspace{1mm}{\sbullet}\hspace{1mm}[\beta]$. En efecto, por levantamiento de 
    homotopías relativas, se sigue que $\widehat{\alpha}(1)=\widehat{\beta}(1)$, lo que implica 
    que
    \begin{equation*}
        \widehat{x}\hspace{1mm}{\sbullet}\hspace{1mm}[\alpha]=\widehat{\alpha}(1)
        =\widehat{\beta}(1)=\widehat{x}\hspace{1mm}{\sbullet}\hspace{1mm}[\beta]
    \end{equation*}

    Además, como $p\widehat{\alpha}=\alpha$, se sigue que $\widehat{\alpha}(1)\in p^{-1}(x_{0})$. 
    Veamos que $\sbullet$ induce una acción por la derecha de $\pi_{1}(X,x_{0})$ en 
    $p^{-1}(x_{0})$. Sea $\widehat{x}\in p^{-1}(x_{0})$ y $ct_{x_{0}}$ el lazo constante basado
    en $x_{0}$. Notemos que el lazo $ct_{\widehat{x}}$ levanta a $ct_{x_{0}}$, luego 
    $\widehat{x}\hspace{1mm}{\sbullet}\hspace{1mm}[ct_{x_{0}}]=ct_{\widehat{x}}(1)=\widehat{x}$.

    \vspace{1mm}
    Sean $\alpha,\beta$ lazos basados en $x_{0}$ y $\widehat{x}\in p^{-1}(x_{0})$. Afirmamos que
    $[p(\widehat{\alpha}^{-1}*\widehat{\alpha*\beta})]=[\beta]$ en $\pi_{1}(X,x_{0})$. En primer
    lugar, vemos que
    \begin{equation*}
        p(\widehat{\alpha}^{-1}*\widehat{\alpha*\beta})(0)=\alpha(1)=x_{0}
        \hhtext{y}
        p(\widehat{\alpha}^{-1}*\widehat{\alpha*\beta})(1)=\beta(1)=x_{0}
    \end{equation*}
    por lo que la expresión tiene sentido. Por otro lado, notemos que 
    $p(\widehat{\alpha}^{-1}*\widehat{\alpha*\beta})=p(\widehat{\alpha}^{-1})*\alpha*\beta$ y
    adicionalmente tenemos que
    \begin{equation*}
        [\alpha]*[p\widehat{\alpha^{-1}}]=[ct_{x_{0}}]=p_{*}[ct_{\widehat{x}}]
        =p_{*}[\widehat{\alpha}*\widehat{\alpha}^{-1}]=[p(\widehat{\alpha}*\widehat{\alpha}^{-1})]
        =[p\widehat{\alpha}*p\widehat{\alpha}^{-1}]=[p\widehat{\alpha}]*[p\widehat{\alpha}^{-1}]
        =[\alpha]*[p\widehat{\alpha}^{-1}]
    \end{equation*}
    lo que implica que $p\widehat{\alpha}^{-1}\sim\alpha^{-1}$, lo que prueba la afirmación. Así,
    se tiene lo siguiente
    \begin{equation*}
        (\widehat{x}\hspace{1mm}{\sbullet}\hspace{1mm}[\alpha])
        \hspace{1mm}{\sbullet}\hspace{1mm}[\beta]
        =\widehat{\alpha}(1)\hspace{1mm}{\sbullet}\hspace{1mm}[\beta]
        =\widehat{\alpha}(1)\hspace{1mm}{\sbullet}\hspace{1mm}[
            p(\widehat{\alpha}^{-1}*\widehat{\alpha*\beta})
        ]=\widehat{\alpha*\beta}(1)=\widehat{x}\hspace{1mm}{\sbullet}\hspace{1mm}[\alpha*\beta]
    \end{equation*}
    Notar que $\widehat{\alpha}^{-1}*\widehat{\alpha*\beta}(0)=\widehat{\alpha}(1)$.
    
    \item Supongamos que $\widehat{X}$ es arcoconexo, sean $\widehat{x}_{1},\widehat{x}_{2}\in 
    p^{-1}(x_{0})$, existe $\widehat{\gamma}:[0,1]\to\widehat{X}$ continua tal que 
    $\widehat{\gamma}(0)=\widehat{x}_{1}$ y $\widehat{\gamma}(1)=\widehat{x}_{2}$. Definimos 
    $\gamma=p\widehat{\gamma}$, un lazo basado en $x_{0}$, entonces
    \begin{equation*}
        \widehat{x}_{1}\hspace{1mm}{\sbullet}\hspace{1mm}[\gamma]
        =\widehat{\gamma}(1)=\widehat{x}_{2}
    \end{equation*}

    Por otro lado, supongamos que $\sbullet$ es transitiva. Sean 
    $\widehat{x},\widehat{y}\in\widehat{X}$, tenemos dos casos, 
    $\widehat{x},\widehat{y}\in p^{-1}(x_{0})$ para algún $x_{0}\in X$, entonces, existe un lazo 
    $\gamma$ basado en $x_{0}$ tal que $\widehat{\gamma}(0)=\widehat{x}$ y
    \begin{equation*}
        \widehat{\gamma}(1)=\widehat{x}\hspace{1mm}{\sbullet}\hspace{1mm}[\gamma]=\widehat{y}
    \end{equation*}
    por lo tanto, $\widehat{\gamma}$ es el camino buscado. En cambio, si 
    $\widehat{x}\in p^{-1}(x)$ e $\widehat{y}\in p^{-1}(y)$ con $x\neq y$, como $X$ es arcoconexo,
    existe un camino $\gamma$ de modo que $\gamma(0)=x$ y $\gamma(1)=y$. Por lema del 
    levantamiento $\widehat{\gamma}(0)=\widehat{x}$ y $\widehat{\gamma}(1)
    =\widehat{y'}\in p^{-1}(y)$ y por el caso anterior concluimos.
    
    \item Debemos probar que dado $\widehat{x}\in\widehat{X}$ se tiene que
    \begin{equation*}
        p_{*}(\pi_{1}(\widehat{X},\widehat{x}))=S_{\widehat{x}}
    \end{equation*}
    donde $S_{\widehat{x}}$ es el estabilizador de $\widehat{x}$. Sea 
    $[\alpha]\in S_{\widehat{x}}$, entonces $\widehat{x}
    =\widehat{x}\hspace{1mm}{\sbullet}\hspace{1mm}[\alpha]=\widehat{\alpha}(1)$, como 
    $p\widehat{\alpha}=\alpha$, concluimos que 
    $[\alpha]\in p_{*}(\pi_{1}(\widehat{X},\widehat{x}))$. Sea 
    $[\alpha]\in p_{*}(\pi_{1}(\widehat{X},\widehat{x}))$, entonces existe un lazo basado en 
    $\widehat{x}$, digamos $\widehat{\alpha}$, tal que $[p\widehat{\alpha}]=[\alpha]$, entonces
    \begin{equation*}
        \widehat{x}\hspace{1mm}{\sbullet}\hspace{1mm}[\alpha]
        =\widehat{x}\hspace{1mm}{\sbullet}\hspace{1mm}[p\widehat{\alpha}]=\widehat{\alpha}(1)
        =\widehat{x}
    \end{equation*}
    
    \item Usando la parte anterior y orbita establizador, resulta que
    \begin{equation*}
        \abs{\frac{\pi_{1}(X,x_{0})}{p_{*}(\pi_{1}(\widehat{X},\widehat{x}))}}
        =\abs{O_{\widehat{x}}}=\abs{p^{-1}(x_{0})}
    \end{equation*}
    donde en la última igualdad usamos que la acción es transitiva, o equivalentemente, que 
    $\widehat{X}$ es arcoconexo.
\end{enumerate}

\section*{Problema 2}

\noindent El problema planteado es similar al problema 1.IV, pero faltan condiciones.

\section*{Problema 3}

\noindent Consideremos la función $f:[0,1]^{2}\to[0,1]^{2}$ definida por
\begin{equation*}
    f(x,y):=\begin{cases}
        (x,1-2y) &\quad\text{si} \hspace{1mm} 0\leq y\leq\frac{1}{2} \\
        (x,2y-1) &\quad\text{si} \hspace{1mm} \frac{1}{2}\leq y\leq1
    \end{cases}
\end{equation*}
y por lema del pegado, vemos que $f$ es continua. Tenemos el diagrama

\vspace{2mm}
\centerline{
    \xymatrix{
        [0,1]^{2} \ar[r]^{f} \ar[d]^{\pi_{\T^{2}}} \ar[rd]^{\phi} & [0,1]^{2} \ar[d]^{\pi_{K}} \\
        \T^{2} & K
    }
}
\vspace{2mm}
\noindent donde $\phi=\pi_{K}\circ f$. Queremos que $\phi$ descienda a un función continua entre
$\T^{2}$ y $K$, veamos que es constante en las fibras de $\pi_{\T^{2}}$. Sea $t\in[0,1]$, sabemos
que $(0,t)\sim(1,t)$. Si $t\leq\frac{1}{2}$, entonces $f(0,t)=(0,1-2t)\sim(1,1-2t)=f(1,t)$, del
mismo modo para $t\geq\frac{1}{2}$. Por otro lado, tenemos que $(t,0)\sim(t,1)$, luego
$f(t,0)=(t,1)=f(t,1)$.

\vspace{1mm}
\noindent Sea $\varphi:\T^{2}\to K$ la función inducida por $\phi$, afirmamos que 
$(T^{2},\varphi)$ es un espacio cubriente regular de $K$. Bajo lo anterior, concluimos que
\begin{equation*}
    \Z^{2}\cong\pi_{1}(\T^{2},x_{0})\cong\varphi_{*}(\pi_{1}(\T^{2},x_{0}))\unlhd\pi_{1}(K,y_{0})
\end{equation*}
con $y_{0}=\varphi(x_{0})$. Lo anterior concluye el ejercicio.

\vspace{1mm}
\noindent Para probar la afirmación, debemos demostrar los siguientes dos puntos
\begin{enumerate}
    \item La tupla $(\T^{2},\varphi)$ es un espacio cubriente de $K$.
    
    \item El cubrimiento es regular, esto es equivalente a probar que el grupo 
    $\triangle(\T^{2},\varphi)$ actúa transitivamente en $\varphi^{-1}(x_{0})$ para algún 
    $x_{0}\in K$. Definimos la función continua $g:[0,1]^{2}\to[0,1]^{2}$ por $g(x,y)=(x,1-y)$, 
    se tiene el diagrama
    
    \vspace{2mm}
    \centerline{
        \xymatrix{
            [0,1]^{2} \ar[r]^{g} \ar[d]^{\pi_{\T^{2}}} \ar[rd]^{\rho}
            & [0,1]^{2} \ar[d]^{\pi_{\T^{2}}} \\
            \T^{2} & \T^{2}
        }
    }
    \vspace{2mm}

    donde $\rho:=\pi_{\T^{2}}\circ g$. Sea $t\in[0,1]$, entonces $g(0,t)=(0,1-t)\sim(1,1-t)
    =g(1,t)$ y además $g(t,1)=(t,0)\sim(t,1)=g(t,0)$, lo que implica que $\rho$ desciende a una
    función continua $r:\T^{2}\to\T^{2}$, más aún, como $g$ es homeomorfismo también lo es $r$.
    Por otro lado, sea $(x,y)$ tal que $y\geq\frac{1}{2}$, se sigue que
    \begin{equation*}
        \varphi([x,y])=[x,2y-1]=[x,1-2(1-y)]=\varphi([x,1-y])=\varphi\circ r([x,y])
    \end{equation*}
    similarmente se tiene para $y\leq\frac{1}{2}$. Concluimos que $r\in\triangle(\T^{2},\varphi)$.
    Consideramos $x_{0}=[\frac{1}{2},\frac{1}{3}]$, luego
    \begin{equation*}
        \varphi^{-1}(x_{0})=\left\{\left[\frac{1}{2},\frac{1}{3}\right], 
        \left[\frac{1}{2},\frac{2}{3}\right]\right\}
    \end{equation*}
    Notar que $r([\frac{1}{2},\frac{1}{3}])=[\frac{1}{2},\frac{2}{3}]$ y viceversa. Por ende, 
    $\triangle(\T^{2},\varphi)$ actúa transitivamente en $\varphi^{-1}(x_{0})$.
\end{enumerate}

%\printbibliography % Quitar el comentado si quiero usar bibliografia

\end{document}
