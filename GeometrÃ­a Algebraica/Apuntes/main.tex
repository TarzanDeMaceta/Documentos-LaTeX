\documentclass{article}
\usepackage{hyperref}
\usepackage{Style}

\nocite{*} % Comentar si quiero citar
%\addbibresource{bibliografia.bib} % Quitar el comentado si quiero usar bibliografia

\begin{document}

\begin{minipage}{2.5cm}
    \includegraphics[width=2cm]{imagen_puc.jpg}
\end{minipage}
\begin{minipage}{14cm}
    {\sc Pontificia Universidad Católica de Chile\\
    Facultad de Matemáticas\\
    Departamento de Matemática\\
    Profesor: Giancarlo Urzúa -- Estudiante: Benjamín Mateluna}
\end{minipage}
\vspace{1ex}

{\centerline{\bf Geometría Algebraica - MAT2824}
\centerline{\bf Apuntes}}
\centerline{\bf 06 de Marzo de 2025}

\newpage
\tableofcontents

\newpage
\section*{Introducción}
\phantomsection
\addcontentsline{toc}{section}{Introducción}

\noindent Habrán tres evaluaciones (I1, I2, I3) cada una vale un $20\%$ y un examen (EX) que vale 
un $40\%$. Las fechas son, 9 de abril, 14 de Mayo, 11 de Junio y 1 de Julio respectivamente.

\newpage
\section{Conjuntos Algebraicos afines}
\subsection{Preliminares algebraicos}
\noindent Sea $R$ un anillo conmutativo con $+,\cdot$ y con $1\neq0$. Si $R,R'$ son anillos, un 
morfismo de anillos es una función $f:R\to R'$ que respeta $+,\cdot$ y $f(1_{R})=1_{R'}$. Un 
dominio $R$ es un anillo en donde $xy=xz$ implica que $y=z$ para todo $x\neq0$.
\vspace{4mm}

\noindent\textbf{Ejemplo} $\Z$ es dominio, pero $\Z/6$ no lo es.
\vspace{4mm}

\noindent Un cuerpo es un dominio donde todo $x\neq0$ tiene un inverso. Dado $R$ dominio, existe el
cuerpo de fracciones $K$ tal que $R\subseteq K$. Dado $R$ anillo, sea $R[x]$ el anillo de 
polinomios con coeficientes en $R$, sus elementos tienen la forma

\begin{equation*}
    f(x)=a_{0}+a_{1}x+\cdots+a_{d}x^{d}\text{ , }a_{d}\neq0
\end{equation*}

\noindent y decimos que $f$ tiene grado $d$ denotado por $gr(f)$. Se define de manera recursiva
$R[x_{1},\cdots,x_{n}]=R[x_{1},\cdots,x_{n-1}][x_{n}]$ el anillo de polinomios en $n$ variables.
Dado $f=\alpha\cdot x_{1}^{\lambda_{1}}\cdots x_{n}^{\lambda_{n}}$ su grado se define como 
$gr(f)=\sum_{i=1}^{n}\lambda_{i}$, para $f$ en general, definimos su grado como 
$gr(f):=max\{\text{grados de monomios}\}$. Dado $f\in R[x_{1},\cdots,x_{n}]$ y $d=gr(f)$ entonces

\begin{equation*}
    f=F_{0}+F_{1}+\cdots+F_{d}\text{ , con }F_{i}\text{ homogeneos, esto es, }
    F_{i}(\lambda x_{x_{1}},\cdots,\lambda x_{n})=\lambda^{i}F(x_{1},\cdots,x_{n})
\end{equation*}

\noindent Si $f\in R[x]$ una raíz (cero) de $f$ es un $r\in R$ tal que $f(r)=0$.

\begin{teo}
    Se tiene que $r$ es cero si y solo si $f(x)=(x-r)g(x)$ para algún $g\in R[x]$.
\end{teo}

\noindent Un cero de $f(x_{1},\cdots,x_{n})$ es un $(a_{1},\cdots,a_{n})\in\R^{n}$ tal que 
$f(a_{1},\cdots,a_{n})=0$.
\vspace{4mm}

\noindent Decimos que $r\in R$ es irreductible si toda descomposición $r=ab$ con $a,b\in R$ se 
tiene que $a$ o $b$ es una unidad. Un anillo $R$ se dice dominio de factorización unica si todo 
elemento no nulo se puede factorizar de manera esencialmente unica en producto de irreductibles.

\begin{lema}
    Si $R$ es dominio de factorización unica entonces $R[x]$ es dominio de factorización unica.
\end{lema}

\begin{lema}
    Si $R$ es un dominio de factorización unica y $K$ su cuerpo de fracciones. 
Dado $f\in R[x]$ irreductible entonces $f$ es irreductible en $K[x]$.
\end{lema}

\noindent Sea $R$ un anillo. Un ideal $I\subset R$ es tal que si $a,b\in I$ entonces $a+b\in I$ y
si $r\in R$ entonces $ra\in I$. Consideramos la función $\pi:R\to R/I$ donde $R/I$ es el anillo 
cociente que es conmutativo. Un ideal es maximal si y solo si $R/I$ es cuerpo.

\begin{teo}
    Sea $R$ un dominio euclideano (se cumple algoritmo de la división) y $a,b\in R$, consideremos
    $mcd(a,b)=d$. Entonces existen $c,e\in R$ tales que $ac+be=d$.
\end{teo}

\begin{teo}
    Si $F$ es un polinomio homogeneo de grado $d$, entonces
    \begin{equation*}
        dF=x_{1}F_{x_{1}}+\cdots+x_{n}F_{x_{n}}
    \end{equation*}
    donde $F_{x_{i}}$ es la derivada formal con respecto a $x_{i}$.
\end{teo}

\subsection{Espacio Afín y Conjuntos Algebraicos}
\begin{dfn}
Sea $k$ un cuerpo. El espacio afín de dim $n$ es $\A_{k}^{n}:=k^{n}$ (generalmente se supondra
que $k=\overline{k}$).
\end{dfn}

\begin{dfn}
    Una hipersuperficie de $\A_{k}^{n}$ es $V(F)=\{p\in\A_{k}^{n}:F(p)=0\}$ para un 
    $F\in k[x_{1},\cdots,x_{n}]$.
\end{dfn}

\noindent\textbf{Ejemplos:}
\begin{itemize}
    \item Sea $k=\R$ consideramos la hipersuperficie $V(y^{2}-x^{2}(x+1))\subseteq\A_{\R}^{2}$ 
    (foto)

    El punto $(0,0)$ se llama nodo.
    \item Veamos la hipersuperficie $V((x^{3}-y^{3})(y^{3}-1)(x^{3}-1))\subseteq\A_{\C}^{2}$.
    \item La hipersuperficie $V(x^{2}+y^{2}-z^{2})\subseteq\A_{\R}^{3}$ es conocida como cono
    (foto)

    Como en el primer ejemplo, el punto $(0,0)$ se llama nodo
    \item Consideremos $V(y^{2}-x^{3})\subseteq\A_{\R}^{2}$ (foto)
    
    En este caso, el punto $(0,0)$ no es un nodo, en este caso se llama cuspide.
    \item Veamos el caso de una hipersuperficie no parametrizable, esta es 
    $V(y^{2}-x(x+1)(x+\lambda))$.
\end{itemize}

\begin{dfn}
    Sea $S\subseteq k[x_{1},\cdots,x_{n}]$ un conjunto arbitrario, se define 
    \begin{equation*}
        V(S):=\{p\in\A_{k}^{n}:F(p)=0\hspace{4mm}\forall F\in S\}=\bigcap_{F\in S} V(F)
    \end{equation*}
    y se dice que es un conjunto algebraico afín.
\end{dfn}

\noindent Propiedades de un conjunto algebraico afín:
\begin{enumerate}
    \item Sea $I=\gen{S}=\left\{\ds\sum_{i=1}^{n}a_{i}s_{i},a_{i}\in k\right\}$, entonces 
    $V(I)=V(S)$.
    \begin{dem}
        Veamos que $V(I)\subseteq V(S)$, si $p\in V(I)$, como $S\subseteq I$ se sigue que 
        $p\in V(S)$. Para $V(S)\subseteq V(I)$ notemos que dado $f\in I$ se tiene que 
        $f=\sum a_{i}s_{i}$, luego si $p\in V(S)$ vemos que $f(p)=\sum a_{i}s_{i}(p)=0$.
    \end{dem}

    \item Sea $\{I_{\alpha}\}$ una colección de ideales, entonces 
    $V(\bigcup_{\alpha}I_{\alpha})=\bigcap_{\alpha}V(I_{\alpha})$.
    \item Si $I\subseteq J$ se sigue que $V(J)\subseteq V(I)$.
    \item Sean $F,G\in k[x_{1},\cdots,x_{n}]$, se tiene que $V(FG)=V(F)\cup V(G)$.
    \item Tenemos las siguientes dos identidades $V(1)=\emptyset$ y $V(0)=\A_{k}^{n}$.
    
    \textbf{observación:} Lo anterior es valido si $k$ es algebraicamente cerrado, de lo 
    contrario, si consideramos $\A_{\R}^{1}$ vemos que $V(x^{2}+1)=\emptyset$.
\end{enumerate}

\subsection{Ideal de un conjunto}
\begin{dfn}
    Sea $X\subseteq\A_{k}^{n}$ un conjunto arbitrario. Se define el ideal de $X$ como
    \begin{equation*}
        I(X):=\left\{F\in k[x_{1},\cdots,x_{n}]:F(p)=0\hspace{4mm}\forall p\in X\right\}
    \end{equation*}
\end{dfn}

\noindent\textbf{observación:} Notemos que si $F^{m}\in I(X)$ entonces $F\in I(X)$. Un ideal con
esta propiedad se dice radical.
\vspace{4mm}

\noindent Propiedades del ideal de un conjunto:
\begin{enumerate}
    \item Si $X\subseteq Y$ se tiene que $I(Y)\subseteq I(X)$.
    \item Se tiene lo siguiente $I(\emptyset)=k[x_{1},\cdots,x_{n}]$ y $I(\A_{k}^{n})=\{0\}$.
    Además, si $k$ es un cuerpo infinito, se tiene que $I(\{a_{1},\cdots,a_{n}\})=
    (x_{1}-a_{1},\cdots,x_{n}-a_{n})$.
\end{enumerate}

\subsection{El Teorema de la Base de Hilbert}
\begin{teo}
    Todo conjunto algebraico corresponde a la intersección finita de hipersuperficies.
\end{teo}

\begin{dem}
    Sea $V(I)$ el conjunto algebraico para algún ideal $I\subseteq k[x_{1},\cdots,x_{n}]$. 
    Basta con probar que $I$ es finitamente generado, en tal caso $I=(F_{1},\cdots,F_{r})$, 
    entonces $V(I)=V(F_{1},\cdots,F_{r})=V(F_{1})\cap\cdots\cap V(F_{r})$.
\end{dem}

\begin{teo}
    Si $R$ es un anillo Noetheriano, entonces $R[X]$ es un anillo Noetheriano.
\end{teo}

\begin{dem}
    Sea $I\subseteq R[X]$ un ideal. Dado $F=a_{0}+a_{1}x+\cdots+a_{d}x^{d}$ con $a_{d}\neq0$
    decimos que $a_{d}$ es el término líder de $F$ denotado por $l(F)$. Sea
    
    \begin{equation*}
        \J:=\{r\in R:r \text{ es término líder de algún } F\in I\}\cup\{0\}
    \end{equation*}
    Afirmamos que $\J$ es ideal, en efecto, sean $l(F),l(G)\in\J$, supongamos sin perdida de 
    generalidad que $gr(F)\leq gr(G)$, luego
    
    \begin{equation*}
        Fx^{gr(G)-deg(F)}+G=H
    \end{equation*}
    donde $l(H)=l(F)+l(G)$. Es claro que $r\cdot l(F)\in\J$ con $r\in R$. Por hipotesis existen 
    $F_{1},\cdots,F_{r}\in I$ tales que $\J=(l(F_{1}),\cdots,l(F_{r}))$. Sea $N>gr(F_{i})$ para
    todo $1\leq i\leq r$. Para cada $m\leq N$ definimos

    \begin{equation*}
        \J_{m}:=\{r\in R:r\text{ es término líder de }F\in I\text{ y }gr(F)\leq m\}
    \end{equation*}
    Notemos que los $\J_{m}$ son ideales en $R$, por ende, son finitamente generados, es decir
    $\J_{m}=(l(F_{m,j}))$. Consideremos el ideal $I'=\gen{F_{m,j}, F_{i}}$, afirmamos que $I'=I$.
    Claramente se tiene que $I'\subset I$. Supongamos, por contradicción, que $I'\neq I$, sea 
    $G\in I'\setminus I$ de menor grado. Tenemos dos consideramos

    \begin{enumerate}
        \item Veamos cuando $gr(G)>N$, existen polinomios $Q_{i}\in R[X]$ tal que $G$ y 
        $\sum Q_{i}F_{i}$ tienen el mismo coeficiente líder. Luego $G-\sum Q_{i}F_{i}\in I'$ pues
        tiene menor grado que $G$, se sigue que $G\in I'$.

        \item El resultado para $gr(G)\leq N$ se obtiene del mismo modo, usando esta vez los 
        $F_{m,j}$.
    \end{enumerate}
\end{dem}

\noindent\textbf{Ejemplo:} Sea $(0,0)\in\A_{\R}^{2}$, entonces $\{(0,0)\}=V(x^{2}+y^{2})$. Pero en
$\C$ tenemos que $\{(0,0)\}\neq V(F)$ para ningún $F\in k[x,y]$.

\subsection{Componentes Irreducibles en un Conjunto Algebraico}

\begin{dfn}
    Un conjunto algebraico $V$ se dice reducible si $V=V_{1}\cup V_{2}$ con $V_{i}$ conjunto
    conjunto algebraico y distinto de $V$.
\end{dfn}

\noindent\textbf{Observación: } Un punto es un conjunto algebraico irreducible, lo que implica que
cualquier conjunto finito es algebraico y reducible.

\noindent\textbf{Ejemplos: }
\begin{itemize}
    \item Notemos que $V(xy)=V(x)\cup V(y)$, es decir $V(xy)$ es reducible.
    \item Consideremos el espacio afín $\A_{\R}^{1}$, entonces el conjunto algebraico 
    $V((x^{2}+1)x)=\{0\}$ es irrducible.
\end{itemize}

\begin{prop}
    Un conjunto algebraico $V$ es irrducible si y solo si el ideal $I(V)$ es primo.
\end{prop}

\begin{dem}\hspace{1mm}
    \begin{itemize}
        \item $\Rightarrow|$ Supongamos que $I(V)$ no es primo, entonces existen $F_{1},F_{2}$ 
        polinomios tales que $F_{1}\cdot F_{2}\in I(V)$ y $F_{1},F_{2}\notin I(V)$. Afirmamos que
        $V=(V\cap V(F_{1}))\cup(V\cap V(F_{2}))$. Sea $p\in V$, entonces $F_{1}(p)\cdot F_{2}(p)=0$
        lo que implica que $p\in(V\cap V(F_{1}))\cup(V\cap V(F_{2}))$, además 
        $V\cap V(F_{i})\neq V$ ya que existe $q_{i}$ tal que $F_{i}(q_{i})\neq 0$.

        \item $\Leftarrow|$ Supongamos que $V$ es reducible. Luego $V=V_{1}\cup V_{2}$ con 
        $V_{i}\neq V$. Entonces existe un polinomio $F_{i}$ tal que $F_{i}(p)=0$ para todo 
        $p\in V_{i}$, pero no para todo punto en $V$. Notemos que $F_{1}\cdot F_{2}\in I(V)$, 
        sin embargo, $F_{i}\notin I(V)$.
    \end{itemize}
\end{dem}

\begin{dfn}
    Una variedad afín $V$ es un conjunto algebraico afín irreducible.
\end{dfn}

\begin{lema}
    Sea $R$ un anillo, las siguientes afirmaciones son equivalentes:
    \begin{enumerate}
        \item $R$ es Noetheriano.
        \item Si  $\mathcal{C}$ es una colección no vacía de ideales en $R$, entonces
        $\mathcal{C}$ tiene un elemento maximal, es decir, existe $I\in\mathcal{C}$ que no está 
        contenido en otro ideal de $\mathcal{C}$.
        \item Toda cadena ascendente de ideales en $R$ se estabiliza.
    \end{enumerate}
\end{lema}

\begin{dem} \hspace{1mm}
    \begin{itemize}
        \item (a) $\Rightarrow$(b) $|$ Necesitamos usar el axioma de elección. Sea $\mathcal{C}$
        una colección de ideales en $R$, para cada subconjunto no vacío de $\mathcal{C}$ elegimos
        un ideal. Sea $I_{0}$ el ideal escogido para $\mathcal{C}$, definimos el conjunto

        \begin{equation*}
            \mathcal{C}_{1}:=\{I\in\mathcal{C}:I_{0}\subset I\}
        \end{equation*}
        Si $\mathcal{C}_{1}=\emptyset$ entonces $I_{0}$ es el ideal maximal. Si no, repetimos el
        proceso. Sea $I\in\mathcal{C}_{1}$ el escogido, definimos

        \begin{equation*}
            \mathcal{C}_{2}:=\{I\in\mathcal{C}_{2}:I_{1}\subset I\}
        \end{equation*}
        Es suficiente demostrar que existe $n$ tal que $\mathcal{C}_{n}=\emptyset$. Sea 
        $I=\bigcup_{n=0}^{\infty}I_{n}$ es ideal, además, notemos que $I_{n}\subset I_{n+1}$. 
        Como $R$ es Noetheriano, entonces $I=(f_{1},\cdots,f_{m})$, luego existe $r$ tal que 
        $f_{1},\cdots,f_{m}\in I_{r}$, lo que implica que $I\subseteq I_{r}$ y por lo tanto 
        $I=I_{r}$ se sigue que $I_{r}=I_{s}$ para todo $s>r$, lo cual es una contradicción.

        \item (b) $\Rightarrow$(c) $|$ Basta tomar $\mathcal{C}$ como nuestra colección de ideales
        en $R$, luego, existe un elemento maximal.

        \item (c) $\Rightarrow$(a) $|$ Sea $I\subseteq R$ un ideal. Si $I=(0)$ estamos listos, de
        lo contrario, sea $f_{1}\in I$, entonces $(f_{1})\subseteq I$. Supongamos que 
        $I\setminus(f_{1})\neq\emptyset$, sea $f_{2}\in I\setminus(f_{1})$, de esta manera
        construimos una cadena ascendente de ideales

        \begin{equation*}
            (f_{1})\subset (f_{1},f_{2})\subset\cdots\subset(f_{1},\cdots,f_{n})\subset\cdots
        \end{equation*}
        para algun $N$ la cadena se estabiliza y por ende $(f_{1},\cdots,f_{N})=I$.
    \end{itemize}
\end{dem}

\begin{prop}
    Cualquier colección de conjuntos algebraicos $\{V_{i}\}_{i\in I}$ en $\A_{k}^{n}$ tiene un
    elemento minimal.
\end{prop}

\begin{dem}
    Dada $\{V_{i}\}_{i\in I}$ obtenemos una colección $\mathcal{C}=\{I(V_{i})\}_{i\in I}$ de
    ideales en $k[x_{1},\cdots,x_{n}]$, el cual es Noetheriano. Luego $\mathcal{C}$ tiene un 
    elemento maximal, digamos $I(V_{*})$, afirmamos que $V_{*}$ es el elemento minimal, de lo
    contrario, existe $V_{i}\subseteq V_{*}$ entonces $I(V_{*})\subseteq I(V_{i})$.
\end{dem}

\begin{teo}
    Sea $V\subseteq\A_{k}^{n}$ un conjunto algebraico. Entonces existen unicos conjuntos 
    algebraicos irreducibles $V_{1},\cdots,V_{m}$ tales que 
    \begin{equation*}
        V=\bigcup_{i}^{m}V_{i}\hspace{4mm}\text{y}\hspace{4mm}V_{i}\not\subset 
        V_{j}\hspace{4mm}\forall i\neq j
    \end{equation*}   
\end{teo}

\begin{dem}
    Sea $\mathcal{C}=\{V\subseteq\A_{k}^{n}\text{ conjunto algebraico}:V\text
    { no es unión finita de irreducibles}\}$. Si $\mathcal{C}$ es vacío estamos listos. Si no lo 
    es, sea $V\in\mathcal{C}$ minimal. Tenemos que $V$ no es irreducible, entonces 
    $V=V_{1}\cup V_{2}$ con $V_{i}\subset V$, lo que implica que algún $V_{i}\in\mathcal{C}$ lo
    cual es una contradicción.
    \vspace{4mm}

    \noindent Sea $V=\bigcup_{i=1}^{m}V_{i}$ con $V_{i}$ irreducibles, asumir que 
    $V_{i}\not\subset V_{j}$ para todo $i\neq j$. Digamos que
    
    \begin{equation*}
        \bigcup_{i=1}^{m}V_{i}=\bigcup_{j=1}^{s}W_{j}\hspace{4mm}\text{con}\hspace{4mm}
        V_{i}\not\subset V_{j}\hspace{4mm}\text{y}\hspace{4mm} W_{i}\not\subset W_{j}
        \hspace{4mm}\text{y}\hspace{4mm}V_{i},W_{j}\neq\emptyset
    \end{equation*}
    Notemos que $V_{1}=V_{1}\cap V=\bigcup_{j=1}^{s}(V_{1}\cap W_{j})$, como $V_{1}$ es 
    irreducible, existe unico $j$ tal que $V_{1}=V_{1}\cap W_{j}$, es decir, 
    $V_{1}\subseteq W_{j}$. Por otro lado, existe unico $i$ tal que $W_{j}\subseteq V_{i}$, lo que
    implica que $V_{1}\subseteq V_{i}$ entonces $i=1$ y así $V_{1}=W_{j}$.
\end{dem}

\subsection{Conjuntos Algebraicos del Plano}

\begin{lema}
    Si $f,g\in k[x,y]$ no tienen factores en común, entonces $V(f,g)$ es un conjunto finito.
\end{lema}

\begin{dem}
    Recordemos que $k(x)[y]$ es dominio euclideano. Por lema de gauss, $f,g$ no tienen factores en
    común en $k(x)[y]$, entonces existen $a,b\in k(x)[y]$ tal que $af+bg=1$. Existe $r(x)$ tal que
    
    \begin{equation*}
        raf+rbg=r
    \end{equation*}
    es una ecuación en $k[x,y]$. Sea $(p,q)\in V(f,g)$, evaluando en la ecuación anterior vemos que
    
    \begin{equation*}
        0=raf(p,q)+rbg(p,q)=r(p)
    \end{equation*}
    por lo tanto la cantidad de valores posibles de $p$ es finita. Haciendo lo mismo para $y$ 
    obtenemos que $q$ solo puede tomar una cantidad finita de valores.
\end{dem}

\begin{cor}
    Si $f\in k[x,y]$ es irreducible con $\abs{V(f)}=\infty$ entonces $I(V(f))=(f)$ y $V(f)$ es 
    irreducible.
\end{cor}

\begin{dem}
    Si $g\in I(V(f))$, entonces $\abs{V(f,g)}=\infty$, luego, $f$ y $g$ tienen factores en común,
    como $f$ es irreducible, entonces $f$ divide a $g$ lo que implica que $g\in(f)$. La otra
    contención es directa.
    \vspace{4mm}

    \noindent Por otro lado, notemos que $(f)$ es primo, pues f es irreducible, así, $V(f)$ es 
    irreducible.
\end{dem}

\begin{cor}
    Supongamos que $k$ es infinito, entonces los conjuntos algebraicos irreducibles de $\A_{k}^{2}$
    son: $\emptyset$, $\A_{k}^{2}$, un punto y los conjuntos $V(f)$ con $f$ irreducible y 
    $\abs{V(f)}=\infty$.
\end{cor}

\begin{dem}
    Sea $V$ un conjunto algebraico irreducible. Si $\abs{V}<\infty$ entonces $V=\emptyset$ o $V$
    es un punto. Si $I(V)=(0)$ entonces $V=\A_{k}^{2}$. Supongamos que $\abs{V}=\infty$ y que
    $(0)\subset I(V)\subset k[x,y]$. Como $I(V)$ es primo, existe un polinomio no constante e
    irreducible tal que $f\in I(V)$.
    \vspace{4mm}

    \noindent Si $g\in I(V)$ y $g\not\in(f)$, entonces $V\subset V(f,g)$, por la proposición, esto
    es una contradicción. De este modo, $I(V)=(f)$. Afirmamos que $V(f)=V$, en efecto, tenemos que
    $V=V(I(V))=V(f)$.
\end{dem}

\begin{cor}
    Supongamos que $k=\overline{k}$. Sea $f\in k[x,y]$ y sea $f=\ds\prod_{i=1}^{m}f_{i}^{\alpha_{i}}$
    con $f_{i}$ irreducible. Entonces
    \begin{equation*}
        V(f)=\bigcup_{i=1}^{m}V(f_{i})
    \end{equation*}
    es su descomposición en irreducibles y además $I(V(f))=(f_{1},\cdots,f_{m})$.
\end{cor}

\begin{dem}
    Como $f_{i},f_{j}$ son coprimos no hay inclusiones entre $V(f_{i})$ y $V(f_{j})$, de lo 
    contrario si existen $i\neq j$ tales que $V(f_{i})\subset V(f_{j})$, entonces
    \begin{equation*}
        (f_{i})=I(V(f_{i}))\supset I(V(f_{j}))=(f_{j})
    \end{equation*}
    lo cual es una contradicción. Luego,
    \begin{equation*}
        I(V(f))=I\left(\bigcup V(f_{i})\right)=\bigcap I(V(f_{i}))=\bigcap(f_{i})=
        (f_{1}\cdots f_{m})
    \end{equation*}
\end{dem}

\subsection{Nullstellensatz de Hilbert}
\noindent En general supondremos que $k=\overline{k}$, a no ser que se diga lo contrario.

\begin{teo}
    Sea $I\subset k[x_{1},\cdots,x_{n}]$ un ideal, entonces $V(I)\neq\emptyset$.
\end{teo}

\begin{dem}
    Podemos suponer que $I$ es maximal. En efecto, recordemos que todo ideal esta contenido en un
    ideal maximal, digamos $M$, entonces $V(M)\subseteq V(I)$. $\{\}$Como $I$ es maximal, esto equivale
    a que $k[x_{1},\cdots,x_{n}]/I\supset k$ es cuerpo. Como $k$ es algebraicamente cerrado,
    podemos asumir que $k[x_{1},\cdots,x_{n}]/I=k$.
    \vspace{4mm}

    \noindent Así, cada variable $x_{i}$ puede ser identificada por un elemento en $k$ digamos
    $a_{i}$, lo que implica que $x_{i}-a_{i}$ es igual $0$ bajo el cociente, se sigue que 
    $x_{i}-a_{i}\in I$, luego $I=(x_{1}-a_{1},\cdots,x_{n}-a_{n})$. (Mejorar escritura)
\end{dem}

\noindent De la demostración surge una pregunta, ¿Por que $k[x_{1},\cdots,x_{n}]/I=k$? El siguiente 
lema lo responde

\begin{lema}
    (Lema de Zariski) Sea $K\subset L$ una extensión de cuerpo tal que $L$ es finitamente generado 
    como $k$-algebra. Entonces $L$ es finitamente generado como $k$-módulo.
\end{lema}

\noindent Exploraremos una demostración menos general del teorema anterior, pero sin usar lema de
Zariski. Para ello supongamos que $k=\C$.

\begin{dem}
    Del mismo modo, supongamos que $I\subset k[x_{1},\cdots,x_{n}]$ es un ideal maximal, luego
    $L:=k[x_{1},\cdots,x_{n}]/I$ es cuerpo, consideramos el morfismo canónico 
    \vspace{2mm}

    \centerline{
        \xymatrix{
            \C[x_{1},\cdots,x_{n}] \ar[r]^-{\pi} & L \\
            \C[x_{i}] \ar[u]^{i} \ar[ru]_{\pi_{i}:=\pi\big|_{\C[x_{i}]}}
        }
    }
    \vspace{2mm}

    Afirmamos que $ker(\pi_{i})=(0)$ o $ker(\pi_{i})=(x_{i}-a_{i})$ para algún 
    $a_{i}\in\C$. En efecto, si $ker(\pi_{i})\neq(0)$, entonces $(0)\subset ker(\pi_{i})
    \subset\C[x_{i}]$, donde la segunda contención es estricta, de lo contrario, $1\in I$ y
    entonces $I=k[x_{1},\cdots,x_{n}]$. Sea $f\in ker(\pi_{i})$, entonces como $\C$ es 
    algebraicamente cerrado, existe $(x_{i}-a_{i})$ factor de $f$ tal que $\pi_{i}(x_{i}-a_{i})=0$.
    \vspace{4mm}

    \noindent Volviendo a la demostración del teorema. Tenemos dos consideramos
    \begin{itemize}
        \item $ker(\pi_{i})=(x_{i}-a_{i})$ para todo $i$. Entonces $(x_{1}-a_{i},\cdots,
        x_{n}-a_{n})\subseteq I$. Como $(x_{1}-a_{i},\cdots,x_{n}-a_{n})$ es ideal maximal e $I$ es
        propio se obtiene el resultado.

        \item Existe $i$ tal que $ker(\pi_{i})=(0)$, entonces $\pi_{i}$ es inyectiva, como $L$ es
        cuerpo $\C(x_{i})$ se incrusta en $L$.
        \vspace{2mm}

        \centerline{
            \xymatrix{
                \C[x_{i}] \ar[r]^-{\pi_{i}} \ar[d]_{i} & L \\
                \C(x_{i}) \ar[ru]_{i_{L}}
            }
        }
        \vspace{2mm}

        \noindent Es decir $\C(x_{i})\subseteq L$. Notemos que $L$ es un espacio vectorial 
        numerable, a saber, la base corresponde a todos los monomios. Notemos que el siguiente 
        conjunto es linealmente independiente
        \begin{equation*}
            S:=\left\{\frac{1}{x_{i}-a_{i}}:a\in\C\right\}
        \end{equation*}
        Notemos que si $\sum_{j=1}^{m}\frac{\lambda_{j}}{x_{i}-a_{j}}=0$ entonces multiplicando 
        por $(x_{i}-a_{1})\cdots(x_{i}-a_{m})$ y evaluando se tiene que $\lambda_{j}=0$ para todo 
        $j$. Esto es una contradicción pues $S$ es no numerable.
    \end{itemize}
\end{dem}

\begin{teo}
    (Teorema de Nullstellensatz) Sea $I\subset k[x_{1},\cdots,x_{n}]$, entonces $I(V(I))=\sqrt{I}$.
\end{teo}

\begin{dem} \hspace{4mm}
    \begin{itemize}
        \item $\supseteq|$ Sea $f\in\sqrt{I}$, entonces $f^{n}\in I$ para algún $n$. Luego 
        $f^{n}(p)=0$ para todo $p\in V(I)$, entonces $f(p)=0$ para todo $p\in V(I)$ lo que implica
        que $f\in I(V(I))$.

        \item $\subseteq|$ (Truco de Rabinowitsch) Sea $f\in I(V(I))$ y digamos que 
        $I=(f_{1},\cdots,f_{m})$. Definimos el ideal $J:=(f_{1},\cdots,f_{m},x_{n+1}f-1)
        \subseteq k[x_{1},\cdots,x_{n+1}]$. Supongamos que $(a_{1},\cdots,a_{n},a_{n+1})\in V(J)$,
        entonces $(a_{1},\cdots,a_{n})\in V(I)$ se sigue que $f(a_{1},\cdots,a_{n})=0$, esto 
        resulta en una contradicción. Concluimos que $V(J)=\emptyset$.
        \vspace{4mm}

        \noindent Por el teorema anterior y como $k$ es algebraicamente cerrado tenemos que 
        $J=k[x_{1},\cdots,x_{n+1}]$, entonces existen $\{g_{i}\}_{i=1}^{m+1}\subseteq 
        k[x_{1},\cdots,x_{n}]$ tales que
        \begin{equation*}
            g_{1}f_{1}+\cdots+g_{m}f_{m}+g_{m+1}(x_{n+1}f-1)=1
        \end{equation*}
        tomando $x_{n+1}=1/f$ obtenemos
        \begin{equation*}
            g_{1}(x_{1},\cdots,x_{n},1/f)f_{1}+\cdots+g_{m}(x_{1},\cdots,x_{n},1/f)f_{m}=1
        \end{equation*}
        existe $n\in\N$ tal que $f^{n}\in I$.
    \end{itemize}
\end{dem}

\begin{cor}
    Hay una correspondecia uno a uno entre puntos en $\A_{k}^{n}$ e ideales maximales.
\end{cor}

\begin{cor}
    Las variedades afines en $\A_{k}^{n}$ estan en correspondecia uno a uno con los ideales primos.
\end{cor}

\begin{cor}
    Las hipersuperficies irreducibles en $\A_{k}^{n}$ se corresponden uno a uno con polinomios
    irreducibles en $k[x_{1},\cdots,x_{n}]$.
\end{cor}

\begin{cor}
    Sea $I\subseteq k[x_{1},\cdots,x_{n}]$ un ideal. Entonces $V(I)$ es un conjunto finito de 
    puntos si y solo si como $k$-espacio vectorial $k[x_{1},\cdots,x_{n}]/I$ tiene dimensión finita.
\end{cor}

\begin{dem}\hspace{4mm}
    \begin{itemize}
        \item $\Leftarrow|$ Sean $p_{1},\cdots,p_{r}\in V(I)\subseteq\A_{k}^{n}$. Consideramos
        $F_{1},\cdots,F_{r}\in k[x_{1},\cdots,x_{n}]$ tales que $F_{i}(p_{j})=0$ para todo 
        $i\neq j$ y $F_{i}(p_{i})=1$. Sea $\overline{F_{i}}$ la imagen de $F_{i}$ en el cociente
        $k[x_{1},\cdots,x_{n}]/I=R$.
        \vspace{4mm}

        \noindent Afirmamos que el conjunto $\{F_{1},\cdots,F_{r}\}$ es linealmente independiente
        en $R$. En efecto, si
        \begin{equation*}
            \sum_{i=1}^{r}\lambda_{i}\overline{F_{i}}=0\hspace{4mm}\text{con}\hspace{4mm}
            \lambda_{i}\in k
        \end{equation*}
        entonces $\sum\lambda_{i}\overline{F_{i}}\in I$, evaluando en $p_{i}$ vemos que 
        $\lambda_{i}=0$ para todo $i$, lo que prueba la afirmación. Así, $r\leq dim_{k}R$.

        \item $\Rightarrow|$ Digamos que $V(I)=\{p_{1},\cdots,p_{r}\}$ y 
        $p_{i}=(a_{i1},\cdots,a_{in})$. Definimos
        \begin{equation*}
            F_{j}:=\prod_{i=1}^{r}(x_{j}-a_{ij})
        \end{equation*}
        Luego $F_{j}\in I(V(I))$, por Nullstellensatz, se tiene que $F_{j}^{N}$ para algún $N$,
        así, $\overline{F_{j}}^{N}=0$ en $R$, es decir, $p(x_{j})+x_{j}^{rN}=0$, con 
        $gr(p_{j})<rN$ entonces $dim_{k}R<\infty$.
    \end{itemize}
\end{dem}

\noindent\textbf{Ejemplos:}
\begin{itemize}
    \item Consideremos los polinomios $x-y,y-x^{2}\in k[x,y]$, se sigue 
    $V((x-y,y-x^{2}))=\{(0,0),(1,1)\}$
    \begin{equation*}
        dim_{k}\left(k[x,y]\Big/(x-y,y-x^{2})\right)=dim_{k}\left(k[x]\Big/(x-x^{2})\right)
        =dim_{k}\left(k\oplus kx\right)=2
    \end{equation*}

    \item Notemos que $V(x-y-1,x-y)=\emptyset$ y por otro lado
    \begin{equation*}
        k[x,y]\Big/(x-y,x-y-1)=k[x,y]\Big/(1)=(0)
    \end{equation*}
    así $dim_{k}R=0$.

    \item Veamos que $V(y,x-y^{3})=\{(0,0)\}$, entonces
    \begin{equation*}
        dim_{k}\left(k[x,y]\Big/(y,x^{3}-y)\right)=dim_{k}\left(k[x]\Big/(x^{3})\right)=3
    \end{equation*}

    \item El conjunto $V(my-x,y-x^{2})$ tiene dos puntos de intersección para todo $m\neq0$,
    \begin{equation*}
        dim_{k}\left(k[x,y]\Big/(my-x,y-x^{2})\right)=dim_{k}\left(k[x]\Big/(mx^{2}-x)\right)=2
    \end{equation*}
    pero si $m=0$, vemos que  $dim_{k}R=1$.
\end{itemize}

\subsection{Modulos y Condiciones de Finitud}
Sea $R$ un anillo, se dice que $M$ es un $R$-módulo, si $M$ es un grupo conmutativo y si viene con
producto escalar, es decir, una función de $R\times M$ a $M$, se denota por $a\cdot m$ que 
satisface lo siguiente
\begin{itemize}
    \item $(a+b)m=am+bm$ para todo $a,b\in R$ y $m\in M$.
    \item $a(m+n)=am+an$ para todo $a\in R$ y $m,n\in M$.
    \item $(ab)m=a(bm)$ para todo $a,b\in R$ y $m\in M$.
    \item $1_{R}\cdot m=m$ para todo $m\in M$
\end{itemize}
Un subgrupo de $N$ de un $R$-módulo $M$ se dice un submodulo si $N$ es un $R$-módulo con el mismo
producto escalar. Dado $S\subseteq M$, definimos el generado de $S$ por
\begin{equation*}
    \gen{S}:=\left\{\sum r_{i}s_{i}\hspace{2mm}|\hspace{2mm}r_{i}\in R,s_{i}\in S\right\}
\end{equation*}
de hecho corresponde al submódulo de $M$ mas pequeño que contiene a $S$. Decimos que $M$ es 
finitamente generado si existe $S\subseteq M$ tal que $\gen{S}=M$.
\vspace{4mm}

\noindent Sea $R$ un subanillo de $S$. Decimos que $S$ es modulo finito sobre $R$, si es 
finitamente generado como $R$-módulo.
\vspace{4mm}

\noindent Sean $v_{1},\cdots,v_{n}\in S$. Sea $\varphi:R[x_{1},\cdots,x_{n}]\to S$ el morfismo
de anillo que manda $x_{i}$ a $v_{i}$. La imagen de $\varphi$ se denota por 
$R[v_{1},\cdots,v_{n}]$ y corresponde a un subanillo de $S$ que contiene a $R$ y 
$v_{1},\cdots,v_{n}$, además, es el subanillo mas pequeño con esta propiedad. Decimos que $S$ es 
un anillo finito sobre $R$ si $S=R[v_{1},\cdots,v_{n}]$ para algunos $v_{1},\cdots,v_{n}\in S$.
\vspace{4mm}

\noindent Sean $K\subset L$ cuerpos. Sean $v_{1},\cdots,v_{n}\in L$ y consideremos 
$K(v_{1},\cdots,v_{n})$ el cuerpo de fracciones de  $K[v_{1},\cdots,v_{n}]$. Al igual que antes,
corresponde al menor subcuerpo de $L$ que contiene a $K$ y $v_{1},\cdots,v_{n}$. El cuerpo $L$ se
dice una extensión finitamente generada de $K$ si $L=K(v_{1},\cdots,v_{n})$ para algunos
$v_{1},\cdots,v_{n}\in L$.

\subsection{Elementos Integrales}
\begin{dfn}
    Sean $R\subset S$ dominios enteros. Decimos que un elemento $v\in S$ es integral sobre $R$ si
    \begin{equation*}
        v^{n}+r_{n-1}v^{n-1}+\cdots+r_{1}v+r_{0}=0
    \end{equation*}
    para algunos $r_{i}\in R$ y $n\in\N$.
\end{dfn}

\begin{prop}
    Sean $R\subset S$ dominios enteros, $v\in S$. Son equivalentes las siguientes afirmaciones
    \begin{enumerate}
        \item $v$ es integral sobre $R$.
        \item $R[v]$ es un $R$-modulo finitamente generado.
        \item Existe un subanillo $R'\subset S$ con $R[v]\subset R'$ y $R'$ un $R$-modulo 
        finitamente generado sobre $R$.
    \end{enumerate}
\end{prop}

\begin{dem}\hspace{4mm}
    \begin{itemize}
        \item (a) $\Rightarrow$(b) $|$ Existe un polinomio monico $f\in R[x]$ tal que $f(v)=0$, 
        luego el $R[v]$ se puede generar por finitos elementos.

        \item (b) $\Rightarrow$(c) $|$ Basta tomar $R'=R[v]$.
        
        \item (c) $\Rightarrow$(a) $|$ Existe $R'$ tal que $R\subset R[v]\subset R'\subset S$. Con
        $R$, $R[v]$, $R'$ finitamente generados como $R$-modulos. Sean $w_{1},\cdots,w_{n}$
        generadores de $R'$. Sabemos que
        \begin{equation*}
            v\cdot w_{i}=a_{i1}w_{1}+\cdots+a_{in}w_{n}
        \end{equation*}
        luego tenemos el sistema
        \begin{align*}
            & (a_{11}-v)w_{1}+a_{12}w_{2}+\cdots+a_{1n}w_{n}=0 \\
            & a_{21}w_{1}+(a_{22}-v)w_{2}+\cdots+a_{2n}w_{n}=0 \\
            & \vdots \\
            & a_{n1}w_{1}+a_{n2}w_{2}+\cdots+(a_{nn}-v)w_{n}=0
        \end{align*}
        Como $R\subset S$ son dominios, podemos verlo dentro del cuerpo de fracciones, entonces
        tiene sentido calcular el determinante de la matriz asociada al sistema de ecuaciones. Por
        otro lado, $(w_{1},\cdots,w_{n})$ es una solución no trivial del sistema y por lo tanto el
        determinante de la matriz asociada es $0$, lo que implica que $v$ es integral sobre $R$.
    \end{itemize}
\end{dem}

\begin{cor}
    Sean $R\subseteq S$ dominios. Entonces los elementos integrales sobre $R$ forman un anillo.
\end{cor}
\begin{dem}
    Sean $a,b\in S$ elementos integrales sobre $R$. Notemos que
    \begin{equation*}
        R\subseteq R[a+b]\subseteq R[a,b]\hspace{4mm}\text{y}\hspace{4mm}
        R\subseteq R[ab]\subseteq R[a,b]
    \end{equation*}
    Como $a$ y $b$ son elementos integrales sobre $R$, $R[a]$ y $R[b]$ son finitamente generados
    por $\{1,a,a^{2},\cdots,a^{n-1}\}$ y $\{1,b,b^{2},\cdots,b^{n-1}\}$. Es claro que
    $R[a,b]$ es generado por $\{a^{i}b^{j}:0\leq i\leq n-1,\hspace{2mm}0\leq j\leq m-1\}$. Por la
    proposición se sigue que $a+b$ y $ab$ son elementos integrales sobre $R$.
\end{dem}

\begin{dfn}
    Sean $R\subseteq S$ dominios. Decimos que $S$ es integral sobre $R$ si todo $s\in S$ es 
    integral sobre $R$.
    \vspace{4mm}

    \noindent Además, $R$ es un dominio integralmente cerrado si para ningún 
    $z\in Frac(R)\setminus R$ es integral.
\end{dfn}

\noindent\textbf{Ejemplos:}
\begin{itemize}
    \item Consideremos $\Z\subseteq\Q$, sea $p/q\in\Q$ con $p$ y $q$ coprimos. Si tenemos la 
    expresión
    \begin{equation*}
        \left(\frac{p}{q}\right)^{n}+a_{n-1}\left(\frac{p}{q}\right)^{n-1}+
        \cdots+a_{1}\left(\frac{p}{q}\right)+a_{0}=0
    \end{equation*}
    Por teorema de la raiz racional, $q$ debe dividir a $1$, luego $q=1$ lo que implica que 
    $p/q\in\Z$. Concluimos que $Z$ es integralmente cerrado.

    \item Veamos el conjunto algebraico $V(y^{2}-x^{3})\subseteq\A_{k}^{2}$ con $k=\overline{k}$.
    Vemos el anillo
    \begin{equation*}
        R=\frac{k[x,y]}{(y^{2}-x^{3})}
    \end{equation*}
    que es un dominio, pues $(y^{2}-x^{3})$ es irreductible. Dentro de $R\subseteq Frac(R)$, vemos
    que se cumple la relación $y^{2}=x^{3}$, que dentro del cuerpo de fracciones es equivalente a
    \begin{equation*}
        \left(\frac{y}{x}\right)^{2}-x=0
    \end{equation*}
    notemos que $\frac{y}{x}\not\in R$ y que $x\in R$. Por lo tanto $R$ no es integralmente 
    cerrado.

    \item Sea $V(y-x^{2})\subseteq\A_{k}^{2}$. Vemos el anillo
    \begin{equation*}
        R=\frac{k[x,y]}{(y-x^{2})}
    \end{equation*}
    por demostrar, $R$ es integralmente cerrado. Consideremos la función 
    $\varphi:\A_{k}^{1}\to\A_{k}^{2}$ dada por $\varphi(t)=(t,t^{2})$. Notemos que 
    $Im(\varphi)=V(y-x^{2})$. La función $\varphi$ induce el isomorfismo
    \begin{align*}
        \frac{k[x,y]}{(y-x^{2})} &\to k[t] \\
        x &\to t \\
        y^{2} &\to t^{2}
    \end{align*}
    Como $k[t]$ es DFU, se sigue que $R$ es integralmente cerrado.
\end{itemize}

%\printbibliography % Quitar el comentado si quiero usar bibliografia

\end{document}
