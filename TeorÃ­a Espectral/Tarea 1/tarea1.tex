\documentclass{article}
\usepackage{hyperref}
\usepackage{Style}

\nocite{*} % Comentar si quiero citar
%\addbibresource{bibliografia.bib} % Quitar el comentado si quiero usar bibliografia

\begin{document}

\begin{minipage}{2.5cm}
    \includegraphics[width=2cm]{imagen_puc.jpg}
\end{minipage}
\begin{minipage}{14cm}
    {\sc Pontificia Universidad Católica de Chile\\
    Facultad de Matemáticas\\
    Departamento de Matemática\\
    Profesora: Amal Taarabt -- Estudiante: Benjamín Mateluna}
\end{minipage}
\vspace{1ex}

{\centerline{\bf Teoría Espectral - MAT2820}
\centerline{\bf Tarea 1}}
\centerline{\bf 15 de septiembre de 2025}

\section{Transformada de Fourier - Versión continua}
\subsection{Propiedades básicas}
\begin{enumerate}
    \item 
    \item 
    \item 
    \item 
    \item 
    \item 
\end{enumerate}

\subsection{Fórmula de inversión}
\begin{enumerate}
    \item Sobre la transformada de Fourier
    \begin{enumerate}
        \item 
        \item 
    \end{enumerate}
    \item\textbf{Aplicación a la resolución de EDP.}
    \begin{enumerate}
        \item 
        \item 
    \end{enumerate}
\end{enumerate}

\subsection{La transformada de Fourier en \texorpdfstring{$L^{2}(\mathbb{R})$}{}}



\subsection{La transformada de Fourier en \texorpdfstring{$\mathcal{S}(\R)$}{}}
\begin{enumerate}
    \item 
    \item 
    \item 
    \item 
    \item 
    \item 
    \item 
\end{enumerate}

\section{Transformada de Fourier - Versión discreta}
\begin{enumerate}
    \item 
    \item 
    \item 
\end{enumerate}

%\printbibliography % Quitar el comentado si quiero usar bibliografia

\end{document}
