\documentclass{article}
\usepackage{hyperref}
\usepackage{Style}
\setcounter{MaxMatrixCols}{20}

\nocite{*} % Comentar si quiero citar
%\addbibresource{bibliografia.bib} % Quitar el comentado si quiero usar bibliografia

\begin{document}

\begin{minipage}{2.5cm}
    \includegraphics[width=2cm]{imagen_puc.jpg}
\end{minipage}
\begin{minipage}{14cm}
    {\sc Pontificia Universidad Católica de Chile\\
    Facultad de Matemáticas\\
    Departamento de Matemática\\
    Profesor: Mauricio Bustamante -- Estudiante: Benjamín Mateluna}
\end{minipage}
\vspace{1ex}

{\centerline{\bf Topología Algebraica - MAT2850}
\centerline{\bf Tarea 2}}
\centerline{\bf 05 de septiembre de 2025}

\section*{Problema 3} 
\noindent Por simplicidad en el argumento, denotaremos los morfismos $A_{i}\to A_{i+1}$ y 
$B_{i}\to B_{i+1}$ como $\partial$, además, como ambas secuencias son exactas, resulta que 
$\partial^{2}a=\partial\circ\partial (a)=0$. Veamos que $\kr{f_{3}}=0$. Sea $a\in\kr{f_{3}}$,
notemos que
\begin{equation*}
    0=\partial f_{3}(a)=f_{4}\partial(a)=0\hhtext{entonces}\partial a=0
\end{equation*}
Como $a\in\kr\partial$, existe $a'\in A_{2}$ tal que $\partial a'=a$, luego $\partial f_{2}(a')
=f_{3}\partial(a')=f_{3}(a)=0$. Por exactitud, existe $b'\in B_{1}$ tal que $\partial b'
=f_{2}(a')$, como $f_{1}$ es isomorfismo, existe $a''\in A_{1}$ tal que $b'=f_{1}(a'')$, como los
diagramas conmutan vemos que
\begin{equation*}
    a''=f^{-1}_{1}(b')\hhtext{entonces}\partial a''=\partial f^{-1}_{1}(b')=f^{-1}_{2}\partial(b')
\end{equation*}
recordemos que $\partial b'=f_{2}(a')$, es decir, $\partial a''=a'$, luego 
$0=\partial^{2}a''=\partial a'=a$.

\vspace{2mm}
\noindent Sea $b\in B_{3}$, consideramos $\partial b\in B_{4}$, entonces 
$f^{-1}_{4}(\partial b)\in A_{4}$, por conmutatividad de los diagramas, vemos que 
$\partial f_{4}^{-1}(\partial b)=f_{5}^{-1}(\partial^{2}b)=0$, luego, por exactitud, existe
$a\in A_{3}$ tal que $\partial a=f_{4}^{-1}(\partial b)$,
\begin{equation*}
    \partial(f_{3}(a)-b)=\partial f_{3}(a)-\partial b=f_{4}\partial(a)-\partial b=0
\end{equation*}
así, existe $b'\in B_{2}$ tal que $\partial b'=f_{3}(a)-b$, definimos 
$a'=f_{2}^{-1}(b')\in A_{2}$, de este modo,
\begin{equation*}
    f_{3}(a)-b=\partial b'=\partial f_{2}(a')=f_{3}(\partial a')
\end{equation*}
en resumen, $f_{3}(a-\partial a')=b$. Concluimos que $f_{3}$ es isomorfismo.

\section*{Problema 4}
\noindent Asignamos el orden a los vértices tal que $i<i+1$.
\begin{enumerate}
    \item\textbf{Definición:} Notemos que el complejo simplicial es conexo, luego, el 
    argumento presentado en el problema 5, es invariante del anillo utilizado, entonces 
    $H_{0}(K)\cong R$ donde $R=\Z,\Z_{2},\Q$. Adicionalmente, $C_{i}=0$ para $i>2$, ya que no hay
    $i-$simplices. Basta calcular $H_{i}(K)$ para $i=1,2$. Tenemos el complejo de cadenas

    \vspace{2mm}
    \centerline{
        \xymatrix{
            0 \ar[r] & C_{2} \ar[r]^{\partial_{2}} & C_{1} \ar[r]^{\partial_{1}} & C_{0} \ar[r] 
            & 0
        }
    }
    Para encontrar los grupos de homología basta calcular $\kr{\partial_{2}},\im{\partial_{2}}$ y
    $\ker{\partial_{1}}$. A cada vértice en $C_{0}(K)$ le asignamos el vector canónico de la 
    siguiente manera $i=e_{i+1}$, a cada $1-$simplice le asignamos un vector como sigue,
    \begin{equation*}
        \begin{array}{llll}
            \gen{0,1}=e_{1} & \gen{1,2}=e_{6} & \gen{2,3}=e_{10} & \gen{3,4}=e_{13} \\
            \gen{0,2}=e_{2} & \gen{1,3}=e_{7} & \gen{2,4}=e_{11} & \gen{3,5}=e_{14} \\
            \gen{0,3}=e_{3} & \gen{1,4}=e_{8} & \gen{2,5}=e_{12} & \gen{4,5}=e_{15} \\
            \gen{0,4}=e_{4} & \gen{1,5}=e_{9} \\
            \gen{0,5}=e_{5}
        \end{array}
    \end{equation*}
    Luego, la acción de $\partial_{1}$ esta representado por la matriz
    
    \vspace{1mm}
    \begin{equation*}
        \begin{pmatrix}
            -1 & -1 & -1 & -1 & -1 & 0 & 0 & 0 & 0 & 0 & 0 & 0 & 0 & 0 & 0 \\
            1 & 0 & 0 & 0 & 0 & -1 & -1 & -1 & -1 & 0 & 0 & 0 & 0 & 0 & 0 \\
            0 & 1 & 0 & 0 & 0 & 1 & 0 & 0 & 0 & -1 & -1 & -1 & 0 & 0 & 0 \\
            0 & 0 & 1 & 0 & 0 & 0 & 1 & 0 & 0 & 1 & 0 & 0 & -1 & -1 & 0 \\
            0 & 0 & 0 & 1 & 0 & 0 & 0 & 1 & 0 & 0 & 1 & 0 & 1 & 0 & -1 \\
            0 & 0 & 0 & 0 & 1 & 0 & 0 & 0 & 1 & 0 & 0 & 1 & 0 & 1 & 1
        \end{pmatrix}
    \end{equation*}

    \vspace{1mm}
    Sean $c_{i}$ las columnas de la matriz, notemos que $c_{i}-c_{1}=c_{i+4}$ para $1<i<6$, 
    $c_{i}-c_{6}=c_{i+3}$ para $6<i<10$, $c_{i}-c_{10}=c_{i+2}$ para $i=11,12$ y 
    $c_{14}-c_{13}=c_{15}$. Lo anterior nos dice que las primeras cinco columnas generan la imagen
    de $\partial_{1}$, luego, el kernel tiene dimensión 10. Además, las relaciones nos dan los 
    vectores que generan y tienen la forma $1,-1,-1$, no necesariamente juntos y el resto son 
    ceros, se sigue que $\kr{\partial_{1}}\cong R^{10}$, donde $R=\Z,\Z_{2},\Q$.

    \vspace{1mm}
    Queda estudiar la acción de $\partial_{2}$.
\end{enumerate}

\section*{Problema 5}
\noindent Dado $K$ un complejo simplicial finito, sean $v,w\in V_{K}$, decimos que $v\sim_{p}w$ si
y solo si $v$ esta conectado a $w$ ó $v=w$, es decir, si existe una sucesión de $1-$simplices
$\gen{w_{0},w_{1}},\cdots,\gen{w_{k-1},w_{k}}$ tales que $w_{0}=v$ y $w_{k}=w$. 

\vspace{1mm}
\noindent Por definición resulta que $x\sim_{p}x$, además, notemos que si $v\sim_{p}w$ entonces 
$w\sim_{p}v$ basta tomar $\omega_{i}:=w_{k-i}$. Por otro lado, si $v\sim_{p}w$ y $w\sim_{p}u$, 
entonces la sucesión
\begin{equation*}
    \gen{w_{0},w_{1}},\cdots,\gen{w_{k-1},w_{k}},\gen{\omega_{0},\omega_{1}},\cdots,
    \gen{\omega_{j-1},\omega_{j}}
\end{equation*}
donde $w_{0}=v$, $w_{k}=w$, $\omega_{0}=w$ y $\omega_{j}=u$ es una sucesión que conecta $v$ con 
$u$, en otras palabras, $v\sim_{p}u$.

\vspace{2mm}
\begin{dfn}[Componente Conexa]
    Sea $K$ un complejo simplicial finito y $v\in V_{K}$, definimos su componente conexa
    como
    \begin{equation*}
        [v]_{c}:=\{\sigma\in K:\sigma=\gen{v_{0},v_{1},\cdots,v_{r}}\hhtext{y}v\sim_{p}v_{i}\}
    \end{equation*}
\end{dfn}

\noindent\textbf{Observación:} Si $[v]_{c}=K$, entonces el complejo simplicial es conexo. Sea 
$w\in V_{K}$ tal que $w\in[v]_{c}$, entonces $v\sim_{p}w$. Luego, dado $\sigma
=\gen{v_{0},\cdots,v_{r}}\in[w]_{c}$ se tiene que $v\sim_{p}v_{i}$, se sigue que 
$[w]_{c}\subseteq[v]_{c}$, de manera similar obtenemos que $[v]_{c}\subseteq[w]_{c}$. Por lo tanto
$[w]_{c}=[v]_{c}$.

\vspace{1mm}
\noindent Veamos que dado $v\in V_{K}$, se tiene que $[v]_{c}$ es un subcomplejo simplicial de 
$K$. En efecto, sea $\sigma\in[v]_{c}$ y $\tau\leq\sigma$, si $v_{i}$ es un vértice de $\tau$ 
entonces es vértice de $\sigma$, luego $v\sim_{p}v_{i}$, lo que implica que $\tau\in[v]_{c}$. La
segunda propiedad se cumple trivialmente. Por lo tanto, una componente conexa es un subcomplejo 
simplicial conexo de $K$ y la unión de dos componentes conexas también es subcomplejo simplicial, 
puesto que un simplice esta en una componente conexa o en la otra, pero no en ambas.

\vspace{1mm}
\noindent Como $\sim_{p}$ es una relación de equivalencia, particiona el conjunto de vértices,
junto con lo anterior hemos probado que las componentes conexas particionan al complejo 
simplicial.

\vspace{2mm}
\noindent Debemos probar lo siguiente:
\begin{enumerate}
    \item \textbf{Si $K$ es conexo, entonces $\abs{K}$ arcoconexo}. Sean $x,y\in\abs{K}$, 
    existen $\sigma_{x},\sigma_{y}\in K$ tales que $x\in\sigma_{x}$ e $y\in\sigma_{y}$, sean 
    $v_{x}\in\sigma_{x}$ y $v_{y}\in\sigma_{y}$ vértices de $K$. Existe una sucesión
    \begin{equation*}
        \gen{w_{0},w_{1}},\cdots,\gen{w_{k-1},w_{k}}
    \end{equation*}
    tal que $w_{0}=v_{x}$ y $w_{k}=v_{y}$, consideramos la función $f_{i}:[0,1]\to\abs{K}$ dada 
    por $f_{i}(t):=(1-t)w_{i}+tw_{i+1}$ que esta bien definida por que $\gen{w_{i},w_{i+1}}$ es 
    convexo y es continua. Del mismo modo, como $\sigma_{x}$ es convexo, la función 
    $f_{x}:[0,1]\to\abs{K}$ dada por $f_{x}(t):=(1-t)x+tv_{x}$ esta bien definida. De manera
    análoga definimos $f_{y}$, pero $f_{y}(0)=v_{y}$ y $f_{y}(1)=y$. Luego,
    \begin{equation*}
        f:=f\sbullet f_{0}\sbullet f_{1}\sbullet\cdots\sbullet f_{k-1}\sbullet f_{y}
    \end{equation*}
    donde $\sbullet$ es la operación de concatenación. Es una función continua, por lema del 
    pegamiento, y tal que $f(0)=x$ y $f(1)=y$.
    
    \vspace{1mm}
    Concluimos que $\abs{K}$ es arcoconexo.

    \item \textbf{Probar que $H_{0}(K)\cong\Z^{\text{\#componentes conexas}}$}. Como $K$ es 
    finito, hay finitas componentes conexas, procederemos por inducción en el número de 
    componentes conexas.

    \vspace{1mm}
    Supongamos que $K$ tiene una componente conexa, entonces $K$ es conexo. Dado $v\in V_{K}$, 
    basta probar que $[v]=[w]$ en $H_{0}(K)$ para todo $w\in V_{K}$. Como $K$ es conexo, existe
    una sucesión
    \begin{equation*}
        \gen{w_{0},w_{1}},\cdots,\gen{w_{k-1},w_{k}}
    \end{equation*}
    tal que $w_{0}=v$ y $w_{k}=w$, entonces
    \begin{equation*}
        \partial\left(\sum_{i=0}^{k-1}\gen{w_{i},w_{i+1}}\right)
        =\sum_{i=0}^{k-1}\partial\gen{w_{i},w_{i+1}}=\sum_{i=0}^{k-1}w_{i+1}-w_{i}=w_{k}-w_{0}=w-v
    \end{equation*}
    luego $w-v\in\im{\partial}$. Entonces $H_{0}(K)\cong\Z$.

    \vspace{1mm}
    Sea $n$ el número de componentes conexas. Sean $M_{i}$ las componentes conexas de $K$, 
    consideramos los subcomplejos simpliciales
    \begin{equation*}
        M=M_{n} \hhtext{y} N=\bigcup_{i=1}^{n-1}M_{i}
    \end{equation*}
    Por Mayer-Vietoris, se tiene la secuencia exacta
    
    \vspace{2mm}
    \centerline{
        \xymatrix{
            0 \ar[r] & H_{0}(M)\oplus H_{0}(N) \ar[r] & H_{0}(K) \ar[r] & 0
        }
    }
    entonces
    \begin{equation*}
        H_{0}(K)\cong H_{0}(M)\oplus H_{0}(N)\cong\Z\oplus\Z^{n-1}\cong\Z^{n}
    \end{equation*}
\end{enumerate}

%\printbibliography % Quitar el comentado si quiero usar bibliografia

\end{document}
