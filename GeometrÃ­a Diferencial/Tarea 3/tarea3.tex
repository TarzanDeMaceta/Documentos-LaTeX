\documentclass{article}
\usepackage{hyperref}
\usepackage{Style}

\nocite{*} % Comentar si quiero citar
%\addbibresource{bibliografia.bib} % Quitar el comentado si quiero usar bibliografia

\begin{document}

\begin{minipage}{2.5cm}
    \includegraphics[width=2cm]{imagen_puc.jpg}
\end{minipage}
\begin{minipage}{14cm}
    {\sc Pontificia Universidad Católica de Chile\\
    Facultad de Matemáticas\\
    Departamento de Matemática\\
    Profesor: Pedro Gaspar -- Estudiante: Benjamín Mateluna}
\end{minipage}
\vspace{1ex}

{\centerline{\bf Geomtría Diferencial - MAT2860}
\centerline{\bf Tarea I3}}
\centerline{\bf 26 de junio de 2025}

\section*{Problema 1}
\noindent Probaremos un resultado previo sobre transporte paralelo y curvas geodésicas en la 
esfera.
\begin{lema}
    Sea $\alpha:I\to\mathbb{S}^{2}$ una curva de la forma $\alpha(t)=usen(t)+vcos(t)$ donde $u,v$ 
    son ortonormales, entonces $P_{\alpha,t_{0},t_{1}}(w)=w+\ip{w}{\alpha'(t_{0})}(\alpha'(t_{1})
    -\alpha'(t_{0}))$.
\end{lema}
\begin{dem}
    Empezamos observando que $\alpha$ es una curva geodésica. Notemos que 
    $\ip{\alpha'(t)}{\alpha(t)}=0$ para todo $t\in I$ y además $\alpha''=-\alpha$. Sea 
    $w\in T_{p}\mathbb{S}^{2}$ donde $p=\alpha(t_{0})$ y sea $Y:I\to\mathbb{S}^{2}$ el único campo 
    paralelo a lo largo de $\alpha$ tal que $Y(t_{0})=w$, como $Y$ en particular es campo tangente 
    se tiene que
    \begin{equation*}
        Y(t)\in T_{\alpha(t)}\mathbb{S}^{2}=(\alpha(t))^{\perp}
    \end{equation*}
    es decir, $\ip{Y(t)}{\alpha(t)}=0$ para todo $t\in I$, lo que implica que $\ip{Y}{\alpha'}
    =-\ip{Y'}{\alpha}$. Notemos que
    \begin{equation*}
        0=\nabla_{\alpha'}Y=Y'-\ip{Y'}{\alpha}\alpha
    \end{equation*}
    tomando producto interno con $\alpha'$ vemos que
    \begin{equation*}
        \ip{Y'}{\alpha'}=\ip{Y'}{\alpha}\ip{\alpha}{\alpha'}=0
    \end{equation*}
    de este modo
    \begin{equation*}
        \dv{}{t}(\ip{Y}{\alpha'})=\ip{Y'}{\alpha'}+\ip{Y}{\alpha''}=-\ip{Y}{\alpha}=0
    \end{equation*}
    luego la función $\ip{Y}{\alpha'}$ es constante, evaluando en $t_{0}$ vemos que 
    $\ip{Y'}{\alpha}=-\ip{Y}{\alpha'}\equiv-\ip{w}{\alpha'(t_{0})}$. Tenemos que
    \begin{equation*}
        Y'(t)=\ip{w}{\alpha'(t_{0})}\alpha''(t)
    \end{equation*}
    integrando a ambos lados vemos que
    \begin{equation*}
        Y(t)-Y(t_{0})=Y(t)-w=\ip{w}{\alpha'(t_{0})}(\alpha'(t)-\alpha'(t_{0}))
    \end{equation*}
    lo que concluye la demostración.
\end{dem}

\noindent Sea $\{e_{i}\}_{i=1}^{3}$ la base canonica de $\R^{3}$. Sea $w_{0}\in 
T_{p}\mathbb{S}^{2}=(0,0,1)^{\perp}$, digamos que $w_{0}=ae_{1}+be_{2}$.
\begin{enumerate}
    \item Notemos que
    \begin{equation*}
        \gamma_{1}=sen(s)e_{1}+cos(s)e_{3}\hspace{4mm}\text{y}\hspace{4mm}
        \gamma_{2}=sen(s)w_{1}+cos(s)e_{3}
    \end{equation*}
    donde $w_{1}=(cos(\theta),sen(\theta),0)$. Claramente $e_{3}$ y $w_{1}$ son ortonormales, 
    además
    \begin{equation*}
        \gamma_{1}'=(cos(s),0,-sen(s))\hspace{4mm}\text{y}\hspace{4mm}
        \gamma_{2}'=(cos(\theta)cos(s),sen(\theta)cos(s),-sen(s))
    \end{equation*}
    Como $P_{\gamma,t_{0},t_{1}}:T_{\gamma(t_{0})}\Sigma\to T_{\gamma(t_{1})}\Sigma$ es una 
    transformación lineal y $\{e_{1},e_{2}\}$ es base de $T_{p}\mathbb{S}^{2}$, se sigue que
    \begin{align*}
        P_{\gamma_{1},0,\pi}(w_{o}) &= P_{\gamma_{1},0,\pi}(ae_{1}+be_{2})
        =aP_{\gamma_{1},0,\pi}(e_{1})+bP_{\gamma_{1},0,\pi}(e_{2})
        =aP_{\gamma_{1},0,\pi}(\gamma_{1}'(0))+bP_{\gamma_{1},0,\pi}(e_{2}) \\[2mm]
        &= a\gamma_{1}'(\pi)+bP_{\gamma_{1},0,\pi}(e_{2})=-ae_{1}+bP_{\gamma_{1},0,\pi}(e_{2})
    \end{align*}
    y usando el lema, notamos que
    \begin{equation*}
        P_{\gamma_{1},0,\pi}(e_{2})=e_{2}+\ip{e_{2}}{\gamma_{1}'(0)}
        (\gamma_{1}'(\pi)-\gamma_{1}'(0))=e_{2}+0(\gamma_{1}'(\pi)-\gamma_{1}'(0))=e_{2}
    \end{equation*}
    entonces
    \begin{equation*}
        w_{1}=P_{\gamma_{1},0,\pi}(w_{o})=-ae_{1}+be_{2}
    \end{equation*}
    Por otro lado, del mismo modo que antes, se tiene que
    \begin{equation*}
        P_{\gamma_{2},0,\pi}(w_{o})=aP_{\gamma_{2},0,\pi}(e_{1})
        +bP_{\gamma_{2},0,\pi}(e_{2})
    \end{equation*}
    por el lema, observamos lo siguiente
    \begin{equation*}
        P_{\gamma_{2},0,\pi}(e_{1})=e_{1}+\ip{e_{1}}{\gamma_{2}'(0)}
        (\gamma_{2}'(\pi)-\gamma_{2}'(0))=\begin{pmatrix}
            1-2cos^{2}(\theta) \\ -2cos(\theta)sen(\theta) \\ 0
        \end{pmatrix}=-\begin{pmatrix}
            cos(2\theta) \\ sen(2\theta) \\ 0
        \end{pmatrix}
    \end{equation*}
    y
    \begin{equation*}
        P_{\gamma_{2},0,\pi}(e_{2})=e_{2}+\ip{e_{2}}{\gamma_{2}'(0)}
        (\gamma_{2}'(\pi)-\gamma_{2}'(0))=\begin{pmatrix}
            -2cos(\theta)sen(\theta) \\ 1-2sen^{2}(\theta) \\ 0
        \end{pmatrix}=\begin{pmatrix}
            -sen(2\theta) \\ cos(2\theta) \\ 0
        \end{pmatrix}
    \end{equation*}
    entonces
    \begin{equation*}
        w_{2}=P_{\gamma_{2},0,\pi}(w_{o})=-a\begin{pmatrix}
            cos(2\theta) \\ sen(2\theta) \\ 0
        \end{pmatrix}+b\begin{pmatrix}
            -sen(2\theta) \\ cos(2\theta) \\ 0
        \end{pmatrix}=\begin{pmatrix}
            cos(2\theta) & -sen(2\theta) & 0 \\ sen(2\theta) & cos(2\theta) & 0 \\ 0 & 0 & 1
        \end{pmatrix}w_{1}
    \end{equation*}
    luego el ángulo entre $w_{1}$ y $w_{2}$ es de $2\theta$.

    \item Notemos que
    \begin{equation*}
        \widehat{\gamma_{2}}=(cos(\theta)sen(\pi-s),sen(\theta)sen(\pi-s),cos(\pi-s))
        =(cos(\theta)sen(s),sen(\theta)sen(s),-cos(s))
    \end{equation*}
    y entonces $\widehat{\gamma_{2}}'=(cos(\theta)cos(s),sen(\theta)cos(s),sen(s))$, además
    $\beta'=(-sen(s),cos(s),0)$.
    
    Debemos calcular
    \begin{equation*}
        v_{0}=P_{\widehat{\gamma_{2}},\pi/2,\pi}(P_{\beta,0,\theta}
        (P_{\gamma_{1},0,\pi/2}(w_{0})))
    \end{equation*}
    Por el mismo argumento que antes, notamos que
    \begin{equation*}
        v_{0}^{1}:=P_{\gamma_{1},0,\pi/2}(w_{0})
        =aP_{\gamma_{1},0,\pi/2}(e_{1})+bP_{\gamma_{1},0,\pi/2}(e_{2})=a\gamma_{1}'(\pi/2)+be_{2}
        =-ae_{3}+be_{2}
    \end{equation*}
    luego
    \begin{equation*}
        v_{0}^{2}:=P_{\beta,0,\theta}(v_{0}^{1})=-aP_{\beta,0,\theta}(e_{3})
        +bP_{\beta,0,\theta}(e_{2})
    \end{equation*}
    usando el lema tenemos que $P_{\beta,0,\theta}(e_{3})=e_{3}$ y además
    \begin{equation*}
        P_{\beta,0,\theta}(e_{2})=e_{2}+\ip{e_{2}}{\beta'(0)}(\beta'(\theta)-\beta'(0))
        =e_{2}+\beta'(\theta)-e_{2}=\beta'(\theta)
    \end{equation*}
    entonces $v_{0}^{2}=-ae_{3}+b\beta'(\theta)$. Queda ver
    \begin{equation*}
        v_{0}=P_{\widehat{\gamma_{2}},\pi/2,\pi}(v_{0}^{2})
        =-aP_{\widehat{\gamma_{2}},\pi/2,\pi}(e_{3})
        +bP_{\widehat{\gamma_{2}},\pi/2,\pi}(\beta'(\theta))
    \end{equation*}
    usando nuevamente el lema, tenemos que
    \begin{equation*}
        P_{\widehat{\gamma_{2}},\pi/2,\pi}(e_{3})=e_{3}
        +\ip{e_{3}}{\widehat{\gamma_{2}}'(\pi/2)}
        (\widehat{\gamma_{2}}'(0)-\widehat{\gamma_{2}}'(\pi/2))
        =e_{3}+\widehat{\gamma_{2}}'(0)-e_{3}=\widehat{\gamma_{2}}'(0)
    \end{equation*}
    y
    \begin{equation*}
        P_{\widehat{\gamma_{2}},\pi/2,\pi}(\beta'(\theta))
        =\beta'(\theta)+\ip{\beta'(\theta)}{\widehat{\gamma_{2}}'(\pi/2)}
        (\widehat{\gamma_{2}}'(0)-\widehat{\gamma_{2}}'(\pi/2))=\beta'(\theta)
    \end{equation*}
    de este modo
    \begin{equation*}
        v_{0}=-a\widehat{\gamma_{2}}'(0)+b\beta'(\theta)=-a\begin{pmatrix}
            -cos(\theta) \\ -sen(\theta) \\ 0
        \end{pmatrix}+b\begin{pmatrix}
            -sen(\theta) \\ cos(\theta) \\ 0
        \end{pmatrix}=\begin{pmatrix}
            cos(\theta) & -sen(\theta) & 0 \\ sen(\theta) & cos(\theta) & 0 \\ 0 & 0 & 1
        \end{pmatrix}w_{0}
    \end{equation*}
    y por lo tanto el ángulo entre $w_{0}$ y $v_{0}$ es $\theta$.
\end{enumerate}

\section*{Problema 2}
\noindent El toro, puede ser parametrizado por $X:(0,2\pi)\times(0,2\pi)\to\R^{3}$ dada por
\begin{equation*}
    X(u,v)=((R+rcos(u))cos(v),(R+rcos(u))sen(v),rsen(u))
\end{equation*}
con derivadas parciales
\begin{align*}
    X_{u}(u,v) &= (-rsen(u)cos(v),-rsen(u)sen(v),rcos(u)) \\
    X_{v}(u,v) &= (-(R+rcos(u))sen(v),(R+rcos(u))cos(v),0)
\end{align*}
entonces el campo normal definido por la parametrización es
\begin{equation*}
    (X_{u}\times X_{v})(u,v)=(-(R+rcos(u))rcos(u)cos(v),-(R+rcos(u))rcos(u)sen(v),
    -(R+rcos(u))rsen(v))
\end{equation*}
y además $\abs{X_{u}\times X_{v}}=r(R+rcos(u))$. Sea $u_{0}\in[0,2\pi]$ tal que $rsen(u_{0})=c$, 
que existe pues $c\in[-r,r]$. Consideramos $\eta:=(R+rcos(u_{0}))$, definimos la curva p.p.a 
$\alpha:(0,2\pi\eta)\to\R^{3}$ dada por
\begin{equation*}
    \alpha(t):=X\left(u_{0},\frac{t}{\eta}\right)=\left(\eta cos\left(\frac{t}{\eta}\right),
    \eta sen\left(\frac{t}{\eta}\right),rsen(u_{0})\right)
\end{equation*}
entonces, suponiendo que la parametrización es positiva
\begin{equation*}
    (N\circ\alpha)(t)=(N^{x}\circ X^{-1}\circ\alpha)(t)=N^{x}\left(u_{0},\frac{t}{\eta}\right)
    =\left(cos(u_{0})cos\left(\frac{t}{\eta}\right),cos(u_{0})sen\left(\frac{t}{\eta}\right),
    sen(u_{0})\right)
\end{equation*}
Por otro lado vemos que
\begin{equation*}
    \alpha'=\left(-sen\left(\frac{t}{\eta}\right),cos\left(\frac{t}{\eta}\right),0\right)
    \hspace{4mm}\text{y entonces}\hspace{4mm}
    \alpha''(t)=-\frac{1}{\eta}\left(cos\left(\frac{t}{\eta}\right),
    sen\left(\frac{t}{\eta}\right),0\right)
\end{equation*}
Nuestro objetivo ahora es determinar $\nabla_{\alpha'}\alpha'(t)$, para ello observemos que
\begin{equation*}
    \ip{\alpha''}{N\circ\alpha}=\frac{1}{\eta}\begin{pmatrix}
        cos(t/\eta) \\ sen(t/\eta) \\ 0
    \end{pmatrix}\cdot\begin{pmatrix}
        cos(u_{0})cos(t/\eta) \\ cos(u_{0})sen(t/\eta) \\ sen(u_{0})
    \end{pmatrix}=\frac{cos(u_{0})}{\eta}
\end{equation*}
de este modo
\begin{align*}
    \nabla_{\alpha'}\alpha' &= \alpha''-\ip{\alpha''}{N\circ\alpha}N\circ\alpha
    =-\frac{1}{\eta}\begin{pmatrix}
        cos(t/\eta) \\ sen(t/\eta) \\ 0
    \end{pmatrix}+\frac{cos(u_{0})}{\eta}\begin{pmatrix}
        cos(u_{0})cos(t/\eta) \\ cos(u_{0})sen(t/\eta) \\ sen(u_{0})
    \end{pmatrix} \\[2mm]
    &= \frac{1}{\eta}\begin{pmatrix}
        cos(t/\eta)(cos^{2}(u_{0})-1) \\ sen(t/\eta)(cos^{2}(u_{0})-1) \\ cos(u_{0})sen(u_{0})
    \end{pmatrix}
\end{align*}
adicionalmente tenemos que
\begin{equation*}
    N\circ\alpha\times\alpha'=\begin{pmatrix}
        cos(t/\eta)sen(u_{0}) \\ sen(t/\eta)sen(u_{0}) \\ -cos(u_{0})
    \end{pmatrix}
\end{equation*}
por último, se tiene que
\begin{align*}
    K_{g} &= [\nabla_{\alpha'}\alpha']=\frac{1}{\eta}\begin{pmatrix}
        cos(t/\eta)(cos^{2}(u_{0})-1) \\ sen(t/\eta)(cos^{2}(u_{0})-1) \\ cos(u_{0})sen(u_{0})
    \end{pmatrix}\cdot\begin{pmatrix}
        cos(t/\eta)sen(u_{0}) \\ sen(t/\eta)sen(u_{0}) \\ -cos(u_{0})
    \end{pmatrix} \\[2mm]
    &= \frac{1}{\eta}\left((cos^{2}(u_{0})-1)sen(u_{0})-cos^{2}(u_{0})sen(u_{0})\right)
    =\frac{-sen(u_{0})}{R+rcos(u_{0})}
\end{align*}

\section*{Problema 3}
\noindent Proponemos la siguiente definición para un poligono geodésico.
\begin{dfn}
    Decimos que $P\subseteq\Sigma$ es un $n$-poligono geodésico si $P$ es homeomorfo a 
    $D=\overline{B_{1}(0)}\subseteq\R^{2}$ y $\partial P$ esta parametrizada por 
    $\gamma:I\to\Sigma$ una curva diferenciable a trozos tal que
    \begin{equation*}
        \partial P=\gamma(I)=\bigcup_{i=1}^{n}\gamma_{i}([a_{i},b_{i}])
    \end{equation*}
    donde $\gamma_{i}:[a_{i},b_{i}]\to\Sigma$ es una curva geodésica tal que $\gamma_{i}(b_{i})
    =\gamma_{i+1}(a_{i+1})$ y $\gamma_{i}'(b_{i}^{-})\neq\pm\gamma_{i+1}'(a_{i+1}^{+})$ para 
    $i=1,\cdots,n$ con la convención de que $i+1=1$ si $i=n$.
\end{dfn}

\noindent Sea $P\subseteq S^{2}(r)$ un $n$-poligono geodésico. Como $D$ no es homeomorfo a la 
$\mathbb{S}^{2}$, pues esta última es una variedad sin borde, se tiene que $P\neq S^{2}(r)$, así,
existe $p\in S^{2}(r)\setminus P$.

\vspace{2mm}
\noindent Consideremos $O\in O(3)$ tal que $O(0,0,r)=p$ y la proyección estereográfica 
$\overline{X}:\R^{2}\to S^{2}(r)\setminus\{(0,0,r)\}$. Definimos 
$X:=O\circ\overline{X}:\R^{2}\to S^{2}(r)$, notemos que $X$ es parametrización ortogonal de 
$S^{2}(r)$, pues $\overline{X}$ es parametrización ortogonal de $S^{2}(r)$ y $O$ es una 
transformación lineal ortogonal, luego, por gauss bonnet, vemos que
\begin{align*}
    2\pi=\sum_{i=1}^{n}\int_{a_{i}}^{b_{i}}K_{g}\hspace{1mm}ds
    +\iint_{P}(K_{S^{2}(r)}\circ X)\sqrt{EG}\hspace{1mm}dudv+\sum_{i=1}^{n}\theta_{i}
\end{align*}
donde $\theta_{i}\in[-\pi,\pi]$ es el ángulo orientado de $\gamma_{i}'(b_{i})\in 
T_{\gamma_{i}(b_{i}^{-})}S^{2}(r)$ a $\gamma_{i+1}(a_{i+1}^{+})\in T_{\gamma_{i+1}}S^{2}(r)$. Como
la curvatura gaussiana es independiente de la parametrización, sabemos que $K_{S^{2}(r)}\circ X
=1/r^{2}$, además, por lo visto en clases, obtenemos que
\begin{equation*}
    2\pi=\iint_{P}\frac{1}{r^{2}}\sqrt{EG}\hspace{1mm}dudv+\sum_{i=1}^{n}\theta_{i}
    =\frac{1}{r^{2}}Area(P)+\sum_{i=1}^{n}\pi-\alpha_{i}
\end{equation*}
donde $\alpha_{i}$ son los ángulos interiores del $n$-poligono geodesico, así
\begin{equation*}
    2\pi+\sum_{i=1}^{n}\alpha_{i}=\frac{1}{r^{2}}Area(P)+n\pi
\end{equation*}
 
\section*{Problema 4}
\begin{enumerate}
    \item Como el producto interno es lineal en la primera entrada entonces $\omega$ es una 
    1-forma. Notemos que
    \begin{align*}
        \omega_{p}(v) &= \ip{v}{\frac{Jp}{\abs{p}^{2}}}=\frac{1}{\abs{p}^{2}}\ip{v}{
            \begin{pmatrix}
                -y \\
                x
            \end{pmatrix}
            }=\frac{1}{\abs{p}^{2}}(-v_{1}y+v_{2}x) \\[2mm]
        &= \frac{1}{\abs{p}^{2}}(-dx(v)y+dy(v)x)=\frac{-y}{x^{2}+y^{2}}dx(v)
        +\frac{x}{x^{2}+y^{2}}dy(v)
    \end{align*}
    conlcuimos que
    \begin{equation*}
        \omega_{p}=\frac{-y}{x^{2}+y^{2}}dx+\frac{x}{x^{2}+y^{2}}dy
    \end{equation*}
    además, para $p\in\R^{2}\setminus\{0\}$ las funciones
    \begin{equation*}
        \omega_{1}:=\frac{-y}{x^{2}+y^{2}}\hspace{4mm}\text{y}
        \hspace{4mm}\omega_{2}:=\frac{x}{x^{2}+y^{2}}
    \end{equation*}
    son diferenciables, y por lo tanto la 1-forma $\omega$ es diferenciable.
    
    \item Veamos la siguiente expresión
    \begin{align*}
        d\omega &= \left(\pdv{\omega_{1}}{x}dx+\pdv{\omega_{1}}{y}dy\right)\land dx
        +\left(\pdv{\omega_{2}}{x}dx+\pdv{\omega_{2}}{y}dy\right)\land dy
        =\left(\pdv{\omega_{2}}{x}dx-\pdv{\omega_{1}}{y}dy\right)dx\land dy \\[2mm]
        &= \left(\frac{y^{2}-x^{2}}{(x^{2}+y^{2})^{2}}
        -\frac{y^{2}-x^{2}}{(x^{2}+y^{2})^{2}}\right)dx\land dy=0
    \end{align*}
    
    \item Para finalizar, tenemos que
    \begin{align*}
        \int_{0}^{1}\omega_{\gamma(t)}(\gamma'(t))\hspace{1mm}dt &= \int_{0}^{1}
        -\frac{\gamma_{2}(t)}{\gamma_{1}^{2}(t)+\gamma_{2}^{2}(t)}\cdot\gamma_{1}'(t)
        +\frac{\gamma_{1}(t)}{\gamma_{1}^{2}(t)+\gamma_{2}^{2}(t)}\cdot\gamma_{2}'(t)
        =\int_{0}^{1}F(\gamma(t))\cdot\gamma'(t)\hspace{1mm}dt \\[2mm]
        &= \int_{\gamma}F(x,y)\hspace{1mm}ds
    \end{align*}
    es decir, la integral corresponde a la integral de línea del campo vectorial 
    $F(x,y)=(\omega_{1},\omega_{2})(x,y)$ sobre la curva $\gamma$.
\end{enumerate}

\section*{Problema 5}
\begin{enumerate}
    \item Veamos que $\omega_{ij}$ es una 1-forma, sea $p\in W$, sean $u,v\in\R^{3}$ y 
    $\alpha\in\R$, entonces
    \begin{align*}
        (\omega_{ij})_{p}(u+\alpha v) &= \ip{DE_{i}(p)(u+\alpha v)}{E_{j}(p)}
        =\ip{DE_{i}(p)u+\alpha DE_{i}(p)v}{E_{j}(p)} \\[2mm]
        &= \ip{DE_{i}(p)u}{E_{j}(p)}+\alpha\ip{DE_{i}(p)v}{E_{j}(p)}
        =(\omega_{ij})_{p}(v)+\alpha(\omega_{ij})_{p}(v)
    \end{align*}
    Por otro lado, notemos que
    \begin{align*}
        (\omega_{ij})_{p}(v) &= \ip{DE_{i}(p)v}{E_{j}(p)}=\ip{\sum_{i=1}^{3}\pdv{E_{i}}
        {x_{k}}(p)v_{k}}{E_{j}(p)}=\sum_{i=1}^{3}\ip{\pdv{E_{i}}{x_{k}}}{E_{j}(p)}v_{k} \\[2mm]
        &= \sum_{i=1}^{3}\ip{\pdv{E_{i}}{x_{k}}}{E_{j}}(p)(dx_{k})_{p}(v)
    \end{align*}
    es decir,
    \begin{equation*}
        \omega_{ij}=\ip{\pdv{E_{i}}{x}}{E_{j}}dx+\ip{\pdv{E_{i}}{y}}{E_{j}}dy
        +\ip{\pdv{E_{i}}{z}}{E_{j}}dz
    \end{equation*}
    como las funciones $E_{i}$ y el producto interno son diferenciables, concluimos que 
    $\omega_{ij}$ es una 1-forma diferenciable. Queda chequear que $\omega_{ij}=-\omega_{ji}$, en
    efecto, sea $p\in W$ y $v\in\R^{3}$, entonces
    \begin{equation*}
        0=D\ip{E_{i}}{E_{j}}(p)v=\ip{DE_{i}(p)v}{E_{j}(p)}+\ip{E_{i}}{DE_{j}(p)v}
        =(\omega_{ij})_{p}(v)+(\omega_{ji})_{p}(v)
    \end{equation*}
    lo que prueba lo pedido.
    
    \item Sea $\{e_{j}\}_{j=1}^{3}$ la base canónica de $\R^{3}$. En primer lugar, tenemos que
    \begin{equation*}
        E_{i}=\sum_{j=1}^{3}\beta_{ij}e_{j}
        \hspace{4mm}\text{donde }\beta_{ij}(p)=e_{j}^{*}(E_{i}(p))
    \end{equation*}
    además, se tiene que $D\beta_{ij}=d\beta_{ij}$, en efecto, dado $p\in W$ y $v\in\R^{3}$
    \begin{equation*}
        (d\beta_{ij})_{p}(v)=\pdv{\beta_{ij}}{x}(p)v_{1}+\pdv{\beta_{ij}}{y}(p)v_{2}
        +\pdv{\beta_{ij}}{z}(p)v_{3}=D\beta_{ij}(p)v
    \end{equation*}
    lo que implica que
    \begin{equation*}
        DE_{i}=D\left(\sum_{j=1}^{3}\beta_{ij}e_{j}\right)=\sum_{j=1}^{3}D\beta_{ij}e_{j}
        =\sum_{j=1}^{3}d\beta_{ij}e_{j}
    \end{equation*}
    Por otro lado, como $(E_{j}(p))_{j=1}^{3}$ es base ortonormal de $\R^{3}$, dado $p\in W$ y 
    $v\in\R^{3}$ se obtiene que
    \begin{equation*}
        DE_{i}(p)v=\sum_{k=1}^{3}\ip{DE_{i}(p)v}{E_{k}(p)}E_{k}(p)
        =\sum_{k=1}^{3}(\omega_{ik})_{p}(v)E_{k}(p)
        \hspace{4mm}\text{es decir, }DE_{i}=\sum_{k=1}^{3}\omega_{ik}E_{k}
    \end{equation*}
    de este modo,
    \begin{equation*}
        DE_{i}=\sum_{k=1}^{3}\omega_{ik}E_{k}
        =\sum_{k=1}^{3}\omega_{ik}\left(\sum_{j=1}^{3}\beta_{kj}e_{j}\right)
        =\sum_{j=1}^{3}\left(\sum_{k=1}^{3}\omega_{ik}\beta_{kj}\right)e_{j}
    \end{equation*}
    comparando las dos expresiones para $DE_{i}$ y usando que $\{e_{j}\}_{j=1}^{3}$ es base, 
    concluimos que
    \begin{equation*}
        d\beta_{ij}=\sum_{k=1}^{3}\omega_{ik}\beta_{kj}
    \end{equation*}
    Además, por la expresión para $E_{i}$, vemos que
    \begin{equation*}
        \omega_{i}=\sum_{j=1}^{3}\beta_{ij}dx_{j}
    \end{equation*}
    derivando se sigue que
    \begin{align*}
        d\omega_{i} &= d\left(\sum_{j=1}^{3}\beta_{ij}dx_{j}\right)
        =\sum_{j=1}^{3}d\beta_{ij}\land dx_{j}+\beta_{ij}d(dx_{j})
        =\sum_{j=1}^{3}d\beta_{ij}\land dx_{j} \\[2mm]
        &= \sum_{j=1}^{3}\left(\sum_{k=1}^{3}\omega_{ik}\beta_{kj}\right)\land dx_{j}
        =\sum_{k=1}^{3}\omega_{ik}\land\left(\sum_{j=1}^{3}\beta_{kj}dx_{j}\right)
        =\sum_{k=1}^{3}\omega_{ik}\land\omega_{k}=\sum_{k=1}^{3}\omega_{k}\land\omega_{ki}
    \end{align*}

    \item Notemos lo siguiente
    \begin{equation*}
        0=d(d\beta_{ij})=\sum_{k=1}^{3}d(\omega_{ik}\beta_{kj})
        =\sum_{k=1}^{3}d\beta_{kj}\land\omega_{ik}+\sum_{k=1}^{3}d\omega_{ik}\beta_{kj}
    \end{equation*}
    entonces
    \begin{equation*}
        \sum_{k=1}^{3}d\omega_{ik}\beta_{kj}=\sum_{k=1}^{3}\omega_{ik}\land d\beta_{kj}
        =\sum_{k=1}^{3}\omega_{ik}\land\sum_{s=1}^{3}\omega_{ks}\beta_{sj}
        =\sum_{s=1}^{3}\left(\sum_{k=1}^{3}\omega_{ik}\land\omega_{ks}\right)\beta_{sj}
        \hspace{4mm}(*)
    \end{equation*}
    Sea $p\in W$ y $v\in\R^{3}$, definimos las matrices $d\Omega:=((d\omega_{ik})_{p}(v))
    _{i,k=1}^{3}$ y $B:=(\beta_{rj}(p))_{r,j=1}^{3}$, como $\{E_{r}\}_{r=1}^{3}$ forma una base
    ortonormal para todo $p\in W$, la matriz $B$ es ortogonal para todo $p\in W$, en particular 
    tenemos que es invertible. Consideremos además la matriz
    \begin{equation*}
        \Omega=\left(\sum_{k=1}^{3}(\omega_{ik})_{p}(v)\land(\omega_{ks})_{p}(v)\right)
        _{i,s=1}^{3}
    \end{equation*}
    luego el producto en $(*)$ puede ser resumido como $(d\Omega B)_{ij}=(\Omega B)_{ij}$, 
    y por ende tenemos la igualdad de matrices $d\Omega B=\Omega B$, lo que implica que
    $d\Omega=\Omega$ y dado que $p$ y $v$ son arbitrarios concluimos que
    \begin{equation*}
        d\omega_{ij}=\sum_{k=1}^{3}\omega_{ik}\land\omega_{kj}
    \end{equation*}
    y se tiene lo pedido.
\end{enumerate}

\vspace{2mm}
\noindent Colaboradores: Sergio Peña, Ricardo Larraín y Felipe Inostroza. Queria dar especiales
agradeciemientos a Felipe Inostroza por el Lema 0.1.

%\printbibliography % Quitar el comentado si quiero usar bibliografia

\end{document}
