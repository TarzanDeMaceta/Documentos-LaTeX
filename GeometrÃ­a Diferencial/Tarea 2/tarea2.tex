\documentclass{article}
\usepackage{hyperref}
\usepackage{Style}

\nocite{*} % Comentar si quiero citar
%\addbibresource{bibliografia.bib} % Quitar el comentado si quiero usar bibliografia

\begin{document}

\begin{minipage}{2.5cm}
    \includegraphics[width=2cm]{imagen_puc.jpg}
\end{minipage}
\begin{minipage}{14cm}
    {\sc Pontificia Universidad Católica de Chile\\
    Facultad de Matemáticas\\
    Departamento de Matemática\\
    Profesor: Pedro Gaspar -- Estudiante: Benjamín Mateluna}
\end{minipage}
\vspace{1ex}

{\centerline{\bf Geometría Diferencial - MAT2860}
\centerline{\bf Interrogación 2}}
\centerline{\bf 30 de mayo de 2025}

\section*{Problema 1}
\noindent Sea
\begin{equation*}
    \Sigma:=\left\{(x,y,z)\in\R^{3}:\frac{x^{2}}{a^{2}}+\frac{y^{2}}{b^{2}}+\frac{y^{2}}{c^{2}}
    =1\right\}
\end{equation*} 
Definimos la función diferenciable $h:\R^{3}\to\R$ dada por
\begin{equation*}
    h(x,y,z):=\frac{x^{2}}{a^{2}}+\frac{y^{2}}{b^{2}}+\frac{y^{2}}{c^{2}}
\end{equation*}
Notemos que $\Sigma=h^{-1}(1)$ con $1$ valor regular de $h$. Luego definimos la orientación 
del elipsoide dada por $N:\Sigma\to\mathbb{S}^{2}$ definida por
\begin{equation*}
    N(p):=\frac{\left(\frac{x}{a^{2}},\frac{y}{b^{2}},\frac{z}{c^{2}}\right)}
    {\sqrt{\frac{x^{2}}{a^{4}}+\frac{y^{2}}{b^{4}}+\frac{z^{2}}{c^{4}}}}
\end{equation*}
Tenemos tres casos:
\begin{enumerate}
    \item Si $r=a=b=c>0$, entonces $\Sigma$ es una esfera de radio $r$ y por ende 
    $N(p)=\frac{1}{r}p$ y por lo tanto $DN_{p}v=\frac{1}{r}v$ para todo $v\in T_{p}\Sigma$ y todo 
    $p\in\Sigma$, luego, todo punto en $\Sigma$ es umbilical.

    \item Si $a>b=c>0$, notemos que este caso es igual que $a=b>c>0$.
\end{enumerate}

\section*{Problema 2}
\begin{enumerate}
    \item Recordemos que se tiene el resultado
    \begin{equation*}
        K_{\Sigma}\circ X=\frac{eg-f^{2}}{EG-F^{2}}
    \end{equation*}
    como $F\equiv0$, vemos que
    \begin{equation*}
        (K_{\Sigma}\circ X)\cdot EG=eg-f^{2}
    \end{equation*}
    Por otro lado, se tienen las siguientes identidades para $E_{vv}$ y $G_{uu}$,
    \begin{align*}
        & E_{vv}=\pdv{^{2}}{v^{2}}\left(\ip{X_{u}}{X_{u}}\right)
        =\pdv{}{v}\left(2\ip{X_{uv}}{X_{u}}\right)
        =2\left(\ip{X_{uvv}}{X_{u}}+\ip{X_{uv}}{X_{uv}}\right) \\
        & G_{uu}=\pdv{^{2}}{u^{2}}\left(\ip{X_{v}}{X_{v}}\right)
        =\pdv{}{u}\left(2\ip{X_{vu}}{X_{v}}\right)
        =2\left(\ip{X_{vuu}}{X_{v}}+\ip{X_{vu}}{X_{vu}}\right)
    \end{align*}
    Para la expresiones $\abs{(X_{uv})^{T}}^{2}$ y $\ip{(X_{uu})^{T}}{(X_{vv})^{T}}$, tenemos que
    \begin{align*}
        \abs{(X_{uv})^{T}}^{2} &= \ip{X_{uv}-\ip{X_{uv}}{N^{x}}N^{x}}
        {X_{uv}-\ip{X_{uv}}{N^{x}}N^{x}} \\
        &= \ip{X_{uv}-fN^{x}}{X_{uv}-fN^{x}}=\abs{X_{uv}}^{2}-f^{2}-f^{2}+f^{2}
        =\abs{X_{uv}}^{2}-f^{2}
    \end{align*}
    y
    \begin{align*}
        \ip{(X_{uu})^{T}}{(X_{vv})^{T}} &= \ip{X_{uu}-\ip{X_{uu}}{N^{x}}N^{x}}{X_{vv}-\ip{X_{vv}}
        {N^{x}}N^{x}} \\
        &= \ip{X_{uu}-eN^{x}}{X_{vv}-gN^{x}}=\ip{X_{uu}}{X_{vv}}-eg-eg+eg=\ip{X_{uu}}{X_{vv}}-eg
    \end{align*}
    recordando que $e=\ip{N^{x}}{X_{uu}}$, $f=\ip{N^{x}}{X_{uv}}$ y $g=\ip{N^{x}}{X_{vv}}$. 
    Además, tenemos lo siguiente como consecuencia de que la parametrización es ortogonal
    \begin{align*}
        0=\pdv{^{2}}{u\partial v}\left(\ip{X_{u}}{X_{v}}\right)
        &= \pdv{}{u}\left(\ip{X_{uv}}{X_{v}}+\ip{X_{u}}{X_{vv}}\right) \\
        &= \ip{X_{uvu}}{X_{v}}+\ip{X_{uv}}{X_{vu}}+\ip{X_{uu}}{X_{vv}}+\ip{X_{u}}{X_{vvu}}
    \end{align*}
    lo que implica que
    \begin{equation*}
        -\ip{X_{uu}}{X_{vv}}=\ip{X_{uvu}}{X_{v}}+\ip{X_{uv}}{X_{vu}}+\ip{X_{u}}{X_{vvu}}
    \end{equation*}
    Usando lo anterior, se sigue que
    \begin{align*}
        & -\frac{1}{2}\left(E_{vv}+G_{uu}\right)+\abs{(X_{uv})^{T}}^{2}-\ip{(X_{uu})^{T}}
        {(X_{vv})^{T}} \\
        &= -\ip{X_{uvv}}{X_{u}}-\ip{X_{uv}}{X_{uv}}-\ip{X_{vuu}}{X_{v}}-\ip{X_{vu}}{X_{vu}}
        +\abs{X_{uv}}^{2}-f^{2}+eg-\ip{X_{uu}}{X_{vv}} \\
        &= eg-f^{2}
    \end{align*}
    y se tiene lo pedido. (Para esta parte trabaje en conjunto con Ricardo Larraín)

    \item Como $\{X_{u},X_{v},N\}$ forman una base de $\R^{3}$, existen $a,b,c\in\R$ tales que
    \begin{equation*}
        X_{uu}=aX_{u}+bX_{v}+cN
    \end{equation*}
    Notemos que $\ip{X_{uu}}{N^{x}}=e$, veamos las siguientes igualdades para $\ip{X_{uu}}{X_{u}}$ 
    y $\ip{X_{uu}}{X_{v}}$
    \begin{equation*}
        \ip{X_{uu}}{X_{u}}=\frac{1}{2}\pdv{}{u}\left(\ip{X_{u}}{X_{u}}\right)=\frac{1}{2}E_{u}
    \end{equation*}
    y usando que $F\equiv0$ vemos que
    \begin{equation*}
        0=\pdv{}{u}\left(\ip{X_{u}}{X_{v}}\right)=\ip{X_{uu}}{X_{v}}+\ip{X_{u}}{X_{vu}}
        =\ip{X_{uu}}{X_{v}}+\frac{1}{2}\pdv{}{v}\left(\ip{X_{u}}{X_{u}}\right)
        =\ip{X_{uu}}{X_{v}}+\frac{1}{2}E_{v}
    \end{equation*}
    entonces tomando producto interno con $X_{u}$, $X_{v}$ y $N^{x}$ respectivamente y usando que
    $F\equiv0$, $N^{x}$ es unitario y ortogonal a $\{X_{u}$,$X_{v}\}$, tenemos que
    \begin{equation*}
        a=\frac{\ip{X_{uu}}{X_{u}}}{\ip{X_{u}}{X_{u}}}=\frac{E_{u}}{2E}\hspace{1.5cm}
        b=\frac{\ip{X_{uu}}{X_{v}}}{\ip{X_{v}}{X_{v}}}=-\frac{E_{v}}{2G}\hspace{1.5cm}
        c=\ip{X_{uu}}{N^{x}}=e
    \end{equation*}
    Del mismo modo que antes, tenemos $X_{uv}=a_{0}X_{u}+b_{0}X_{v}+c_{0}N$, veamos que 
    $\ip{X_{uv}}{N^{x}}=f$ y además
    \begin{align*}
        \ip{X_{uv}}{X_{v}} &= \frac{1}{2}\pdv{}{u}\left(\ip{X_{v}}{X_{v}}\right)
        =\frac{1}{2}G_{u} \\
        \ip{X_{uv}}{X_{u}} &= \frac{1}{2}\pdv{}{v}\left(\ip{X_{u}}{X_{u}}\right)
        =\frac{1}{2}E_{v}
    \end{align*}
    luego
    \begin{equation*}
        a_{0}=\frac{\ip{X_{uv}}{X_{u}}}{\ip{X_{u}}{X_{u}}}=\frac{E_{v}}{2E}\hspace{1.5cm}
        b_{0}=\frac{\ip{X_{uv}}{X_{v}}}{\ip{X_{v}}{X_{v}}}=\frac{G_{u}}{2G}\hspace{1.5cm}
        c_{0}=\ip{X_{uv}}{N^{x}}=f
    \end{equation*}
    Queda ver $ X_{vv}=a_{1}X_{u}+b_{1}X_{v}+c_{1}N$. Recordemos que $\ip{X_{vv}}{N^{x}}=g$ y
    adicionalmente
    \begin{equation*}
        \ip{X_{vv}}{X_{v}}=\frac{1}{2}\pdv{}{v}\left(\ip{X_{v}}{X_{v}}\right)=\frac{1}{2}G_{v}
    \end{equation*}
    por otro lado se tiene que
    \begin{equation*}
        0=\pdv{}{v}\left(\ip{X_{u}}{X_{v}}\right)=\ip{X_{uv}}{X_{v}}+\ip{X_{u}}{X_{vv}}
        =\frac{1}{2}G_{u}+\ip{X_{u}}{X_{vv}}
    \end{equation*}
    entonces
    \begin{equation*}
        a_{0}=\frac{\ip{X_{vv}}{X_{u}}}{\ip{X_{u}}{X_{u}}}=-\frac{G_{u}}{2E}\hspace{1.5cm}
        b_{0}=\frac{\ip{X_{vv}}{X_{v}}}{\ip{X_{v}}{X_{v}}}=\frac{G_{v}}{2G}\hspace{1.5cm}
        c_{0}=\ip{X_{vv}}{N^{x}}=g
    \end{equation*}
    lo que demuestra lo pedido.

    \item Del item anterior tenemos las siguientes igualdades
    \begin{align*}
        (X_{uu})^{T} &= \frac{E_{u}}{2E}X_{u}-\frac{E_{v}}{2G}X_{v} \\
        (X_{uv})^{T} &= \frac{E_{v}}{2E}X_{u}+\frac{G_{u}}{2G}X_{v} \\
        (X_{vv})^{T} &= -\frac{G_{u}}{2E}X_{u}+\frac{G_{v}}{2G}X_{v}
    \end{align*}
    Por otro lado también se tiene lo siguiente
    \begin{align*}
        \abs{(X_{uv})^{T}}^{2} &= \ip{\frac{E_{v}}{2E}X_{u}+\frac{G_{u}}{2G}X_{v}}
        {\frac{E_{v}}{2E}X_{u}+\frac{G_{u}}{2G}X_{v}}=\frac{E_{v}^{2}}{4E}+\frac{G_{u}^{2}}{4G} \\
        \ip{(X_{uu})^{T}}{(X_{vv})^{T}} &= \ip{\frac{E_{u}}{2E}X_{u}-\frac{E_{v}}{2G}X_{v}}
        {-\frac{G_{u}}{2E}X_{u}+\frac{G_{v}}{2G}X_{v}}=-\frac{E_{u}G_{u}}{4E}-\frac{E_{v}G_{v}}{4G}
    \end{align*}
    Luego
    \begin{align*}
        &-\frac{1}{2\sqrt{EG}}\left[\left(\frac{G_{u}}{\sqrt{EG}}\right)_{u}+\left(\frac{E_{v}}
        {\sqrt{EG}}\right)_{v}\right] \\[2mm]
        &= -\frac{1}{2\sqrt{EG}}\left(\frac{G_{uu}\sqrt{EG}-G_{u}
        \frac{E_{u}G+EG_{u}}{2\sqrt{EG}}+E_{vv}\sqrt{EG}-E_{v}\frac{E_{v}G+EG_{v}}{2\sqrt{EG}}}{EG}
        \right) \\[2mm]
        &= -\frac{1}{4EG}\left(\frac{2G_{uu}EG-G_{u}(E_{u}G+EG_{u})+2E_{vv}EG-E_{v}
        (E_{v}G+EG_{v})}{EG}\right) \\[2mm]
        &= -\frac{1}{4EG}\left(2(E_{vv}+G_{uu})-\frac{G_{u}E_{u}}{E}-\frac{G_{u}^{2}}{G}
        -\frac{E_{v}^{2}}{E}-\frac{E_{v}G_{v}}{G}\right) \\[2mm]
        &= \frac{1}{EG}\left(-\frac{1}{2}(E_{vv}+G_{uu})+\frac{E_{v}^{2}}{4E}+\frac{G_{u}^{2}}{4G}
        +\frac{E_{u}G_{u}}{4E}+\frac{E_{v}G_{v}}{4G}\right) \\[2mm]
        &=\frac{1}{EG}\left(-\frac{1}{2}\left(E_{vv}+G_{uu}\right)+\abs{(X_{uv})^{T}}^{2}
        -\ip{(X_{uu})^{T}}{(X_{vv})^{T}}\right)=\frac{1}{EG}(K_{\Sigma}\circ X)EG=K_{\Sigma}\circ X
    \end{align*}    
\end{enumerate}

\section*{Problema 3}
\begin{enumerate}
    \item Sea $\gamma:[a,b]\to\Sigma$ una curva continua. Definimos el conjunto
    \begin{equation*}
        A:=\{c\in[a,b]:\text{Existe }N_{c}:[a,c]\to\mathbb{S}^{2}\text{ continua, }
        N_{c}(a)=n\text{ y }N_{c}(t)\in(T_{\gamma(t)}\Sigma)^{\perp}\}
    \end{equation*}
    Notemos que $A$ es no vacío, en efecto, $a\in A$ tomando $N_{a}(a)=n$ y es continua por ser
    constante. 
    
    \noindent Afirmamos que $A$ es abierto, sea $c\in A$. Existe $N_{c}:[a,c]\to\mathbb{S}^{2}$
    continua tal que $N_{c}(a)=n$ y $N_{c}(t)\in(T_{\gamma(t)}\Sigma)^{\perp}$. Sea $(\U,X)$ una 
    carta de $\gamma(c)$. Sea $\varepsilon>0$ tal que $(c-\varepsilon,c+\varepsilon)
    =:V\subseteq\gamma^{-1}(X(\U))$. Dado $0<\delta<\varepsilon$ definimos $N_{\delta}^{x}:
    \U\to\mathbb{S}^{2}$ como sigue
    \begin{equation*}
        N_{\delta}^{x}(u,v):=\frac{X_{u}\times X_{v}}{\abs{X_{u}\times X_{v}}}(u,v)
    \end{equation*}
    Consideramos la función $N_{\delta}^{c}:=N_{\delta}^{x}\circ X^{-1}\circ
    \gamma:V\to\mathbb{S}^{2}$ tal que $N_{\delta}^{c}(c)=N_{c}(c)$. Definimos 
    $N_{\delta}:[0,\delta]\to\mathbb{S}^{2}$ como
    \begin{equation*}
        N_{\delta}(t):=\begin{cases}
            N_{c}(t) &\quad\text{si }t\in[a,c] \\
            N_{\delta}^{c}(t) &\quad\text{si }t\in[c,c+\delta]
        \end{cases}
    \end{equation*}
    Por lema del pegamiento $N_{\delta}$ es una función continua, ya que 
    $N_{\delta}^{c}(c)=N_{c}(c)$. Además, por construcción, cumple las hipotesis necesarias, luego
    $[a,c+\varepsilon)\subset A$.

    \noindent Veamos que $A$ es cerrado. Sea $(c_{n})_{n}\subseteq A$ tal que 
    $c_{n}$ converge a $c\in[a,b]$. Sea $(\U,X)$ una carta de $\gamma(c)$, por 
    continuidad, existe $n\in\N$ tal que $\gamma(c_{n})\in X(\U)$. 
    
    Definimos $N_{c}^{x}$ del mismo modo que antes. Sea $V:=\gamma^{-1}(X(\U))$ consideremos 
    $N_{c}^{0}:=N_{c}^{x}\circ X^{-1}\circ\gamma:V\to\mathbb{S}^{2}$ tal que $N_{c}^{0}(c_{n})
    =N_{c_{n}}(c_{n})$. Se define $N_{c}:[a,c]\to\mathbb{S}^{2}$ por
    \begin{equation*}
        N_{c}(t):=\begin{cases}
            N_{c_{n}}(t) &\quad\text{si }t\in[a,c_{n}] \\
            N_{c}^{0}(t) &\quad\text{si }t\in[c_{n},c]
        \end{cases}
    \end{equation*}
    Al igual que antes, esta función es continua, por lema de pegamientos y cumple con las 
    hipotesis necesarias por construcción y por lo tanto $c\in A$. Así, $A\subseteq[a,b]$ es clopen
    y por lo tanto $A=[a,b]$.

    \noindent Veamos que la función $N_{\gamma}$ es unica, supongamos que existe $N'$ que 
    satisface las mismas condiciones, luego $(N_{\gamma}-N')^{-1}(0)$ y $(N_{\gamma}+N')^{-1}(0)$
    son cerrados disjuntos que separan $[a,b]$, pues $N_{\gamma}(t)=\pm N'(t)$. Como 
    $N_{\gamma}(a)=N'(a)$, se sigue que $N_{\gamma}(t)=N'(t)$ para todo $t\in[a,b]$ lo que prueba 
    la unicidad.

    \item Supongamos que $\Sigma$ es orientable, entonces existe un campo normal unitario continuo
    $N:\Sigma\to\mathbb{S}^{2}$. Sea $\gamma:[a,b]\to\Sigma$ una curva continua y cerrada. Por la
    parte anterior existe una única función y continua $N_{\gamma}:[a,b]\to\mathbb{S}^{2}$ tal que 
    \begin{equation*}
        N(\gamma(a))=N_{\gamma}(a)\hspace{4mm}\text{y}\hspace{4mm}
        N_{\gamma}(t)\in(T_{\gamma(t)}\Sigma)^{\perp}
    \end{equation*}
    
    \vspace{2mm}
    Consideremos la función continua $N\circ\gamma:[a,b]\to\mathbb{S}^{2}$. Notemos que 
    $N\circ\gamma$ cumple las mismas propiedades que $N_{\gamma}$ por definición de $N$, luego,
    por unicidad se sigue que
    \begin{equation*}
        N_{\gamma}(a)=N(\gamma(a))=N(\gamma(b))=N_{\gamma}(b)
    \end{equation*}

    
    \noindent Supongamos, sin perdida de generalidad, que $\Sigma$ es conexa, en caso contrario 
    basta ver el resultado para cada componente conexa. Supongamos que para toda curva $\gamma$ 
    continua y cerrada se cumple que $N_{\gamma}(a)=N_{\gamma}(b)$. 
    
    Sea $q\in\Sigma$ consideremos su vector normal $n$ unitario, definimos 
    $N:\Sigma\to\mathbb{S}^{2}$ como sigue
    \begin{equation*}
        N(p):=N_{\gamma}(1)
    \end{equation*}
    donde $\gamma:[0,1]\to\Sigma$ es una curva continua tal que $\gamma(0)=q$ y $\gamma(1)=p$, 
    además $N_{\gamma}(0)=n$ para toda curva $\gamma$. Veamos que $N$ esta bien definida, es decir,
    es independiente de la curva representante. Sean $\gamma,\gamma_{0}$ como antes, definimos la
    curva
    \begin{equation*}
        \alpha(t):=\begin{cases}
            \gamma(1-2t) &\quad\text{si }t\in[0,1/2] \\
            \gamma_{0}(2t-1) &\quad\text{si }t\in[1/2,1]
        \end{cases}
    \end{equation*}
    y su campo normal $N_{\alpha}:[0,1]\to\mathbb{S}^{2}$ como
    \begin{equation*}
        N_{\alpha}(t):=\begin{cases}
            N_{\gamma}(1-2t) &\quad\text{si }t\in[0,1/2] \\
            N_{\gamma_{0}}(2t-1) &\quad\text{si }t\in[1/2,1]
        \end{cases}
    \end{equation*}
    es continuo pues $N_{\gamma}(0)=N_{\gamma_{0}}(0)$ y es única tal que $N_{\alpha}(0)
    =N_{\gamma}(1)$, luego como $\alpha$ es cerrada
    \begin{equation*}
        N_{\gamma}(1)=N_{\alpha}(0)=N_{\alpha}(1)=N_{\gamma_{0}}(1)
    \end{equation*}
    Veamos que $N$ es continua. Sea $p_{0}\in\Sigma$, sea $\gamma$ un camino que une $q$ con 
    $p_{0}$, entonces por continuidad de $N_{\gamma}$ se sigue que $\lim\limits_{t\to b}
    N(\gamma(t))=\lim\limits_{t\to b}N_{\gamma}(t)=N_{\gamma(b)}=N(p_{0})$. Como lo anterior es 
    independiente del camino, conluimos que
    \begin{equation*}
        \lim\limits_{p\to p_{0}}N(p)=N(p_{0})
    \end{equation*}
\end{enumerate}

\section*{Problema 4}
\noindent Si $p\in S_{r}$, es claro que $\Phi(p)=\phi(p)$, adicionalmente esta es la única extensión. Veamos que $\Phi:\R^{3}\setminus\{0\}\to
\R^{3}\setminus\{0\}$ es invertible, como $\phi$ es isometría, en particular es invertible, 
definimos $\psi:\R^{3}\setminus\{0\}\to\R^{3}\setminus\{0\}$ como sigue
\begin{equation*}
    \psi(p):=\frac{\abs{p}}{r}\phi^{-1}\left(\frac{p}{\abs{p}}r\right)
\end{equation*}
luego
\begin{align*}
    \Phi\circ\psi(p) &= \Phi(\psi(p))=\Phi\left(\frac{\abs{p}}{r}\phi^{-1}
    \left(\frac{p}{\abs{p}}r\right)\right)=\frac{\abs{\frac{\abs{p}}{r}\phi^{-1}\left(\frac{p}
    {\abs{p}}r\right)}}{r}\phi\left(\frac{\frac{\abs{p}}{r}\phi^{-1}\left(\frac{p}
    {\abs{p}}r\right)}{\abs{\frac{\abs{p}}{r}\phi^{-1}\left(\frac{p}{\abs{p}}r\right)}}r\right) \\
    &= \frac{\abs{p}}{r}\phi\left(\phi^{-1}\left(\frac{p}{\abs{p}}r\right)\right)=p
\end{align*}
del mismo modo se sigue que $\psi\circ\Phi(p)=p$. Veamos que $D\Phi_{p}$ es isometría lineal y por 
ende un isomorfismo lineal. Sea $p\in\R^{3}\setminus\{0\}$ y sea $w\in\R^{3}$, entonces
\begin{align*}
    D\Phi_{p}w &= \dv{}{t}\Big|_{t=0}\Phi(p+tw)=\dv{}{t}\Big|_{t=0}\left(\frac{\abs{p+tw}}{r}\phi
    \left(r\frac{p+tw}{\abs{p+tw}}\right)\right) \\[2mm]
    &= \left(\frac{\ip{p+tw}{w}}{r\abs{p+tw}}\phi\left(r\frac{p+tw}{\abs{p+tw}}\right)
    +\frac{\abs{p+tw}}{r}D\phi_{\frac{p}{\abs{p}}r}\left(r\frac{w\abs{p+tw}-(p+tw)
    \frac{\ip{p+tw}{w}}{\abs{p+tw}}}{\abs{p+tw}^{2}}\right)\right)\Big|_{t=0} \\[2mm]
    &= \frac{\ip{p}{w}}{r\abs{p}}\phi\left(r\frac{p}{\abs{p}}\right)+\abs{p}
    D\phi_{\frac{p}{\abs{p}}r}\left(\frac{w}{\abs{p}}-\frac{p\ip{p}{w}}{\abs{p}^{3}}\right)
\end{align*}
notemos que
\begin{align*}
    \abs{D\Phi_{p}w}^{2} &= \abs{\frac{\ip{p}{w}}{r\abs{p}}\phi\left(r\frac{p}{\abs{p}}\right)+
    \abs{p}D\phi_{\frac{p}{\abs{p}}r}\left(\frac{w}{\abs{p}}-\frac{p\ip{p}{w}}{\abs{p}^{3}}
    \right)}^{2} \\
    &= \frac{\ip{p}{w}^{2}}{\abs{p}^{2}}+\abs{w-p\frac{\ip{p}{w}}{\abs{p}^{2}}}^{2}
    =\frac{\ip{p}{w}^{2}}{\abs{p}^{2}}+\abs{w}^{2}-2\frac{\ip{p}{w}^{2}}{\abs{p}^{2}}
    +\frac{\ip{p}{w}^{2}}{\abs{p}^{2}}=\abs{w}^{2}
\end{align*}
donde la segunda igualdad se debe a que dado $v\in S_{r}$ se tiene que $\ip{D\phi_{v}w}{v}=0$, 
además usamos el hecho de que $D\phi_{v}$ es isometría lineal. Por teorema de la función inversa
se sigue que $\Phi$ es difeomorfismo local y como es biyectiva es difeomorfismo, así por ejercicio
visto en clase existe un único movimiento rígido $F:\R^{3}\setminus\{0\}\to\R^{3}\setminus\{0\}$ 
tal que $\Phi=F\big|_{\R^{3}\setminus\{0\}}$.

\noindent Notemos que $F(S_{r})=\Phi(S_{r})=\phi(S_{r})=S_{r}$ y por lo tanto $F$ es una isometría 
lineal de $\R^{3}$ y $\phi=\Phi\big|_{S_{r}}=F\big|_{S_{r}}$. Por otro lado por un ejemplo visto 
en clase, dada una isometría lineal $F$ tenemos que $F(S_{r})=S_{r}$ lo que implica que
$\phi:=F\big|_{S_{r}}$ es un difeomorfismo y $D\phi_{p}$ es isometría lineal.

\vspace{2mm}
\noindent Por último, queda ver que esta correspondencia es, de hecho, un morfismo de grupos. Sean
$\phi,\psi$ isometrías de $S_{r}$ y $F,G$ sus correspondientes isometrías lineales de $\R^{3}$, es 
claro que $F\circ G$ es isometría y dado $p\in S_{r}$ notamos que $(F\circ G)(p)=F(G(p))=F(\psi(p))
=\phi(\psi(p))$, por unicidad $F\circ G$ es la isometría que corresponde a $\phi\circ\psi$. 
Concluimos que $Isom(S_{r})$ es isomorfo al grupo de isometrías lineales de $\R^{3}$.

\section*{Problema 5}
\begin{enumerate}
    \item Consideramos la parametrización $X:(0,2\pi)\times(0,2\pi)\to\R^{3}$ dada por
    \begin{equation*}
        X(u,v):=((R+rcos(u))cos(v),(R+rcos(u))sen(v),rsen(u))
    \end{equation*}
    calculamos sus derivadas parciales para encontrar los coeficientes de la primera forma 
    fundamental
    \begin{align*}
        F &= \ip{X_{u}}{X_{v}}=(R+rcos(u))rsen(u)cos(v)sen(v)-(R+rcos(u))rsen(u)cos(v)sen(v)=0 
        \\[2mm]
        E &= \ip{X_{u}}{X_{u}}=r^{2}sen^{2}(u)cos^{2}(v)+r^{2}sen^{2}(u)sen^{2}(v)+r^{2}cos^{2}(u)
        =r^{2}sen(u)+r^{2}cos(u)=r^{2} \\[2mm]
        G &= \ip{X_{v}}{X_{v}}=(R+rcos(u))^{2}sen^{2}(v)+(R+rcos(u))^{2}cos^{2}(v)=(R+rcos(u))^{2}
    \end{align*}
    Como $F\equiv0$, la parametrización es ortogonal y por el problema tenemos que
    \begin{equation*}
        (K_{\Sigma}\circ X)\sqrt{EG}=-\frac{1}{2}\left[\left(\frac{G_{u}}{\sqrt{EG}}\right)_{u}
        +\left(\frac{E_{v}}{\sqrt{EG}}\right)_{v}\right]
    \end{equation*}
    donde $E_{v}\equiv0$, pues $E$ es constante, además $\sqrt{EG}=(R+rcos(u))r$ y $G_{u}
    =-2(R+rcos(u))rsen(u)$, luego
    \begin{equation*}
        \left(\frac{G_{u}}{\sqrt{EG}}\right)_{u}=\left(\frac{-2(R+rcos(u))rsen(u)}{(R+rcos(u))r}
        \right)_{u}=(-2sen(u))_{u}=-2cos(u)
    \end{equation*}
    Utilizando lo anterior, vemos que
    \begin{equation*}
        (K_{\Sigma}\circ X)\sqrt{EG}=-\frac{1}{2}\left[\left(\frac{G_{u}}{\sqrt{EG}}\right)_{u}
        +\left(\frac{E_{v}}{\sqrt{EG}}\right)_{v}\right]=-\frac{1}{2}(-2cos(u)+0)=cos(u)
    \end{equation*}

    \item Así
    \begin{equation*}
        \int_{0}^{2\pi}\int_{0}^{2\pi}(K_{\Sigma}\circ X)\sqrt{EG}\hspace{1mm}dudv
        =\int_{0}^{2\pi}\int_{0}^{2\pi}cos(u)\hspace{1mm}du\hspace{1mm}dv=0
    \end{equation*}
\end{enumerate}

%\printbibliography % Quitar el comentado si quiero usar bibliografia

\end{document}
