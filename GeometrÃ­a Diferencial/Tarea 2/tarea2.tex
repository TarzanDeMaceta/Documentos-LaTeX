\documentclass{article}
\usepackage{hyperref}
\usepackage{Style}

\nocite{*} % Comentar si quiero citar
%\addbibresource{bibliografia.bib} % Quitar el comentado si quiero usar bibliografia

\begin{document}

\begin{minipage}{2.5cm}
    \includegraphics[width=2cm]{imagen_puc.jpg}
\end{minipage}
\begin{minipage}{14cm}
    {\sc Pontificia Universidad Católica de Chile\\
    Facultad de Matemáticas\\
    Departamento de Matemática\\
    Profesor: Pedro Gaspar -- Estudiante: Benjamín Mateluna}
\end{minipage}
\vspace{1ex}

{\centerline{\bf Geometría Diferencial - MAT2860}
\centerline{\bf Tarea 2}}
\centerline{\bf 30 de mayo de 2025}

\section*{Problema 1}
\begin{enumerate}
    \item Recordemos que se tiene el resultado
    \begin{equation*}
        K_{\Sigma}\circ X=\frac{eg-f^{2}}{EG-F^{2}}
    \end{equation*}
    como $F\equiv0$, vemos que
    \begin{equation*}
        (K_{\Sigma}\circ X)\cdot EG=eg-f^{2}
    \end{equation*}
    Por otro lado, se tienen las siguientes identidades para $E_{vv}$ y $G_{uu}$,
    \begin{align*}
        & E_{vv}=\pdv{^{2}}{v^{2}}\left(\ip{X_{u}}{X_{u}}\right)
        =\pdv{}{v}\left(2\ip{X_{uv}}{X_{u}}\right)
        =2\left(\ip{X_{uvv}}{X_{u}}+\ip{X_{uv}}{X_{uv}}\right) \\
        & G_{uu}=\pdv{^{2}}{u^{2}}\left(\ip{X_{v}}{X_{v}}\right)
        =\pdv{}{u}\left(2\ip{X_{vu}}{X_{v}}\right)
        =2\left(\ip{X_{vuu}}{X_{v}}+\ip{X_{vu}}{X_{vu}}\right)
    \end{align*}
    Para la expresiones $\abs{(X_{uv})^{T}}^{2}$ y $\ip{(X_{uu})^{T}}{(X_{vv})^{T}}$, tenemos que
    \begin{align*}
        \abs{(X_{uv})^{T}}^{2} &= \ip{X_{uv}-\ip{X_{uv}}{N^{x}}N^{x}}
        {X_{uv}-\ip{X_{uv}}{N^{x}}N^{x}} \\
        &= \ip{X_{uv}-fN^{x}}{X_{uv}-fN^{x}}=\abs{X_{uv}}^{2}-f^{2}-f^{2}+f^{2}
        =\abs{X_{uv}}^{2}-f^{2}
    \end{align*}
    y
    \begin{align*}
        \ip{(X_{uu})^{T}}{(X_{vv})^{T}} &= \ip{X_{uu}-\ip{X_{uu}}{N^{x}}N^{x}}{X_{vv}-\ip{X_{vv}}
        {N^{x}}N^{x}} \\
        &= \ip{X_{uu}-eN^{x}}{X_{vv}-gN^{x}}=\ip{X_{uu}}{X_{vv}}-eg-eg+eg=\ip{X_{uu}}{X_{vv}}-eg
    \end{align*}
    recordando que $e=\ip{N^{x}}{X_{uu}}$, $f=\ip{N^{x}}{X_{uv}}$ y $g=\ip{N^{x}}{X_{vv}}$. 
    Además, tenemos lo siguiente como consecuencia de que la parametrización es ortogonal
    \begin{align*}
        0=\pdv{^{2}}{u\partial v}\left(\ip{X_{u}}{X_{v}}\right)
        &= \pdv{}{u}\left(\ip{X_{uv}}{X_{v}}+\ip{X_{u}}{X_{vv}}\right) \\
        &= \ip{X_{uvu}}{X_{v}}+\ip{X_{uv}}{X_{vu}}+\ip{X_{uu}}{X_{vv}}+\ip{X_{u}}{X_{vvu}}
    \end{align*}
    lo que implica que
    \begin{equation*}
        -\ip{X_{uu}}{X_{vv}}=\ip{X_{uvu}}{X_{v}}+\ip{X_{uv}}{X_{vu}}+\ip{X_{u}}{X_{vvu}}
    \end{equation*}
    Usando lo anterior, se sigue que
    \begin{align*}
        & -\frac{1}{2}\left(E_{vv}+G_{uu}\right)+\abs{(X_{uv})^{T}}^{2}-\ip{(X_{uu})^{T}}
        {(X_{vv})^{T}} \\
        &= -\ip{X_{uvv}}{X_{u}}-\ip{X_{uv}}{X_{uv}}-\ip{X_{vuu}}{X_{v}}-\ip{X_{vu}}{X_{vu}}
        +\abs{X_{uv}}^{2}-f^{2}+eg-\ip{X_{uu}}{X_{vv}} \\
        &= eg-f^{2}
    \end{align*}
    y se tiene lo pedido. (Para esta parte trabaje en conjunto con Ricardo Larraín)
\newpage

    \item Como $\{X_{u},X_{v},N\}$ forman una base de $\R^{3}$, existen $a,b,c\in\R$ tales que
    \begin{equation*}
        X_{uu}=aX_{u}+bX_{v}+cN
    \end{equation*}
    Notemos que $\ip{X_{uu}}{N^{x}}=e$, veamos las siguientes igualdades para $\ip{X_{uu}}{X_{u}}$ 
    y $\ip{X_{uu}}{X_{v}}$
    \begin{equation*}
        \ip{X_{uu}}{X_{u}}=\frac{1}{2}\pdv{}{u}\left(\ip{X_{u}}{X_{u}}\right)=\frac{1}{2}E_{u}
    \end{equation*}
    y usando que $F\equiv0$ vemos que
    \begin{equation*}
        0=\pdv{}{u}\left(\ip{X_{u}}{X_{v}}\right)=\ip{X_{uu}}{X_{v}}+\ip{X_{u}}{X_{vu}}
        =\ip{X_{uu}}{X_{v}}+\frac{1}{2}\pdv{}{v}\left(\ip{X_{u}}{X_{u}}\right)
        =\ip{X_{uu}}{X_{v}}+\frac{1}{2}E_{v}
    \end{equation*}
    entonces tomando producto interno con $X_{u}$, $X_{v}$ y $N^{x}$ respectivamente y usando que
    $F\equiv0$, $N^{x}$ es unitario y ortogonal a $\{X_{u}$,$X_{v}\}$, tenemos que
    \begin{equation*}
        a=\frac{\ip{X_{uu}}{X_{u}}}{\ip{X_{u}}{X_{u}}}=\frac{E_{u}}{2E}\hspace{1.5cm}
        b=\frac{\ip{X_{uu}}{X_{v}}}{\ip{X_{v}}{X_{v}}}=-\frac{E_{v}}{2G}\hspace{1.5cm}
        c=\ip{X_{uu}}{N^{x}}=e
    \end{equation*}
    Del mismo modo que antes, tenemos $X_{uv}=a_{0}X_{u}+b_{0}X_{v}+c_{0}N$, veamos que 
    $\ip{X_{uv}}{N^{x}}=f$ y además
    \begin{align*}
        \ip{X_{uv}}{X_{v}} &= \frac{1}{2}\pdv{}{u}\left(\ip{X_{v}}{X_{v}}\right)
        =\frac{1}{2}G_{u} \\
        \ip{X_{uv}}{X_{u}} &= \frac{1}{2}\pdv{}{v}\left(\ip{X_{u}}{X_{u}}\right)
        =\frac{1}{2}E_{v}
    \end{align*}
    luego
    \begin{equation*}
        a_{0}=\frac{\ip{X_{uv}}{X_{u}}}{\ip{X_{u}}{X_{u}}}=\frac{E_{v}}{2E}\hspace{1.5cm}
        b_{0}=\frac{\ip{X_{uv}}{X_{v}}}{\ip{X_{v}}{X_{v}}}=\frac{G_{u}}{2G}\hspace{1.5cm}
        c_{0}=\ip{X_{uv}}{N^{x}}=f
    \end{equation*}
    Queda ver $ X_{vv}=a_{1}X_{u}+b_{1}X_{v}+c_{1}N$. Recordemos que $\ip{X_{vv}}{N^{x}}=g$ y
    adicionalmente
    \begin{equation*}
        \ip{X_{vv}}{X_{v}}=\frac{1}{2}\pdv{}{v}\left(\ip{X_{v}}{X_{v}}\right)=\frac{1}{2}G_{v}
    \end{equation*}
    por otro lado se tiene que
    \begin{equation*}
        0=\pdv{}{v}\left(\ip{X_{u}}{X_{v}}\right)=\ip{X_{uv}}{X_{v}}+\ip{X_{u}}{X_{vv}}
        =\frac{1}{2}G_{u}+\ip{X_{u}}{X_{vv}}
    \end{equation*}
    entonces
    \begin{equation*}
        a_{0}=\frac{\ip{X_{vv}}{X_{u}}}{\ip{X_{u}}{X_{u}}}=-\frac{G_{u}}{2E}\hspace{1.5cm}
        b_{0}=\frac{\ip{X_{vv}}{X_{v}}}{\ip{X_{v}}{X_{v}}}=\frac{G_{v}}{2G}\hspace{1.5cm}
        c_{0}=\ip{X_{vv}}{N^{x}}=g
    \end{equation*}
    lo que demuestra lo pedido.

    \item
\end{enumerate}

\section*{Problema 2}

\section*{Problema 3}

\section*{Problema 4}

\section*{Problema 5}


%\printbibliography % Quitar el comentado si quiero usar bibliografia

\end{document}
