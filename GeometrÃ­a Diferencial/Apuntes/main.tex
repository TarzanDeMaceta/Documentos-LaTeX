\documentclass{article}
\usepackage{hyperref}
\usepackage{Style}

\nocite{*} % Comentar si quiero citar
%\addbibresource{bibliografia.bib} % Quitar el comentado si quiero usar bibliografia

\begin{document}

\begin{minipage}{2.5cm}
    \includegraphics[width=2cm]{imagen_puc.jpg}
\end{minipage}
\begin{minipage}{14cm}
    {\sc Pontificia Universidad Católica de Chile\\
    Facultad de Matemáticas\\
    Departamento de Matemática\\
    Profesor: Pedro Gaspar -- Estudiante: Benjamín Mateluna}
\end{minipage}
\vspace{1ex}

{\centerline{\bf Geometría Diferencial - MAT2860}
\centerline{\bf Apuntes}}
\centerline{\bf 06 de Marzo de 2025}

\newpage
\tableofcontents

\newpage
\section*{Introducción}
\addcontentsline{toc}{section}{Introducción}

Habrán tres interrogaciones (I1, I2, I3) cada una vale un $25\%$ y un examen (EX) que vale un 
$25\%$. Las fechas son 14 de abril, 19 de Mayo, 16 de Junio y 3 de Julio respectivamente.

\newpage
\section{Curvas en \texorpdfstring{$\R^{n}$}{}}
\subsection{Curvas parametrizadas}
\noindent Consideramos $\R^{n}:=\{v=(v_{1},\cdots,v_{n}):v_{i}\in\R\}$. Un espacio vectorial sobre 
$\R$ de dimensión n, con el producto escalar dado por
\begin{equation*}
    \ip{v}{w}=\ds\sum_{i=1}^{n}v_{i}w_{i}\hspace{4mm}\text{con }v,w\in\R^{n}
\end{equation*}

\begin{dfn}
    Una curva parametrizada en $\R^{n}$ es una función continua $\alpha:I\subseteq\R\to\R^{n}$ con 
    $I$ un intervalo abierto. Escribimos $\alpha(t)=(\alpha_{1}(t),\cdots,\alpha_{n}(t))$.
\end{dfn}

\noindent Diremos que $\alpha$ es diferenciable si sus funciones coordenadas 
$\alpha_{i}\in\mathcal{C}^{\infty}$. En tal caso, el vector 
$\alpha'(t)=(\alpha_{1}'(t),\cdots,\alpha_{n}'(t))$ se llama vector tangente a la curva $\alpha$ en
$t\in I$

\begin{dfn}
    La traza de una curva parametrizada $\alpha:I\subseteq\R\to\R^{n}$ es $\alpha(I)=im(\alpha)$.
\end{dfn}

\noindent\textbf{Ejemplos}
\begin{enumerate}
    \item Si $p,v\in\R^{n}$ con $v\neq0$, la curva parametrizada $\alpha(t)=tv+p$ con $t\in\R$ 
    que describe una recta que pasa por $p=\alpha(0)$ con vector tangente $\alpha'(t)=v$.
    \item Sea $\beta:\R\to\R^{3}$ dada por $b(t):=t^{3}\cdot\overrightarrow{e}_{1}$ es una curva
    parametrizada diferenciable con $\beta'(t)=3t^{2}\cdot\overrightarrow{e}_{1}$.
    \item Sea $p\in\R^{2}$ y $r>0$ consideramos $\alpha(t)=(rcos(t),rsen(t))+p$, una curva
    parametrizada diferenciable cuya traza es $\alpha(\R)=\{(x,y)\in\R^{2}:\abs{(x,y)}{p}=r\}$
    \item Sean $a,b\in\R\setminus{\{0\}}$. La curva parametrizada $\alpha:\R\to\R^{3}$ dada por
    $\alpha(t)=(acos(t),asen(t),bt)$ con $t\in\R$ se llama una helice circular. 
    Además $\alpha'(t)=(-asen(t),acos(t),b)$.
    \item Sea $\alpha:\R\to\R^{2}$ dada por $\alpha(t)=(t^{3}-4t,t^{2}-4)$ es una curva 
    parametrizada diferenciable con $\alpha(-2)=\alpha(2)=0$, pero $\alpha'(-2)\neq\alpha(2)$.
\end{enumerate}

\subsection{Longitud y Parametro de Arco}
\noindent Sea $\alpha:I\subseteq\R\to\R^{n}$ una curva parametrizada, consideremos $[a,b]\subseteq I$.
Buscamos medir la longitud de $\alpha([a,b])$. Una estrategia, dada una partición $P:=\{a=t_{0}
<t_{1}<\cdots<t_{n}=b\}$ de $[a,b]$ calculamos

\begin{equation*}
    \ds\sum_{i=1}^{k}\abs{\alpha(t_{i})-\alpha(t_{i-1})}=:L_{a}^{b}(\alpha,P)
\end{equation*}

\noindent esta suma corresponde a la longitud de una curva poligonal que pasa por los puntos 
$\alpha(t_{i})$. Si $Q\supseteq P$ es otra partición de $[s,b]$, entonces 
$L_{a}^{b}(\alpha,Q)\geq L_{a}^{b}(\alpha,P)$.

\begin{dfn}
    La longitud de una curva parametrizada $\alpha$ sobre $[a,b]\subseteq I$ es
    \begin{equation*}
        L_{a}^{b}(\alpha)=sup\{L_{a}^{b}(\alpha,P):P\text{ es partición de }[a,b]\}.
    \end{equation*}
\end{dfn}

\noindent Si $\alpha$ es diferenciable sobre $[a,b]$ y hacemos $\abs{P}=max\{t_{i}-t_{i-1}\}$ muy
pequeña, esperariamos que $\abs{\alpha(t_{i})-\alpha(t_{i-1})}\approx
\abs{\alpha'(\overline{t_{i}})}(t_{i}-t_{i-1})$.

\begin{prop}
    Si $\alpha:I\subseteq\R\to\R^{n}$ es una curva parametrizada diferenciable sobre $[a,b]
    \subseteq I$, entonces 
    \begin{equation*}
        L_{a}^{b}(\alpha)=\int_{a}^{b}\abs{\alpha'(t)}dt
    \end{equation*}
    (Para la demostración revisar Montiel-Ros, página 5)
\end{prop}

\begin{cor}
    Tenemos que $\abs{\alpha(a)-\alpha(b)}\leq L_{a}^{b}(\alpha)$.
\end{cor}

\begin{cor}
    Si $F:\R^{n}\to\R^{n}$ cumple $\abs{DF(p)v}=\abs{v}$ para todo $p,v\in\R^{n}$, entonces
    $L_{a}^{b}(F\circ\alpha)=L_{a}^{b}(\alpha)$.
\end{cor}

\noindent De hecho, $F\circ\alpha: I\to\R^{n}$ es una curva parametrizada diferenciable, con

\begin{equation*}
    \abs{(F\circ\alpha)'(t)}=\abs{DF(\alpha(t))\alpha'(t)}=\abs{\alpha'(t)}
\end{equation*}

\noindent para todo $t\in I$, basta con integrar sobre $[a,b]$. Si $p_{0}\in\R^{n}$ y 
$A:\R^{n}\to\R^{n}$ es una transformación lineal ortogonal , esto es, $\ip{Au}{Av}=\ip{u}{v}$ 
para todo $u,v\in\R^{n}$, entonces $F:\R^{n}\to\R^{n}$ dada por $F(p)=Ap+p_{0}$ cumple

\begin{equation*}
    DF(p)v=\dv{}{t}F(p+tv)\big|_{t=0}=\dv{}{t}(A(p+tv)+p_{0})\big|_{t=0}=Av
\end{equation*}

\noindent Por lo tanto $\abs{DF(p)v}=\abs{Av}=\abs{v}$.

\begin{cor}
    Si $h:J\subseteq\R\to I\subseteq\R$ es un difeomorfismo y $\alpha: I\to\R$ es una curva 
    parametrizada diferenciable, entonces
    \begin{equation*}
        L_{a}^{b}(\alpha\circ h)=L_{c}^{d}(\alpha)
    \end{equation*}
    donde $h([a,b])=[c,d]$ para todo $[a,b]\subseteq J$.
\end{cor}

\noindent Por regla de la cadena tenemos que $(\alpha\circ h')(t)=h'(t)\alpha'(h(t))$. La curva 
$\alpha\circ h$ tiene la misma traza que $\alpha$, en efecto $(\alpha\circ h)(J)=\alpha(h(J))=
\alpha(I)$. Decimos que $\alpha\circ h$ es una reparametrización de la curva $alpha$.

\begin{dem}
    Como $h$ y $h^{-1}$ son diferenciables, se tiene que $h'(t)\neq0$ para todo $t\in J$. Veamos
    que
    \begin{equation*}
        1=\dv{}{t}(t)=(h^{-1}\circ h)'(t)=(h^{-1})'(h(t))h'(t)
    \end{equation*}
    Luego como J es un intervalo y $h'$ es continua, tenemos que $h'<0$ o $h>0$.
    \begin{itemize}
        
        \item Si $h'<0$, entonces $h(a)=c$, $h(b)=d$,
        \begin{equation*}
            \int_{a}^{b}\abs{(\alpha\circ h)'(t)}dt=\int_{a}^{b}\abs{\alpha'(h(t))}\abs{h'(t)}dt=
            \int_{c}^{d}\abs{\alpha'(s)}ds=L_{c}^{d}(\alpha)
        \end{equation*}

        \item Si $h'>0$, entonces $h(b)=c$, $h(a)=d$,
        \begin{equation*}
            \int_{a}^{b}\abs{(\alpha\circ h)'(t)}dt=\int_{a}^{b}\abs{\alpha'(h(t))}\abs{h'(t)}dt=
            \int_{d}^{c}-\abs{\alpha'(s)}ds=\int_{c}^{d}\abs{\alpha'(s)}ds=L_{c}^{d}(\alpha)
        \end{equation*}

    \end{itemize}
\end{dem}

\begin{dfn}
    Se dice que una curva parametrizada diferenciable $\alpha: I\subseteq\R\to\R^{n}$ es regular si
    $\alpha'(t)\neq0$ para todo $t\in I$. Si además $\abs{\alpha'(t)}=1$ para todo $t\in I$ se dice
    que $\alpha$ esta parametrizada por el arco.
\end{dfn}

\noindent Una curva $\alpha$ parametrizada por el arco tiene las siguientes propiedades

\begin{itemize}
    \item $\alpha'(t)$ es ortogonal a $\alpha''(t)$ para todo $t\in I$, en efecto
    
    \begin{equation*}
        0=\dv{}{t}(\abs{\alpha'(t)}^{2})=\dv{}{t}(\ip{\alpha'(t)}{\alpha'(t)})=2\ip{\alpha'(t)}
        {\alpha''(t)}
    \end{equation*}

    \item Se tiene que $L_{a}^{b}(\alpha)=\int_{a}^{b}\abs{\alpha'(t)}dt=b-a$.
\end{itemize}

\begin{teo}
    Si $\alpha: I\to\R^{n}$ es una curva parametrizada diferenciable regular, entonces $\alpha$
    admite una parametrización por arco. Concretamente , si $t_{0}\in I$ y definimos $s: I\to\R$ 
    por
    \begin{equation*}
        s(t):=\int_{t_{0}}^{t}\abs{\alpha'(t)}dt
    \end{equation*}
    entonces $s$ es un difeomorfismo sobre $J\subseteq\R$ y $\alpha\circ s^{-1}:J\to\R^{n}$ esta
    parametrizada por el arco.
\end{teo}

\begin{dem}
    Por TFC, sabemos que $s$ es diferenciable, mas aun, $s'(t)=\abs{\alpha'(t)}$ para todo 
    $t\in I$. Luego, $s'>0$, es decir, $s$ es creciente y $s(I)=J$ es un intervalo abierto. Además,
    por teorema de la función inversa, vemos que

    \begin{equation*}
        (s^{-1})'(r)=\frac{1}{s'(s^{-1}(r))}=\frac{1}{\abs{\alpha'(s^{-1}(r))}}\hspace{4mm}
        \forall r\in J
    \end{equation*}

    \noindent Por lo tanto $\abs{(\alpha\circ s^{-1})'(r)}=1$ para todo $r\in J$, luego 
    $\alpha\circ s^{-1}$ esta parametrizada por el arco.
\end{dem}

\noindent\textbf{Ejemplos}
\begin{enumerate}
    \item Sea $\alpha(t)=tv+p_{0}$ con $p_{0},v\in\R^{n}$ y $v\neq0$. Como $\alpha'(t)=v$, tenemos
    \begin{equation*}
        s(t)=\int_{0}^{t}\abs{v}dx=t\abs{v}
    \end{equation*}
    entonces $\alpha\circ s^{-1}(x)=x\cdot\frac{v}{\abs{v}}+p_{0}$ es una parametrización por el
    arco de $\alpha$.

    \item Consideremos $\alpha(t)=(rcost,rsent)+p_{0}$ con $p_{0}\in\R^{2}$ y $r>0$. Como 
    $\alpha'(t)=(-rsent,rcost)$ entonces $\abs{\alpha'(t)}=r$, tenemos que
    \begin{equation*}
        s(t)=\int_{0}^{t}rdx=rt
    \end{equation*}
    y $(\alpha\circ s^{-1})(x)=(rcos(\frac{x}{r}),rsen(\frac{x}{r}))+p_{0}$ es una curva 
    parametrizada por el arco para $\alpha$.

    \item Definimos $\alpha(t)=(acost,asent,bt)$ con $a,b\in\R\setminus\{0\}$. Como $\abs
    {\alpha'(t)}=\sqrt{a^{2}+b^{2}}$ una curva parametrizada por el arco es
    \begin{equation*}
        (\alpha\circ s^{-1})(x)=\left(acos\left(\frac{x}{\sqrt{a^{2}+b^{2}}}\right),
        asen\left(\frac{x}{\sqrt{a^{2}+b^{2}}}\right),\frac{bt}{\sqrt{a^{2}+b^{2}}}\right)
    \end{equation*}
\end{enumerate}

\subsection{Curvatura de una Curva Regular (Teoría Local de Curvas)}

\noindent\textbf{Notación:} Notamos por $\mathcal{J}$ a la función $\mathcal{J}:\R^{2}\to\R^{2}$
dada por $\mathcal{J}(x,y)=(-y,x)$ que cumple lo siguientes

\begin{itemize}
    \item $\mathcal{J}$ es una transformación lineal ortogonal.
    \item $\ip{u,\mathcal{J}u}=0$ y $\mathcal{J}(\mathcal{J}u)=-u$ para todo $u\in\R^{2}$.
    \item Si $\abs{u}=1$, entonces $\{u,\mathcal{J}u\}$ es una base ortonormal positiva de 
    $\R^{2}$.
    \item Si $A:\R^{2}\to\R^{2}$ es una transformación lineal ortogonal, entonces 
    $\mathcal{J}A=det(A)A\mathcal{J}$.
\end{itemize}

\noindent Nuestro objetivo es asociar a una curva parametrizada regular $\alpha: I\to\R^{n}$ una 
cantidad geometrica, para ello queremos definir una función $K(=K_{\alpha}): I\to\R$ tal que
\begin{enumerate}
    \item $K$ es invariante bajo movimientos rigidos.
    \item $K$ es invariante por parametrizaciones.
    \item $K\equiv0$ si y solo si $\alpha$ corresponde a un segmento de recta.
\end{enumerate}

\noindent Si tenemos $\alpha: I\to\R^{2}$ una curva parametrizada por el arco, definimos la función 
$T: I\to\R^{2}$ dada por $T(s):=\alpha'(s)$ y $N: I\to\R^{2}$ como $N(s):=\mathcal{J}T(s)$.
Recordemos que $\{T(s),N(s)\}$ es una base ortonormal en $\R^{n}$ para cada $s\in I$ 
(Diedro de Frenet).
\vspace{4mm}

\noindent Notemos que $N(s)\perp T(s)$ y $T'(s)\perp T(s)$, luego, existe un $k(s)\in\R$ tal que 
$T'(s)=K(s)N(s)$. La función $K_{\alpha}=K: I\to\R$ se llama la curva de $\alpha$. Tomando el 
producto con $N(s)$,
\begin{equation*}
    K(s)=\ip{K(s)N(s)}{N(s)}
\end{equation*}
Por lo tanto $K(s)=\ip{T'(s)}{N(s)}$. Por otro lado, observemos que
\begin{equation*}
    N'(s)=\dv{}{s}\left(\J T(s)\right)=\J\dv{}{s}(T(s))=\J(K(s)N(s))=\J(K(s)\J T(s))=-K(s)T(s)
\end{equation*}

\begin{prop}
    Para una curva parametrizada por el arco $\alpha: I\to\R^{2}$ vale que $T'=KN$ y $N'=-KT$.
\end{prop}

\noindent\textbf{Ejemplos:}
\begin{enumerate}
    \item Una recta parametrizada por el arco $\alpha(s):=s\cdot \frac{v}{\abs{v}}+p_{0}$ con
    $v\in\R^{2}\setminus\{0\}$, tenemos que
    \begin{equation*}
        T(s)=\frac{v}{\abs{v}}\text{ , }N(s)=\frac{\J v}{\abs{v}}=\frac{\J v}{\abs{\J v}}
        \text{ y }K(s)=0\hspace{4mm}\forall s\in\R
    \end{equation*}

    \item Si $\alpha: I\to\R^{2}$ esta parametrizada y $K\equiv0$, entonces $T'(s)=0$ para todo
    $s\in I$, es decir, $\alpha''(s)=0$ para todo $s\in I$. Integrando dos veces concluimos que
    cada coordenada de $\alpha$ es una función lineal, luego $\alpha$ es un segmento de recta.

    \item Sea 
    $\alpha(s):=\left(rcos\left(\frac{s}{r}\right),rsen\left(\frac{s}{r}\right)\right)+p_{0}$,
    entonces
    \begin{equation*}
        T(s)=\left(-sen\left(\frac{s}{r}\right),cos\left(\frac{s}{r}\right)\right)\text{ y }
        N(s)=\left(-cos\left(\frac{s}{r}\right),-sen\left(\frac{s}{r}\right)\right)
    \end{equation*}
    Notemos que
    \begin{equation*}
        T'(s)=\left(-\frac{1}{r}cos\left(\frac{s}{r}\right),-\frac{1}{r}sen\left(
        \frac{s}{r}\right)\right)=\frac{1}{r}N(s)
    \end{equation*}
    Por lo tanto $K(s)=\frac{1}{r}\ip{N(s)}{N(s)}=\frac{1}{r}$.
\end{enumerate}

\noindent Consideremos ahora una curva regular $\beta:\widetilde{I}\to\R^{2}$ y una 
reparametrización $\alpha=\beta\circ h: I\to\R^{2}$ parametrizada por el arco, donde 
$h: I\to\widetilde{I}$ es un difeomorfismo con $h'>0$. Con esto
\begin{equation*}
    \abs{\beta'(t)}=\abs{(\beta\circ h\circ h^{-1})'(t)}=\abs{(\alpha\circ h^{-1})'(t)}=
    (h^{-1})'(t)
\end{equation*}
Así, definimos el diedro de Frenet de la curva $\alpha$ por
\begin{equation*}
    T_{\beta}(t):=\frac{\beta'(t)}{\abs{\beta'(t)}}=
    \frac{(\alpha\circ h^{-1})'(t)}{\abs{(\alpha\circ h^{-1})'(t)}}=
    \frac{\alpha'(h^{-1}(t))h^{-1}(t)}{\abs{\alpha'(h^{-1}(t))h^{-1}(t)}}=
    T_{\alpha}(h^{-1}(t))
\end{equation*}
Por otro lado
\begin{equation*}
    N_{\beta}=\J T_{\beta}(t)=\J T_{\alpha}(h^{-1}(t))=N_{\alpha}(h^{-1}(t))
\end{equation*}
y definimos la curvatura de la curva $\beta$ por
\begin{equation*}
    K_{\beta}(t):=K_{\alpha}(h^{-1}(t))\text{ , }t\in\widetilde{I}
\end{equation*}
Como $\beta'(t)=\abs{\beta'(t)}T_{\alpha}(h^{-1}(t))$ se tiene que
\begin{equation*}
    \beta''=(\abs{\beta'})'T_{\alpha}\circ h^{-1}+\abs{\beta'}^{2}(T'_{\alpha}\circ h^{-1})
\end{equation*}
y además $N_{\alpha}\circ h^{-1}=\J T_{\beta}=\frac{\J\beta'}{\abs{\beta'}}$ se sigue que
\begin{equation*}
    \frac{\ip{\beta''}{\J\beta'}}{\abs{\beta'}}=
    \ip{(\abs{\beta'})'T_{\alpha}\circ h^{-1}+\abs{\beta'}^{2}(T'_{\alpha}\circ h^{-1})}
    {N_{\alpha}\circ h^{-1}}=\abs{\beta'}^{2}\ip{T'_{\alpha}\circ h^{-1}}{N_{\alpha}\circ h^{-1}}
    =\abs{\beta'}^{2}K_{\alpha}\circ h^{-1}
\end{equation*}
Concluimos que $K_{\beta}=\frac{\ip{\beta''}{\J\beta'}}{\abs{\beta'}^{3}}$.

\begin{prop}
    Sea $\alpha: I\to\R^{2}$ una curva regular, entonces
    \begin{enumerate}
        \item Si $\phi:\widetilde{I}\to I$ es un difeomorfismo entonces $K_{\alpha\circ\phi}=
        sgn(\phi')K_{\alpha}\circ\phi$.
        
        \item Si $F:\R^{2}\to\R^{2}$ es un movimiento rigido, entonces $K_{F\circ\alpha}=
        (detDF)K_{\alpha}$.
    \end{enumerate}
\end{prop}

\begin{dem} Sea $\alpha: I\to\R^{2}$ una curva regular
    \begin{enumerate}
        \item Como $(\alpha\circ\phi)'(t)=\phi'(t)\alpha'(\phi(t))$, se sigue que 
        $\abs{(\alpha\circ\phi)'(t)}=\abs{\phi'(t)}\abs{\alpha'(\phi(t))}$, escrito de otro modo
        
        \begin{equation*}
            \abs{(\alpha\circ\phi)}=sgn(\phi')\cdot\phi'\abs{\alpha'\circ\phi}
        \end{equation*}
        Luego
        \begin{align*}
            K_{\alpha\circ\phi} &= \frac{\ip{(\alpha\circ\phi)''}{\J(\alpha\circ\phi)'}}
            {\abs{(\alpha\circ\phi)'}^{3}}=
            \frac{\ip{\phi''(\alpha'\circ\phi)+(\phi')^{2}\alpha''\circ\phi}
            {\phi'\J(\alpha'\circ\phi)}}{sgn(\phi')(\phi')^{3}\abs{\alpha'\circ\phi}^{3}} \\
            &= \frac{(\phi')^{3}\ip{\alpha''\circ\phi}{\J\alpha'\circ\phi}}
            {(\phi')^{3}\abs{\alpha'\circ\phi}^{3}}sgn(\phi')=sgn(\phi')K_{\alpha}\circ\phi
        \end{align*}
        
        \item Sabemos que $F(p)=Ap+p_{0}$, entonces $DF=A$. Luego,
        \begin{align*}
            \ip{(F\circ\alpha)''}{\J(F\circ\alpha)'} &= \ip{(DF(\alpha)\alpha')'}
            {\J(DF(\alpha)\alpha')}=\ip{(A\alpha')'}{\J(A\alpha')} \\
            &= \ip{A\alpha''}{(detA)A\J\alpha'}=detA\ip{\alpha''}{\J\alpha'}
        \end{align*}
        Además $\abs{(F\circ\alpha)'}=\abs{A\alpha'}=\abs{\alpha'}$. Juntando lo anterior vemos que
        \begin{equation*}
            K_{F\circ\alpha}=\frac{\ip{(F\circ\alpha)''}{\J(F\circ\alpha)'}}
            {\abs{(F\circ\alpha)'}^{3}} = detA\cdot K_{\alpha}
        \end{equation*}
    \end{enumerate}
\end{dem}

\begin{prop}
    Sea $\alpha: I\to\R^{2}$ una curva parametrizada por el arco. Supongamos que existe una función
    diferenciable $\theta: I\to\R$ tal que $T(s)=(cos(\theta(s)),sen(\theta(s)))$. Entonces 
    $K_{\alpha}=\dv{\theta}{s}$.
\end{prop}

\begin{dem}
    Recordemos que
    \begin{equation*}
        K_{\alpha}=\ip{T'_{\alpha}}{\J T_{\alpha}}=
        \ip{\left(-\dv{\theta}{s}sen\theta,\dv{\theta}{s}cos\theta\right)}
        {(-sen\theta,cos\theta)}=\dv{\theta}{s}\abs{(-sen\theta,cos\theta)}^{2}=\dv{\theta}{s}
    \end{equation*}
\end{dem}

\begin{teo}
    Sea $K: I\to\R$ una función diferenciable, entonces existe una unica curva parametrizada por 
    el arco $\alpha: I\to\R$, salvo por movimientos rigidos, tal que $K_{\alpha}=K$.
\end{teo}

\subsection{Teoría Local de Curvas en el Espacio}

\begin{dfn}
    Sea $\alpha: I\to\R^{3}$ parametrizada por el arco. La curvatura de $\alpha$ en $s\in I$ es
    \begin{equation*}
        K_{\alpha}:=\abs{T'(s)}
    \end{equation*}
\end{dfn}

\noindent\textbf{Observación:} Para curvas en $\R^{3}$, $K_{\alpha}\geq0$. Además, 
$K_{\alpha}\equiv0$ si y solo si $\alpha$ es un segmento de recta.

\begin{dfn}
    Sea $\alpha: I\to\R^{3}$ una curva parametrizada por el arco, tal que $K_{\alpha}>0$. Definimos
    \begin{equation*}
        N(s):=\frac{T'(s)}{\abs{T'(s)}}
    \end{equation*}
\end{dfn}

\noindent\textbf{Observación:} Como $T(s)\perp T'(s)$, pues $\abs{T}=1$, está definición se condice
con el caso en $\R^{2}$, además de manera directa, obtenemos que $K_{\alpha}N(s)=T(s)$.

\begin{dfn}
    Sea $\alpha: I\to\R^{3}$ parametrizada por el arco. Definimos el vector binormal de $\alpha$
    en $s\in I$ por
    \begin{equation*}
        B(s)=T(s)\times N(s)
    \end{equation*}
\end{dfn}

\noindent\textbf{Observación:} Por definición del producto cruz el conjunto $\{T,N,B\}$ es una base
ortonormal positiva de $\R^{3}$ para todo $s\in I$ llamada el tiedro de Frenet de $\alpha$ en 
$s\in I$.
\vspace{4mm}

\noindent Notemos que $B'(s)=\dv{}{s}(T(s)\times N(s))=T'(s)\times N(s)+T(s)\times N'(s)=
T(s)\times N'(s)$. Además, $\abs{B}=\abs{T}\abs{N}=1$ y por lo tanto $B'\perp B$, por otro lado
$\ip{B'}{T}=\ip{T\times N'}{T}=0$, osea $B'\perp T$. Por lo tanto, existe $\tau(s)\in I$ tal quiero
\begin{equation*}
    B'(s)=\tau(s)N(s)
\end{equation*}
Se dice que $\tau(s)=:\tau_{\alpha}(s)$ es la torsión de $\alpha$ en $s\in I$. Finalmente, como
$N'\perp N$, tenemos que
\begin{equation*}
    N'(s)=aT(s)+bB(s)
\end{equation*}
donde

\begin{align*}
    a\ip{T}{T} &= \ip{N'}{T}=\ip{N'}{T}+\ip{N}{T'}-\ip{N}{T'} \\
    &= \dv{}{s}\ip{N}{T}-\ip{N}{T'}=-\ip{N}{KN}=-K
\end{align*}
y similarmente obtenemos que $b=\ip{N'}{B}=-\tau(s)$.

\begin{prop}
    Ecuaciones de Frenet-Serret
    \begin{itemize}
        \item $T'(s)=K(s)N(s)$
        \item $N'(s)=-K(s)T(s)-\tau(s) B(s)$
        \item $B'(s)=\tau(s) N(s)$
    \end{itemize}
\end{prop}

\noindent\textbf{Ejemplos:}
\begin{enumerate}
    \item Sea $\alpha: I\to\R^{3}$ una curva parametrizada por el arco. Supongamos que 
    $\alpha(I)\subseteq P$ con $P$ un plano. Podemos describir el plano con la ecuación
    $\ip{x-p_{0}}{u}=0$, donde $p_{0},u\in\R^{3}$ con $u$ unitario y perpendicular al plano. 
    Entonces $\ip{\alpha(s)-p_{0}}{u}=0$ para todo $s\in I$, derivando vemos que
    \begin{equation*}
        \ip{\alpha'(s)}{u}=\ip{T(s)}{u}=0\hspace{4mm}\forall s\in I
    \end{equation*}
    En ese caso, $K(s)$ es el valor absoluto de la curvatura de $\alpha$ como una curva plana.
    Supongamos que $K(s)>0$ para todo $s\in I$. Entonces
    \begin{equation*}
        0=\dv{}{s}=\ip{T}{u}=\ip{T'}{u}=K(s)\ip{N}{u}
    \end{equation*}
    lo que implica que $N\perp u$ para todo $s\in I$. Luego, $B(s)=\pm u$ para todo $s\in I$, se
    sigue que $\tau(s)=\ip{B'}{N}=0$.

    \item Supongamos que $\alpha$ es una curva parametrizada por el arco tal que 
    $\tau_{\alpha}\equiv0$, entonces $B'=\tau\cdot N=0$ para todo $s\in I$ y por lo tanto $B=u$,
    con $u\in\R^{3}$ y $\abs{u}=1$, así $T\times N=u$ para todo $s\in I$.
    \vspace{4mm}

    \noindent Ahora, usando las ecuaciones de frenet vemos que $T\perp u$ y $N\perp u$ para todo
    $s\in I$ y concluimos que
    \begin{equation*}
        \ip{\alpha(s)-\alpha(s_{0})}{u}=\ip{\int_{s_{0}}^{s}T(x)dx}{u}=
        \int_{s_{0}}^{s}\ip{T(x)}{u}dx=0\hspace{4mm}\forall s\in I
    \end{equation*}
\end{enumerate}

\begin{prop}
    Sea $\alpha: I\to\R^{3}$ una curva parametrizada por el arco, $p_{0}\in\R^{3}$, 
    $A:\R^{3}\to\R^{3}$ lineal, ortogonal y positiva. Sea $F:\R^{3}\to\R^{3}$ con 
    $F(p)=Ap+p_{0}$. Entonces
    \begin{align*}
        & K_{F\circ\alpha}=K_{\alpha}\hspace{4mm}\text{,}\hspace{4mm}\tau_{F\circ\alpha}=
        \tau_{\alpha} \\
        & T_{F\circ\alpha}=AT_{\alpha}\hspace{4mm}\text{,}\hspace{4mm}
        N_{F\circ\alpha}=AN_{\alpha}\hspace{4mm}\text{,}\hspace{4mm}
        B_{F\circ\alpha}=AB_{\alpha}
    \end{align*}
\end{prop}

\noindent Podemos extender las definiciones de curvatura, torsión y del tiedro de frenet para
curvas regulares $\beta: I\to\R^{3}$ por
\begin{equation*}
    K_{\beta}(t):=K_{\alpha}(h^{-1}(t))
\end{equation*}
donde $\alpha=\beta\circ h$ es una parametrización por el arco, con $h$ difeomorfismo, $h'>0$ y
$K_{\beta}>0$. Se cumple lo siguiente
\begin{itemize}
    \item $T_{\beta}(t)=T_{\alpha}(h^{-1}(t))$
    \item $N_{\beta}(t)=N_{\alpha}(h^{-1}(t))$
    \item $B_{\beta}(t)=B_{\alpha}(h^{-1}(t))$
    \item $\tau_{\beta}(t)=\tau_{\alpha}(h^{-1}(t))$
\end{itemize}

\begin{prop}
    Sea $\beta: I\to\R^{3}$ una curva regular, entonces
    \begin{enumerate}
        \item $K_{\beta}=\dfrac{\abs{\beta'\times\beta''}}{\abs{\beta'}^{3}}$
        \item $\tau_{\beta}=\dfrac{-det(\beta',\beta'',\beta''')}{\abs{\beta'\times\beta''}^{2}}=
        -\dfrac{\ip{\beta'}{\beta''\times\beta'''}}{\abs{\beta'\times\beta''}^{2}}$
        \item $T_{\beta}=\dfrac{\beta'}{\abs{\beta'}}$
        \item $B_{\beta}=\dfrac{\beta'\times\beta''}{\abs{\beta'\times\beta''}}$
        \item $N_{\beta}=\dfrac{\abs{\beta'}^{2}\beta''-\ip{\beta'}{\beta''}\beta'}
        {\abs{\abs{\beta'}^{2}\beta''-\ip{\beta'}{\beta''}\beta'}}$
    \end{enumerate}
\end{prop}

\begin{teo}
    (Teorema Fundamental de las curvas en el Espacio)
    
    Sea $K,\tau: I\subseteq\R\to\R$ funciones diferenciables con $K(s)>0$ para todo $s\in I$.
    Entonces existe $\alpha:I\to\R^{3}$ parametrizada por el arco tal que
    \begin{equation*}
        K_{\alpha}=K\hspace{4mm}\text{y}\hspace{4mm}\tau_{\alpha}=\tau
    \end{equation*}
    Además, si $\beta:I\to\R^{3}$ es parametrizada por el arco tal que $K_{\beta}=K$ y 
    $\tau_{\beta}=\tau$. Entonces existe un movimiento rigido $F:\R^{3}\to\R^{3}$ tal que
    $F\circ\beta=\alpha$.
\end{teo}

\begin{dem}
    El sistema
    \begin{equation*}
        (FS):\begin{pmatrix}
            T \\ N \\ B
        \end{pmatrix}'
        =A
        \begin{pmatrix}
            T \\ N \\ B
        \end{pmatrix}
        \hspace{4mm}\text{y}\hspace{4mm}A(s)=
        \begin{pmatrix}
            0 & K(s) & 0 \\
            -K(s) & 0 & -\tau(s) \\
            0 & \tau(s) & 0
        \end{pmatrix}
    \end{equation*}
    para cada $\{T_{0},N_{0},B_{0}\}\subseteq\R^{3}$ y $s_{0}\in I$, existe una única solución del
    sistema, $\{T, N, B\}$, definida en $I$ tal que $T(s_{0})=T_{0}$, $N(s_{0})=N_{0}$ y 
    $B(s_{0})=B_{0}$. Veamos que $\{T,N,B\}$ son ortonormales para cada  $s\in I$. Sea
    $\{T_{0},N_{0},B_{0}\}$ una base ortonormal positiva de $\R^{3}$. Consideremos la función
    \begin{equation*}
        M(s)=
        \begin{pmatrix}
            \ip{T}{T} & \ip{T}{N} & \ip{T}{B} \\
            \ip{N}{T} & \ip{N}{N} & \ip{N}{B} \\
            \ip{B}{T} & \ip{B}{N} & \ip{B}{B}
        \end{pmatrix}
        =
        \begin{pmatrix}
            T & N & B
        \end{pmatrix}^{T}\cdot
        \begin{pmatrix}
            T & N & B
        \end{pmatrix}
    \end{equation*}
    Por otro lado
    \begin{align*}
        M'(s) &=
        \begin{pmatrix}
            T & N & B
        \end{pmatrix}'^{T}\cdot
        \begin{pmatrix}
            T & N & B
        \end{pmatrix}+
        \begin{pmatrix}
            T & N & B
        \end{pmatrix}^{T}\cdot
        \begin{pmatrix}
            T & N & B
        \end{pmatrix}' \\
        &= A
        \begin{pmatrix}
            T & N & B
        \end{pmatrix}^{T}\cdot
        \begin{pmatrix}
            T & N & B
        \end{pmatrix}+
        \begin{pmatrix}
            T & N & B
        \end{pmatrix}^{T}
        \begin{pmatrix}
            T & N & B
        \end{pmatrix}A^{T} \\
        &= AM-MA
    \end{align*}
    La matriz $M_{0}(s)=I_{3}$ con $s\in I$ es solución del sistema, además
    $M_{0}(s_{0})=I_{3}=M(s_{0})$ (pues $T_{0},N_{0},B_{0}$ son ortonormales). Por unicidad de la 
    solución $M(s)\equiv I_{3}$ para todo $s\in I$.
    \vspace{4mm}

    \noindent La matriz $(T\hspace{1mm}N\hspace{1mm}B)$ tiene determinante $1$ o $-1$. Como $I$ es
    conexo y el determinante una función continua, entonces es constante. Como vale $1$ en 
    $s=s_{0}$ pues $\{T_{0},N_{0},B_{0}\}$ es base positiva, vale $1$ sobre $I$.
    \vspace{4mm}

    \noindent Definimos $\alpha:I\to\R^{3}$ por
    \begin{equation*}
        \alpha(x)=\int_{s_{0}}^{s}T(x)dx
    \end{equation*}
    Por TFC, $\alpha'(s)=T(s)$ unitario, luego $\alpha$ es una curva parametrizada por el arco.
    Además
    \begin{equation*}
        K_{\alpha}(s)=\abs{T'(s)}=\abs{K(s)N(s)}=K(s)\abs{N(s)}=K(s)\hspace{4mm}\forall s\in I
    \end{equation*}
    De ahí,
    \begin{equation*}
        N_{\alpha}(s)=\frac{T'_{\alpha}(s)}{\abs{T'_{\alpha}(s)}}=\frac{T'(s)}{\abs{T'(s)}}=
        \frac{K(s)N(s)}{\abs{K(s)N(s)}}=N(s)
    \end{equation*}
    y $B_{\alpha}(s)=T_{\alpha}(s)\times N_{\alpha}(s)=T(s)\times N(s)=B(s)$, ya que 
    $T(s),N(s),B(s)$ es base ortonormal positiva. Por tanto,
    \begin{equation*}
        \tau_{\alpha}=\ip{B'_{\alpha}(s)}{N_{\alpha}(s)}=\ip{B'(s)}{N(s)}=\ip{\tau N}{N}=\tau(s)
    \end{equation*}
    Sea $A:\R^{3}\to\R^{3}$ ortogonal tal que
    \begin{align*}
        & AT_{\beta}(s_{0})=T_{\alpha}(s_{0}) \\
        & AN_{\beta}(s_{0})=N_{\alpha}(s_{0}) \\
        & AB_{\beta}(s_{0})=B_{\alpha}(s_{0})
    \end{align*}
    y $p_{0}=\alpha(s_{0})-A\beta(s_{0})$. Luego, $F:\R^{3}\to\R^{3}$ con $F(p)=Ap+p_{0}$. Defina
    $\gamma=F\circ\beta:I\to\R^{3}$. Queremos ver que $\gamma\equiv\alpha$. Como $F$ es movimiento
    rigido $\alpha$ y $\gamma$ tienen curvatura $K$ y torsión $\tau$ y tiedro
    \begin{align*}
        & T_{\gamma}=T_{F\circ\beta}=AT_{\beta} \\
        & N_{\gamma}=N_{F\circ\beta}=AN_{\beta} \\
        & B_{\gamma}=B_{F\circ\beta}=AB_{\beta}
    \end{align*}
    Luego
    $f(s)=\abs{T_{\gamma}(s)-T_{\alpha}(s)}^{2}+
    \abs{N_{\gamma}(s)-N_{\alpha}(s)}^{2}+\abs{B_{\gamma}(s)-B_{\alpha}(s)}^{2}$
    vale $0$ en $s=s_{0}$. Por otro lado
    \begin{equation*}
        f'(s)=2\ip{T_{\gamma}-T_{\alpha}}{T'_{\gamma}-T'_{\alpha}}
        +2\ip{N_{\gamma}-N_{\alpha}}{N'_{\gamma}-N'_{\alpha}}
        +2\ip{B_{\gamma}-B_{\alpha}}{B'_{\gamma}-B'_{\alpha}}=0\hspace{4mm}\forall s\in I
    \end{equation*}
    por lo tanto $f$ es constante y por lo mencionado $f\equiv0$. De este modo, 
    $\gamma'=T_{\gamma}\equiv T_{\alpha}=\alpha'$. Como 
    \begin{equation*}
        \gamma(s_{0})=F(\beta{s_{0}})=A\beta(s_{0})+p_{0}=\alpha(s_{0})
    \end{equation*}
    concluimos que $\gamma\equiv\alpha$.
\end{dem}

\newpage

\section{Superficies Regulares}
\subsection{Definición y ejemplos}
\begin{dfn}
    Sea $\Sigma\subseteq\R^{3}$, decimos que $\Sigma$ es una superficie regular si para todo
    $p\in\Sigma$ existe un abierto $V\subseteq\R^{3}$ con $p\in V$ y una función diferenciable
    \begin{equation*}
        \varphi:\mathcal{V}\subseteq\R^{2}\to\R^{3}
    \end{equation*}
    tal que
    \begin{itemize}
        \item $\varphi(\mathcal{V})=V\cap\Sigma$
        \item $\varphi$ es homeomorfismo de $\mathcal{V}$ sobre $V\cap\Sigma$
        \item $D\varphi(q):\R^{3}\to\R^{3}$ es inyectiva, es decir, si $\varphi=\varphi(u,v)$, 
        entonces
        \begin{align*}
            & D\varphi(q)\cdot e_{1}=\dv{}{t}\varphi(q+te_{1})\big|_{t=0}=\varphi_{u}(q) \\
            & D\varphi(q)\cdot e_{2}=\dv{}{t}\varphi(q+te_{2})\big|_{t=0}=\varphi_{v}(q)
        \end{align*}
        son linealmente independientes, en otras palabras 
        $\varphi_{u}(q)\times\varphi_{v}(q)\neq0$. Decimos que $\varphi$ es una parametrización 
        local para $\Sigma$
    \end{itemize}
\end{dfn}

\noindent\textbf{Ejemplos: }
\begin{itemize}
    \item Sea $f:\V\subseteq\R^{2}\to\R$ diferenciable, consideramos $\Sigma:=
    \{(x,y,(f(x,y)))\in\R^{3}:(x,y)\in\V\}$ Tomamos $V=\R^{3}$, definimos la función $\varphi:
    \V\to\R^{3}$ dada por $\varphi(u,v)=(u,v,f(u,v))$, entonces
    \begin{enumerate}
        \item $\varphi(\V)=\Sigma=\Sigma\cap V$.
        \item $\varphi$ tiene inversa, a saber, $\varphi^{-1}(x,y,z)=(x,y)$ que es la restricción
        de una función continua, luego $\varphi^{-1}$ es continua.
        \item $\varphi_{u}(u,v)=\left(1,0,\pdv{f}{u}(u,v)\right)$ y $\varphi_{v}(u,v)=
        \left(0,1,\pdv{f}{v}(u,v)\right)$ son linealmente independientes.
    \end{enumerate}
    Por lo tanto, $\varphi$ es una parametrización local con $\varphi(\V)=\Sigma$

    \item Veamos la esfera unitaria $\mathbb{S}^{2}$. Si $(x,y,z)\in\mathbb{S}^{2}$, entonces 
    $x\neq0$ o $y\neq0$ o $z\neq0$. Consideramos
    \begin{equation*}
        \mathbb{S}^{2}=\mathbb{S}^{2}\cap\left(V_{1}^{+}\cup V_{2}^{+}\cup V_{3}^{+}
        \cup V_{1}^{-}\cup V_{2}^{-}\cup V_{3}^{-}\right)
    \end{equation*}
    donde $V_{i}^{\pm}:=\{(x,y,z)\in\R^{3}:\pm x_{i}>0\}$. Definimos la función 
    $\varphi_{1}^{\pm}:B_{1}(0)\subseteq\R^{2}\to\R^{3}$ dada por
    \begin{equation*}
        \varphi_{1}^{\pm}(u,v):=(\pm\sqrt{1-u^{2}-v^{2}},u,v)
    \end{equation*}
    luego, $\varphi_{1}^{\pm}(B_{1}(0))=V_{1}^{\pm}\cap\mathbb{S}^{2}$, 
    $(\varphi_{1}^{\pm})^{-1}:V_{1}^{\pm}\cap\mathbb{S}^{2}\to B_{1}(0)$ que manda $(x,y,z)$ en 
    $(y,z)$ es continua y además $(\varphi_{1}^{\pm})^{-1}_{u}(q)$ y 
    $(\varphi_{1}^{\pm})^{-1}_{v}(q)$ son linealmente independientes para todo $q\in B_{1}(0)$. Un
    argumento similar se utiliza para $V_{2}^{\pm}$ y $V_{3}^{\pm}$.
\end{itemize}

\begin{dfn}
    Una superficie parametrizada diferenciable es una aplicación  diferenciable $\varphi:\V
    \subseteq\R^{2}\to\R^{3}$ con $\V$ abierto. Se dice que $\varphi$ es regular si 
    $D\varphi(q):\R^{2}\to\R^{2}$ es inyectiva para todo $q\in\V$.
\end{dfn}

\noindent\textbf{Ejemplos:} 
\begin{itemize}
    \item Toda parametrización local de una superficie regular es una superficie parametrizada 
    regular.

    \item Sea $\alpha:I\subseteq\R\to\R^{2}$ una curva parametrizada diferenciable. Definimos
    $\varphi:I\times\R\to\R^{3}$ por $\varphi(u,v)=(\alpha(u),v)$. Esta superficie parametrizada 
    diferenciable se llama cilindro sobre $\alpha$.
    \vspace{4mm}

    \noindent Como $\varphi_{u}(u,v)=(\alpha'(u),0)$ y $\varphi_{v}(u,v)=(0,0,1)$ son linealmente
    independientes si y solo si $\alpha'\not\equiv0$, es decir, $\varphi$ es regular si y solo si
    $\alpha$ es una curva regular.

    \item Si $I=\R$ y existe $T\in\R$ tal que $\alpha(t+T)=\alpha(t)$ entonces 
    $\varphi(I\times\R)=\alpha(\R)\times\R$ es una superficie regular.

    \item Si $\alpha$ es inyectiva y para todo $t\in I$ existen abiertos $V\subseteq\R^{2}$ y 
    $J\subseteq I$ con $t\in J$ tales que $\alpha(I)\cap V=\alpha(J)$ entonces $\varphi(I\times\R)$
    es una superficie regular.
\end{itemize}

\begin{teo}
    (Teorema de la Función Implicita) Sea $h:W\subseteq\R^{3}\to\R$ continua diferenciable, 
    $p_{0}=(x_{0},y_{0},z_{0})\in W$ tal que $\pdv{h}{z}(p_{0})\neq0$. Entonces existen abiertos
    $\V\subseteq\R^{2}$, $I\subseteq\R$ y $f:\V\to I$ continua diferenciable tales que
    \begin{itemize}
        \item El punto $p_{0}\in\V\times I$
        \item Se tiene la igualdad de conjuntos $h^{-1}(h_{p_{0}})\cap(\V\times I)=
        \{(x,y,f(x,y)):(x,y)\in\V\}$
    \end{itemize}
    Además se tiene que
    \begin{itemize}
        \item
        \begin{equation*}
            \pdv{f}{x}(x,y)=\frac{-\pdv{h}{x}(x,y,f(x,y))}{-\pdv{h}{z}(x,y,f(x,y))}
        \end{equation*}
        \item
        \begin{equation*}
            \pdv{f}{y}(x,y)=\frac{-\pdv{h}{y}(x,y,f(x,y))}{-\pdv{h}{z}(x,y,f(x,y))}
        \end{equation*}
    \end{itemize}
    Si $h$ es suave entonces $f$ también lo es.
\end{teo}

\begin{dfn}
    Sea $F:W\subseteq\R^{n}\to\R^{m}$ diferenciable. Se dice que $q\in\R^{m}$ es un valor regular
    para $F$, si $F^{-1}(q)=\emptyset$ o si para todo $p\in F^{-1}(q)$ se tiene que 
    $DF(p):\R^{n}\to\R^{m}$ es sobreyectiva.
\end{dfn}

\noindent Por ejemplo si $h:W\subseteq\R^{n}\to\R$ entonces $Dh(p):\R^{n}\to\R$
\begin{align*}
    Dh(p)e_{i} &= \dv{}{t}h(p+te_{i})\big|_{t=0} \\
    &= \dv{}{t}h(p_{1},\cdots, p_{i-1},p_{i}+t,p_{i+1},\cdots,p_{n})\big|_{t=0} \\
    &= \pdv{h}{x_{i}}(p)
\end{align*}
Luego $q\in\R$ es valor regular par a $h$ si y solo si para todo $p\in h^{-1}(q)$, 
$\pdv{h}{x_{i}}(p)\neq0$ para algún $i$, o sea, $\nabla h(p)\neq0$ para todo $p\in h^{-1}(q)$.

\begin{teo}
    Sea $h:W\subseteq\R^{3}\to\R$ una función diferenciable. Si $c\in\R$ es un valor regular para
    $h$, entonces $h^{-1}(c)$ es una superficie regular.
\end{teo}
\begin{dem}
    Si $c$ es valor regular, entonces para todo $p\in h^{-1}(c)$ se sigue que $\nabla h(p)\neq0$,
    es decir,
    \begin{equation*}
        \pdv{h}{x}(p)\neq0\hspace{4mm}\text{ó}\hspace{4mm}
        \pdv{h}{y}(p)\neq0\hspace{4mm}\text{ó}\hspace{4mm}
        \pdv{h}{z}(p)\neq0
    \end{equation*}
    Supongamos sin perdida de generalidad que $\pdv{h}{z}\neq0$. Por teo de la función Implicita,
    existen abiertos $\V\subseteq\R^{2}$, $I\subseteq\R$ y una función suave $f:\V\to I$ tales que
    $p\in\V\times I$ y $h^{-1}(c)\cap(\V\times I)=Graf(f)$.
    \vspace{4mm}

    \noindent Por lo visto al inicio de la sección, existe parametrización local $\varphi:\V\to
    \R^{3}$ con $\varphi(\V)=h^{-1}(c)\cap(\V\times I)$.
\end{dem}

\subsection{Cambio de Coordenadas}
\begin{lema}
    Sea $\varphi:\V\subseteq\R^{2}\to\R^{3}$ superficie parametrizada regular
    \begin{equation*}
        \varphi(u,v)=(x(u,v),y(u,v),z(u,v))
    \end{equation*}
    Entonces para todo punto $(u_{0},v_{0})\in\V$ se tiene que $D(\pi\circ\varphi)(u_{0},v_{0}):
    \R^{2}\to\R^{2}$ es un isomorfismo lineal, donde $\pi:\R^{3}\to\R^{2}$ es una de las 
    proyecciones a los planos $xy$, $xz$ o $yz$.
    \vspace{4mm}

    \noindent Consecuentemente existe $\V_{0}\subseteq\V$ abierto con $(u_{0},v_{0})\in\V_{0}$ tal
    que $\pi\circ\varphi(\V_{0})=W_{0}\subseteq\R^{2}$ es abierto y $\pi\circ\varphi\big|_{\V_{0}}
    :\V_{0}\to\ W_{0}$ es un difeomorfismo.
\end{lema}

\begin{dem}
    La matriz $D\varphi(u_{0},v_{0}):\R^{2}\to\R^{3}$ es
    \begin{equation*}
        \begin{pmatrix}
            \pdv{x}{u} & \pdv{x}{v} \\[1mm]
            \pdv{y}{u} & \pdv{y}{v} \\[1mm]
            \pdv{z}{u} & \pdv{z}{v}
        \end{pmatrix}(u_{0},v_{0})
    \end{equation*}
    Como la superficie es regular, las columnas son linealmente independientes. Luego la matriz 
    tiene una submatriz $2\times2$ invertible. Pero estas submatrices son las matrices de
    \begin{equation*}
        D(\pi\circ\varphi)(u_{0},v_{0}):\R^{2}\to\R^{2}
    \end{equation*}
    La última parte es consecuencia directa del teorema de la función inversa.
\end{dem}

\noindent\textbf{Observación:} La función $\psi=\varphi\circ(\pi\circ\varphi)^{-1}:W_{0}\to\R^{3}$
es también una superficie parametrizada regular con
\begin{equation*}
    \psi(W_{0})=\varphi((\pi\circ\varphi)^{-1}(W_{0}))=\varphi(\V_{0})
\end{equation*}
Además, $\pi\circ\psi=id_{W_{0}}$, osea, $\psi$ es la grafica de una función $f:W_{0}\to\R$ 
diferenciable.

\begin{cor}
    Si $\Sigma\subseteq\R^{3}$ es una superficie regular, entonces para todo $p\in\Sigma$ existe 
    parametrización local cuya imagen contiene a $p$ y que es grafica.
\end{cor}

\begin{teo}
    Si $\varphi_{i}:\V_{i}\subseteq\R^{2}\to\Sigma$ son parametrizaciones locales de $\Sigma$ con
    $U:=\varphi_{1}(\V_{1})\cap \varphi_{2}(\V_{2})\neq\emptyset$. Entonces la aplicación
    \begin{equation*}
        \varphi_{2}^{-1}\circ \varphi_{1}:\varphi_{1}^{-1}(U)\subseteq\R^{2}\to 
        \varphi_{2}^{-1}(U)\subseteq\R^{2}
    \end{equation*}
    es un difeomorfismo. Se dice que $\varphi_{2}^{-1}\circ \varphi_{1}$ es un cambio de 
    coordenadas.
\end{teo}

\begin{dem}
    Como $\varphi_{i}$ son homeomorfismos, basta demostrar que $\varphi_{2}^{-1}\circ \varphi_{1}$ 
    es diferenciable en cada $p_{1}\in \varphi_{1}^{-1}(U)$. Sean
    \begin{equation*}
        q=\varphi_{1}(p_{1})\hspace{4mm}\text{y}\hspace{4mm}p_{2}=\varphi_{2}^{-1}(q)=\varphi_{2}^{-1}(\varphi_{1}(p_{1}))
    \end{equation*}
    Por el lema, existe proyección $\pi:\R^{3}\to\R^{2}$ y un abierto $V_{2}\subseteq 
    \varphi_{2}^{-1}(U)$ con $p_{2}\in V_{2}$ tal que
    \begin{equation*}
        \pi\circ \varphi_{2}:V_{2}\to\pi(\varphi_{2}(V_{2}))=:W\subseteq\R^{2}\hspace{2mm}
        \text{un abierto}
    \end{equation*}
    es un difeomorfismo. Sea $V_{1}:=(\varphi_{2}^{-1}\circ \varphi_{1})^{-1}(V_{2})=
    \varphi_{1}^{-1}(\varphi_{2}(V_{2}))$, entonces
    \begin{itemize}
        \item $p_{1}\in V_{1}$ pues $\varphi_{2}^{-1}\circ \varphi_{1}(p_{1})=\varphi_{2}^{-1}(q)
        =p_{2}\in V_{2}$.
        
        \item Si $p\in V_{1}$ entonces $\varphi_{1}(p)\in\varphi_{2}(V_{2})$ y por ende
        $\pi\circ\varphi_{1}(p)\in\pi\circ\varphi_{2}(V_{2})=W$
        
        \item $V_{1}$ es abierto
    \end{itemize}
    Por lo tanto esta bien definida la función $(\pi\circ\varphi_{2})^{-1}\circ\pi\circ\varphi_{1}:
    V_{1}\to\V_{2}$. La cual cumple que $(\pi\circ\varphi_{2})^{-1}\circ\pi\circ\varphi_{1}=
    \varphi_{2}^{-1}\circ\varphi_{1}$ en su dominio. Como $(\pi\circ\varphi_{2})^{-1}$ y
    $(\pi\circ\varphi_{1})$ son diferenciables, $\varphi_{2}^{-1}\circ\varphi_{1}$ es diferenciable 
    en $p_{1}\in V_{1}$.
\end{dem}

\subsection{Aplicaciones Diferenciables}
\begin{dfn}
    Se dice que $f:\Sigma\to\R^{d}$ es diferenciable en $p\in\Sigma$ si existe una parametrización
    local $\varphi:\V\subseteq\R^{2}\to\Sigma$ con $p\in\varphi(\V)$ y tal que $f\circ\varphi$ es 
    diferenciable en $\varphi^{-1}(p)\in\V$.
\end{dfn}

\begin{dfn}
    Se dice que
    \begin{equation*}
        \gamma:V\subseteq\R^{d}\to\Sigma\subseteq\R^{3}
    \end{equation*}
    con $\Sigma$ una superficie parametrizada regular, es diferenciable en $q\in V$. Si existe una
    parametrización local $\varphi:\V\subseteq\R^{2}\to\Sigma$ con $\gamma(q)\in\varphi(\V)$ tal 
    que
    \begin{equation*}
        \varphi^{-1}\circ\gamma:\gamma^{-1}(\varphi(\V))\subseteq\R^{d}\to\R^{2}
    \end{equation*}
    es diferenciable en $q\in\gamma^{-1}(\varphi(\V))$.
\end{dfn}

\noindent\textbf{Observación:}
\begin{enumerate}
    \item La definición de diferenciabilidad de $f:\Sigma\to\R^{d}$ no depende de la 
    parametrización
    \begin{equation*}
        f\circ\widetilde{\varphi}=(f\circ\varphi)\circ(\varphi^{-1}\circ\widetilde{\varphi})
    \end{equation*}
    entonces $f\circ\widetilde{\varphi}$ es diferenciable si y solo si $f\circ\varphi$ es 
    diferenciable.

    \item Esta noción de diferenciabilidad es local, es decir, si $p\in U\subseteq\Sigma$, con
    $U$ abierto, entonces $f$ es diferenciable si y solo si $f\big|_{U}:U\to\R^{d}$ es 
    diferenciable en $p$.
    
    \item Si $f:\Sigma\to\R^{d}$ es diferenciable entonces $f$ es continua, en efecto
    \begin{equation*}
        f=(f\circ\varphi)\circ\varphi^{-1}
    \end{equation*}
    es composición de mapeos continuos.

    \item Observaciones análogas se cumplen para $\gamma:V\subseteq\R^{2}\to\Sigma$.
\end{enumerate}

\noindent\textbf{Ejemplos:}
\begin{itemize}
    \item Si $\varphi:\V\subseteq\R^{2}\to\Sigma$ es una parametrización local, entonces 
    $\varphi$ y $\varphi^{-1}$ son diferenciables y $\varphi^{-1}\circ\varphi$ es la identidad.

    \item Si $h:W\subseteq\R^{3}\to\R$ es diferenciable con $W$ abierto y si $\Sigma\subseteq W$
    es una superficie parametrizada regular, entonces $h\big|_{\Sigma}:\Sigma\to\R$ es 
    diferenciable. Para toda parametrización local $\varphi:\V\subseteq\R^{2}\to\Sigma$ tenemos que
    $h\big|_{\Sigma}$ es la composición de $\varphi$ y $h$.

    \item Función altura, $h:\Sigma\to\R$ dada por $h(p)=\ip{u}{p-p_{0}}$. Esta función mide la 
    altura del punto $p_{0}\Sigma$ al plano $p_{0}+u^{\perp}$, donde $\abs{u}=1$.
    
    \item El cuadrado de la distancia a un $p_{0}\in\R^{3}$. Es decir, $f\Sigma\to\R$ dada por
    $\abs{p-p_{0}}^{2}$. Si $p_{0}\not\in\Sigma$ entonces $\abs{p-p_{0}}$ también es diferenciable.
\end{itemize}

\begin{lema}\hspace{1mm}
    \begin{enumerate}
        \item Sean $\gamma:\V\subseteq\R^{d}\to\Sigma$ y $f:\Sigma\to\R^{m}$ tales que $\gamma$ es
        diferenciable en $q$ y $f$ es diferenciable en $\gamma(q)$ entonces $f\circ\gamma$ es
        diferenciable en $q$.

        \item Sean $f:\Sigma\to\R^{m}$ y $\phi:W\subseteq\R^{m}\to\R^{d}$ con 
        $f(\Sigma)\subseteq W$ tales que $f$ es diferenciable en $p$ y $\phi$ es diferenciable en
        $\phi\circ f$ es diferenciable en $p$.
    \end{enumerate}
\end{lema}

\begin{dem}\hspace{1mm}
    \begin{enumerate}
        \item Sea $\varphi:\V\subseteq\R^{2}\to\R^{3}$ una parametrización local con 
        $\gamma(q)\in\varphi(\V)$. Entonces $\varphi\circ\gamma$ es diferenciable en $q$ y 
        $f\circ\varphi$ es diferenciable en $\varphi^{-1}(\gamma(q))$. Luego
        \begin{equation*}
            f\circ\gamma=(f\circ\varphi)\circ(\varphi^{-1}\circ\gamma)
        \end{equation*}
        es diferenciable en $q$ por ser composición de funciones diferenciables.

        \item Sea $\varphi:\V\subseteq\R^{2}\to\R^{3}$ una parametrización local de 
        $p\in\varphi(\V)$, entonces $f\circ\varphi:\V\to\R^{m}$ es diferenciable en 
        $\varphi^{-1}(p)$. Además $f\circ\varphi(\V)\subseteq f(\Sigma)\subseteq W$ y $\phi$ es
        diferenciable en $f\circ\varphi(\varphi^{-1}(p))=f(p)$. Luego
        \begin{equation*}
            (\phi\circ f)\circ\varphi=\phi\circ(f\circ\varphi)
        \end{equation*}
        es diferenciable en $\varphi^{-1}$. Por lo tanto, $\phi\circ f$ es diferenciable en 
        $p\in\Sigma$.
    \end{enumerate}
\end{dem}

\begin{cor}
    Una aplicación $\gamma:V\subseteq\R^{d}\to\Sigma$ es diferenciable $q\in V$ si y solo si
    sus coordenadas $\gamma_{1},\gamma_{2},\gamma_{3}$ son funciones diferenciables de $V$ a $\R$ 
    en $q$.
\end{cor}

\begin{dfn}
    Sean $\Sigma_{1},\Sigma_{2}$ superficies regulares. Se dice que
    \begin{equation*}
        F:\Sigma_{1}\to\Sigma_{2}
    \end{equation*}
    es diferenciable en $p\in\Sigma_{1}$. Si existen parametrizaciones locales 
    $\varphi_{i}:\V_{i}\subseteq\R^{2}\to\R^{3}$ para $\Sigma_{i}$ con $p\in\varphi_{1}(\V_{1})$ y
    $F(p)\in\varphi_{2}(\V_{2})$ tales que
    \begin{equation*}
        \varphi_{2}^{-1}\circ F\circ\varphi_{1}:(F\circ\varphi_{1})^{-1}(\varphi(\V_{2}))\to\V_{2}
    \end{equation*}
    es diferenciable en $q$.
\end{dfn}

\begin{prop}
    Sea $F:\Sigma_{1}\to\Sigma_{2}\subseteq\R^{3}$ y escribimos
    \begin{equation*}
        F(p)=(F_{1}(p),F_{2}(P),F_{3}(p))
    \end{equation*}
    donde $F_{i}:\Sigma\to\R$. Entonces $F$ es diferenciable en $p\in\Sigma_{1}$ si y solo si 
    $F_{i}$ son diferenciables en $p\in\Sigma_{1}$.
\end{prop}

\begin{dfn}
    Se dice que $F:\Sigma_{1}\to\Sigma_{2}$ entre superficies regulares es un difeomorfismo si
    \begin{itemize}
        \item $F$ es diferenciable, es decir, $F$ es diferenciable para todo $p\in\Sigma_{1}$.
        \item $F$ es una biyección y $F^{-1}$ es diferenciable
    \end{itemize}
\end{dfn}

\begin{teo}
    Sean $F:\Sigma_{1}\to\Sigma_{2}$ y $G:\Sigma_{2}\to\Sigma_{3}$ aplicaciones diferenciables entre
    superficies regulares. Si $F$ es diferenciable en $p\in\Sigma_{1}$ y $G$ es diferenciable en
    $F(p)\in\Sigma_{2}$ entonces $G\circ F$ es diferenciable en $p\in\Sigma_{1}$.
\end{teo}

\begin{dem}
    Escribimos $G(p)=(G_{1}(p),G_{2}(p),G_{3}(p))$ donde $G_{i}:\Sigma_{2}\to\R$ son diferenciables 
    en $F(p)$ por la proposición anterior. Por el lema anterior tenemos que 
    $G_{i}\circ F:\Sigma_{1}\to\R$ son diferenciables en $p\in\Sigma_{1}$.
    Como
    \begin{equation*}
        G\circ F(p)=(G_{1}\circ F(p),G_{2}\circ F(p),G_{3}\circ F(p))
    \end{equation*}
    por la proposición anterior, $G\circ F$ es diferenciable en $p\in\Sigma_{1}$.
\end{dem}

\noindent Del teorema anterior se sigue que $\Sigma_{1}$ es difeomorfo a $\Sigma_{2}$ define una 
relación de equivalencia entre superficies regulares. Notemos que
\begin{equation*}
    id_{\Sigma_{1}}:\Sigma_{1}\to\Sigma_{1}
\end{equation*}
es un difeomorfismo. Si $\varphi_{1},\varphi_{2}$ son parametrizaciones locales entonces 
$\varphi_{2}^{-1}\circ id_{\Sigma_{1}}\circ\varphi_{1}=\varphi_{2}^{-1}\circ\varphi_{1}$ es un
cambio de coordenadas.
\vspace{4mm}

\noindent\textbf{Ejemplo:} Consideremos las superficies regulares $\mathbb{S}^{2}$ y
\begin{equation*}
    \Sigma:=\left\{(x,y,z)\in\R^{3}:
    \left(\frac{x}{a}\right)^{2}+\left(\frac{y}{b}\right)^{2}+\left(\frac{z}{c}\right)^{2}=1
    \right\}
\end{equation*}
Afirmamos que $\mathbb{S}^{2}$ y $\Sigma$ son difeomorfas. En efecto, definimos 
$\phi:\R^{3}\to\R^{3}$ por $\phi(x,y,z)=(ax,by,cz)$ como $\phi$ es lineal e invertible $\phi$ y 
$\phi^{-1}$ son diferenciables y luego $\phi$ es un difeomorfismo. Además si 
$(x,y,z)\in\mathbb{S}^{2}$ entonces
\begin{equation*}
    \phi(x,y,z)=(ax,by,cz)\in\Sigma
\end{equation*}
Por lo tanto $\phi(\mathbb{S}^{2})\subseteq\Sigma$. similarmente $\phi^{-1}(\Sigma)\subseteq
\mathbb{S}^{2}$. Claramente $\phi:\mathbb{S}^{2}\to\Sigma$ es una biyección. Además como $\phi$ 
es diferenciable en todo punto
\begin{equation*}
    \phi\big|_{\mathbb{S}^{2}}:\mathbb{S}^{2}\to\R^{3}
\end{equation*}
es diferenciable. Por la proposición anterior tenemos que $\phi\big|_{\mathbb{S}^{2}}:
\mathbb{S}^{2}\to\Sigma$ es diferenciable. similarmente para $\phi^{-1}$.
\vspace{4mm}

\noindent La misma idea demuestra, en general, que si $\phi:U_{1}\subseteq\R^{3}
\to U_{2}\subseteq\R^{3}$ es difeomorfismo entre abiertos y $\Sigma\subseteq U_{1}$ es una 
superficie regular, entonces
\begin{equation*}
    \phi\big|_{\Sigma}:\Sigma\to\phi(\Sigma)
\end{equation*}
donde $\phi(\Sigma)$ es una superficie regular, es un difeomorfismo entre superficies.

\subsection{El Plano Tangente}
Sea $\Sigma\subseteq\R^{3}$ una superficie regular, $p\in\Sigma$. Si $\varphi_{i}:\V_{i}\subseteq
\R^{2}\to\R^{3}$ son parametrizaciones locales para $\Sigma$ con $p\in\varphi_{1}(\V_{1})\cap
\varphi_{2}(\V_{2})$. Vimos que $\varphi_{2}^{-1}\circ\varphi_{1}$ es un difeomorfismo entre 
abiertos de $\R^{2}$.
\vspace{4mm}

\noindent Luego, $D(\varphi_{2}^{-1}\circ\varphi_{1})(\varphi_{1}^{-1}(p))$ es un isomorfismo 
lineal. De ahí,
\begin{equation*}
    D\varphi_{1}(\varphi_{1}^{-1}(p))(\R^{2})=D(\varphi_{2}\circ(\varphi_{2}^{-1}\circ
    \varphi_{1}))(\R^{2})=D\varphi_{2}(\varphi_{2}^{-1}(p))\circ D(\varphi_{2}^{-1}\circ
    \varphi_{1})(\varphi_{1}^{-1}(p))(\R^{2})=D\varphi_{2}(\varphi_{2}^{-1}(p))
\end{equation*}
y cualquier parametrización local en $p$ tiene derivada con la misma imagen en $\varphi^{-1}(p)$.
\begin{dfn}
    El plano tangente a $\Sigma$ en $p\in\Sigma$ es el subespacio vectorial
    \begin{equation*}
        D\varphi(\varphi^{-1}(p))(\R^{2})\subseteq\R^{3}
    \end{equation*}
    con $\varphi$ una parametrización local en $p$. Lo denotaremos por $T_{p}\Sigma$.
\end{dfn}
\noindent Denotamos por $\varphi_{u}:=D\varphi(\varphi^{-1}(p))e_{1}$ y $\varphi_{v}:=
D\varphi(\varphi^{-1}(p))e_{2}$.
\vspace{4mm}

\noindent\textbf{Observación:} Geometricamente $T_{p}\Sigma$ es un plano en $\R^{3}$ que pasa por 
$0\in\R^{3}$.
\begin{prop}
    Para $p\in\Sigma$ y $w\in\R^{3}$ tenemos que $w\in T_{p}\Sigma$ si y solo si existe una curva
    parametrizada diferenciable $\alpha:(-\varepsilon,\varepsilon)\subseteq\R\to\R^{3}$ tal que
    \begin{itemize}
        \item $\alpha(0)=p$.
        \item $\alpha(t)\in\Sigma$ para todo $t\in(-\varepsilon,\varepsilon)$.
        \item $\alpha'(0)=w$.
    \end{itemize}
\end{prop}

\noindent\textbf{Ejemplos:}
\begin{itemize}
    \item Consideremos un plano en $\R^{3}$, es decir, $P=q+span\{w_{1},w_{2}\}$. Su 
    parametrización local es $\varphi:\R^{2}\to\R^{3}$ dada por $\varphi(u,v)=q+uw_{1}+vw_{2}$, 
    luego
    \begin{align*}
        \varphi_{u}(u,v)=w_{1} \\
        \varphi_{v}(u,v)=w_{2}
    \end{align*}
    es decir $T_{\varphi(u,v)}P=span\{w_{1},w_{2}\}$.

    \item Sea $f:\V\subseteq\R^{2}\to\R$ una función diferenciable. Tomemos $Graf(f)=
    \{(x,f(x)):x\in\V\subseteq\R^{2}\}$. Su parametrización local es $\varphi(u,v)=(u,v,f(u,v))$,
    entonces
    \begin{align*}
        \varphi_{u}(u,v)=\left(1,0,\pdv{f}{u}\right) \\
        \varphi_{v}(u,v)=\left(0,1,\pdv{f}{v}\right)
    \end{align*}
    se sigue que $T_{\varphi(u,v)}Graf(f)=\left\{(a,b,a\pdv{f}{u}+b\pdv{f}{v}):a,b\in\R\right\}$.
\end{itemize}

\begin{dem} Sea $p\in\Sigma$, $\varphi:\V\subseteq\R^{2}\to\R^{3}$ una parametrización local en 
    $p$ y $p_{0}=\varphi^{-1}(p)$.
    \begin{itemize}
        \item $\Rightarrow|$ Existen $a,b\in\R$ tales que
        \begin{equation*}
            w=D\varphi(p_{0})(a,b)=a\varphi_{u}(p_{0})+b\varphi_{v}(p_{0})
        \end{equation*}
        Como $p_{0}\in\V$ y $\V$ es abierto, tenemos que $p_{0}+t(a,b)\in\V$ para 
        $t\in(-\varepsilon,\varepsilon)$ y $\varepsilon>0$ suficientemente pequeño. Definimos
        \begin{equation*}
            \alpha:(-\varepsilon,\varepsilon)\to\R^{3}\hspace{4mm}\text{por}\hspace{4mm}
            \alpha(t)=\varphi(p_{0}+t(a,b))
        \end{equation*}
        entonces $\alpha$ es diferenciable y $\alpha((-\varepsilon,\varepsilon))\subseteq\Sigma$.
        Por otro lado $\alpha(0)=\varphi(p_{0})=p$ y
        \begin{align*}
            \alpha'(0) &= \dv{}{t}\Big|_{t=0}\varphi(p_{0}+t(a,b)) \\
            &= D\varphi(p_{0})\left(\dv{}{t}\Big|_{t=0}(p_{0}+t(a,b))\right)=
            D\varphi(p_{0})(a,b) \\
            &= w
        \end{align*}

        \item $\Leftarrow|$ Como $\alpha(0)=p\in\varphi(\V)$ y $\varphi(\V)$ es abierto en 
        $\Sigma$, tenemos que $\alpha(t)\in\varphi(\V)$ para $t$ suficientemente pequeño. Luego,
        esta bien definida
        \begin{equation*}
            \alpha_{0}:=\varphi^{-1}\circ\alpha
        \end{equation*}
        y es una curva diferenciable en $\V$. Sea $(a,b)=\alpha_{0}'(0)\in\R^{2}$, entonces
        \begin{equation*}
            D\varphi(p_{0})(a,b)=D\varphi(\alpha_{0}(0))\alpha_{0}'(0)=
            (\varphi\circ\alpha_{0})'(0)=\alpha'(0)=w
        \end{equation*}
    \end{itemize}
\end{dem}

\noindent\textbf{Ejemplo:} Sea $h:W\subseteq\R^{3}\to\R$ diferenciable y $c\in\R$ un valor regular
para $h$. Vimos que $h^{-1}(c)=\Sigma$ es una superficie regular. Sea $p\in\Sigma$, y 
$w\in T_{p}{\Sigma}$, existe $\alpha$ curva diferenciable tal que $\alpha'(0)=w$, $\alpha(0)=p$ y
$\alpha(t)\in\Sigma$ para todo $t\in(-\varepsilon, \varepsilon)$. Veamos que 
$h\circ\alpha\equiv c$. Entonces
\begin{equation*}
    0=(h\circ\alpha)'(0)=Dh(\alpha(0))\alpha'(0)=Dh(p)w
\end{equation*}
luego $T_{p}\Sigma\subseteq ker(Dh(p))=(\nabla h(p))^{\perp}$. Por lo tanto $T_{p}\Sigma=
(\nabla h(p))^{\perp}$. Hemos concluido que el gradiente de la función es perpendicular al plano 
tangente $T_{p}\Sigma$.
\vspace{4mm}

\noindent Por ejemplo $h(x,y,z)=x^{2}+y^{2}+z^{2}$ entonces $\nabla h=(2x,2y,2z)$, es decir, 
$\nabla h=2p$ para todo $p\in\R^{3}$. Como $\mathbb{S}^{2}=h^{-1}(1)$ tenemos 
$T_{p}\mathbb{S}^{2}=(\nabla h(p))^{\perp}=p^{\perp}$

\subsection{El Diferencial de una Aplicación Diferenciable}
\begin{dfn}
    Sea $f:\Sigma\to\R^{m}$ diferenciable en $p\in\Sigma$. Definimos la derivada o diferencial de
    $f$ en $p\in\Sigma$ se define por
    \begin{align*}
        Df_{p}:T_{p}\Sigma &\to \R^{m} \\
        w=\alpha'(0) &\to (f\circ\alpha)'(0)\in\R^{m}
    \end{align*}
\end{dfn}

\begin{prop}
    La derivada de una función diferenciable no depende de la elección de la curva y es lineal.
\end{prop}
\begin{dem}
    Sea $\varphi:\V\subseteq\R^{2}\to\R^{3}$ parametrización local para $\Sigma$ con 
    $p\in\varphi(\V)$. Escriba $p_{0}=\varphi^{-1}(p)\in\V$. Sea $\alpha:
    (-\varepsilon, \varepsilon)\to\R^{3}$ una curva diferenciable, con $\alpha(0)=p$, 
    $\alpha'(0)=w\in T_{p}\Sigma$ y $\alpha(t)\in\varphi(\V)$ para todo 
    $t\in(-\varepsilon, \varepsilon)$. Definimos 
    $\alpha_{0}=\varphi^{-1}\circ\alpha:(-\varepsilon, \varepsilon)\to\V$ es una curva parametrizada
    diferenciable con $\alpha_{0}(0)=\varphi^{-1}(\alpha(0))=p_{0}$.
    \vspace{4mm}

    \noindent Notemos que $w=D\varphi(p_{0})(\alpha_{0}'(0))$. Luego
    \begin{align*}
        Df_{p}(w) &= (f\circ\alpha)'(0)=(f\circ\varphi\circ\varphi^{-1}\circ\alpha)'(0)=
        ((f\circ\varphi)\circ\alpha_{0})'(0) \\
        &= D(f\circ\varphi)(\alpha_{0}(0))\alpha_{0}'(0) \\
        &= D(f\circ\varphi)(p_{0})\circ (D\varphi)^{-1}(p_{0})w
    \end{align*}
    para todo $w\in T_{p}\Sigma$. Es decir,
    \begin{equation*}
        Df_{p}=D(f\circ\varphi)(p_{0})\circ (D\varphi)^{-1}(p_{0})
    \end{equation*}
    entonces, $Df_{p}$ es lineal y no depende de $\alpha$.
\end{dem}

\noindent\textbf{Ejemplos:}
\begin{itemize}
    \item Sea $f:\Sigma\to\R^{m}$ dada por $f\equiv c$. Para todo $p\in\Sigma$ y todo 
    $w\in T_{p}\Sigma$, $w=\alpha'(0)$, luego
    \begin{equation*}
        Df_{p}(w)=(f\circ\alpha)'(0)=\dv{}{t}\Big|_{t=0}(f\circ\alpha)(t)=0
    \end{equation*}

    \item Sea $i_{\Sigma}:\Sigma\to\R^{3}$ la inclusión, es decir, $i_{\Sigma}(p)=p$. Para 
    $p\in\Sigma$ con $w=\alpha'(0)\in T_{p}\Sigma$ tenemos que
    \begin{equation*}
        D(i_{\Sigma})_{p}w=(i_{\Sigma}\circ\alpha)'(0)=\alpha'(0)=w
    \end{equation*}

    \item Sea $F:\R^{3}\to\R^{m}$ diferenciable, entonces $F\big|_{\Sigma}=:f$ es diferenciable.
    Además, si $p\in\Sigma$, $w=\alpha'(0)\in T_{p}\Sigma$ se sigue que
    \begin{equation*}
        Df_{p}(w)=(f\circ\alpha)'(0)=(F\circ\alpha)'(0)=DF(\alpha(0))\alpha'(0)=DF(p)w
    \end{equation*}
    es decir, $Df_{p}=DF(p)\big|_{T_{p}\Sigma}$.

    \item Sea $h:\Sigma\to\R$ dada por $h(p)=\ip{p-p_{0}}{u}$ con $p_{0},u\in\R^{3}$ y $\abs{u}=1$,
    entonces $Dh_{p}w=\ip{w}{u}$.

    \item Sea $f:\Sigma\to\R$ dada por $f(p)=\abs{p-p_{0}}^{2}$, luego
    $Df_{p}w=2\ip{w}{p-p_{0}}$.

    \item Sea $\pi_{i}:\R^{3}\to\R$ la proyección en la coordenada i-esima, entonces
    $D\left(\pi_{i}\big|_{\Sigma}\right)w=w_{i}$.
\end{itemize}

\noindent (Corregir hacia atras)
%-------------------------------------------------------------------%

\noindent Dada $\gamma:\V\subseteq\R^{n}\to\Sigma$ diferenciable en $q\in\V$ luego su derivada 
está bien definida como aplicación lineal $D_{\gamma}(q):\R^{n}\to\R^{3}$, notemos que
\begin{equation*}
    D_{\gamma}(q)w=\dv{}{t}\Big|_{t=o}\gamma(q+tw)=(\gamma\circ\beta)'(0)
\end{equation*}
donde $\beta(t)=q+tw$, como $\gamma\circ\beta\in\Sigma$ vemos que $(\gamma\circ\beta)'(0)\in
T_{\gamma(q)}\Sigma$, definimos su diferencial como
\begin{equation*}
    D_{\gamma_{q}}:=D_{\gamma}(q):\R^{n}\to T_{\gamma(q)}\Sigma
\end{equation*}

\begin{dfn}
    Sea $F:\Sigma_{1}\to\Sigma_{2}$ diferenciable en $p\in\Sigma_{1}$. Dado $v\in T_{p}\Sigma$ 
    definimos el \textbf{diferencial F en p} como $DF_{p}:T_{p}\Sigma_{1}\to T_{F(p)}
    \Sigma_{2}$ dada por
    \begin{equation*}
        DF_{p}(v):=(F\circ\alpha)'(0)
    \end{equation*}
    donde $\alpha$ es una curva diferenciable tal que $\alpha\subset\Sigma_{1}$ y es tangente a 
    $\alpha'(0)=v$.
\end{dfn}

\noindent\textbf{Observación:} El diferencial de $F$ esta bien definido, como $\alpha\subset
\Sigma_{1}$ se tiene que $F\circ\alpha\subset\Sigma_{2}$ y pasa por el punto $F(p)$, luego 
$(F\circ\alpha)'(0)\in T_{F(p)}\Sigma_{2}$. Además, por la discusión anterior, el valor no depende
de la elección de la curva $\alpha$ y es un mapeo lineal.

\vspace{2mm}
\noindent\textbf{Ejemplo:} Sea $A\R^{3}\to\R^{3}$ lineal y $\Sigma\subseteq\R^{3}$ superficie 
regular tal que $A(\Sigma)\subseteq\Sigma$. Luego esta bien definida $A:\Sigma\to\Sigma$ y es 
diferenciable en $p\in\Sigma$, pues es restricción de una aplicación diferenciable. Queremos 
calcular
\begin{equation*}
    DA_{p}:T_{p}\Sigma\to T_{A(p)}\Sigma
\end{equation*}
Sea $v\in T_{p}\Sigma$ y $v=\alpha'(0)$ para $\alpha\subset\Sigma$ tangente a $v$, entonces
\begin{equation*}
    DA_{p}(v)=\dv{}{t}\Big|_{t=0}(A\circ\alpha)(t)=A\left(\dv{}{t}\Big|_{t=0}\alpha(t)\right)=
    A(\alpha'(0))=Av
\end{equation*}
Por lo tanto $DA_{p}=A\Big|_{T_{p}\Sigma}$.

\begin{teo}
    (\textbf{Regla de la Cadena}) Sean $F:\Sigma_{1}\to\Sigma_{2}$ y $G:\Sigma_{2}\to\Sigma_{3}$ 
    aplicaciones diferenciables en $p\in\Sigma_{1}$ y $F(p)\in\Sigma_{2}$ respectivamente, 
    entonces
    \begin{equation*}
        D(G\circ F)_{p}=DG_{F(p)}\circ DF(p)
    \end{equation*}
\end{teo}

\begin{dem}
    Sea $v\in T_{p}\Sigma_{1}$ y $\alpha\subset\Sigma_{1}$ tangente a $\alpha'(0)=v$. Notemos que
    \begin{equation*}
        D(G\circ F)_{p}(\alpha'(0))=(G\circ F\circ \alpha)'(0)=DG_{F(p)}(F\circ\alpha)'(0)=
        DG_{F(p)}\circ DF_{p}(\alpha'(0))
    \end{equation*}
\end{dem}

\begin{cor}
    Sea $F:\Sigma_{1}\to\Sigma_{2}$ un difeomorfismo, entonces $D(F^{-1})_{F(p)}=(DF_{p})^{-1}$ 
    para todo $p\in\Sigma_{1}$
\end{cor}

\begin{teo}
    (\textbf{Teorema de la Función Inversa}) Sea $F:\Sigma_{1}\to\Sigma_{2}$ diferenciable. Sea 
    $p\in\Sigma_{1}$ tal que $Df_{p}$ es isomorfismo, entonces existe $U\subseteq\Sigma_{1}$ 
    abierto con $p\in U$ tal que $F(U)\subseteq\Sigma_{2}$ es abierto y
    \begin{equation*}
        F\Big|_{U}:U\to F(U)
    \end{equation*}
    es un difeomorfismo.
\end{teo}

\begin{dfn}
    Sea $f:\Sigma\to\R$ diferenciable. Decimos que $p\in\Sigma$ es un \textbf{punto crítico} de 
    $f$ si
    \begin{equation*}
        0\equiv Df_{p}:T_{p}\Sigma\to\R
    \end{equation*}
    en otras palabras, no es sobreyectiva. Se dice que $c\in\R$ es un \textbf{valor regular} si 
    $f^{-1}(c)$ no contiene puntos críticos.
\end{dfn}

\begin{prop}
    Sea $f:\Sigma\to\R$ diferenciable y $p\in\Sigma$ un punto maximo o minimo (local o global),
    entonces $p$ es punto crítico.
\end{prop}
\begin{dem}
    Sea $v\in T_{p}\Sigma$ y $\alpha\subset\Sigma$ tangente a $\alpha'(0)=v$, luego
    \begin{equation*}
        Df_{p}(v)=(f\circ\alpha)'(0)=0
    \end{equation*}
    Concluimos que $p$ es punto crítico de $f$.
\end{dem}

\begin{teo}
    Sea $f:\Sigma\to\R$ diferenciable, si $c\in\R$ es valor regular de $f$, entonces $f^{-1}(c)$
    es una curva regular.
\end{teo}

\noindent\textbf{Observación:} Se dice que $S\subset\Sigma$ es una \textbf{curva regular} si para
todo $p\in\Sigma$ existe un abierto $U\subseteq\Sigma$ y una curva parametrizada regular
$\alpha:I\subseteq\R\to\Sigma$ tal que $\alpha(I)=U\cap S$.

\vspace{4mm}
\noindent Sea $f:\Sigma\to\R^{m}$ diferenciable en $p\in\Sigma$. Si $\varphi:\V\subseteq\R^{2}\to\R^{3}$ es
parametrización local para $\Sigma$ con $p\in\varphi(\V)$ tenemos una base de $T_{p}\Sigma$, a 
saber, $\varphi_{u}(p_{0}),\varphi_{v}(p_{0})$ donde $p_{0}=\varphi^{-1}(p)$. Vimos que
$Df_{p}\circ D\varphi_{p_{0}}=D(f\circ\varphi)_{p_{0}}$, luego
\begin{equation*}
    Df_{p}(a\varphi_{u}(p_{0})+b\varphi_{v}(p_{0}))=Df_{p}(D\varphi_{p_{0}}(a,b))=
    D(f\circ\varphi)_{p_{0}}(a,b)
\end{equation*}
es decir, la matriz de $Df_{p}$ respecto a $\{\varphi_{u}(p_{0}),\varphi_{v}(p_{0})\}$ es la matriz
jacobiana de $D(f\circ\varphi)_{p_{0}}$.

\newpage
\section{La Segunda Forma Fundamental}
\subsection{Campos Vectoriales y Orientación}

\begin{dfn}
    Sea $\Sigma$ una superficie regular. Una aplicación continua $V:\Sigma\to\R^{3}$ se dice
    \textbf{campo vectorial}. Se dice que $V$ es:
    \begin{enumerate}
        \item \textbf{Campo tangente} si $V(p)\in T_{p}\Sigma$ para todo $p\in\Sigma$.
        \item \textbf{Campo normal} si $V(p)\in\left(T_{p}\Sigma\right)^{\perp}$ para todo $p\in\Sigma$.
        \item \textbf{Campo unitario} si $\abs{V(p)}=1$ para todo $p\in\Sigma$.
    \end{enumerate}
\end{dfn}

\noindent\textbf{Ejemplo:} Sea $\varphi:\V\subseteq\R^{2}\to\R^{3}$ una parametrización local para
$\Sigma$ y $U=\varphi(\V)$ entonces
\begin{align*}
    \varphi_{u}\circ\varphi^{-1}:U\to\R^{3} \\
    \varphi_{v}\circ\varphi^{-1}:U\to\R^{3}
\end{align*}
son campos tangentes. Sea $N^{x}:\V\to\R^{3}$ dada por
\begin{equation*}
    N^{x}(u,v):=\frac{\varphi_{u}(u,v)\times\varphi_{v}(u,v)}
    {\abs{\varphi_{u}(u,v)\times\varphi_{v}(u,v)}}
\end{equation*}
es diferenciable y $N^{x}\perp T_{\varphi(u,v)}\Sigma$. Luego $N=N^{x}\circ\varphi^{-1}$ es campo
unitario, normal y diferenciable.

\begin{dfn}
    Se dice que una superficie regular $\Sigma$ es \textbf{orientable} si existe un campo normal, 
    unitario y continuo $N:\Sigma\to\R^{3}$. En ese caso, se dice que $N$ define una Orientación.
    Además,
    \begin{itemize}
        \item Sea $\{w_{1},w_{2}\}$ base de $T_{p}\Sigma$, se dice \textbf{positiva} si 
        $\{w_{1},w_{2},N(p)\}$ es una base positiva de $\R^{3}$.
        
        \item Una parametrización $\varphi$ es \textbf{positiva} si $N^{x}=N\circ\varphi$.
    \end{itemize}
\end{dfn}

\noindent\textbf{Observaciones:}
\begin{enumerate}
    \item Si $\Sigma$ es conexa y esta orientada por $N:\Sigma\to\R^{3}$, hay exactamente dos
    orientaciones en $\Sigma$, que son $N$ y $-N$. En efecto, sea $V:\Sigma\to\R^{3}$ normal, 
    unitario y continuo, entonces
    \begin{align*}
        A=\{p\in\Sigma:V(p)=N(p)\}=(V-N)^{-1}(0), \\
        B=\{p\in\Sigma:V(p)=-N(p)\}=(V+N)^{-1}(0)
    \end{align*}
    son cerrados. Dado $p\in\Sigma$ se tiene que $V(p)\in\left(T_{p}\Sigma\right)^{\perp}=
    span\{N(p)\}$ y $\abs{V(p)}=1$, luego $V(p)=N(p)$ ó $V(p)=-N(p)$, es decir $A\cup B=\Sigma$. 
    Además, $A\cap B=\emptyset$. Así, por conexidad, tenemos que $A=\Sigma$ y por lo tanto 
    $V\equiv N$ o bien $B=\Sigma$ y $V\equiv -N$.

    \item Dada $N:\Sigma\to\R^{3}$ una orientación. Veamos que $N$ es diferenciable. Sea 
    $p\in\Sigma$, existe una parametrización local $\varphi:\V\subseteq\R^{2}\to\R^{3}$ para 
    $\Sigma$ con $p\in\varphi(\V)$ y $\varphi(\V)$ conexo. De ahí, $N\big|_{\varphi(\V)}$ define 
    una orientación para $\varphi(\V)$. Por (a) tenemos que $(N^{x}\circ\varphi^{-1})\equiv N$ ó
    $(N^{x}\circ\varphi^{-1})\equiv -N$ y por lo tanto $N\big|_{\varphi(\V)}$ es diferenciable. 
    Concluimos que $N$ es diferenciable.

    \item Si $\Sigma_{1}$ y $\Sigma_{2}$ son difeomorfos. Entonces $\Sigma_{1}$ es orientable si y
    solo si $\Sigma_{2}$ es orientable.
\end{enumerate}

\noindent\textbf{Ejemplos:}
\begin{itemize}
    \item Si $\Sigma=\varphi(\V)$ para una parametrización, entonces $\Sigma$ 
    es orientable.

    \item Sea $\Sigma=Graf(f)$ con $f:\V\subseteq\R^{2}\to\R^{3}$ diferenciable, entonces 
    $\varphi(u,v)=(u,v,f(u,v))$ es parametrización local y $\varphi(\V)=\Sigma$. Luego
    \begin{align*}
        \varphi_{u}(u,v)=(1,0,\partial_{u}f) \\
        \varphi_{v}(u,v)=(0,1,\partial_{v}f)
    \end{align*}
    entonces
    \begin{equation*}
        \varphi_{u}\times\varphi_{v}=(-\partial_{u}f,-\partial_{v}f,1)=(-\nabla f,1)
    \end{equation*}
    consideramos
    \begin{equation*}
        N^{x}(u,v)=\frac{1}{\sqrt{1+\abs{\nabla f}^{2}}}(-\nabla f,1)
    \end{equation*}
    se sigue que $N^{x}\circ\varphi^{-1}$ define una orientación para $\Sigma$.

    \item (Niveles Regulares) Sea $h:W\subseteq\R^{3}\to\R$ diferenciable y $c\in\R$ un valor 
    regular para $h$, luego $\Sigma=h^{-1}(c)$ es superficie regular. Sabemos que $T_{p}\Sigma=
    (\nabla h(p))^{\perp}$. Si $W_{0}=\{p\in W:\nabla h(p)\neq0\}$, entonces $W_{0}$ es abierto y
    $\Sigma\subset W_{0}$. Definimos $N:W_{0}\to\R^{3}$ dada por
    \begin{equation*}
        N(p):=\frac{\nabla h(p)}{\abs{\nabla h(p)}}
    \end{equation*}
    es continua y dado $p\in\Sigma$ se tiene que $N(p)\perp T_{p}\Sigma$. Por lo tanto 
    $N\big|_{\Sigma}$ define una orientación.
\end{itemize}

\begin{prop}
    Sea $\Sigma$ una superficie regular. Entonces $\Sigma$ es orientable si y solo si existen
    $\{\varphi_{i}:\V_{i}\subseteq\R^{2}\}_{i}$ parametrizaciones locales para $\Sigma$ tales que
    \begin{itemize}
        \item $\bigcup_{i}\varphi_{i}(\V_{i})=\Sigma$
        \item $\varphi_{j}^{-1}\circ\varphi_{i}$ tiene determinante jacobiano positivo para todo 
        $i,j$.
    \end{itemize}
\end{prop}

\subsection{Formas Fundamentales y Aplicación de Gauss}
\begin{dfn}
    Sea $\Sigma$ una superficie regular. La \textbf{primera forma fundamental} de $\Sigma$ en 
    $p\in\Sigma$ es la restricción del producto interno a $T_{p}\Sigma$,
    \begin{equation*}
        \ip{.}{.}=I_{p}:T_{p}\Sigma\times T_{p}\Sigma\to\R,\hspace{4mm}
        \ip{w_{1}}{w_{2}}_{p}=I_{p}(w_{1},w_{2})\to\ip{w_{1}}{w_{2}}
    \end{equation*}
\end{dfn}

\noindent Sea $\varphi:\V\subseteq\R^{2}\to\R^{3}$ una parametrización local para $\Sigma$, 
entonces las funciones diferenciables $E,F,G:\V\to\R$ dadas por
\begin{equation*}
    E(u,v):=\abs{\varphi_{u}(u,v)}^{2},\hspace{4mm}
    F(u,v):=\ip{\varphi_{u}(u,v)}{\varphi_{v}(u,v)},\hspace{4mm}
    G(u,v):=\abs{\varphi_{v}(u,v)}^{2},\hspace{4mm}
\end{equation*}
se llaman los \textbf{coeficientes de la primera forma fundamental} en las coordenadas $(u,v)$. Se
dice que $\varphi$ es una parametrización ortogonal si $F\equiv0$. Podemos escribir $\ip{.}{.}_{p}$
en $p=\varphi(p_{0})$ en términos de la base $\{\varphi_{u}(p_{0}),\varphi_{v}(p_{0})\}$ como
\begin{equation*}
    I_{p}(w_{1},w_{2})=\ip{a_{1}\varphi_{u}(p_{0})+a_{2}\varphi_{v}(p_{0})}
    {b_{1}\varphi_{u}(p_{0})+b_{2}\varphi_{v}(p_{0})} = a_{1}b_{1}E(p_{0})+
    (a_{1}b_{2}+a_{2}b_{1})F(p_{0})+a_{2}b_{2}G(p_{0})
\end{equation*}

\vspace{4mm}
\noindent\textbf{Observación:} Si denotamos las variables de $\varphi$ por $x_{i}$ también se usa
la notación $g_{ij}:=\ip{\varphi_{x_{i}}}{\varphi_{x_{j}}}$, es decir, $g_{11}=E$, $g_{12}=g_{21}=F$
y $g_{22}=G$.

\vspace{4mm}
\noindent\textbf{Ejemplos:}
\begin{itemize}
    \item Sea $P=p_{0}+w^{\perp}\subseteq\R^{3}$ un plano, podemos escoger $w_{1},w_{2}\in 
    w^{\perp}$ ortonormales y definir una parametrización $\varphi:\R^{2}\to\R^{3}$ por
    $\varphi(u,v):=p_{0}+uw_{1}+vw_{2}$, con lo cual $\varphi_{u}\equiv w_{1}$ y 
    $\varphi_{v}\equiv w_{2}$. Luego
    \begin{equation*}
        \begin{pmatrix}
            E & F \\
            F & G
        \end{pmatrix}(u,v)=
        \begin{pmatrix}
            1 & 0 \\
            0 & 1
        \end{pmatrix}\hspace{4mm}\forall(u,v)\in\R^{2}
    \end{equation*}

    \item Para una gráfica $\Sigma=Graf(f)$ de una función $f:\V\subseteq\R^{2}\to\R$ diferenciable
    la parametrización local $\varphi_{u}=(1,0,\partial_{u}f)$ y $\varphi_{v}=(0,1,\partial_{v}f)$,
    luego
    \begin{equation*}
        E(u,v)=1+(\partial_{u}f(u,v))^{2},\hspace{4mm}
        F(u,v)=\ip{\partial_{u}f(u,v)}{\partial_{v}f(u,v)},\hspace{4mm}
        G(u,v)=1+(\partial_{v}f(u,v))^{2}
    \end{equation*}

    \item Considere el cilindro $\{(x,y,z)\in\R^{3}:x^{2}+y^{2}=r^{2}\}$ con parametrización local
    $\varphi(u,v)=(rcosu,rsenu,v)$. Tenemos $\varphi_{u}(u,v)=(-rsenu,rcosu,0)$ y 
    $\varphi_{v}(u,v)=(0,0,1)$, luego
    \begin{equation*}
        E(u,v)=r^{2},\hspace{4mm}
        F(u,v)=0,\hspace{4mm}
        G(u,v)=1
    \end{equation*}
    Vemos que la parametrización es ortogonal.
\end{itemize}

\vspace{2mm}
\noindent\textbf{Observación:} Veamos lo siguiente
\begin{align*}
    \abs{\varphi_{u}\times\varphi_{v}}^{2}(u,v) &= \abs{\varphi_{u}(u,v)}^{2}\cdot
    \abs{\varphi_{v}(u,v)}^{2}-\ip{\varphi_{u}(u,v)}{\varphi_{v}(u,v)}^{2} \\
    &= E(u,v)\cdot G(u,v)-F^{2}(u,v)=det
    \begin{vmatrix}
        E & F \\
        F & G
    \end{vmatrix}(u,v)
\end{align*}
Luego, la matriz $\begin{pmatrix}
    E & F \\
    F & G
\end{pmatrix}$ es invertible y es definida positiva.

\begin{prop}
    Sea $f:\Sigma\to\R$ diferenciable. Existe un campo tangente diferenciable 
    $\nabla^{\Sigma}f:\Sigma\to\R^{3}$ tal que
    \begin{equation*}
        \ip{\nabla^{\Sigma}f(p)}{w}_{p}=Df_{p}(w)\hspace{4mm}\forall w\in T_{p}\Sigma
    \end{equation*}
\end{prop}

\begin{dfn}
    El campo $\nabla^{\Sigma}f$ se llama \textbf{campo gradiente} de $f$.
\end{dfn}

\noindent\textbf{Observación:} Para una parametrización local $\varphi:\V\subseteq\R^{2}\to\R^{3}$ 
de $\Sigma$, se escribe como
\begin{equation*}
    \nabla^{\Sigma}f(p)=\frac{f_{u}\cdot G-f_{v}\cdot F}{EG-F^{2}}(p_{0})\cdot\varphi_{u}(p_{0})+
    \frac{f_{v}\cdot E-f_{u}\cdot F}{EG-F^{2}}(p_{0})\cdot\varphi_{v}(p_{0})
\end{equation*}
para todo $p=\varphi(p_{0})\in\varphi(\V)$, donde $f_{u}=(f\circ\varphi)_{u}$ y 
$f_{v}=(f\circ\varphi)_{v}$.

\vspace{2mm}
\noindent Supongamos que $\Sigma$ es una superficie regular orientable y sea $N:\Sigma\to\R^{3}$
un campo normal unitario diferenciable. Observe que $N$ define una aplicación 
$N:\Sigma\to\mathbb{S}^{2}$ llamada la \textbf{aplicación de Gauss} de $\Sigma$. Como en el caso 
de curvas, esperamos describir la geometría de $\Sigma$ usando la derivada de $N$.

\begin{prop}
    Sea $p\in\Sigma$, tenemos $T_{p}\Sigma$ y la derivada de $N$ es una aplicación lineal 
    $DN_{p}:T_{p}\Sigma\to T_{p}\Sigma$ la cual es simétrica (autoadjunta) respecto a 
    $\ip{.}{.}_{p}$.
\end{prop}

\begin{dem}
    La primera afirmación es consecuencia de $T_{p}\Sigma=N(p)^{\perp}=T_{N(p)}\mathbb{S}^{2}$,
    donde usamos que $T_{q}\mathbb{S}^{2}=q^{\perp}$ para todo $q\in\mathbb{S}^{2}$. Resta ver que
    $DN(p)$ es simétrica. Sea $\varphi:\V\subseteq\R^{2}\to\R^{3}$ es una parametrización local de
    $\Sigma$ con $p=\varphi(p_{0})$, entonces
    \begin{align*}
        \ip{DNp(\varphi_{u}(p_{0}))}{\varphi_{v}(p_{0})}_{p} &= \ip{D(N\circ\varphi)_{p_{0}}e_{1}}
        {\varphi_{v}(p_{0})}=\ip{(N\circ\varphi)_{u}}{\varphi_{v}}(p_{0}) \\
        &= \pdv{}{u}\ip{N,\varphi_{v}}(p_{0})-\ip{N}{\varphi_{vu}}(p_{0})=
        -\ip{N}{\varphi_{vu}}(p_{0})
    \end{align*}
    del mismo modo se tiene que $\ip{DNp(\varphi_{v}(p_{0}))}{\varphi_{u}(p_{0})}_{p}=
    -\ip{N}{\varphi_{uv}}(p_{0})$. Como $\varphi_{uv}=\varphi_{vu}$ y $\{\varphi_{u}(p_{0}),
    \varphi_{v}(p_{0})\}$ es una base para $T_{p}\Sigma$, concluimos que $DN_{p}$ es simétrica.
\end{dem}

\begin{dfn}
    Se dice que $A_{p}:=-DN_{p}:T_{p}\Sigma\to T_{p}\Sigma$ es el \textbf{operador de Weingarten}
    de $\Sigma$ en $p$. La forma bilineal $\mathbb{I}_{p}:T_{p}\Sigma\times T_{p}\Sigma\to\R$
    asociada al operador $A_{p}$ mediante $\ip{.}{.}_{p}$ se llama la \textbf{segunda forma 
    fundamental} de $\Sigma$ en $p$. Concretamente
    \begin{equation*}
        \I_{p}(w_{1},w_{2})=\ip{A_{p}w_{1}}{w_{2}}=-\ip{DN_{p}w_{1}}{w_{2}}=
        -\ip{w_{1}}{DN_{p}w_{2}}
    \end{equation*}
    para todo $w_{1},w_{2}\in T_{p}\Sigma$.
\end{dfn}

\noindent\textbf{Ejemplos:}
\begin{itemize}
    \item Para un plano $P=p_{0}+w^{\perp}\subseteq\R^{3}$ la aplicación de Gauss
    $N:P\to\mathbb{S}^{2}$ es constante y vale $N(p)=\frac{w}{\abs{w}}$. Luego $A_{p}=-DN_{p}\equiv0$
    y $\I_{p}\equiv0$ para todo $p\in\Sigma$.
    
    \item Sea $\mathbb{S}^{2}(r):=\{(x,y,z)\in\R^{3}:x^{2}+y^{2}+z^{2}=r^{2}\}$. Luego 
    $N(f)=-\frac{1}{r}p$. Sea $w\in T_{p}\mathbb{S}^{2}(r)$, con $w=\alpha'(0)$ y $\alpha\subset
    \mathbb{S}^{2}(r)$, entonces
    \begin{equation*}
        DN_{p}(w)=(N\circ\alpha)'(0)=\dv{}{t}\Big|_{t=0}N(\alpha(t))=\dv{}{t}\Big|_{t=0}
        \left(-\frac{1}{r}p\right)=-\frac{1}{r}\alpha'(0)=-\frac{1}{r}w
    \end{equation*}
    Por lo tanto $A_{p}w=\frac{1}{r}w$, entonces la segunda forma fundamental tiene la forma
    $\I_{p}(v,w)=\frac{1}{r}\ip{v}{w}_{p}$ y
    \begin{equation*}
        A_{p}=\begin{pmatrix}
            \frac{1}{r} & 0 \\
            0 & \frac{1}{r}
        \end{pmatrix}
    \end{equation*}

    \item Consideremos $C:=\{(x,y,z)\in\R^{3}:x^{2}+y^{2}=r^{2}\}=h^{-1}(r^{2})$ con 
    $h(x,y,z)=x^{2}+y^{2}$, luego $T_{p}C=(\nabla h(p))^{\perp}=(2x,2y,0)^{\perp}$. Sean $w_{1}=
    (-y,x,0)=\alpha'(0)$ y $w_{2}=(0,0,1)=\beta'(0)$ donde
    \begin{align*}
        & \alpha(t)=\left(cos\left(\frac{t}{r}\right)x-sen\left(\frac{t}{r}\right)y,
        sen\left(\frac{t}{r}\right)x+cos\left(\frac{t}{r}\right)y,z\right) \\
        & \beta(t)=(x,y,z+t)
    \end{align*}
    Tomamos $N(p)=-\frac{1}{r}(x,y,0)$, luego
    \begin{align*}
        & DN_{p}w_{1}=\dv{}{t}\Big|_{t=0}N(\alpha(t))=-\frac{1}{r}w_{1} \\
        & DN_{p}w_{2}=\dv{}{t}\Big|_{t=0}N(\beta(t))=0
    \end{align*}
    por lo tanto
    \begin{equation*}
        A_{p}=\begin{pmatrix}
            \frac{1}{r} & 0 \\
            0 & 0
        \end{pmatrix}
    \end{equation*}
\end{itemize}

\begin{dfn}
    Los autovalores de $A_{p}$ se llaman las \textbf{curvaturas principales} de $\Sigma$ en $p$ y
    y las denotamos $k_{1}(p)\leq k_{2}(p)$. Los autovectores $\{e_{1},e_{2}\}$ que corresponden a 
    $k_{1},k_{2}$ se llaman \textbf{direcciones principales} de $\Sigma$ en $p$. Además se definen
    \begin{equation*}
        k_{\Sigma}(p):=det(A_{p})\hspace{4mm}H_{\Sigma}(p):=\frac{1}{2}tr(A_{p})
    \end{equation*}
    la \textbf{curvatura gaussiana} y \textbf{curvatura media} respectivamente.
\end{dfn}
\noindent\textbf{Observación:} La curvatura gaussiana es la unica que no depende de la orientación,
además, se tiene lo siguiente
\begin{equation*}
    k_{1}=H_{\Sigma}-\sqrt{H_{\Sigma}^{2}-k_{\Sigma}}\hspace{4mm}\text{ y }\hspace{4mm}
    k_{2}=H_{\Sigma}+\sqrt{H_{\Sigma}^{2}-k_{\Sigma}}
\end{equation*}
lo anterior esta bien definido gracias a la desigualdad de las medias.
\begin{prop}
    Sea $\Sigma$ una superficie regular. Sea $\alpha:(-\varepsilon,\varepsilon)\to\Sigma$ una 
    curva parametrizada por el arco con $\alpha(0)=p$ y $\alpha'(0)\in T_{p}\Sigma$. Sea 
    $k_{\alpha}$ la curvatura de $\alpha$ en $p$ y $N_{\alpha}$ el normal unitario. Entonces
    \begin{equation*}
        \I_{p}(\alpha'(0),\alpha'(0))=\ip{N_{\Sigma}(p)}{k_{\alpha}N_{\alpha}}
    \end{equation*}
\end{prop}
\begin{dem}
    Como $\alpha'(t)\in T_{p}\Sigma$ se sigue que $\ip{N\circ\alpha(t)}{\alpha'(t)}=0$, entonces
    \begin{equation*}
        \ip{DN_{\alpha(t)}(\alpha'(t))}{\alpha'(t)}+\ip{N(\alpha(t))}{\alpha''(t)}=0
    \end{equation*}
    lo que implica la afirmación.
\end{dem}
Veamos que $\ip{N(\alpha(t))}{\alpha''(t)}$ solo depende de la dirección tangente $\alpha'(0)\in 
T_{p}\Sigma$.

\begin{dfn}
    Sea $v\in T_{p}\Sigma$ unitario, definimos la \textbf{curvatura normal} de $\Sigma$ en $p$ en
    la dirección de $v$ por $k_{v}:=\I_{p}(v,v)$.
\end{dfn}
\noindent Sea $\{e_{1},e_{2}\}$ base ortonormal de $T_{p}\Sigma$ tal que $A_{p}(e_{i})=
k_{i}(p)e_{i}$. Entonces $v=\ip{v}{e_{1}}e_{1}+\ip{v}{e_{2}}e_{2}$ y también
\begin{align*}
    k_{v} &= \I_{p}(v,v)=\I_{p}(\ip{v}{e_{1}}e_{1}+\ip{v}{e_{2}}e_{2},
    \ip{v}{e_{1}}e_{1}+\ip{v}{e_{2}}e_{2}) \\
    &= \ip{v}{e_{1}}^{2}\I_{p}(e_{1},e_{1})+2\ip{v}{e_{1}}\ip{v}{e_{2}}\I_{p}(e_{1},e_{2})+
    \ip{v}{e_{2}}^{2}\I_{p}(e_{2},e_{2}) \\
    &= \ip{v}{e_{1}}^{2}k_{1}+\ip{v}{e_{2}}^{2}k_{2}
\end{align*}
\begin{cor}
    $k_{1}(p)$ y $k_{2}(p)$ son el maximo y minimo de las curvaturas normales $k_{v}$, para 
    $v\in T_{p}\Sigma$ unitario.
\end{cor}

\begin{dfn}
    Sea $\Sigma$ una superficie regular y $N:\Sigma\to\mathbb{S}^{2}$ aplicación de Gauss. Dado 
    $p\in\Sigma$ se dice un punto
    \begin{itemize}
        \item \textbf{eliptico} si $k_{\Sigma}(p)>0$.
        \item \textbf{hiperbolico} si $k_{\Sigma}(p)<0$.
        \item \textbf{parabolico} si $k_{\Sigma}(p)=0$ y $H_{\Sigma}(p)\neq0$.
        \item \textbf{plano} si $k_{1}(p)=0$ y $k_{2}(p)=0$
    \end{itemize}
\end{dfn}
\noindent\textbf{Observación:}
\begin{enumerate}
    \item Si $p$ es eliptico entonces todas las curvaturas normales en $p$ tienen el mismo signo y
    toda sección normal en $p$ esta contenida en un lado de $T_{p}\Sigma$.

    \item Si $p$ es hiperbolico, $k_{1}(p)<0<k_{2}(p)$ y $\Sigma$ tiene puntos en ambos lados de
    $T_{p}\Sigma$.
\end{enumerate}
Estas conclusiones se pueden profundizar estudiando la hessiana en $p$ de la función 
$f:\Sigma\to\R$ dada por $f(q):=\ip{q-p}{N(p)}$.

\noindent Sea $\varphi=\varphi(u,v):\V\subseteq\R^{2}\to\R^{3}$ parametrización local. Luego
$\varphi_{u}(u,v)$ y $\varphi_{v}(u,v)$ son campos tangentes, definimos
\begin{align*}
    e(u,v) &:= \I_{\varphi(u,v)}(\varphi_{u}(u,v),\varphi_{u}(u,v))=
    \ip{A_{\varphi(u,v)}\varphi_{u}(u,v)}{\varphi_{u}(u,v)} \\
    &= -\ip{DN_{\varphi(u,v)}(\varphi_{u}(u,v))}{\varphi_{u}(u,v)}=
    -\ip{(N\circ\varphi)_{u}(u,v)}{\varphi_{u}(u,v)} \\
    &= \ip{N\circ\varphi(u,v)(u,v)}{\varphi_{uu}(u,v)}
\end{align*}
Del mismo modo se tiene que
\begin{align*}
    f(u,v) &:= \I_{\varphi(u,v)}(\varphi_{u}(u,v),\varphi_{v}(u,v))=\ip{N\circ\varphi(u,v)}
    {\varphi_{uv}(u,v)} \\
    g(u,v) &:= \I_{\varphi(u,v)}(\varphi_{v}(u,v),\varphi_{v}(u,v))=\ip{N\circ\varphi(u,v)}
    {\varphi_{vv}(u,v)}
\end{align*}
Estas funciones se llaman \textbf{coeficientes de la segunda forma fundamental} en 
$\varphi(u,v)\in\Sigma$ respecto a $\varphi$.

\begin{teo}
    Se tienen las siguientes igualdades
    \begin{equation*}
        k_{\Sigma}\circ\varphi=det\left(\begin{pmatrix}
            E & F \\
            F & G
        \end{pmatrix}^{-1}\begin{pmatrix}
            e & f \\
            f & g
        \end{pmatrix}\right)=\frac{eg-f^{2}}{EG-F^{2}}
    \end{equation*}
    \begin{equation*}
        H_{\Sigma}\circ\varphi=\frac{1}{2}tr\left(\begin{pmatrix}
            E & F \\
            F & G
        \end{pmatrix}^{-1}\begin{pmatrix}
            e & f \\
            f & g
        \end{pmatrix}\right)=\frac{Eg+Ge-2Ff}{EG-F^{2}}
    \end{equation*}
\end{teo}
\begin{dem}
    Sea $p=\varphi(p_{0})$ con $p_{0}\in\V$ y escribamos
    \begin{equation*}
        A_{p}=\begin{pmatrix}
            a_{11} & a_{12} \\
            a_{21} & a_{22}
        \end{pmatrix}
    \end{equation*}
    para la matriz de $A_{p}$ en la base $\{\varphi_{u}(p_{0}),\varphi_{v}(p_{0})\}$. 
    Basta demostrar
    \begin{equation*}
        \begin{pmatrix}
            E & F \\
            F & G
        \end{pmatrix}\begin{pmatrix}
            a_{11} & a_{12} \\
            a_{21} & a_{22}
        \end{pmatrix}=\begin{pmatrix}
            e & f \\
            f & g
        \end{pmatrix}
    \end{equation*}
    Notemos que
    \begin{equation*}
        A_{p}(\varphi_{u}(p_{0}))=a_{11}\varphi_{u}(p_{0})+a_{21}\varphi_{v}(p_{0})
        \hspace{4mm}\text{ y }\hspace{4mm}
        A_{p}(\varphi_{v}(p_{0}))=a_{12}\varphi_{u}(p_{0})+a_{22}\varphi_{v}(p_{0})
    \end{equation*}
    Tomando producto interno de cada ecuación con $\varphi_{u}$ y $\varphi_{v}$ vemos que
    \begin{align*}
        e(p_{0}) &= \ip{A_{p}(\varphi_{u}(p_{0}))}{\varphi_{u}(p_{0})}=
        \ip{a_{11}\varphi_{u}(p_{0})+a_{21}\varphi_{v}(p_{0})}{\varphi_{u}(p_{0})} \\
        &= a_{11}E(p_{0})+a_{21}F(p_{0})
    \end{align*}
    Del mismo modo vemos que
    \begin{align*}
        f(p_{0}) &= \ip{A_{p}(\varphi_{u}(p_{0}))}{\varphi_{v}(p_{0})}
        =a_{11}F(p_{0})+a_{21}G(p_{0}) \\
        f(p_{0}) &= \ip{A_{p}(\varphi_{v}(p_{0}))}{\varphi_{u}(p_{0})}
        =a_{12}E(p_{0})+a_{22}F(p_{0}) \\
        g(p_{0}) &= \ip{A_{p}(\varphi_{v}(p_{0}))}{\varphi_{v}(p_{0})}
        =a_{12}F(p_{0})+a_{22}G(p_{0})
    \end{align*}
\end{dem}
\begin{cor}
    Como resultado se tiene que $k_{1},k_{2}:\Sigma\to\R$ son funciones continuas y son 
    diferenciables en todo $p\in\Sigma$ donde $k_{1}(p)\neq k_{2}(p)$.
\end{cor}
\noindent\textbf{Ejemplo:} Sea $\Sigma=Graf(h)$ con $h:\V\subseteq\R^{2}\to\R$ diferenciable con 
parametrización $\varphi(u,v)=(u,v,h(u,v))$. Recordemmos que
\begin{equation*}
    \varphi_{u}=(1,0,h_{u})\text{ ,}\hspace{4mm}\varphi_{v}=(0,1,h_{v})\text{ ,}\hspace{4mm}
    N=(-\nabla h,1)\frac{1}{(1+h_{u}^{2}+h_{v}^{2})^{1/2}}=\frac{1}{\sqrt{1+h_{u}^{2}+h_{v}^{2}}}
    (-h_{u},h_{v},1)
\end{equation*}
Además, tenemos que
\begin{equation*}
    E=1+h_{u}^{2}\text{ ,}\hspace{4mm}F=h_{u}h_{v}\text{ ,}\hspace{4mm}G=1+h_{v}^{2}
\end{equation*}
y por lo tanto
\begin{equation*}
    EG-F^{2}=(1+h_{u}^{2})(1+h_{v}^{2})-(h_{u}h_{v})^{2}=1+h_{u}^{2}+h_{v}^{2}
\end{equation*}
Buscamos calcular $e,g,f$, para ello veamos que
\begin{equation*}
    \varphi_{uu}=(0,0,h_{uu})\text{ ,}\hspace{4mm}\varphi_{uv}=(0,0,h_{uv})\text{ ,}\hspace{4mm}
    \varphi_{vv}=(0,0,h_{vv})
\end{equation*}
luego
\begin{equation*}
    e=\ip{N}{\varphi_{uu}}=\frac{h_{uu}}{\sqrt{1+h_{u}^{2}+h_{v}^{2}}}\text{ ,}\hspace{4mm}
    f=\ip{N}{\varphi_{uv}}=\frac{h_{uv}}{\sqrt{1+h_{u}^{2}+h_{v}^{2}}}\text{ ,}\hspace{4mm}
    g=\ip{N}{\varphi_{vv}}=\frac{h_{vv}}{\sqrt{1+h_{u}^{2}+h_{v}^{2}}}\text{ ,}
\end{equation*}
entonces
\begin{equation*}
    eg-f^{2}=\frac{h_{uu}h_{vv}-h_{uv}^{2}}{\sqrt{1+h_{u}^{2}+h_{v}^{2}}}
\end{equation*}
Concluimos que
\begin{equation*}
    k_{\Sigma}=\frac{h_{uu}h_{vv}-h_{uv}^{2}}{(1+h_{u}^{2}+h_{v}^{2})^{3/2}}
\end{equation*}
por otro lado
\begin{equation*}
    eG+gE-2Ff=\frac{h_{uu}(1+h_{v}^{2})}{\sqrt{1+h_{u}^{2}+h_{v}^{2}}}+
    \frac{h_{vv}(1+h_{u}^{2})}{\sqrt{1+h_{u}^{2}+h_{v}^{2}}}-
    \frac{2h_{u}h_{v}h_{uv}}{\sqrt{1+h_{u}^{2}+h_{v}^{2}}}
\end{equation*}
y Finalmente
\begin{equation*}
    H_{\Sigma}=\frac{h_{uu}(1+h_{v}^{2})+h_{vv}(1+h_{u}^{2})-2h_{u}h_{v}h_{uv}}
    {(1+h_{u}^{2}+h_{v}^{2})^{3/2}}
\end{equation*}

\noindent\textbf{Ejemplo:} Sea $\varphi(u,v)=(ucosv,usenv,v)$ luego $E=1$, $F=0$ y $G=1+u^{2}$.
Además, $\varphi_{uu}=(0,0,0)$, $\varphi_{uv}=(-senv,cosv,0)$ y $\varphi_{vv}=(-ucosv,-usenv,0)$.
Lo que implica que $e=g=0$ y $f=-\frac{1}{\sqrt{1+u^{2}}}$. Finalmente
\begin{equation*}
    k_{\Sigma}=\frac{0-\left(\frac{1}{1+u^{2}}\right)}{1+u^{2}}=-\frac{1}{(1+u^{2})^{2}}<0
    \text{ ,}\hspace{4mm}H_{\Sigma}=0
\end{equation*}

%\printbibliography % Quitar el comentado si quiero usar bibliografia

\end{document}
