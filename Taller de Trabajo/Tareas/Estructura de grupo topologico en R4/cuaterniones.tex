\documentclass{article}
\usepackage{hyperref}
\usepackage{Style}

\nocite{*} % Comentar si quiero citar
%\addbibresource{bibliografia.bib} % Quitar el comentado si quiero usar bibliografia

\title{Estructura de grupo topológico en $\mathbb{S}^{3}$ y $\mathbb{P}^{3}_{\R}$}
\author{}
\date{}

\begin{document}
\maketitle

\noindent Queremos definir una operación binaria en $\R^{4}\setminus\{0\}$ que es multiplicativa 
con la norma, para ello buscamos definir un producto en $\R^{4}$, un análogo a la multiplicación 
compleja en $\R^{2}$. Dado $x\in\R^{4}$ lo denotamos por $x=(a,u)$ donde $a\in\R$ y $u\in\R^{3}$. 
Sean $(a,u),(b,v)\in\R^{4}$, definimos
\begin{equation*}
    (a,u)(b,v):=(a,u)\cdot(b,v):=(ab-\ip{u}{v},av+bu+u\times v)
\end{equation*}
En primer lugar, demostramos la identidad $(\lambda(a,u))(\mu(b,v))=\lambda\mu((a,u)(b,v))$.
En efecto,
\begin{align*}
    (\lambda(a,u))(\mu(b,v)) &= (\lambda a\mu b-\ip{\lambda u}{\mu v},\lambda a\mu v
    +\mu b\lambda u+(\lambda u)\times(\mu v)) \\[2mm]
    &= \lambda\mu((a,u)(b,v))
\end{align*}
Por otro lado, en busca de nuestro objetivo, notemos que
\begin{align*}
    \norm{(a,u)(b,v)}^{2} &= \norm{(ab-\ip{u}{v},av+bu+u\times v)}^{2} \\[2mm]
    &= a^{2}b^{2}-2ab\ip{u}{v}+\ip{u}{v}^{2}+a^{2}\norm{v}^{2}+b^{2}\norm{u}^{2}+2ab\ip{u}{v}
    +\norm{u}^{2}\norm{v}^{2}-\ip{u}{v}^{2} \\[2mm]
    &= (a^{2}+\norm{u}^{2})(b^{2}+\norm{v}^{2})=\norm{(a,u)}^{2}\norm{(b,v)}^{2}
\end{align*}
y además
\begin{align*}
    ((a,u)+(b,v))(c,w) &= (a+b,u+v)(c,w)=((a+b)c-\ip{u+v}{w},(a+b)w+c(u+v)+(u+v)\times w) \\[2mm]
    &= (a,u)(c,w)+(b,v)(c,w)
\end{align*}
Donde la ultima igualdad se obtiene de que el producto cruz distribuye sobre la suma. Similarmente 
se tiene para $(c,w)((a,u)+(b,v))=(c,w)(a,u)+(c,w)(b,v)$. Ahora que tenemos las propiedades 
deseadas veamos que con esta operación $\R^{4}\setminus\{0\}$ forma un grupo. Afirmamos que el 
elemento neutro es el $(1,0)$. Sea
$(a,u)\in\R^{4}$, entonces
\begin{equation*}
    (a,u)(1,0)=(a-\ip{u}{0},a0+u+u\times0)=(a,u)
\end{equation*}
del mismo modo se tiene que $(1,0)(a,u)=(a,u)$. El inverso viene dado por 
$\frac{1}{\norm{(a,u)}^{2}}(a,-u)$, por lo mencionado anteriormente vemos que
\begin{equation*}
    (a,u)\frac{(a,-u)}{\norm{(a,u)}^{2}}=\frac{1}{\norm{(a,u)}^{2}}(a,u)(a,-u)
    =\frac{1}{\norm{(a,u)}^{2}}(\norm{(a,u)}^{2},-au+au-u\times u)=(1,0)
\end{equation*}
analogamente se tiene la otra igualdad. Resta ver la asociatividad, sean $(b,v),(c,w)\in\R^{4}$, 
luego
\begin{align*}
    ((a,u)(b,v))(c,w) &= (ab-\ip{u}{v},av+bu+u\times v)(c,w) \\[2mm]
    &= ((ab-\ip{u}{v})c-\ip{(av+bu+u\times v)}{w}, \\[2mm]
    &\hspace{5mm}(ab-\ip{u}{v})w+c(av+bu+u\times v)+(av+bu+u\times v)\times w)
\end{align*}
Usando la anticonmutatividad del producto cruz y la identidad $A\times(B\times C)
=B\ip{A}{C}-C\ip{A}{B}$ vemos que
\begin{equation*}
    (u\times v)\times w=-w\times(u\times v)=-u\ip{w}{v}+v\ip{w}{u}
\end{equation*}
entonces $v\ip{u}{w}-w\ip{u}{v}=u\times(v\times w)$, usando la alternancia del determinante 
resulta que $\ip{w}{u\times v}=\ip{u}{v\times w}$ y dado que el producto cruz distribuye sobre la
suma se sigue que
\begin{align*}
    ((a,u)(b,v))(c,w) &= (a(bc-\ip{v}{w})-\ip{u}{cv+bw+v\times w}, \\[2mm]
    &\hspace{5mm}(bc-\ip{v}{w})u+a(bw+cv+v\times w)+u\times(bw+cv+v\times w)) \\[2mm]
    &= (a,u)((b,v)(c,w))
\end{align*}

\noindent Con la topología usual de $\R^{4}$ probaremos que esta operación le da estructura de 
grupo topológico a $\R^{4}\setminus\{0\}$. Como este último, en particular es espacio métrico
inducido por la norma euclideana, dotamos al espacio 
$\R^{4}\setminus\{0\}\times\R^{4}\setminus\{0\}$ de la métrica del máximo. Sea $\tau_{max}$ la 
topología inducida por esta métrica, adicionalmente, consideramos $\tau_{prod}$ la topología 
producto inducida por el espacio topológico $\R^{4}\setminus\{0\}$. Probaremos que estas 
topologías son iguales.

\vspace{2mm}
\noindent Sea $(x_{0},y_{0})\in\R^{4}\setminus\{0\}\times\R^{4}\setminus\{0\}$ y $r,r'\in\R^{+}$. 
En primer lugar, afirmamos que $B_{r}(x_{0},y_{0})=B_{r}(x_{0})\times B_{r}(y_{0})$ según las 
métricas correspondientes. En efecto, se tiene que $(x,y)\in B_{r}(x_{0},y_{0})$ si y solo si 
$\norm{x_{0}-x}<r$ y $\norm{y_{0}-y}<r$ o equivalentemente $(x,y)\in B_{r}(x_{0})\times 
B_{r}(y_{0})$. Lo anterior implica que $\tau_{max}\subseteq\tau_{prod}$. Por otro lado, 
consideramos el abierto básico de $\tau_{prod}$ dado por $B_{r}(x_{0})\times B_{r'}(y_{0})$ veamos 
que es abierto según $\tau_{max}$, sea $(x,y)\in B_{r}(x_{0})\times B_{r'}(y_{0})$, existe 
$d\in\R^{+}$ tal que
\begin{equation*}
    B_{d}(x)\subseteq B_{r}(x_{0})
    \hspace{4mm}\text{y además}\hspace{4mm}
    B_{d}(y)\subseteq B_{r'}(y_{0})
\end{equation*}
de este modo $B_{d}(x,y)\subseteq B_{r}(x_{0})\times B_{r'}(y_{0})$. Concluimos que 
$\tau_{max}=\tau_{prod}$.

\vspace{2mm}
\noindent Sean $(x_{0},y_{0}),(x,y)\in\R^{4}\setminus\{0\}\times\R^{4}\setminus\{0\}$. Notemos lo 
siguiente
\begin{align*}
    \norm{(a_{0},u_{0})(b_{0},v_{0})-(a,u)(b,v)} &= \norm{(a_{0},u_{0})(b_{0},v_{0})
    +(a_{0},u_{0})(b,v)-(a_{0},u_{0})(b,v)-(a,u)(b,v)} \\[2mm]
    &\leq \norm{(a_{0},u_{0})}\cdot\norm{(b_{0},v_{0})-(b,v)}
    +\norm{(b,v)}\cdot\norm{(a_{0},u_{0})-(a,u)}
\end{align*}
así, tomando
\begin{equation*}
    \delta:=\min\left\{\frac{\varepsilon}{2\norm{(a_{0},u_{0})}},
    \frac{\varepsilon}{2(\norm{(b_{0},v_{0})}+1)},1\right\}
\end{equation*}

\vspace{2mm}
\noindent se tiene la continuidad de la función $m:\R^{4}\setminus\{0\}\times\R^{4}\setminus\{0\}
\to\R^{4}\setminus\{0\}$ dada por $m(x,y):=xy$. Por otro lado, como la proyección es continua, la
norma también y esta no se anula en $\R^{4}\setminus\{0\}$ entonces la función $\varphi:
\R^{4}\setminus\{0\}\to\R^{4}\setminus\{0\}$ dada por $\varphi(x):=x^{-1}$ es continua.

\vspace{2mm}
\noindent En $\mathbb{S}^{3}$ definimos $m:\mathbb{S}^{3}\times\mathbb{S}^{3}\to\mathbb{S}^{3}$
por $m(x,y)=xy$ que esta bien definida ya que $\norm{xy}=\norm{x}\cdot\norm{y}=1$ y resulta 
continua por la propiedad universal de la topología de subespacio. Por otro lado consideramos 
$\varphi:\mathbb{S}^{3}\to\mathbb{S}^{3}$ dada por $\varphi(x)=x^{-1}$ que esta bien definida 
puesto que
\begin{equation*}
    1=\norm{(1,0)}=\norm{x\cdot x^{-1}}=\norm{x}\cdot\norm{x^{-1}}=\norm{x^{-1}}
\end{equation*}
y por la misma razón que $m$ es una función continua.

\vspace{2mm}
\noindent Consideremos el diagrama

\centerline{
    \xymatrix{
        \R^{4}\setminus\{0\}\times\R^{4}\setminus\{0\} \ar[d]_{\pi\times\pi} \ar[rd]^{\pi\circ m} 
        \\
        \mathbb{P}^{3}_{\R}\times\mathbb{P}^{3}_{\R} & \mathbb{P}^{3}_{\R}
    }
}
\vspace{2mm}
\noindent Notemos que la función $\pi\times\pi$ es cociente y sabemos que $(\lambda x)(\mu y)
=\lambda\mu (xy)$, luego la función $\pi\circ m$ es constante en las fibras de $\pi\times\pi$ y 
por la propiedad universal de la topología cociente, existe $\widetilde{m}:\mathbb{P}^{3}_{\R}
\times\mathbb{P}^{3}_{\R}\to\mathbb{P}^{3}_{\R}$ continua tal que $\widetilde{m}([x],[y])=[xy]$. 
Del mismo modo la función $\widetilde{\varphi}:\mathbb{P}^{3}_{\R}\to\mathbb{P}^{3}_{\R}$ dada por
$\widetilde{\varphi}([x])=[x^{-1}]$ es continua. 

\vspace{2mm}
\noindent La estructura de grupo en ambos casos se tiene del hecho de que $\cdot$ da estructura de 
grupo a $\R^{4}\setminus\{0\}$.

%\printbibliography % Quitar el comentado si quiero usar bibliografia

\end{document}