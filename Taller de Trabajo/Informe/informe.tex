\documentclass{article}
\usepackage{hyperref}
\usepackage{Style}

\nocite{*} % Comentar si quiero citar
%\addbibresource{bibliografia.bib} % Quitar el comentado si quiero usar bibliografia

\begin{document}

\begin{minipage}{2.5cm}
    \includegraphics[width=2cm]{imagen_puc.jpg}
\end{minipage}
\begin{minipage}{14cm}
    {\sc Pontificia Universidad Católica de Chile\\
    Facultad de Matemáticas\\
    Departamento de Matemática\\
    Profesor: Mauricio Bustamante -- Estudiante: Benjamín Mateluna}
\end{minipage}
\vspace{1ex}

{\centerline{\bf Taller de Trabajo - MAT3094}
\centerline{\bf Informe Taller de Trabajo}}
\centerline{\bf 05 de agosto de 2025}

\newpage
\tableofcontents
% Nueva página que contiene el índice, que referencia todo el contenido visto
% durante el taller.

\newpage
\section*{Introducción}
\phantomsection
\addcontentsline{toc}{section}{Introducción}
% Agregar texto sobre el contexto del taller del trabajo, su motivación, algunos preambulos
% y resultados o resumen de lo trabajado durante el taller. Mencionar los libros en los que o
% fuentes en los que se basa el informe.

\newpage
\section{Homología y Cohomología}
\subsection{Complejos de Cadenas}
\begin{dfn}
    Un \textbf{complejo de cadenas} es una secuencia de grupos abelianos y homomorfismos
    \begin{equation*}
        \cdots\xrightarrow[]{}C_{3}\xrightarrow[]{d_{3}}C_{2}\xrightarrow[]{d_{2}}C_{1}
        \xrightarrow[]{d_{1}}C_{0}\xrightarrow[]{d_{0}}0
    \end{equation*}
    tales que $d_{i}\circ d_{i+1}=0$ para todo $i$. Se denota por la tupla $(C_{*},d_{*}^{C})$
\end{dfn}

\begin{dfn}
    Un \textbf{complejo de cocadenas} es una secuencia de grupos abelianos y homomorfismos
    \begin{equation*}
        0\xrightarrow[]{}C^{0}\xrightarrow[]{d^{0}}C^{1}\xrightarrow[]{d^{1}}C^{2}
        \xrightarrow[]{d^{2}}C^{3}\to\cdots
    \end{equation*}
    tal que $d^{i+1}\circ d^{i}=0$ para todo $i$. Se denota por la tupla $(C^{*},d_{C}^{*})$.
\end{dfn}

\noindent\textbf{Observación:} Los morfismos $d_{i}$ y $d^{i}$ se conocen como diferenciales. Por 
la condición dada, observamos que $im\hspace{1mm}d_{i+1}\subseteq ker\hspace{1mm}d_{i}$ (resp.
$im\hspace{1mm}d^{i}\subseteq ker\hspace{1mm}d_{i+1}$). Los elementos en $\kr{d_{i}}$ se dicen
\textbf{ciclos} (resp. \textbf{cociclos}) y los elementos de $\im{d_{i}}$ se llaman 
\textbf{fronteras} (resp. \textbf{cofronteras}).

\begin{dfn}
    La \textbf{homología} de un complejo de cadenas $C$ es
    \begin{equation*}
        H_{i}(C_{*})=\frac{\kr{d_{i}}}{\im{d_{i+1}}}
    \end{equation*}
    Un elemento en $H_{i}(C_{*})$ se conoce como \textbf{clase de homología}.
\end{dfn}

\begin{dfn}
    La \textbf{cohomología} de un complejo de cocadenas $C$ es
    \begin{equation*}
        H_{i}(C^{*})=\frac{\kr{d_{i}}}{\im{d_{i-1}}}
    \end{equation*}
    Un elemento en $H_{i}(C^{*})$ se conoce como \textbf{clase de cohomología}.
\end{dfn}

\begin{dfn}
    Sean $(C,d_{C})$ y $(D,d_{D})$ complejos de cadenas, un \textbf{mapeo de cadena}, denotado por 
    $C_{*}\to D_{*}$ es una colección de morfismos $f_{n}:C_{n}\to D_{n}$ tales que $d_{n}^{D}
    \circ f_{n}=f_{n-1}\circ d_{n}^{C}$. En otras palabras, el siguiente diagrama conmuta
    
    \vspace{2mm}
    \centerline{
        \xymatrix{
            C_{n} \ar[d]^{d_{n}^{C}} \ar[r]^{f_{n}} & D_{n} \ar[d]^{d_{n}^{D}} \\
            C_{n-1} \ar[r]^{f_{n-1}} & D_{n-1}
        }
    }
\end{dfn}

\vspace{2mm}
\noindent La definición para \textbf{mapeo de cocadena} es análogo.

\begin{lema}
    Sea $f_{*}:C_{*}\to D_{*}$ un mapeo de cadena, entonces $f_{*}:H_{n}(C_{*})\to H_{n}(D_{*})$
    dado por $[x]\to [f_{n}(x)]$ es un morfismo bien definido.
\end{lema}
\begin{proof}
    En primer lugar, debemos verificar que dado un ciclo entonces $f$ lo mapea a un ciclo. 
    Sea $x\in\kr{d_{n}^{C}}$. Luego,
    \begin{equation*}
        d_{n}^{C}(f_{n}(x))=f_{n-1}(d_{n}^{C}(x))=f_{n-1}(0)=0
    \end{equation*}
    Por lo tanto, $f_{n}(x)\in\kr{d_{n}^{D}}$. Veamos que esta bien definido, sean 
    $x,y\in\kr{d_{n}^{C}}$ tales que $[x]=[y]$, entonces $x-y=d_{n+1}^{C}(z)$ para $z\in C_{i+1}$.
    De este modo, $f_{n}(x)-f_{n}(y)=f_{n}(d_{n+1}^{C}(z))=d_{n+1}^{D}(f_{n+1}(z))$ es una 
    frontera. Así, $[f_{n}(x)]=[f_{n}(y)]$.
\end{proof}

\noindent Un resultado análogo se tiene para un mapeo de cocadena.

\subsection{Homología y Cohomología Singular}
\noindent La idea ahora es definir, dado un espacio topologico $X$, un complejo de cadenas 
$C_{*}(X)$, y para cada función continua $f:X\to Y$, construir un mapeo de cadena $f_{*}:C_{*}(X)
\to C_{*}(Y)$. Para luego, obtener los grupos de homología $H_{*}(X)$ de cada espacio y los mapeos
$f_{*}:H_{*}(X)\to H_{*}(Y)$.

\noindent Para esta construcción se utilizara homología singular. La ventaja de este vía, es que
la definicón es una proppiedad instrinseca del espacio, sin embargo, resulta casi imposible 
calcular los complejos de cadenas y por ende, los grupos de homología.

\noindent Todos los resultados de la sección pueden ser dualizados, tomando el funtor Hom.

\begin{dfn}
    Un \textbf{n-simplice estándar} es
    \begin{equation*}
        \Delta^{n}=\left\{
            (t_{0},\cdots,t_{n})\in\R^{n+1}:t_{i}\geq0,\hspace{1mm}\sum t_{i}=1
        \right\}
    \end{equation*}
\end{dfn}

\noindent\textbf{Observación:} La \textbf{i-esima cara} de $\Delta^{n}$ es 
$\Delta^{n}_{i}=\{(t_{0},\cdots,t_{n})\in\Delta^{n}:t_{i}=0\}$. Notamos que un $n-$simplice tiene
$n+1$ caras. Cada cara luce como un $(n-1)-$simplice. De hecho, mediante el mapeo
\begin{align*}
    \delta_{i}:\Delta^{n-1} &\to \Delta^{n} \\
    (t_{0},\cdots,t_{n-1}) &\to (t_{0},\cdots,t_{i-1},0,t_{i},\cdots,t_{n-1})
\end{align*}
resulta ser homeomorfa a $\Delta^{n-1}$.

\begin{dfn}
    Sea $X$ un espacio topologico. Un \textbf{n-simplice singular} en $X$ es una función continua
    $\sigma:\Delta^{n}\to X$.
\end{dfn}

\noindent Dado un espacio $X$, tenemos una colección de $n-$simplices, vamos a considerar el grupo
abeliano libre generado por este conjunto para construir un complejo de cadenas y así los grupos
de homología.

\begin{dfn}
    (\textbf{Complejo de cadena singular}) Definimos
    \begin{equation*}
        C_{n}(X):=\left\{\sum n_{\sigma}\sigma\hspace{1mm}|\hspace{1mm}\sigma:\Delta^{n}\to X
        \text{, }n_{\sigma}\in\Z\hspace{1mm}
        \text{ nulo salvo finitos casos}\right\}
    \end{equation*}
    Junto con los diferenciales $d_{n}:C_{n}(X)\to C_{n-1}(X)$ dada por
    \begin{equation*}
        \sigma\to\sum_{i=0}^{n}(-1)^{i}\sigma\circ\delta_{i}
    \end{equation*}
    que se extiende linealmente.
\end{dfn}

\noindent Para chequear que efectivamente tenemos una complejo de cadenas, tenemos los siguientes 
resultados
\begin{lema}
    Sea $i<j$, entonces $\delta_{j}\circ\delta_{i}=\delta_{i}\circ\delta_{j-1}$.
\end{lema}
\begin{proof}
    Sea $x=(t_{0},\cdots,t_{n-2})\in\Delta^{n-2}$, notemos que
    \begin{equation*}
        \delta_{j}\circ\delta_{i}(x)
        =(t_{0},\cdots,t_{i-1},0,t_{i},\cdots,t_{j-2},0,t_{j-1},\cdots,t_{n-2})
        =\delta_{i}\circ\delta_{j-1}(x)
    \end{equation*}
    lo que prueba el resultado.
\end{proof}

\begin{cor}
    El morfismo $d_{n-1}\circ d_{n}:C_{n}(X)\to C_{n-2}(X)$ es trivial.
\end{cor}
\begin{proof}
    Basta probar para cada elemento en la base, es decir, $\sigma:\Delta^{n}\to X$. Tenemos que
    \begin{equation*}
        d_{n-1}\circ d_{n}(\sigma)=\sum_{i=0}^{n-1}(-1)^{i}\sum_{j=0}^{n}(-1)^{j}
        \sigma\circ\delta_{j}\circ\delta_{i}
    \end{equation*}
    Usando el lema anterior, vemos que
    \begin{align*}
        d_{n-1}\circ d_{n}(\sigma) &= \sum_{i<j}(-1)^{i+j}\sigma\circ\delta_{j}\circ\delta_{i}
        +\sum_{i\geq j}(-1)^{i+j}\sigma\circ\delta_{j}\circ\delta_{i} \\
        &= \sum_{i<j}(-1)^{i+j}\sigma\circ\delta_{i}\circ\delta_{j-1}
        +\sum_{i\geq j}(-1)^{i+j}\sigma\circ\delta_{j}\circ\delta_{i}
    \end{align*}
    haciendo un cambio de índice en la primera expresión, resulta que
    \begin{equation*}
        d_{n-1}\circ d_{n}(\sigma)=\sum_{i\leq j}(-1)^{i+j+1}\sigma\circ\delta_{i}\circ\delta_{j}
        +\sum_{i\geq j}(-1)^{i+j}\sigma\circ\delta_{j}\circ\delta_{i}=0
    \end{equation*}
\end{proof}

\noindent Con lo anterior podemos definir la homología singular de un espacio

\begin{dfn}
    La \textbf{homología singular} de un espacio topologico $X$ es la homología del complejo de
    cadenas $C_{*}(X)$, es decir,
    \begin{equation*}
        H_{i}(X):=H_{i}(C_{*}(X),d_{*})=\frac{\kr{d_{i}}}{\im{d_{i+1}}}
    \end{equation*}
\end{dfn}

\noindent Para definir un complejo de cocadenas debemos trabajar un poco mas. Definimos el grupo 
abeliano
\begin{equation*}
    C^{n}(X)=Hom(C_{n}(X),\Z)
\end{equation*}
con la operación de grupo dada por $(\psi+\varphi)(a)=\psi(a)+\varphi(a)$ con $\psi,\varphi\in 
C_{n}(X)$, se verifica que $\psi+\varphi$ es morfismo. Consideramos el operador dual de $d_{n}$,
denotado por $d^{n}:C^{n}(X)\to C^{n+1}(X)$ y dado por $d^{n}(\varphi):=\varphi\circ d_{n+1}$. 
Observamos que
\begin{equation*}
    0\xrightarrow[]{}C^{0}(X)\xrightarrow[]{d^{0}}C^{1}(X)\xrightarrow[]{d^{1}}\cdots
\end{equation*}
es un complejo de cocadenas, ya que $d^{n+1}(d^{n}(\varphi))=\varphi\circ d_{n+1}\circ d_{n+2}
=\varphi\circ0=0$. Así, la \textbf{cohomología singular} se define como
\begin{equation*}
    H^{i}(X):=H^{i}(C^{*}(X),d^{*})=\frac{\kr{d^{i}}}{\im{d^{i-1}}}
\end{equation*}

\begin{prop}
    Sea $f:X\to Y$ una función continua de espacios topologicos, entonces las funciones
    \begin{align*}
        f_{n}:C_{n}(X) &\to C_{n}(Y)\htext{dada por} \\
        \sigma &\to f\circ\sigma
    \end{align*}
    da un mapeo de cadena. Como consecuencia del primer lema, induce un mapeo en la homología 
    (cohomología).
\end{prop}

%\printbibliography % Quitar el comentado si quiero usar bibliografía

\end{document}
