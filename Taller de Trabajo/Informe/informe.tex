\documentclass{article}
\usepackage{hyperref}
\usepackage{Style}

\nocite{*} % Comentar si quiero citar
%\addbibresource{bibliografia.bib} % Quitar el comentado si quiero usar bibliografia

\begin{document}

\begin{minipage}{2.5cm}
    \includegraphics[width=2cm]{imagen_puc.jpg}
\end{minipage}
\begin{minipage}{14cm}
    {\sc Pontificia Universidad Católica de Chile\\
    Facultad de Matemáticas\\
    Departamento de Matemática\\
    Profesor: Mauricio Bustamante -- Estudiante: Benjamín Mateluna}
\end{minipage}
\vspace{1ex}

{\centerline{\bf Taller de Trabajo - MAT3094}
\centerline{\bf Informe Taller de Trabajo}}
\centerline{\bf 04 de agosto de 2025}

\newpage
\tableofcontents
% Nueva página que contiene el índice, que referencia todo el contenido visto
% durante el taller.

\newpage
\section*{Introducción}
% Agregar texto sobre el contexto del taller del trabajo, su motivación, algunos preambulos
% y resultados o resumen de lo trabajado durante el taller. Mencionar los libros en los que o
% fuentes en los que se basa el informe.

\newpage
\section{Preliminares}
\subsection{Simplices} % Munkres
\subsection{Complejos Simpliciales y Mapas Simpliciales} % Munkres
\subsection{Complejos Simpliciales Abstractos} % Munkres
\subsection{Revisión de Grupos Abelianos} % MUnkres
\subsection{Grupos de Homología} % Munkres
\subsection{Categorías y Funtores} % Munkres

% Escribir sobre los preliminares para poder definir el grupo de cohomología y hablar del funtor
% hom.

\newpage
\section{Cohomología}
\subsection{El Funtor Hom}
\begin{dfn}
    Sean $A$ y $G$ grupos abelianos, entonces el conjunto $Hom(A,G)$ que consiste en todos los
    morfismos de $A$ a $G$ es un grupo abeliano con la operación $(\varphi+\psi)(a)
    :=\varphi(a)+\psi(a)$.
\end{dfn}

\vspace{2mm}
\noindent Hay que probar que esta operación esta bien definida, esto es, debemos verificar que 
$\varphi+\psi$ es un morfismo de $A$ a $G$, notemos que
\begin{equation*}
    (\varphi+\psi)(a+b)=\varphi(a+b)+\psi(a+b)=\varphi(a)+\psi(a)+\varphi(b)+\psi(b)
    =(\varphi+\psi)(a)+(\varphi+\psi)(b)
\end{equation*}
La identidad en $Hom(A,G)$ resulta ser el morfismo trivial y el inverso de $\varphi$ es 
$-\varphi$.

\vspace{2mm}
\noindent\textbf{Ejemplo:} % Desarrollar el ejemplo de munkres

\begin{dfn}
    Un morfismo $f:A\to B$ da resultado a un \textbf{morfismo dual} $\widetilde{f}:Hom(B,G)\to 
    Hom(A,G)$. Donde $\widetilde{f}$ le asigna a un morfismo $\varphi:B\to G$ el morfismo 
    $\widetilde{f}(\varphi)=\varphi\circ f$.
\end{dfn}

\vspace{2mm}
\noindent Se verifica que $\widetilde{f}$ esta bien definido, en otras palabras, que 
$\widetilde{f}(\varphi)$ es un morfismo. Para $G$ fijo, la asignación $A\to Hom(A,G)$ y 
$f\to\widetilde{f}$ define un funtor contravariante de la categoría de grupos abelianos y morfismos
a si mismo.

\subsection{Grupo de Cohomología Simplicial}

%\printbibliography % Quitar el comentado si quiero usar bibliografía

\end{document}
