\documentclass[11pt, %t,
	aspectratio=169
]{beamer}

\usepackage{booktabs} % Se usa para \toprule, \midrule y \bottomrule mejorar reglas en las tablas
\usepackage{palatino}
\usepackage[default]{opensans}
\usepackage[spanish]{babel}

\usepackage[all]{xy} %% Para los diagramas conmutativos

\usepackage{animate}
\usepackage{tikz}

%% Comandos o definiciones nuevas
\theoremstyle{plain}
\newtheorem{axioma}{Axioma}
\newtheorem{af}[axioma]{Afirmación}
\newtheorem{teo}{Teorema}[section]
\newtheorem{lema}[teo]{Lema}
\newtheorem{prop}[teo]{Proposición}
\newtheorem{cor}[teo]{Corolario}

\theoremstyle{remark}
\newtheorem{dfn}[teo]{Definición}
\newtheorem*{ej}{Ejemplo}
\newtheorem*{obs}{Observación}

%% Nuevos comandos específicos, necesarios para simplificar la escritura

%% Norma y valor absoluto
\newcommand{\norm}[1]{\left \lVert #1\right \rVert}
\newcommand{\abs}[1]{\left|#1 \right|}

%% Derivada y derivada parcial
\newcommand{\dv}[2]{\frac{d#1}{d#2}}
\newcommand{\pdv}[2]{\frac{\partial#1}{\partial#2}}

%% Producto interno y conjunto generador
\newcommand{\ip}[2]{\left\langle{#1},{#2}\right\rangle}
\newcommand{\gen}[1]{\left\langle{#1}\right\rangle}

%% Otros
\newcommand{\htext}[1]{\hspace{4mm}\text{#1 }}
\newcommand{\hhtext}[1]{\hspace{4mm}\text{#1}\hspace{4mm}}
\newcommand{\im}[1]{im\hspace{1mm}#1}
\newcommand{\kr}[1]{ker\hspace{1mm}#1}
\newcommand\sbullet[1][.5]{\mathbin{\vcenter{\hbox{\scalebox{#1}{$\bullet$}}}}}

%% Simplificar comandos
\def \ds {\displaystyle}
\def \C {\mathbb{C}}
\def \R {\mathbb{R}}
\def \Q {\mathbb{Q}}
\def \Z {\mathbb{Z}}
\def \N {\mathbb{N}}
\def \S {\mathbb{S}}
\def \H {\mathbb{H}}
\def \P {\mathbb{P}}
\def \T {\mathbb{T}}

\def \U {\mathcal{U}}

%% Indica la carpeta donde se ubican las imágenes
\graphicspath{{Images/}{Images/homotopía/}}

%% Tema del Beamer
\usetheme{Berlin}

%% Otras configuraciones
\usefonttheme{default}
\useinnertheme{circles}
\setbeamertemplate{footline}

%% Presentación de la disertación/defensa
\title{
	Obstrucciones homotópicas a la existencia de estructuras de grupo topológico
}

\subtitle{Tesis de Pregrado}
\author{Benjamín Mateluna Medina}
\institute{
	Pontificia Universidad Católica de Chile \\ \smallskip \textit{bmateluna@uc.cl}
}
\date{\today}

%% Inicio del código...
\begin{document}

%% Diapositiva --- Presentación
\begin{frame}
	\titlepage
\end{frame}

%% Resumen del trabajo/tesis
\begin{frame} %% Esta parte esta lista
	\frametitle{Resumen}
	Estudiaremos la noción de grupo topológico, analizaremos obstrucciones topológicas para que un
	espacio admita dicha estructura. Se usarán herramientas de la topología algebraica, tales como
	\pause

	\begin{itemize}
		\itemsep0.5em
		
		\item Cohomología singular \pause
		
		\item Espacios cubrientes \pause
		
		\item Grupo fundamental \pause
	\end{itemize}
	Uno de los resultados principales, consecuencias de otros tres, es que la única superficie
	(una $2-$variedad) compacta y conexa que es grupo topológico es el toro. También veremos que
	ninguna esfera de dimensión par lo es.
\end{frame}

%% Diapositiva --- Estructura de la Disertación
\begin{frame}
	\frametitle{Esquema de la Presentación}
	
	\tableofcontents
\end{frame}

%% Dar a conocer el objetivo de la tesis e introducir la noción de grupos topologicos
\section{Introducción}

%% Definimos el objeto principal a estudiar.
\subsection{Grupo topológico} %% Esta parte esta lista

\begin{frame}
	\frametitle{Grupo Topológico}
	¿Qué es un grupo topológico? 
	
	\vspace{1mm}
	Un \emph{grupo topológico} es un espacio $X$, equipado con una aplicación continua,
	\[
		m:X\times X\to X
	\]
	y un elemento $e\in X$ que actúa como la identidad, de modo que $(X,m,e)$ es grupo. Además,
	exigimos que que el mapa de inversión
	\[
		\varphi:X\to X, \qquad \varphi(x)=x^{-1}
	\]
	sea continuo.
\end{frame}

%% Ejemplos para familiarizarse con el concepto
\subsection{Ejemplos} %% Esta parte esta lista

\begin{frame}
	\frametitle{Ejemplos}
	Los primeros ejemplos surgen de manera natural, como $\R$, $\C$, $GL_{n}(\R)$ ó $GL_{n}(\C)$ 
	son grupos topológicos. Menos trivial, también tenemos los espacios
	
	%% Ejemplos menos triviales, el círculo y el toro, explicar ejemplos
	\begin{figure}
		\begin{minipage}[b]{0.4\linewidth}
			\centering
			\includegraphics[scale=0.15]{circulo.png}
			\caption{El círculo ó $\S^{1}$}
		\end{minipage}
		\hspace{1mm}
		\begin{minipage}[b]{0.4\linewidth}
			\centering
			\includegraphics[scale=0.1]{toro.png}
			\caption{El toro ó $\T^{2}$}
		\end{minipage}
	\end{figure}
\end{frame}

%% Ejemplos no triviales
\begin{frame}
	Para los siguientes ejemplos requerimos de los cuaterniones, denotados por $\H$, que 
	corresponden a los elementos de la forma
	\[
		q=a+bi+cj+dk \qquad \text{con }a,b,c,d\in\R
	\]
	donde $i^{2}=j^{2}=k^{2}=ijk=-1$. \pause

	\vspace{1mm}
	Se suman coordenada a coordenada y se pueden multiplicar usando las relaciones dadas (como los 
	números complejos). Forman un álgebra de división, equipado con una norma, que corresponde a 
	la norma en $\R^{4}$, por lo que la identificación
	\[
		(a,b,c,d)\mapsto q=a+bi+cj+dk
	\]
	es isometría.
\end{frame}

\begin{frame}
	Al igual que con $\S^{1}$, la norma es multiplicativa y por lo tanto $\S^{3}\subseteq
	\H\setminus\{0\}$ es subgrupo. La multiplicación y el mapa de inversión son continuos, luego
	$\S^{3}$ es grupo topológico. \pause

	\vspace{1mm}
	Podemos, además, definir una estructura de grupo topológico en $\R\P^{3}=\S^{3}/\{\pm1\}$ 
	usando la multiplicación en $\S^{3}$, que desciende a una en $\R\P^{3}$, ya que $\{\pm1\}$ 
	esta en el centro.

	\centerline{
    \xymatrix{
            \S^{3}\times\S^{3} \ar[d]_{\pi\times\pi} 
            \ar[rd]^{\pi\circ m} \\
            \R\P^{3}\times\R\P^{3} \ar@{-->}[r] & \R\P^{3}
        }
    }
    \vspace{2mm}
	La construcción de los últimos dos ejemplos no es sencilla.
\end{frame}

\section*{}
%% Conectar el tema anterior, ejemplos, con el siguiente, obstrucciones topológicas
\begin{frame}
	\begin{center}
		{\huge \textbf{¿Qué obstrucciones topológicas surgen para ser grupo topológico?}}
		
		\vspace{2mm}
		Buscamos condiciones necesarias que debe cumplir un espacio para tener dicha estructura.
	\end{center}
\end{frame}

%% Obstrucciones para ser grupo topológico
\section{Obstrucciones}

\subsection{Grupo fundamental} %% Esta parte esta lista

%% Grupo fundamental
\begin{frame}
	\begin{center}
		{\huge \textbf{Grupo Fundamental}}
	\end{center}

	\centering
	\includegraphics[scale=0.3]{lazo basado.png}
	%\animategraphics[loop,autoplay,width=0.3\textwidth]{45}{frame-}{001}{141}
\end{frame}

%% El grupo fundamental es abeliano
\begin{frame}
	La primera obstrucción que aparece y también la más accesible se enuncia en el teorema que 
	sigue:

	\begin{alertblock}{Teorema}
		Sea $X$ un grupo topológico arcoconexo. Entonces $\pi_{1}(X,e)$ es abeliano.
	\end{alertblock}

	El primer no ejemplo que surge es $\S^{1}\vee\S^{1}$, cuyo grupo fundamental es $F_{2}$. 
	También, es posible calcular los grupos fundamentales de las superficies
	\[
		\T^{2}\#\T^{2}\#\cdots\#\T^{2} \qquad y \qquad \R\P^{2}\#\R\P^{2}\#\cdots\#\R\P^{2}
	\]
	Que corresponden a las superficies de género $g>1$, en ambos casos, los grupos fundamentales
	no son abelianos.
\end{frame}

\subsection{Espacios cubrientes} %% Esta parte esta lista

%% Espacios cubrientes
\begin{frame}
	\begin{center}
		{\huge \textbf{Espacios cubrientes}}
	\end{center}

	\centering
	\includegraphics[scale=0.25]{espacio-cubriente.png}
\end{frame}

%% El cubriente universal es grupo topológico
\begin{frame}
	De la mano de la primera obstrucción, surge la teoría de espacios cubrientes como una manera 
	de calcular el grupo fundamental de un espacio. Sin embargo, nuestro interés radica en el 
	siguiente teorema:

	\begin{alertblock}{Teorema}
		El cubriente universal de un grupo topológico es grupo topológico.
	\end{alertblock}

	Para nuestro objetivo buscamos el cubriente universal del plano proyectivo ($\R\P^{2}$), que
	resulta ser $\S^{2}$ y más adelante veremos que este último no es grupo topológico. Así, el 
	plano proyectivo tampoco es grupo topológico.
\end{frame}

\subsection{Cohomología singular} %% Esta parte esta lista

%% Cohomología singular
\begin{frame}
	\begin{center}
		{\huge \textbf{Cohomología singular}}
	\end{center}

	\centering
	\includegraphics[scale=0.2]{simplice-singular.png}
\end{frame}

%% El anillo de cohomología es álgebra de hopf
\begin{frame}{Anillo de Cohomología}
	Dado un anillo conmutativo $R$, a un espacio $X$ le asignamos un $R-$módulo, denotado por 
	$H^{n}(X;R)$, para cada $n\in\N$. En conjunto, forman el anillo de cohomología
	\[
		H^{*}(X;R):=\bigoplus_{n\in\N}H^{n}(X;R)
	\]
	Que tiene estructura de álgebra graduada. \pause Por ejemplo, los anillos de cohomología de 
	$\S^{n}$ y $\T^{2}$ son isomorfos, como algebras graduadas, a lo siguiente:
	\[
		H^{*}(\S^{n};\Z)\cong\Z[x]/(x^{2}) \qquad \text{donde } \abs{x}=n
	\]
	y
	\[
		H^{*}(\T^{2};\Z)\cong\Z[x,y]/(x^{2},y^{2},xy+yx) \qquad \text{donde } \abs{x},\abs{y}=1
	\]
\end{frame}

\begin{frame}{Algebras de Hopf}
	\begin{dfn}
		Sea $A$ un álgebra graduada, se dice álgebra de Hopf si cumple:
		\begin{enumerate}
			\item Existe una unidad $1\in A_{0}$ tal que el morfismo $R\to A_{0}$ dado por 
			$r\mapsto r\cdot1$ es isomorfismo.

			\item Existe un morfismo de algebras graduadas $\triangle:A\to A\otimes A$, llamado
			coproducto, que satisface
			\[
				\triangle(\alpha)=\alpha\otimes1+1\otimes\alpha
				+\sum_{i}\alpha'_{i}\otimes\alpha''_{n-i}
			\]
			donde $\abs{\alpha'_{i}}$, $\abs{\alpha''_{n-i}}>0$ para todo $\alpha\in A_{n}$ con
			$n>0$.
		\end{enumerate}
	\end{dfn}
\end{frame}

\begin{frame}
	\begin{alertblock}{Teorema}
		Sea $X$ un grupo topológico arcoconexo tal que $H^{n}(X;R)$ es un $R-$módulo libre f.g. 
		para todo $n\in\N$. Entonces $H^{*}(X;R)$ es un álgebra de Hopf.
	\end{alertblock} \pause
	En el caso de $\S^{n}$ el coproducto viene dado por $\triangle(x)=x\otimes1+1\otimes x$.
	Como $\triangle$ es morfismo de algebras, resulta que
	\[
		0=\triangle(x^{2})=2\cdot x\otimes x \hspace{4mm}\text{cuando $\abs{x}$ par}
		\qquad y \qquad
		\triangle(x^{2})=0 \hspace{4mm}\text{cuando $\abs{x}$ impar}
	\]
	En el primer caso vemos que $2=0$ en $\Z$. \pause Concluimos que
	\begin{alertblock}{Teorema}
		Sea $n\in\N$. Si $\S^{n}$ es grupo topológico entonces $n$ es impar.
	\end{alertblock}
\end{frame}

%% Implicancias de las obstrucciones en superficies
\section{Superficies} %% Esta parte esta lista

\begin{frame}{Superficies}
	Hasta ahora tenemos que el toro $\T^{2}$ es una superficie compacta y conexa que es grupo
	topológico. Por otro lado, hemos descartado otras superficies como $\S^{2}$, $\R\P^{2}$ y 
	superficies de género $g>1$, el siguiente teorema las clasifica:

	\begin{exampleblock}{Teorema de clasificación de superficies}
		Toda superficie, no vacía, compacta y conexa es homeomorfa a una de las siguientes
		\begin{enumerate}
			\item La $2-$esfera, es decir, $\S^{2}$.
			\item Superficies orientables de género $g$.
			\item Superficies no orientables de género $g$.
		\end{enumerate}
	\end{exampleblock}
\end{frame}

\section{Generalización} %% Esta parte esta lista

%% Nota al pie de página
\begin{frame}{Nota al pie de página}
	En un ánimo de generalizar los resultados, podemos relajar la noción de grupo topológico hasta 
	homotopía. Introducimos los $H-$espacios:

	\begin{dfn}
		Sea $X$ un espacio topológico. Decimos que $X$ es un $H$-espacio si existe una aplicación
		continua $\mu: X\times X \to X$ y un elemento $e\in X$ (identidad) tales que los mapeos
		\begin{equation*}
			\phi(x)=\mu(x,e), \qquad \psi(x)=\mu(e,x)
		\end{equation*}
		son homotópicos a la identidad, relativos a $\{e\}$.
	\end{dfn}
\end{frame}

%% 1min y 40seg por diapositiva

\end{document} 