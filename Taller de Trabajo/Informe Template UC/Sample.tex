\documentclass[aop]{imsart2}

%% Packages
\RequirePackage{amsthm,amsmath,amsfonts,amssymb}
\RequirePackage[numbers]{natbib}
\RequirePackage[colorlinks, citecolor=blue, urlcolor=blue]{hyperref}
\RequirePackage{graphicx}

\usepackage[spanish]{babel}
\usepackage[utf8]{inputenc}

%% Arregla error al intentar usar un tamaño de font no disponible
\usepackage{anyfontsize}

\usepackage{tikz} %% Se utiliza para hacer gráficos
\usepackage[all]{xy} %% Para los diagramas conmutativos
\usetikzlibrary{calc}

\usepackage{graphicx} %% Requerido para insertar imagenes

\startlocaldefs

\theoremstyle{plain}
\newtheorem{axioma}{Axioma}
\newtheorem{af}[axioma]{Afirmación}
\newtheorem{teo}{Teorema}[section]
\newtheorem{lema}[teo]{Lema}
\newtheorem{prop}[teo]{Proposición}
\newtheorem{cor}[teo]{Corolario}

\theoremstyle{remark}
\newtheorem{dfn}[teo]{Definición}
\newtheorem*{ej}{Ejemplo}
\newtheorem*{obs}{Observación}

%% Nuevos comandos específicos, necesarios para simplificar la escritura

%% Norma y valor absoluto
\newcommand{\norm}[1]{\left \lVert #1\right \rVert}
\newcommand{\abs}[1]{\left|#1 \right|}

%% Derivada y derivada parcial
\newcommand{\dv}[2]{\frac{d#1}{d#2}}
\newcommand{\pdv}[2]{\frac{\partial#1}{\partial#2}}

%% Producto interno y conjunto generador
\newcommand{\ip}[2]{\left\langle{#1},{#2}\right\rangle}
\newcommand{\gen}[1]{\left\langle{#1}\right\rangle}

%% Otros
\newcommand{\htext}[1]{\hspace{4mm}\text{#1 }}
\newcommand{\hhtext}[1]{\hspace{4mm}\text{#1}\hspace{4mm}}
\newcommand{\im}[1]{im\hspace{1mm}#1}
\newcommand{\kr}[1]{ker\hspace{1mm}#1}
\newcommand\sbullet[1][.5]{\mathbin{\vcenter{\hbox{\scalebox{#1}{$\bullet$}}}}}

%% Simplificar comandos

\def \ds {\displaystyle}
\def \C {\mathbb{C}}
\def \R {\mathbb{R}}
\def \Q {\mathbb{Q}}
\def \Z {\mathbb{Z}}
\def \N {\mathbb{N}}
\def \S {\mathbb{S}}

\endlocaldefs

\begin{document}

\begin{titlepage}
    \hspace*{-3cm}
    \raisebox{-1cm}[0pt][0pt]{
        \includegraphics[width=0.4\textwidth]{logo_universidad2.png}
    }

    \vspace{2cm}
    \begin{center}
        {\Large Pontificia Universidad Católica de Chile}
        
        \vspace{2cm}
        {\Huge \bfseries Título del Documento}
        
        \vspace{1.5cm}
        {\Large Benjamín Mateluna}
        
        \vspace{1.5cm}
        {\Large Docente Guía: Mauricio Bustamante}
        \vfill
        
        {\large \today \par}
    \end{center}
\end{titlepage}

\tableofcontents

\newpage
\section{Resumen}

\section{Introduction}

\newpage
\section{Homología}
Para empezar, debemos estudiar algunas definiciones y resultados de álgebra homológica y así 
establecer . Se comenzará por complejos de cadenas y cocadenas y algunos resultados menores. En la 
parte de ``Resultados de Homología'' se verán herramientas importantes para trabajar con los 
grupos de homología.

\subsection{Complejos de Cadenas}

\begin{dfn}
    Un complejo de cadenas es una sucesión de grupos abelianos y homomorfismos
    
    \vspace{2mm}
    \centerline{
        \xymatrix{
            \cdots \ar[r] & C_{3} \ar[r]^{\partial_{3}} & C_{2} \ar[r]^{\partial_{2}} & 
            C_{1} \ar[r]^{\partial_{1}} & C_{0} \ar[r] & 0
        }
    }
    \vspace{2mm}
    \noindent tal que $\partial_{i}\circ \partial_{i+1}=0$ para todo $i$. Se denota por 
    $(C_{\sbullet},\partial_{\sbullet})$.
\end{dfn}

\begin{obs}
    Notemos que $\im{\partial_{i+1}}\subseteq\kr{\partial_{i}}\subseteq C_{i}$. Dado que los 
    grupos son abelianos, esta observación permite definir el siguiente grupo.
\end{obs}

\begin{dfn}
    El $i-$ésimo grupo de homología de $(C_{\sbullet},\partial_{\sbullet})$ se define por
    \begin{equation*}
        H_{i}(C_{\sbullet}):=\frac{\kr{\partial_{i}}}{\im{\partial_{i+1}}}
    \end{equation*}
\end{dfn}

\begin{ej}
    Veamos que
    
    \vspace{2mm}
    \centerline{
        \xymatrix{
            \cdots \ar[r] & 0 \ar[r]^{\cdot0} & \Z \ar[r]^{\cdot2} & \Z \ar[r]^{\cdot0} 
            & \Z \ar[r] & 0
        }
    }
    \vspace{2mm}

    \noindent es un complejo de cadenas, donde los grupos de homología asociados son
    $H_{0}(C_{\sbullet})=\Z$, $H_{1}(C_{\sbullet})=\Z_{2}$ y $H_{k}(C_{\sbullet})=0$ para 
    $k\neq0,1$.
\end{ej}

\noindent Los elementos en $\kr{\partial_{i}}$ e $\im{\partial_{i}}$ se llaman ciclos y fronteras,
respectivamente. Un elemento en $H_{i}(C_{\sbullet})$ se dice clase de homología. Los elementos en
los grupos abelianos $C_{i}$ se conocen como cadenas y los morfismos $\partial_{i}$ como 
diferenciales.

\vspace{1mm}
\noindent Ahora que hemos introducido lo básico sobre homología, queremos estudiar como 
interactúan estos objetos entre sí, buscamos una noción de morfismo entre las cadenas que además
induzca uno entre los grupos de homología, esto motiva la siguiente definición.

\begin{dfn}
    Sean $(C_{\sbullet},\partial_{\sbullet})$ y $(D_{\sbullet},\partial_{\sbullet})$ dos complejos 
    de cadenas. Un mapeo de cadenas es una colección de homomorfismos $f_{n}:C_{n}\to D_{n}$ 
    tal que $\partial_{n}f_{n}=f_{n-1}\partial_{n}$ para todo $n$, es decir, el siguiente diagrama 
    conmuta

    \vspace{2mm}
    \centerline{
        \xymatrix{
            C_{n} \ar[r]^{\partial_{n}} \ar[d]^{f_{n}} & C_{n-1} \ar[d]^{f_{n-1}} \\
            D_{n} \ar[r]^{\partial_{n}} & D_{n-1}
        }
    }
    \noindent y se denota por $f_{\sbullet}:C_{\sbullet}\to D_{\sbullet}$.
\end{dfn}

\begin{prop}
    Si $f_{\sbullet}:C_{\sbullet}\to D_{\sbullet}$ es un mapeo de cadenas, entonces la asignación 
    $f_{*}:H_{n}(C_{\sbullet})\to H_{n}(D_{\sbullet})$ dada por
    \begin{equation*}
        f_{*}([x])=[f_{n}(x)]
    \end{equation*}
    esta bien definida y es un homomorfismo de grupos.
\end{prop}

\begin{proof}
    Sea $x\in ker\partial_{n}$ entonces $\partial_{n}f_{n}(x)=f_{n-1}\partial_{n}(x)
    =f_{n-1}(0)=0$. Así, $f_{n}(x)\in\kr{\partial_{n}}$ y por lo tanto la expresión tiene sentido. 
    
    \vspace{1mm}
    \noindent Si $[x]=[y]$ entonces $x-y=\partial_{n+1}(z)$ para $z\in C_{n+1}$, se sigue que 
    $f_{n}(x)-f_{n}(y)=f_{n}\partial_{n+1}(z)=\partial_{n+1}f_{n+1}(z)$. Concluimos que 
    $[f_{n}(x)]=[f_{n}(y)]$. Que sea homomorfismo es directo de la definición.
\end{proof}

\begin{ej}
    Consideremos la siguiente situación

    \vspace{2mm}
    \centerline{
        \xymatrix{
            \cdots \ar[r] & 0 \ar[r]^{0} \ar[d] & \Z \ar[r]^{3} \ar[d]^{id} & 
            \Z \ar[r]^{0} \ar[d]^{\pi} & \Z \ar[r] \ar[d]^{id} & 0 
            & C_{\sbullet} \ar[d]^{f_{\sbullet}} \\
            \cdots \ar[r] & 0 \ar[r]^{0} & \Z \ar[r]^{3} & \Z_{3} \ar[r]^{0} & \Z \ar[r] & 0 
            & D_{\sbullet}
        }
    }
    \vspace{2mm}
    \noindent Con $f_{\sbullet}$ un mapeo de cadenas. Entonces $f_{*}:H_{2}(C_{\sbullet})\to 
    H_{2}(D_{\sbullet})$ es el morfismo trivial, ya que $H_{2}(C_{\sbullet})=0$. Mientras que 
    $\pi_{*}:H_{1}(C_{\sbullet})=\Z_{3}\to H_{1}(D_{\sbullet})=\Z_{3}$ es la identidad.
\end{ej}

\begin{obs}
    Notemos que si $g_{\sbullet}:D_{\sbullet}\to G_{\sbullet}$ es un mapeo de cadenas, entonces la 
    colección de morfimos $(g\circ f)_{\sbullet}:C_{\sbullet}\to G_{\sbullet}$ es un mapeo de 
    cadenas y el siguiente diagrama conmuta
    
    \vspace{2mm}
        \centerline{
            \xymatrix{
                H_{n}(C_{\sbullet}) \ar[rr]^{(g\circ f)_{*}} \ar[rd]^{f_{*}} & 
                & H_{n}(G_{\sbullet}) \\
                & H_{n}(D_{\sbullet}) \ar[ru]^{g_{*}}
            }
        }
    \vspace{2mm}
    \noindent En efecto, $\partial_{n}g_{n} f_{n}=g_{n-1}\partial_{n}f_{n}=g_{n-1}f_{n-1}
    \partial_{n}$. Por otro lado, tenemos que $(g\circ f)_{*}([x])=[(g\circ f)(x)]=g_{*}([f(x)])=
    (g_{*}\circ f_{*})([x])$, lo que prueba la afirmación.
\end{obs}

\begin{dfn}
    Sean $f_{\sbullet},g_{\sbullet}:C_{\sbullet}\to D_{\sbullet}$ mapeos de cadenas. Una 
    homotopía de cadenas es una colección de morfismos
    \begin{align*}
        h_{n}:C_{n}\to C_{n+1}\htext{tales que} \\
        f_{n}-g_{n}=\partial h_{n}+h_{n-1}\partial
    \end{align*}
    Lo denotamos como $f_{\sbullet}\sim g_{\sbullet}$.
\end{dfn}

\begin{prop}
    Sea $f_{\sbullet}\sim g_{\sbullet}$ entonces $f_{*}=g_{*}$.
\end{prop}

\begin{proof}
    Sea $[x]\in H_{n}(C_{\bullet})$, por definición, sabemos que $\partial x=0$, luego
    \begin{equation*}
        (f_{*}-g_{*})([x])=[(f-g)(x)]=[(\partial h+h\partial)(x)]=[\partial hx]=0
    \end{equation*}
    lo que prueba la afirmación.
\end{proof}

\subsection{Resultados sobre Homología}
Los siguientes resultados van a permitir trabajar en mejor modo con los grupos de homología de
un complejo de cadenas y también permitirán establecer, más adelante, la invarianza homotópica.

\vspace{1mm}
\noindent A partir de este momento los índices para indicar los morfismos $\partial_{i}$ y entre 
complejos de cadenas no se escribirán, a no ser que se de una definición, para mayor comodididad. 

\begin{dfn}
    Sean $i_{\sbullet}:A_{\sbullet}\to B_{\sbullet}$ y $j_{\sbullet}:B_{\sbullet}\to C_{\sbullet}$
    dos mapeos de cadenas. Decimos que forman una secuencia exacta corta si la secuencia

    \vspace{2mm}
    \centerline{
        \xymatrix{
            0 \ar[r] & A_{n} \ar[r]^{i_{n}} & B_{n} \ar[r]^{j_{n}} & C_{n} \ar[r] & 0
        }
    }
    \vspace{2mm}
    \noindent es una secuencia exacta y corta de grupos abelianos libres para todo $n\in\N$. Lo 
    denotamos como $0\xrightarrow[]{} A_{\sbullet}\xrightarrow[]{i_{\sbullet}} 
    B_{\sbullet}\xrightarrow[]{j_{\sbullet}} C_{\sbullet}\xrightarrow[]{}0$.
\end{dfn}

\begin{teo}[Lema de la serpiente]
    Sea $0\to A_{\sbullet}\to B_{\sbullet}\to C_{\sbullet}\to0$ una secuencia de complejos de 
    cadenas, entonces existen morfismos
    \begin{equation*}
        \delta_{n}:H_{n}(C_{\sbullet})\to H_{n-1}(A_{\sbullet})
    \end{equation*}
    tales que la siguiente secuencia es exacta
    
    \vspace{2mm}
    \centerline{
        \xymatrix{
            \ar[r]^-{\delta_{n+1}} & H_{n}(A_{\sbullet}) \ar[r]^{i_{*}} 
            & H_{n}(B_{\sbullet}) \ar[r]^{j_{*}} & H_{n}(C_{\sbullet}) 
            \ar `[dl] `[l] `[llld]_{\delta_{n}} `[d] [dll] \\
            & H_{n-1}(A_{\sbullet}) \ar[r]^{i_{*}} & H_{n-1}(B_{\sbullet}) \ar[r]^{j_{*}} 
            & H_{n-1}(C_{\sbullet}) \ar[r] & \cdots \\
            & \cdots \ar[r] & H_{0}(B_{\sbullet}) \ar[r] & H_{0}(C_{\sbullet}) \ar[r] & 0 
        }
    }
\end{teo}

\begin{proof}
    Vamos a hacer lo que se conoce como una cacería de diagramas. Consideremos el diagrama

    \vspace{2mm}
    \centerline{
        \xymatrix{
            & \vdots \ar[d] & \vdots \ar[d] & \vdots \ar[d] & \\
            0 \ar[r] & A_{n+1} \ar[r]^{i} \ar[d]^{\partial} & B_{n+1} \ar[r]^{j} \ar[d]^{\partial} 
            & C_{n+1} \ar[r] \ar[d]^{\partial} & 0 \\
            0 \ar[r] & A_{n} \ar[r]^{i} \ar[d]^{\partial} & B_{n} \ar[r]^{j} \ar[d]^{\partial} 
            & C_{n} \ar[r] \ar[d]^{\partial} & 0 \\
            0 \ar[r] & A_{n-1} \ar[r]^{i} \ar[d] & B_{n-1} \ar[r]^{j} \ar[d] 
            & C_{n-1} \ar[r] \ar[d] & 0 \\
            & \vdots & \vdots & \vdots &
        }
    }
    \vspace{2mm}
    \noindent Primero debemos definir $\delta_{n}:H_{n}(C_{\sbullet})\to H_{n-1}(A_{\sbullet})$.
    Sea $[c]\in H_{n}(C_{\sbullet})$ entonces $c\in\kr{\partial}\subseteq C_{n}$. Como $j$ es 
    sobre, existe $b\in B_{n}$ tal que $j(b)=c$. Consideramos $\partial b$ y notamos que
    
    \begin{equation*}
        j\partial(b)=\partial j(b)=\partial c=0
    \end{equation*}
    
    \noindent entonces existe un único $a\in A_{n-1}$ tal que $i(a)=\partial b$. Verificamos que 
    $i\partial(a)=\partial i(a)=\partial^{2}b=0$ y como $i$ es inyectiva vemos que 
    $\partial a=0$. Afirmamos que $\delta_{n}:H_{n}(C_{\sbullet})\to H_{n-1}(A_{\sbullet})$ por
    \begin{equation*}
        \delta_{n}([c])=[a]
    \end{equation*}
    cumple lo buscado. Debemos demostrar lo siguiente
    \begin{enumerate}
        \item No depende de la elección de $b$. Sea $b'$ tal que $j(b')=c$ entonces 
        $j(b'-b)=c-c=0$, existe único $a_{0}$ tal que $i(a_{0})=b'-b$. Por otro lado, existe $a'$
        tal que
        \begin{equation*}
            i(a')=\partial b'=\partial b+\partial i(a_{0})=\partial b+i\partial(a_{0})
        \end{equation*}
        entonces $i(a'-\partial a_{0})=\partial b=i(a)$, por inyectividad, $a'-\partial a_{0}=a$,
        lo que implica que $[a]=[a']$.

        \item No depende de la elección del representante de $[c]$. Sea $c'=c+\partial c''
        =j(b)+\partial j(b'')=j(b)+j\partial(b'')$, diremos $b'=b+\partial b''$, notemos que
        $\partial b'=\partial b+\partial^{2}b''=\partial b$. El mismo $a\in A_{n-1}$ satisface
        $i(a)=\partial b'$. Entonces $\delta_{n}[c]=[a]=\delta_{n}[c']$.

        \item La función $\delta_{n}$ es morfismo, es decir
        \begin{equation*}
            \delta_{n}([c]+[c'])=\delta_{n}[c]+\delta_{n}[c']
        \end{equation*}
        Notar que si $j(b)=c$ y $j(b')=c'$ entonces $j(b+b')=c+c'$, existen únicos 
        $a,a'\in A_{n-1}$ tales que $i(a+a')=\partial(b+b'')$ y así
        \begin{equation*}
            \delta_{n}([c+c'])=[a+a']=[a]+[a']
        \end{equation*}

        \item Exactitud en $H_{n}(C_{\sbullet})$ y $H_{n}(A_{\sbullet})$. Veamos que 
        $\im{j_{*}}\subseteq\kr{\delta_{n}}$. Sea $j_{*}[b]$ con $\partial b=0$. Entonces
        \begin{equation*}
            \delta_{n}j_{*}[b]=\delta_{n}[j(b)]
        \end{equation*}
        Existe único $a\in A_{n-1}$ tal que $i(a)=\partial b=0$, entonces $a=0$ y por lo tanto
        $\delta_{n}j_{*}[b]=[a]=0$. Queda ver que $\kr{\delta_{n}}\subseteq\im{j_{*}}$. Sea
        $[c]\in\kr{\delta_{n}}$ con $\partial c=0$. Por definición de $\delta_{n}$, para cada
        $b$ tal que $j(b)=c$ hay un único $a\in A_{n-1}$ tal que $i(a)=\partial b$.

        \vspace{1mm}
        \noindent Como $\delta_{n}[c]=[a]=0$ se sigue que $a=\partial a'$ y entonces 
        $\partial b=i(a)=i\partial(a')=\partial i(a')$, así $b-i(a')\in\kr{\partial}$, es decir
        $b-i(a')$ representa una clase de homología.

        \vspace{1mm}
        \noindent Ahora $j(b-i(a'))=j(b)=c$, por ende, $j_{*}[b-i(a')]=[c]$. Para 
        $H_{n}(A_{\sbullet})$ la demostración es similar.

        \item Exactitud en $H_{n}(B_{\sbullet})$. Sea $[a]\in\im{i_{*}}$ con $\partial a=0$, 
        entonces
        \begin{equation*}
            j_{*}i_{*}[a]=[j_{n}i_{n}(a)]=0
        \end{equation*}
        y por lo tanto $\im{i_{*}}\subseteq\kr{j_{*}}$. Sea $[b]\in\kr{j_{*}}$ con 
        $\partial b=0$, entonces $j_{*}[b]=[j(b)]=0$, lo que implica que $j(b)=\partial c'
        =\partial j(b')=j\partial(b')$, existe único $a\in A_{n-1}$ tal que $b-\partial b'=i(a)$,
        además
        \begin{equation*}
            i\partial(a)=\partial i(a)=\partial b+\partial^{2}b'=0
        \end{equation*}
        entonces $\partial a=0$. Luego $i_{*}[a]=[b]$. Concluimos que $\im{i_{*}}=\kr{j_{*}}$.
    \end{enumerate}
    Lo que concluye el teorema.
\end{proof}

\begin{lema}
    Sean $0\xrightarrow[]{} A_{\sbullet}\xrightarrow[]{i} B_{\sbullet}\xrightarrow[]{j} 
    C_{\sbullet}\xrightarrow[]{}0$ y
    $0\xrightarrow[]{} A'_{\sbullet}\xrightarrow[]{i'} B'_{\sbullet}\xrightarrow[]{j'} 
    C'_{\sbullet}\xrightarrow[]{}0$ secuencias exactas cortas. Además, el diagrama

    \vspace{2mm}
    \centerline{
        \xymatrix{
            A_{n} \ar[r]^{i} \ar[d]^{f_{1}} & B_{n} \ar[r]^{j} \ar[d]^{f_{2}} 
            & C_{n} \ar[d]^{f_{3}} \\
            A'_{n} \ar[r]^{i} & B'_{n} \ar[r]^{j} & C'_{n}
        }
    }
    \vspace{2mm}
    \noindent es conmutativo, donde $f_{i}$ es mapeo de cadenas, entonces el diagrama,
    
    \vspace{2mm}
    \centerline{
        \xymatrix{
            H_{n+1}(C_{\sbullet}) \ar[r]^{\delta} \ar[d]^{f_{3*}} 
            & H_{n}(A_{\sbullet}) \ar[r]^{i_{*}} \ar[d]^{f_{1*}} 
            & H_{n}(B_{\sbullet}) \ar[r]^{j_{*}} \ar[d]^{f_{2*}} 
            & H_{n}(C_{\sbullet}) \ar[d]^{f_{3*}} \\
            H_{n+1}(C_{\sbullet}) \ar[r]^{\delta} & H_{n}(A'_{\sbullet}) \ar[r]^{i_{*}} 
            & H_{n}(B'_{\sbullet}) \ar[r]^{j_{*}} & H_{n}(C'_{\sbullet})
        }
    }
    \noindent es conmutativo.
\end{lema}

\noindent En otras palabras, la secuencia exacta del lema de la serpiente es natural.

\begin{proof}
    Sea $c\in C_{n+1}$ tal que $\partial c=0$. Existen $b\in B_{n+1}$ y $b'\in B'_{n+1}$ tales que
    \begin{equation*}
        j(b)=c
        \hhtext{y}
        j'(b')=f_{3}(c)
    \end{equation*}
    Existen $a\in A_{n}$ y $a'\in A'_{n}$ tales que $i(a)=\partial b$ y $i'(a')=\partial b'$. 
    Veamos que $f_{1}(a)-a'\in\im{\partial}$. Notemos que
    \begin{align*}
        i'(f_{1}(a)-a') &= i'f_{1}(a)-i'(a')=f_{2}i(a)-\partial b'=f_{2}\partial b-\partial b' \\
        &= \partial(f_{2}b-b')
    \end{align*}
    Notemos que $f_{2}b-b'\in\kr{j'}$, en efecto,
    \begin{equation*}
        j'f_{2}(b)-j'(b')=f_{3}j(b)-f_{3}(c)=f_{3}(c)-f_{3}(c)=0
    \end{equation*}
    Luego, existe $a''\in A_{n+1}$ tal que $i'(a'')=f_{2}(b)-b'$, así 
    $i'(f_{1}(a)-a')=\partial i'(a'')=i'\partial a''$. Por inyectividad, vemos que 
    $f_{1}(a)-a'=\partial a''$. Hemos probado que $f_{1*}\delta=\delta' f_{3*}$, el resto es 
    trivial.
\end{proof}

\begin{prop}[Lema del 5]
    Considerar el siguiente diagrama conmutativo
    
    \vspace{2mm}
    \centerline{
        \xymatrix{
            A_{1} \ar[r] \ar[d]^{f_{1}} & A_{2} \ar[r] \ar[d]^{f_{2}} 
            & A_{3} \ar[r] \ar[d]^{f_{3}} & A_{4} \ar[r] \ar[d]^{f_{4}} & A_{5} \ar[d]^{f_{5}} \\
            B_{1} \ar[r] & B_{2} \ar[r] & B_{3} \ar[r] & B_{4} \ar[r] & B_{5}
        }
    }

    \vspace{2mm}
    \noindent donde las filas son secuencias exactas y cada cuadrado conmuta. Si 
    $f_{1},f_{2},f_{4}$ y $f_{5}$ son isomorfismos, entonces $f_{3}$ es isomorfismo.
\end{prop}

\begin{proof}
    Por simplicidad del argumento, denotaremos los morfismos $A_{i}\to A_{i+1}$ y 
    $B_{i}\to B_{i+1}$ como $\partial$. Debido a que ambas secuencias son exactas, resulta que 
    $\partial^{2}a=\partial\circ\partial (a)=0$. Veamos que $\kr{f_{3}}=0$. Sea $a\in\kr{f_{3}}$,
    notemos que
    \begin{equation*}
        0=\partial f_{3}(a)=f_{4}\partial(a)\hhtext{entonces}\partial a=0
    \end{equation*}
    Como $a\in\kr\partial$, existe $a'\in A_{2}$ tal que $\partial a'=a$, luego 
    $\partial f_{2}(a')=f_{3}\partial(a')=f_{3}(a)=0$. Por exactitud, existe $b'\in B_{1}$ tal que 
    $\partial b'=f_{2}(a')$, puesto que $f_{1}$ es isomorfismo, existe $a''\in A_{1}$ tal que 
    $b'=f_{1}(a'')$, usando que los diagramas conmutan vemos que
    \begin{equation*}
        a''=f^{-1}_{1}(b')\hhtext{entonces}\partial a''=\partial f^{-1}_{1}(b')
        =f^{-1}_{2}\partial(b')
    \end{equation*}
    recordemos que $\partial b'=f_{2}(a')$, es decir, $\partial a''=a'$, luego 
    $0=\partial^{2}a''=\partial a'=a$.

    \vspace{2mm}
    \noindent Sea $b\in B_{3}$, consideramos $\partial b\in B_{4}$, entonces 
    $f^{-1}_{4}(\partial b)\in A_{4}$, por conmutatividad del diagrama se sigue que
    $\partial f_{4}^{-1}(\partial b)=f_{5}^{-1}(\partial^{2}b)=0$, luego, por exactitud, existe
    $a\in A_{3}$ tal que $\partial a=f_{4}^{-1}(\partial b)$. Observemos que,
    \begin{equation*}
        \partial(f_{3}(a)-b)=\partial f_{3}(a)-\partial b=f_{4}\partial(a)-\partial b=0
    \end{equation*}
    Así, existe $b'\in B_{2}$ tal que $\partial b'=f_{3}(a)-b$, definimos 
    $a'=f_{2}^{-1}(b')\in A_{2}$, de este modo,
    \begin{equation*}
        f_{3}(a)-b=\partial b'=\partial f_{2}(a')=f_{3}(\partial a')
    \end{equation*}
    En resumen, $f_{3}(a-\partial a')=b$. Concluimos que $f_{3}$ es isomorfismo.
\end{proof}

\subsection{Homología Singular}
La idea ahora es definir, dado un espacio topologico $X$, un complejo de cadenas 
$C_{*}(X)$, y para cada función continua $f:X\to Y$, construir un mapeo de cadena $f_{*}:C_{*}(X)
\to C_{*}(Y)$, para luego obtener los grupos de homología $H_{*}(X)$ de cada espacio y los mapeos
$f_{*}:H_{*}(X)\to H_{*}(Y)$.

\vspace{1mm}
\noindent Para esta construcción se utilizará homología singular. La ventaja de esta vía, es que
la definicón es una proppiedad instrínseca del espacio, sin embargo, resulta casi imposible 
calcular los complejos de cadenas y por ende, los grupos de homología.

\begin{dfn}
    Un $n-$simplice estándar es
    \begin{equation*}
        \Delta^{n}=\left\{
            (t_{0},\cdots,t_{n})\in\R^{n+1}:t_{i}\geq0,\hspace{1mm}\sum t_{i}=1
        \right\}
    \end{equation*}
\end{dfn}

\begin{obs}
    La $i-$ésima cara de $\Delta^{n}$ es $\Delta^{n}_{i}
    =\{(t_{0},\cdots,t_{n})\in\Delta^{n}:t_{i}=0\}$. Notamos que un $n-$simplice estándar tiene
    $n+1$ caras, donde cada una luce como un $(n-1)-$simplice estándar. De hecho, mediante el 
    mapeo
    \begin{align*}
        \delta_{i}:\Delta^{n-1} &\to \Delta^{n}_{i} \\
        (t_{0},\cdots,t_{n-1}) &\to (t_{0},\cdots,t_{i-1},0,t_{i},\cdots,t_{n-1})
    \end{align*}
    resulta ser homeomorfa a $\Delta^{n-1}$.
\end{obs}

\begin{dfn}
    Sea $X$ un espacio topologico. Un $n-$simplice singular en $X$ es una función continua
    $\sigma:\Delta^{n}\to X$.
\end{dfn}

\noindent Dado un espacio $X$, tenemos una colección de $n-$simplices, vamos a considerar el grupo
abeliano libre generado por este conjunto para construir un complejo de cadenas y así los grupos
de homología.

\begin{dfn}
    Definimos
    \begin{equation*}
        C_{n}(X):=\left\{\sum n_{\sigma}\sigma\hspace{1mm}|\hspace{1mm}\sigma:\Delta^{n}\to X
        \text{, }n_{\sigma}\in\Z\hspace{1mm}
        \text{ nulo salvo finitos casos}\right\}
    \end{equation*}
    Junto con los diferenciales $\partial_{n}:C_{n}(X)\to C_{n-1}(X)$ dada por
    \begin{equation*}
        \partial_{n}(\sigma)=\sum_{i=0}^{n}(-1)^{i}\sigma\circ\delta_{i}
    \end{equation*}
    que se extiende linealmente.
\end{dfn}

\noindent Al complejo $(C_{\sbullet}(X),\partial_{\sbullet})$ se le llama complejo de cadena 
singular. Aunque debemos verificar que $\partial^{2}=0$, lo que se muestra en el siguiente lema.

\begin{lema}
    El morfismo $d_{n-1}\circ d_{n}:C_{n}(X)\to C_{n-2}(X)$ es trivial.
\end{lema}

\begin{proof}
    Basta probar para cada elemento en la base, es decir, para $\sigma:\Delta^{n}\to X$. Tenemos 
    lo siguiente
    \begin{equation*}
        \partial_{n-1}\circ \partial_{n}(\sigma)=\sum_{i=0}^{n-1}(-1)^{i}\sum_{j=0}^{n}(-1)^{j}
        \sigma\circ\delta_{j}\circ\delta_{i}
    \end{equation*}
    Usando que $\delta_{j}\circ\delta_{i}=\delta_{i}\circ\delta_{j-1}$ si $i<j$, vemos que
    \begin{align*}
        \partial_{n-1}\circ \partial_{n}(\sigma) &= \sum_{i<j}(-1)^{i+j}\sigma\circ\delta_{j}
        \circ\delta_{i}+\sum_{i\geq j}(-1)^{i+j}\sigma\circ\delta_{j}\circ\delta_{i} \\
        &= \sum_{i<j}(-1)^{i+j}\sigma\circ\delta_{i}\circ\delta_{j-1}
        +\sum_{i\geq j}(-1)^{i+j}\sigma\circ\delta_{j}\circ\delta_{i}
    \end{align*}
    haciendo un cambio de índice en la primera expresión, resulta que
    \begin{equation*}
        \partial_{n-1}\circ \partial_{n}(\sigma)=\sum_{i\leq j}(-1)^{i+j+1}\sigma\circ\delta_{i}
        \circ\delta_{j}+\sum_{i\geq j}(-1)^{i+j}\sigma\circ\delta_{j}\circ\delta_{i}=0
    \end{equation*}
\end{proof}

\noindent Con lo anterior podemos definir la homología singular de un espacio topológico 
arbitrario.

\begin{dfn}
    La homología singular de un espacio topológico $X$, es la homología del complejo de
    cadenas $C_{\sbullet}(X)$, es decir,
    \begin{equation*}
        H_{i}(X):=H_{i}(C_{\sbullet}(X),\partial_{\sbullet})
        =\frac{\kr{\partial_{i}}}{\im{\partial_{i+1}}}
    \end{equation*}
\end{dfn}

\noindent Hemos cumplido con una parte del objetivo inicial, ahora nos gustaría, dada una función
continua $f:X\to Y$ entre espacios topológicos, un morfismo entre sus grupos de homología, para 
ello la siguiente proposición.

\begin{prop}
    Sea $f:X\to Y$ un función continua entre espacios topológicos, entonces las funciones
    \begin{equation*}
        f_{n}:C_{n}(X)\to C_{n}(Y)\hhtext{dadas en la base por}f_{n}(\sigma)=f\circ\sigma
    \end{equation*}
    forman un mapeo de cadenas $f_{\sbullet}:C_{\sbullet}(X)\to C_{\sbullet}(Y)$.
\end{prop}

\begin{proof}
    Es directo de la definición que para todo $n\in\N$ la función $f_{n}$ es morfismo de grupos 
    abelianos, por otro lado
    \begin{equation*}
        \partial_{n} f_{n}(\sigma)=\partial_{n}(f\circ\sigma)
        =\sum_{i}(-1)^{i}f\circ\sigma\circ\delta_{i}
        =\sum_{i}(-1)^{i}f_{n-1}(\sigma\circ\delta_{i})=f_{n-1}\partial_{n}(\sigma)
    \end{equation*}
\end{proof}

\begin{obs}
    Juntando lo anterior y un resultado previo, la función $f$ induce un morfismo de grupos
    $f_{*}:H_{n}(X)\to H_{n}(Y)$, más aún, si $f$ es homeomorfismo, entonces $f_{*}$ es 
    isomorfismo. Por ende, los grupos de homología son un invariante topológico.
\end{obs}

\noindent Lo bueno de esta construcción, a diferencia de otras construcciones, es que las 
definiciones no dependen de otra cosa más que el espacio topológico, es decir, de los abiertos del 
espacio. Pero esto viene con un precio a pagar, es prácticamente imposible calcular estos grupos 
usando únicamente la definición. Sin embargo, podemos calcular unos pocos ejemplos que resultarán 
útiles mas adelante.

\begin{ej}
    Sea $X=pt=\{*\}$, el espacio que consiste de un punto. Entonces, existe un único simplice
    singular, digamos, $\sigma_{n}:\Delta^{n}\to pt$, lo que implica que 
    $\sigma_{n}\circ\delta_{i}=\sigma_{n-1}$, luego
    \begin{equation*}
        \partial_{n}(\sigma_{n})=\sum_{i=0}^{n}(-1)^{i}\sigma_{n}\circ\delta_{i}
        =\begin{cases}
            \sigma_{n-1} &\quad\text{ si }n\text{ es par} \\
            0 &\quad\text{ si }n\text{ es impar}
        \end{cases}
    \end{equation*}
    para $i>0$, vemos que

    \vspace{2mm}
    \centerline{
        \xymatrix{
            \cdots \ar[r] & C_{i+1}(pt) \ar[r]^{\partial_{i+1}} 
            & C_{i}(pt) \ar[r]^{\partial_{i}} & C_{i-1}(pt) \ar[r] & \cdots
        }
    }
    \noindent entonces
    \begin{equation*}
        H_{i}(pt)=\frac{\kr{\partial_{i}}}{\im{\partial_{i+1}}}=\frac{C_{i}(pt)}{C_{i}(pt)}=0
        \htext{si }i\text{ es impar}
    \end{equation*}
    y $H_{i}(pt)=0$ para $i$ par, ya que $\partial_{i}$ es inyectivo. Para $i=0$ se tiene que 
    $H_{0}(pt)=\Z$, por que $C_{0}(pt)\cong\Z$.
\end{ej}

\begin{lema}
    Sea $X$ un espacio arcoconexo no vacío, entonces $H_{0}(X)=\Z$.
\end{lema}

\begin{proof}
    Se define el morfismo $\varphi:C_{0}(X)\to\Z$ dado por
    \begin{equation*}
        \varphi\left(\sum n_{\sigma}\sigma\right)=\sum n_{\sigma}
    \end{equation*}
    Como $X$ es no vacío, el morfismo es sobreyectivo. Afirmamos que 
    $\kr{\varphi}=\im{\partial_{1}}$, en efecto, sea $\sigma\in C_{1}(X)$, luego
    \begin{equation*}
        \varphi(\partial_{1}(\sigma))=\varphi(\sigma\circ\delta_{0}-\sigma\circ\delta_{1})=1-1=0
    \end{equation*}
    Por otro lado, supongamos que $\sum n_{\sigma}\sigma\in\kr{\varphi}$. Sea $x_{0}\in X$, 
    consideremos el $0-$simplice singular $\sigma:\Delta^{0}\to X$. Como el espacio es arcoconexo, 
    existe $\tau_{\sigma}:\Delta^{1}\to X$ de modo que $\tau_{\sigma}\circ\delta_{0}=\sigma$ y
    $\tau_{\sigma}\circ\delta_{1}=x_{0}$, así
    \begin{equation*}
        \partial_{1}\left(\sum n_{\sigma}\tau_{\sigma}\right)=\sum n_{\sigma}(\sigma-x_{0})
        =\sum n_{\sigma}\sigma-\left(\sum n_{\sigma}\right)x_{0}=\sum n_{\sigma}\sigma
    \end{equation*}
    por lo tanto,
    \begin{equation*}
        H_{0}(X)=\frac{C_{0}(X)}{\im{\partial_{1}}}=\frac{C_{0}(X)}{\kr{\varphi}}\cong\Z
    \end{equation*}
\end{proof}

\noindent Lo anterior da pie a la siguiente proposición.

\begin{prop}
    Sea $X=\bigsqcup_{\alpha\in A}X_{\alpha}$ es la unión disjunta de componentes arcoconexas,
    entonces
    \begin{equation*}
        H_{n}(X)\cong\bigoplus_{\alpha\in A}H_{n}(X_{\alpha})
    \end{equation*}
\end{prop}

\begin{proof}
    La inclusión $i_{\alpha}:X_{\beta}\to X$ es continua para todo $\alpha\in A$, por propiedad 
    universal de topología disjunta, luego tenemos la colección de morfismos, 
    \begin{equation*}
        i_{\alpha*}:H_{n}(X_{\alpha})\to H_{n}(X)
    \end{equation*}
    por la propiedad universal de la suma directa, se induce un único morfismo 
    \begin{equation*}
        i_{*}:\bigoplus_{\alpha\in A}H_{n}(X_{\alpha})\to H_{n}(X)
    \end{equation*}
    por continuidad, es isomorfismo.
\end{proof}

\section{Herramientas de Homología Singular}
En esta sección veremos sin demostración, o al menos no detallada, teoremas que son fundamentales
en la teoría de homología de espacios topológicos, como lo son la secuencia de Mayer-Vietoris,
homología relativa, escición e invarianza homotópica.

\subsection{Invarianza Homotópica}

\begin{teo}
    Sean $f,g:X\to Y$ funciones homotópicas. Entonces, inducen el mismo morfismo en homología, 
    es decir,
    \begin{equation*}
        f_{*}=g_{*}:H_{*}(X)\to H_{*}(Y)
    \end{equation*}
\end{teo}

\begin{cor}
    Sea $f:X\to Y$ una equivalencia homotópica, entonces
    \begin{equation*}
        f_{*}:H_{*}(X)\to H_{*}(Y)
    \end{equation*}
    es isomorfismo.
\end{cor}
\begin{proof}
    Sea $g:Y\to X$ una inversa homotópica, entonces
    \begin{equation*}
        g_{*}\circ f_{*}=(g\circ f)_{*}=(id_{X})_{*}=id_{H_{*}(X)}
    \end{equation*}
    Igualmente, tenemos que $f_{*}\circ g_{*}=id_{H_{*}(Y)}$. Entonces $f_{*}$ es un isomorfismo
    con inversa $g_{*}$.
\end{proof}

\begin{ej}
Como $\R^{n}$ es homotópico a un punto, resulta que
    \begin{equation*}
        H_{i}(\R^{n})=\begin{cases}
            \Z &\quad\text{si }i=0 \\
            0 &\quad\text{si }i>0
        \end{cases}
    \end{equation*}
\end{ej}

\subsection{Mayer-Vietoris}

\begin{teo}[Mayer-Vietoris]
    Sea $X=A\cup B$ con $A$ y $B$ conjuntos abiertos. Se tienen las inclusiones
    
    \vspace{2mm}
    \centerline{
        \xymatrix{
            A\cap B \ar[r]^-{i_{A}} \ar[d]^-{i_{B}} & A \ar[d]^-{j_{A}} \\
            B \ar[r]^-{j_{B}} & X
        }
    }
    \vspace{2mm}
    \noindent Entonces, existen homomorfismos $\delta:H_{n}(X)\to H_{n-1}(A\cap B)$ tales 
    que la siguiente secuencia es exacta
    
    \vspace{2mm}
    \centerline{
        \xymatrixcolsep{3pc}\xymatrix{
            \ar[r]^-{\delta} & H_{n}(A\cap B) \ar[r]^-{i_{A*}\oplus i_{B*}} 
            & H_{n}(A)\oplus H_{n}(B) \ar[r]^-{j_{A*}-j_{B*}} 
            & H_{n}(X) \ar `[dl] `[l] `[llld]_{\delta} `[dr] [dlllr] & \\
            & H_{n-1}(A\cap B) \ar[r]^-{i_{A*}\oplus i_{B*}} 
            & H_{n-1}(A)\oplus H_{n-1}(B) \ar[r]^-{j_{A*}-j_{B*}} & H_{n-1}(X) \ar[r] & \cdots \\
            & \cdots \ar[r] & H_{0}(A)\oplus H_{0}(B) \ar[r]^-{j_{A*}-j_{B*}} & H_{0}(X) \ar[r] 
            & 0
        }
    }
    \vspace{2mm}
    \noindent Es mas, la secuencia de Mayer-Vietoris es natural, es decir, si 
    $f:X=A\cup B\to Y=U\cap V$ cumple que $f(A)\subseteq U$ y $f(B)\subseteq V$, entonces el 
    diagrama
    
    \vspace{2mm}
    \centerline{
        \xymatrixcolsep{3pc}\xymatrix{
            H_{n+1}(X) \ar[d]^{f_{*}} \ar[r]^-{\delta} 
            & H_{n}(A\cap B) \ar[d]^{f|_{A\cap B*}} \ar[r]^-{i_{A*}\oplus i_{B*}}
            & H_{n}(A)\oplus H_{n}(B) \ar[d]^{f|_{A*}\oplus f|_{B*}} \ar[r]^-{j_{A*}-j_{B*}} 
            & H_{n}(X) \ar[d]^{f_{*}} \\
            H_{n+1}(Y) \ar[r]^-{\delta} & H_{n}(U\cap V) \ar[r]^-{i_{U*}\oplus i_{V*}}
            & H_{n}(U)\oplus H_{n}(V) \ar[r]^-{j_{U*}-j_{V*}} & H_{n}(Y)
        }
    }
    \noindent conmuta.
\end{teo}

\begin{proof}
    Basta notar la siguiente secuencia

    \vspace{2mm}
    \centerline{
        \xymatrix{
            0 \ar[r] & C_{n}(A\cap B) \ar[r]^-{i_{A}\oplus i_{B}} 
            & C_{n}(A)\oplus C_{n}(B) \ar[r]^-{j_{B}-j_{A}} & C_{n}(X) \ar[r] & 0
        }
    }
    \vspace{2mm}
    \noindent es exacta para todo $n\in\N$. Luego, aplicar el lema de la serpiente y naturalidad 
    del mismo.
\end{proof}

\subsection{Homología Relativa} \hspace{1mm}

\vspace{2mm}
\noindent La inclusión $i:A\to X$ induce un morfismo a nivel de cadenas que resulta ser inyectivo,
así, podemos ver $C_{n}(A)$ como un subgrupo de $C_{n}(X)$, además, el diagrama

\vspace{2mm}
\centerline{
    \xymatrix{
        C_{n}(A) \ar[r]^{i_{n}} \ar[d]^{\partial_{n}} & C_{n}(X) \ar[d]^{\partial_{n}} \\
        C_{n-1}(A) \ar[r]^{i_{n-1}} & C_{n-1}(X) \\
    }
}
\vspace{2mm}
\noindent conmuta, lo que motiva la siguiente definición.

\begin{dfn}
    Sea $A\subseteq X$. Definimos
    \begin{equation*}
        C_{n}(X,A):=\frac{C_{n}(X)}{C_{n}(A)}
    \end{equation*}
    el diferencial $\partial_{n}:C_{n}(X,A)\to C_{n-1}(X,A)$, dado por $\partial_{n}([c])
    :=[\partial_{n}(c)]$. Del complejo de cadenas $(C_{\sbullet}(X,A),\partial_{\sbullet})$, 
    definimos la homología relativa de $X$ respecto a $A$ como
    \begin{equation*}
        H_{n}(X,A):=H_{n}(C_{\sbullet}(X,A))
    \end{equation*}
\end{dfn}

\begin{teo}
    Existen homomorfismos $\delta:H_{n}(X,A)\to H_{n-1}(A)$. Mas aún, la siguiente secuencia es 
    exacta
    
    \vspace{2mm}
    \centerline{
        \xymatrix{
            \cdots \ar[r]^-{\delta} & H_{n}(A) \ar[r]^-{i_{*}} & H_{n}(X) \ar[r]^-{q_{*}} 
            & H_{n}(X,A) \ar `[dl] `[l] `[llld]_{\delta} `[dl] [dll] \\
            & H_{n-1}(A) \ar[r]^-{i_{*}} & H_{n-1}(X) \ar[r]^-{q_{*}} & H_{n-1}(X,A) \ar[r] 
            & \cdots \\
            & \cdots \ar[r] & H_{0}(X) \ar[r]^-{q_{*}} & H_{0}(X,A) \ar[r] & 0
        }
    }
    \vspace{2mm}
    donde $q_{*}$ es el morfismo inducido por la proyección $q:C_{\sbullet}(X)\to 
    C_{\sbullet}(X,A)$.
\end{teo}

\begin{proof}
    La siguiente secuencia

    \vspace{2mm}
    \centerline{
        \xymatrix{
            0 \ar[r] & C_{n}(A) \ar[r]^{i_{n}} & C_{n}(X) \ar[r]^{q_{n}} & C_{n}(X,A) \ar[r] & 0
        }
    }
    \vspace{2mm}
    \noindent es exacta para todo $n\in\N$. Concluimos por el lema de la serpiente.
\end{proof}

\begin{dfn}
    Una tupla $(X,A)$ se dice par de espacios si $X$ y $A$ son espacios topológicos tales que 
    $A\subseteq X$ y $A$ con la topología de subespacio. Sean $(X,A)$ y $(Y,B)$ pares de espacios. 
    Un mapeo de pares es una función continua $f:X\to Y$ tal que $f(A)\subseteq B$.
\end{dfn}

\vspace{2mm}
\noindent Tales mapas inducen un morfismo $f_{*}:H_{n}(X,A)\to H_{n}(X,A)$ tal que la secuencia 
exacta de homología relativa es natural.

\begin{teo}[Teorema de Escisión]
    Sea $(X,A)$ un par de espacios y $Z\subseteq A$ tal que $\overline{Z}\subseteq int(A)$ tomando
    la clausura en $X$. Entonces
    \begin{equation*}
        H_{n}(X\setminus Z,A\setminus Z)\cong H_{n}(X,A)
    \end{equation*}
\end{teo}

\newpage
\begin{thebibliography}{1}
    \bibitem{1}
        \textsc{Billingsley, P.} (1999). 
        \textit{Convergence of Probability Measures}, 2nd ed. Wiley, New York.
    \end{thebibliography}
\end{document}