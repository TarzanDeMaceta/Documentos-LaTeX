\documentclass[aop]{imsart2}

%% Packages
\RequirePackage{amsthm,amsmath,amsfonts,amssymb}
\RequirePackage[numbers]{natbib}
\RequirePackage[colorlinks, citecolor=blue, urlcolor=blue]{hyperref}
\RequirePackage{graphicx}

\usepackage[spanish]{babel}
\usepackage[utf8]{inputenc}

%% Arregla error al intentar usar un tamaño de font no disponible
\usepackage{anyfontsize}

\usepackage{tikz} %% Se utiliza para hacer gráficos
\usepackage[all]{xy} %% Para los diagramas conmutativos
\usetikzlibrary{calc}

\usepackage{graphicx} %% Requerido para insertar imagenes

\startlocaldefs

\theoremstyle{plain}
\newtheorem{axioma}{Axioma}
\newtheorem{af}[axioma]{Afirmación}
\newtheorem{teo}{Teorema}[section]
\newtheorem{lema}[teo]{Lema}
\newtheorem{prop}[teo]{Proposición}
\newtheorem{cor}[teo]{Corolario}

\theoremstyle{remark}
\newtheorem{dfn}[teo]{Definición}
\newtheorem*{ej}{Ejemplo}
\newtheorem*{obs}{Observación}

%% Nuevos comandos específicos, necesarios para simplificar la escritura

%% Norma y valor absoluto
\newcommand{\norm}[1]{\left \lVert #1\right \rVert}
\newcommand{\abs}[1]{\left|#1 \right|}

%% Derivada y derivada parcial
\newcommand{\dv}[2]{\frac{d#1}{d#2}}
\newcommand{\pdv}[2]{\frac{\partial#1}{\partial#2}}

%% Producto interno y conjunto generador
\newcommand{\ip}[2]{\left\langle{#1},{#2}\right\rangle}
\newcommand{\gen}[1]{\left\langle{#1}\right\rangle}

%% Otros
\newcommand{\htext}[1]{\hspace{4mm}\text{#1 }}
\newcommand{\hhtext}[1]{\hspace{4mm}\text{#1}\hspace{4mm}}
\newcommand{\im}[1]{im\hspace{1mm}#1}
\newcommand{\kr}[1]{ker\hspace{1mm}#1}
\newcommand\sbullet[1][.5]{\mathbin{\vcenter{\hbox{\scalebox{#1}{$\bullet$}}}}}

%% Simplificar comandos

\def \ds {\displaystyle}
\def \C {\mathbb{C}}
\def \R {\mathbb{R}}
\def \Q {\mathbb{Q}}
\def \Z {\mathbb{Z}}
\def \N {\mathbb{N}}
\def \S {\mathbb{S}}
\def \H {\mathbb{H}}
\def \P {\mathbb{P}}
\def \T {\mathbb{T}}

\def \U {\mathcal{U}}

\endlocaldefs

\begin{document}

\begin{titlepage}
    \hspace*{-3cm}
    \raisebox{-1cm}[0pt][0pt]{
        \includegraphics[width=0.4\textwidth]{logo_universidad2.png}
    }

    \vspace{2cm}
    \begin{center}
        {\Large Pontificia Universidad Católica de Chile}
        
        \vspace{2cm}
        {\Huge \bfseries 
        Obstrucciones homotópicas a la existencia de estructuras de grupo topológico
        }
        
        \vspace{1.5cm}
        {\Large Benjamín Mateluna}
        
        \vspace{1.5cm}
        {\Large Docente Guía: Mauricio Bustamante}
        \vfill
        
        {\large \today \par}
    \end{center}
\end{titlepage}

\tableofcontents

\newpage
\section*{Resumen} \hspace{1mm} %% Esta parte esta lista

\vspace{1mm}
\noindent En este trabajo estudiamos la noción de grupo topológico y analizamos obstrucciones
topológicas para que un espacio admita esta estructura. Empleamos herramientas de la topología
algebraica, entre ellas la cohomología, los espacios cubrientes y el grupo fundamental. Entre los 
resultados principales está la única superficie compacta y conexa que puede ser un grupo 
topológico es el $2-$toro, y que ninguna esfera de dimensión par admite una estructura de grupo 
topológico.


\section{Introducción} \hspace{1mm} %% Esta parte esta lista

\vspace{1mm}
\noindent El objetivo de esta tesis es estudiar los grupos topológicos y las obstrucciones que 
impiden que un espacio lo sea. Un \emph{grupo topológico} es un espacio topológico $X$ equipado 
con una aplicación continua
\begin{equation*}
    m:X\times X\to X
\end{equation*}
y un elemento $e\in X$ que actúa como identidad, de modo que $(X,m,e)$ es un grupo. Además 
exigimos que el mapa de inversión
\begin{equation*}
    \varphi:X\to X,\qquad \varphi(x)=x^{-1},
\end{equation*}
sea continuo. A continuación presentamos algunos ejemplos para introducir esta noción y mostrar 
que aparece de manera natural en distintos contextos.

\begin{ej}
    Los primeros espacios que surgen como grupos topológicos son $\R$ y $\C$. Con la norma
    usual en $\R$ y $\C$, la suma y multiplicar por $-1$ son aplicaciones continuas; además,
    $(\R,+,0)$ y $(\C,+,0)$ son grupos.
\end{ej}

\begin{ej}
    Consideremos $GL_{n}(\R)$, el grupo general lineal de orden $n$, es decir, el subconjunto de
    las matrices $n\times n$ con coeficientes en $\R$ que son invertibles. Equipado con la norma
    euclidiana en $\R^{n^2}$, la multiplicación de matrices es continua porque cada entrada es
    un polinomio en las entradas de las matrices; de forma similar, la aplicación 
    $A\mapsto A^{-1}$ es continua. Análogamente, $GL_{n}(\C)$ es un grupo topológico.
\end{ej}

\noindent Como mencionamos, los ejemplos anteriores surgen de forma natural y son fundamentales.
No obstante, no es obvio qué estructura de grupo asignar a un espacio topológico dado; por 
ejemplo, más adelante se demostrará que $\S^{2}$ no admite tal estructura. Para caracterizar los 
espacios que pueden ser grupos topológicos estudiaremos obstrucciones para serlo: es decir, 
encontraremos condiciones necesarias que debe cumplir un espacio para poder ser un grupo 
topológico. El siguiente teorema proporciona una primera restricción.

\begin{teo}
    Sea $X$ un grupo topológico arcoconexo. Entonces $\pi_{1}(X,e)$ es abeliano.
\end{teo}

\noindent De este resultado se desprende que $\S^{1}\vee\S^{1}$, el producto \emph{wedge}, no 
admite estructura de grupo topológico, ya que $\pi_{1}(\S^{1}\vee\S^{1},1)\cong F_{2}$, el grupo 
libre en dos generadores. Este cálculo se obtiene mediante el teorema de Seifert--van Kampen. 
Combinando ambos resultados descartamos que espacios como $\#^{n}\T^{2}$ o $\#^{n}\R\P^{2}$, para 
$n>1$, sean grupos topológicos.

\vspace{1mm}
\noindent Por otro lado, $\pi_{1}(\S^{1},1)=\Z$ y $\pi_{1}(\S^{n},x_{0})=0$ para $n>1$, de modo 
que estos espacios cumplen la primera restricción. En el caso de $\S^{1}$ se utiliza la 
multiplicación en $\C$: la multiplicación es continua y $\S^{1}\subset\C^{*}$ es un subgrupo 
multiplicativo, por lo que $\S^{1}$ es un grupo topológico. Como corolario, $\T^{2}
=\S^{1}\times\S^{1}$ es un grupo topológico.

\newpage
\noindent Para $n=3$ consideramos los cuaterniones $\H$. El conjunto de cuaterniones unitarios se 
identifica con $\S^{3}$ y, bajo la multiplicación, forman un grupo topológico; por tanto $\S^{3}$
es un grupo topológico. Dado que $\R\P^{3}=\S^{3}/\{\pm1\}$ y $\{\pm1\}$ es un subgrupo central,
la multiplicación desciende al cociente, de modo que $\R\P^{3}$ también es un grupo topológico
(de hecho, es isomorfo a $SO(3)$).

\vspace{1mm}
\noindent Las esferas $\S^{1}$ y $\S^{3}$ aparecen como subconjuntos de álgebras de división que 
son, además, espacios topológicos; esto facilita la construcción de sus estructuras de grupo 
topológico. En cambio, $\S^{2}$ no admite tal estructura. Nos interesa entonces determinar para 
qué valores de $n\in\N$ la esfera $\S^{n}$ puede ser grupo topológico. Existen restricciones 
fuertes sobre $n$, el siguiente teorema nos brinda una respuesta

\begin{teo}\label{teo:esferas}
    Sea $n\in\N$. Si $\S^{n}$ es grupo topológico entonces $n$ es impar.
\end{teo}

\noindent Es más, en la literatura clásica se conoce que los casos posibles son excepcionales 
(los casos de Hopf: $n=1,3$), y la tesis entra en las condiciones que conducen a dichas 
restricciones. Hasta ahora sabemos que el toro es un grupo topológico, que $\S^{2}$ no lo es, y 
que los espacios homeomorfos a $\#^{n}\T^{2}$ y $\#^{n}\R\P^{2}$ con $n>1$ tampoco lo son. El 
siguiente teorema permite además descartar a $\R\P^{2}$ como grupo topológico.

\begin{teo}
    El cubriente universal de un grupo topológico es grupo topológico.
\end{teo}

\noindent Como $\S^{2}$ es el cubriente universal de $\R\P^{2}$, el teorema de clasificación de
superficies permite concluir lo siguiente:

\begin{teo}
    La única superficie compacta y conexa que es grupo topológico es $\T^{2}$.
\end{teo}

\noindent Veremos más adelante que estos resultados se pueden generalizar a una clase más amplia de
espacios, los llamados $H$-espacios. Esto ocurre porque los invariantes que utilizaremos son
invariantes homotópicos. En este sentido, un $H$-espacio puede verse como un grupo topológico “hasta
homotopía”. La definición formal es la siguiente:

\begin{dfn}
    Sea $X$ un espacio topológico. Decimos que $X$ es un $H$-espacio si existe una aplicación
    continua $\mu: X\times X \to X$ y un elemento $e\in X$ (identidad) tales que los mapeos
    \begin{equation*}
        \phi(x)=\mu(x,e), \qquad \psi(x)=\mu(e,x)
    \end{equation*}
    son homotópicos a la identidad, relativos a $\{e\}$.
\end{dfn}

\noindent Observemos que la definición implica $\mu(e,e)=e$. También podemos añadir más 
estructura a un $H$-espacio:

\begin{enumerate}
    \itemsep0.5em

    \item Si $\mu$ es tal que $\mu(x,\mu(y,z))$ y $\mu(\mu(x,y),z)$ son homotópicas relativo a 
    $\{e,e,e\}$, decimos que $X$ es \emph{homotópicamente asociativo}.
    
    \item Si existe $\varphi:X\to X$ con $\varphi(e)=e$ tal que los mapas
    \begin{equation*}
        x\longmapsto \mu(x,\varphi(x))\qquad\text{y}\qquad x\longmapsto \mu(\varphi(x),x)
    \end{equation*}
    son homotópicos a la función constante $e$, relativos a $\{e\}$, entonces decimos que $X$ 
    tiene \emph{inversa homotópica}.
\end{enumerate}

\noindent Un $H$-grupo es un $H$-espacio que es homotópicamente asociativo y posee inversa 
homotópica. En particular, todo grupo topológico es un $H$-grupo.

\vspace{1mm}
\noindent Estas consideraciones motivan el presente trabajo. El objetivo es analizar, mediante
herramientas de topología algebraica, las obstrucciones que impiden a un espacio admitir una
estructura de grupo topológico. Para ello introduciremos los fundamentos algebraicos necesarios,
en particular nociones básicas de álgebra homológica; a continuación estudiaremos con detalle los
ejemplos mencionados, revisaremos el papel del grupo fundamental y construiremos los grupos de
cohomología singulares. De manera central examinaremos el anillo de cohomología, que será la 
herramienta principal para descartar a las $n$-esferas como posibles grupos topológicos.

\vspace{1mm}
\noindent En esta tesis se asumen definiciones y resultados básicos de topología (entre ellos la
construcción del grupo fundamental). También se emplearán nociones de álgebra conmutativa y teoría
de grupos: grupos libres, presentaciones, productos libres y amalgamados, tensores y sumas 
directas. Algunos resultados se enunciarán sin demostración; en esos casos se remite al lector a 
la bibliografía indicada.

\newpage

\section{Preliminares: Álgebra Homológica} \hspace{1mm} %% Esta parte esta lista

\vspace{1mm}
\noindent Para dar comienzo a la búsqueda de nuestro objetivo necesitamos herramientas algebraicas 
para trabajar con las cuales tendremos una noción formal de las siguientes secciones. Comenzaremos 
con la homología y cohomología, para luego revisar resultados que serán importantes más adelante.

\begin{dfn}
    Un complejo de cadenas es una sucesión de grupos abelianos y homomorfismos
    
    \centerline{
        \xymatrix{
            \cdots \ar[r] & C_{3} \ar[r]^{\partial_{3}} & C_{2} \ar[r]^{\partial_{2}} & 
            C_{1} \ar[r]^{\partial_{1}} & C_{0} \ar[r] & 0
        }
    }
    \vspace{2mm}
    \noindent tal que $\partial_{i}\circ \partial_{i+1}=0$ para todo $i$. Se denota por 
    $(C_{\sbullet},\partial_{\sbullet})$.
\end{dfn}

\begin{obs}
    Notemos que $\im{\partial_{i+1}}\subseteq\kr{\partial_{i}}\subseteq C_{i}$. Dado que los 
    grupos son abelianos, esta observación permite definir los siguientes grupos.
\end{obs}

\begin{dfn}
    El $i-$ésimo grupo de homología de $(C_{\sbullet},\partial_{\sbullet})$ se define por
    \begin{equation*}
        H_{i}(C_{\sbullet}):=\frac{\kr{\partial_{i}}}{\im{\partial_{i+1}}}
    \end{equation*}
\end{dfn}

\begin{ej}
    Veamos que
    
    \centerline{
        \xymatrix{
            \cdots \ar[r] & 0 \ar[r]^{\cdot0} & \Z \ar[r]^{\cdot2} & \Z \ar[r]^{\cdot0} 
            & \Z \ar[r] & 0
        }
    }
    \vspace{2mm}

    \noindent es un complejo de cadenas, donde los grupos de homología asociados son
    $H_{0}(C_{\sbullet})=\Z$, $H_{1}(C_{\sbullet})=\Z_{2}$ y $H_{k}(C_{\sbullet})=0$ para 
    $k\neq0,1$.
\end{ej}

\noindent Por otro lado, para la cohomología damos la siguiente definición

\begin{dfn}
    Un complejo de cocadenas es una secuencia de grupos de abelianos y homomorfismos
    
    \vspace{1mm}
    \centerline{
        \xymatrix{
            0 \ar[r] & C^{0} \ar[r]^{\partial^{0}} & C^{1} \ar[r]^{\partial^{1}} 
            & C^{2} \ar[r]^{\partial^{2}} & C^{3} \ar[r]^{\partial^{3}} & \cdots
        }
    }
    \vspace{1mm}
    \noindent tales que $\partial^{i+1}\circ\partial^{i}=0$. Se denota por 
    $(C^{\sbullet},\partial^{\sbullet})$.
\end{dfn}

\begin{dfn}
    El $i-$ésimo grupo de cohomología de $(C^{\sbullet},\partial^{\sbullet})$ se define por 
    \begin{equation*}
        H^{i}(C^{\sbullet}):=\frac{\kr{\partial^{i}}}{\im{\partial^{i-1}}}
    \end{equation*}
    Un elemento en $H^{i}(C^{\sbullet})$ se conoce como clase de cohomología.
\end{dfn}

\begin{ej}
    Vimos el complejo de cadenas

    \vspace{1mm}
    \centerline{
        \xymatrix{
            \cdots \ar[r] & 0 \ar[r] & \Z \ar[r]^{\cdot2} & \Z \ar[r]^{\cdot0} & \Z \ar[r] & 0
        }
    }

    \vspace{1mm}
    \noindent Donde $H_{0}(C_{\sbullet})=\Z$, $H_{1}(C_{\sbullet})=\Z_{2}$ y 
    $H_{k}(C_{\sbullet})=0$. Definimos $C^{i}:=Hom(C_{i},\Z)$ y los diferenciales 
    $\partial^{i}(\varphi):=\varphi\circ\partial_{i+1}$, notemos que 
    $\partial^{i+1}\circ\partial^{i}(\varphi)=\varphi\circ\partial_{i+1}\circ\partial_{i+2}=0$. 
    Así, tenemos el complejo de cocadenas

    \vspace{1mm}
        \centerline{
            \xymatrix{
                0 \ar[r] & C^{0} \ar[r]^{\cdot0} & C^{1} \ar[r]^{\cdot{2}} & C^{2} \ar[r]^{\cdot0} 
                & 0 \ar[r] & \cdots
            }
        }

    \vspace{1mm}
    \noindent Como $C^{i}\cong\Z$ para $i=0,1,2$, entonces $H^{0}(C^{\sbullet})=\Z$, 
    $H^{1}(C^{\sbullet})=0$ y $H^{2}(C^{\sbullet})=\Z_{2}$.
\end{ej}

\noindent Los elementos en $\kr{\partial_{i}}$ e $\im{\partial_{i}}$ se llaman ciclos 
(resp. cociclos) y fronteras (resp. cofronteras) respectivamente. Un elemento en 
$H_{i}(C_{\sbullet})$ se dice clase de homología (resp. clase de cohomología). Los elementos en
los grupos abelianos $C_{i}$ se conocen como cadenas (resp. cocadenas) y los morfismos 
$\partial_{i}$ como diferenciales.

\vspace{1mm}
\noindent Nos gustaría estudiar como interactúan estos objetos enter sí, buscamos una noción de 
morfismo de cadenas (resp. cocadenas) que además induzca uno entre los grupos de homología 
(resp. cohomología), esto motiva la siguiente definición

\begin{dfn}
    Sean $(C_{\sbullet},\partial_{\sbullet})$ y $(D_{\sbullet},\partial_{\sbullet})$ dos complejos 
    de cadenas. Un mapeo de cadenas es una colección de homomorfismos $f_{n}:C_{n}\to D_{n}$ 
    tal que $\partial_{n}f_{n}=f_{n-1}\partial_{n}$ para todo $n$, es decir, el siguiente diagrama 
    conmuta

    \centerline{
        \xymatrix{
            C_{n} \ar[r]^{\partial_{n}} \ar[d]^{f_{n}} & C_{n-1} \ar[d]^{f_{n-1}} \\
            D_{n} \ar[r]^{\partial_{n}} & D_{n-1}
        }
    }
    \noindent y se denota por $f_{\sbullet}:C_{\sbullet}\to D_{\sbullet}$.
\end{dfn}

\begin{prop}
    Si $f_{\sbullet}:C_{\sbullet}\to D_{\sbullet}$ es un mapeo de cadenas, entonces la asignación 
    $f_{*}:H_{n}(C_{\sbullet})\to H_{n}(D_{\sbullet})$ dada por
    \begin{equation*}
        f_{*}([x])=[f_{n}(x)]
    \end{equation*}
    esta bien definida y es un homomorfismo de grupos.
\end{prop}

\begin{proof}
    Sea $x\in ker\partial_{n}$ entonces $\partial_{n}f_{n}(x)=f_{n-1}\partial_{n}(x)
    =f_{n-1}(0)=0$. Así, $f_{n}(x)\in\kr{\partial_{n}}$ y por lo tanto la expresión tiene sentido. 
    
    \vspace{1mm}
    \noindent Si $[x]=[y]$ entonces $x-y=\partial_{n+1}(z)$ para $z\in C_{n+1}$, se sigue que 
    $f_{n}(x)-f_{n}(y)=f_{n}\partial_{n+1}(z)=\partial_{n+1}f_{n+1}(z)$. Concluimos que 
    $[f_{n}(x)]=[f_{n}(y)]$. Que sea homomorfismo es directo de la definición.
\end{proof}

\begin{ej}
    Consideremos la siguiente situación

    \centerline{
        \xymatrix{
            \cdots \ar[r] & 0 \ar[r]^{0} \ar[d] & \Z \ar[r]^{3} \ar[d]^{id} & 
            \Z \ar[r]^{0} \ar[d]^{\pi} & \Z \ar[r] \ar[d]^{id} & 0 
            & C_{\sbullet} \ar[d]^{f_{\sbullet}} \\
            \cdots \ar[r] & 0 \ar[r]^{0} & \Z \ar[r]^{3} & \Z_{3} \ar[r]^{0} & \Z \ar[r] & 0 
            & D_{\sbullet}
        }
    }
    \vspace{2mm}
    \noindent Con $f_{\sbullet}$ un mapeo de cadenas. Entonces $f_{*}:H_{2}(C_{\sbullet})\to 
    H_{2}(D_{\sbullet})$ es el morfismo trivial, ya que $H_{2}(C_{\sbullet})=0$. Mientras que 
    $\pi_{*}:H_{1}(C_{\sbullet})=\Z_{3}\to H_{1}(D_{\sbullet})=\Z_{3}$ es la identidad.
\end{ej}

\begin{obs}
    Notemos que si $g_{\sbullet}:D_{\sbullet}\to G_{\sbullet}$ es un mapeo de cadenas, entonces la 
    colección de morfimos $(g\circ f)_{\sbullet}:C_{\sbullet}\to G_{\sbullet}$ es un mapeo de 
    cadenas y el siguiente diagrama conmuta
    
        \centerline{
            \xymatrix{
                H_{n}(C_{\sbullet}) \ar[rr]^{(g\circ f)_{*}} \ar[rd]^{f_{*}} & 
                & H_{n}(G_{\sbullet}) \\
                & H_{n}(D_{\sbullet}) \ar[ru]^{g_{*}}
            }
        }
    \noindent En efecto, $\partial_{n}g_{n} f_{n}=g_{n-1}\partial_{n}f_{n}=g_{n-1}f_{n-1}
    \partial_{n}$. Por otro lado, tenemos que $(g\circ f)_{*}([x])=[(g\circ f)(x)]=g_{*}([f(x)])=
    (g_{*}\circ f_{*})([x])$, lo que prueba la afirmación.
\end{obs}

\noindent La definición de mapeo de cocadena para cohomología es análoga, es decir, se debe 
cumplir que $f^{n+1}\partial^{n}=\partial^{n}f^{n}$ y los resultados son similares.

\vspace{1mm}
\noindent Los resultados vistos en esta parte serán vitales para más adelante. A partir de este
momento los índices para indicar los morfismos $\partial_{i}$ entre complejos de cadenas no se
escribirán, esto para mayor comodidad.

\begin{dfn}
    Sean $f_{\sbullet},g_{\sbullet}:C_{\sbullet}\to D_{\sbullet}$ mapeos de cadenas. Una 
    homotopía de cadenas es una colección de morfismos
    \begin{align*}
        h_{n}:C_{n}\to C_{n+1}\htext{tales que} \\
        f_{n}-g_{n}=\partial h_{n}+h_{n-1}\partial
    \end{align*}
    Lo denotamos como $f_{\sbullet}\sim g_{\sbullet}$.
\end{dfn}

\begin{prop}
    Sea $f_{\sbullet}\sim g_{\sbullet}$ entonces $f_{*}=g_{*}$.
\end{prop}

\begin{proof}
    Sea $[x]\in H_{n}(C_{\bullet})$, por definición, sabemos que $\partial x=0$, luego
    \begin{equation*}
        (f_{*}-g_{*})([x])=[(f-g)(x)]=[(\partial h+h\partial)(x)]=[\partial hx]=0
    \end{equation*}
    lo que prueba la afirmación.
\end{proof}

\noindent Para cohomología, una homotopía de cocadena se define como la colección de morfismos 
$h^{n}:C^{n}\to C^{n-1}$ tales que $f^{n}-g^{n}=\partial h^{n}+h^{n+1}\partial$, los resultados de
homología se extienden a cohomología.

\begin{dfn}
    Sean $i_{\sbullet}:A_{\sbullet}\to B_{\sbullet}$ y $j_{\sbullet}:B_{\sbullet}\to C_{\sbullet}$
    dos mapeos de cadenas. Decimos que forman una secuencia exacta corta si la secuencia

    \vspace{2mm}
    \centerline{
        \xymatrix{
            0 \ar[r] & A_{n} \ar[r]^{i_{n}} & B_{n} \ar[r]^{j_{n}} & C_{n} \ar[r] & 0
        }
    }
    \vspace{2mm}
    \noindent es una secuencia exacta y corta de grupos abelianos libres para todo $n\in\N$. Lo 
    denotamos como $0\xrightarrow[]{} A_{\sbullet}\xrightarrow[]{i_{\sbullet}} 
    B_{\sbullet}\xrightarrow[]{j_{\sbullet}} C_{\sbullet}\xrightarrow[]{}0$.
\end{dfn}

\begin{teo}[Lema de la serpiente]
    Sea $0\to A_{\sbullet}\xrightarrow[]{i_{\sbullet}} B_{\sbullet}\xrightarrow[]{j_{\sbullet}} 
    C_{\sbullet}\to0$ una secuencia exacta corta de complejos de cadenas, entonces existen 
    morfismos
    \begin{equation*}
        \delta_{n}:H_{n}(C_{\sbullet})\to H_{n-1}(A_{\sbullet})
    \end{equation*}
    tales que la siguiente secuencia es exacta
    
    \vspace{2mm}
    \centerline{
        \xymatrix{
            \ar[r]^-{\delta_{n+1}} & H_{n}(A_{\sbullet}) \ar[r]^{i_{*}} 
            & H_{n}(B_{\sbullet}) \ar[r]^{j_{*}} & H_{n}(C_{\sbullet}) 
            \ar `[dl] `[l] `[llld]_{\delta_{n}} `[d] [dll] \\
            & H_{n-1}(A_{\sbullet}) \ar[r]^{i_{*}} & H_{n-1}(B_{\sbullet}) \ar[r]^{j_{*}} 
            & H_{n-1}(C_{\sbullet}) \ar[r] & \cdots \\
            & \cdots \ar[r] & H_{0}(B_{\sbullet}) \ar[r] & H_{0}(C_{\sbullet}) \ar[r] & 0 
        }
    }
\end{teo}

\noindent La demostración de este resultado se encuentra en \cite{Hatcher} (\textit{Hatcher, 
Teorema 2.16}) y la construcción del morfismo $\delta$, el morfismo conector, se encuentra un poco 
antes. Se puede aplicar también para cocadenas.

\newpage

\section{Grupos topológicos y superficies} \hspace{1mm}

\subsection{Ejemplos} \hspace{1mm} %% Esta parte esta lista

\vspace{1mm}
\noindent Queremos familiarizarnos con la noción de grupo topológico, por lo que trabajaremos con 
algunos de los ejemplos mencionados durante la introducción. Comenzamos con el mas sencillo,

\begin{ej}
    El espacio $\S^{1}$ es grupo topológico, pensamos el espacio en $\C$, luego la operación de 
    grupo viene dada por la multiplicación en $\C$, recordemos que
    \begin{equation*}
        \abs{z\cdot\omega}=\abs{z}\cdot\abs{\omega}
        \htext{para todo }z,\omega\in\C
    \end{equation*}
    lo que implica que $z\cdot\omega\in\S^{1}$. Por otro lado, como 
    $\S^{1}\subseteq\C\setminus\{0\}$, la función que mapea $z$ en $\frac{1}{z}$ es continua.
    Concluimos que $\S^{1}$ es grupo topológico.
\end{ej}

\noindent A partir de este ejemplo daremos estructura de grupo topológico a $\T^{2}$, aprovechando
el hecho de que $\T^{2}=\S^{1}\times\S^{1}$.

\begin{ej}
    Definimos la operación de grupo en $\T^{2}=\S^{1}\times\S^{1}$ como
    \begin{equation*}
        (a,b)\cdot(c,d)=(ac,bd)
    \end{equation*}
    que esta bien definida por el ejemplo anterior, tiene neutro $(e,e)$, es asociativa y 
    multiplicar es continuo por la topología producto. Además, el inverso, esta dado por
    \begin{equation*}
        (z,\omega)^{-1}:=\left(\frac{1}{z},\frac{1}{\omega}\right)
    \end{equation*}
    nuevamente, el mapa es continuo por la topología producto. Por lo tanto, $\T^{2}$ es un grupo
    topológico.
\end{ej}

\noindent Pasando a ejemplos mas sofisticados, en primer lugar, debemos definir los cuaterniones,
un cuaternión es un elemento de la forma
\begin{equation*}
    q=a+bi+cj+dk
    \htext{con }a,b,c,d\in\R
\end{equation*}
y $i,j,k$ son unidades imaginarias tales que $i^{2}=j^{2}=k^{2}=ijk=-1$, son una $\R-$álgebra de
división no conmutativa, es decir, cada elemento tiene un inverso, la multiplicación es 
asociativa, pero no es conmutativa. El conjunto de los cuaterniones se denota como $\H$ y esta 
dotado de la norma
\begin{equation*}
    \abs{q}=\sqrt{q\cdot\overline{q}}
    \hhtext{donde}
    \overline{q}:=a-bi-cj-dk
\end{equation*}
es un cálculo directo que $q\cdot\overline{q}=a^{2}+b^{2}+c^{2}+d^{2}$, que corresponde a la norma
en $\R^{4}$, por lo que la identificación
\begin{equation*}
    (a,b,c,d)\mapsto q=a+bi+cj+dk
\end{equation*}
en realidad es una isometría.

\begin{ej}
    Pensamos $\S^{3}$ en $\H$, se verifica facilmente que $\overline{q\cdot h}
    =\overline{q}\cdot\overline{h}$, lo que implica que
    \begin{equation*}
        \abs{q\cdot h}=\abs{q}\cdot\abs{h}
    \end{equation*}
    es decir, $q\cdot h\in\S^{3}$ para todo $q,h\in\S^{3}$, por lo que la operación de grupo esta
    bien definida, por otro lado, si multiplicamos dos cuaterniones, nos damos cuenta de que es 
    polinomial en cada entrada, por ende, es continua. El inverso viene dado por
    \begin{equation*}
        q^{-1}=1-bi-cj-dk
    \end{equation*}
    que nuevamente es continuo y esta bien definido. Concluimos que $\S^{3}$ es grupo topológico.
\end{ej}

\newpage
\begin{ej}
    Por último, veamos que $\R\P^{3}$ tambíen es grupo topológico, tenemos el diagrama
    
    \centerline{
    \xymatrix{
            \S^{3}\times\S^{3} \ar[d]_{\pi\times\pi} 
            \ar[rd]^{\pi\circ m} \\
            \R\P^{3}\times\R\P^{3} \ar@{-->}[r] & \R\P^{3}
        }
    }
    \vspace{2mm}
    \noindent Donde $\pi:\S^{3}\to\R\P^{3}$ es la proyección y $m$ la multiplicación en $\S^{3}$.
    En primer lugar, notamos que $\pi\times\pi$ es cociente, luego, para definir un producto basta
    probar que $\pi\circ m$ es constante en las fibras de $\pi\times\pi$. En efecto,
    \begin{equation*}
        [(\lambda q)(\mu h)]=[(\lambda\mu)(qh)]=[qh]
    \end{equation*}
    la multiplicación hereda las propiedades de la multiplicación en $\S^{3}$ y con lo anterior,
    es continua. De manera análoga, la función
    \begin{equation*}
        [q]\mapsto [q]^{-1}:=[q^{-1}]
    \end{equation*}
    esta bien definida y es continua.
\end{ej}

\begin{obs}
    Los dos ejemplos en realidad se deben a que $\H$ es un álgebra de división y la 
    multiplicación es continua bajo la norma euclideana, al igual que el mapa que manda un 
    elemento a su inverso multiplicativo y que $\{\pm1\}$ esta en el centro de $\H$.
\end{obs}

\subsection{Superficies} \hspace{1mm} %% Esta parte esta lista

\vspace{1mm}
\noindent Hablaremos un poco sobre superficies y las consecuencias a las que llegaremos durante el
desarrollo de la tesis.

\begin{dfn}
    Sea $X$ un espacio, decimos que es una superficie si es una $2-$variedad, es decir, un espacio
    Hausdorff, segundo contable y localmente euclideano de dimensión 2.
\end{dfn}

\noindent Hasta ahora, conocemos tres espacios que cumplen esta definición, que son $\S^{2}$, 
$\T^{2}$ y $\R\P^{2}$, todas compactas y conexas. Lo sorpresivo, es que estas superficies junto
con las sumas conexas (que pronto definiremos) son las únicas superficies, salvo homeomorfismo.

\begin{teo}[Teorema de clasificación de superficies]
    Toda superficie, no vacía, compacta y conexa es homeomorfa a una de las siguientes
    \begin{enumerate}
        \item La $2-$esfera, es decir, $\S^{2}$
        \item La suma conexa de $n$ copias de $\T^{2}$.
        \item La suma conexa de $n$ copias de $\R\P^{2}$.
    \end{enumerate}
\end{teo}

\noindent La demostración se puede encontrar en \cite{Lee} (\textit{Lee, Teorema 6.15}). Como 
consecuencia inmediata del teorema, resulta que $\T^{2}$ es la única superficie, compacta y conexa 
que es $H-$espacio, porque como veremos más adelante, ninguna de las otras superficies cumplen lo 
necesario.

\begin{teo}
    La única superficie compacta y conexa que es $H-$espacio es $\T^{2}$.
\end{teo}

\noindent La demostración se irá desmenuzando durante la tesis, pero en resumen, la $2-$esfera
presenta una obstrucción en el anillo de cohomología, el plano proyectivo en el hecho de que su
cubriente universal es $\S^{2}$ y por lo tanto no puede ser $H-$espacio y los espacios 
$\#^{n}\T^{2}$ (suma conexa de toros) y $\#^{n}\R\P^{2}$ (suma conexa de planos proyectivos) 
tienen grupo fundamental no abeliano para $n>1$.

\newpage

\vspace{1mm}
\noindent La manera usual de definir la suma conexa de superficies es eliminando un abierto 
homeomorfo a una bola abierta de $\R^{2}$ a ambos espacios y ``pegandolos'' en la frontera que 
produce esta eliminación, sin embargo, daremos una definición equivalente que resultará más útil 
para calcular el grupo fundamental de estas superficies.

\begin{dfn}
    Sea $n>1$. Consideremos el espacio obtenido de un $4n-$polígono regular con la siguiente 
    identificación en sus aristas
    \[
        (a_{1}b_{1}a_{1}^{-1}b_{1}^{-1})(a_{2}b_{2}a_{2}^{-1}b_{2}^{-1})
        \cdots(a_{n}b_{n}a_{n}^{-1}b_{n}^{-1})
    \]
    Este espacio se conoce como la suma conexa de $n-$copias del toro y se denota por 
    $\#^{n}\T^{2}$.
\end{dfn}

\begin{ej}
    Para $n=2$, se representa en la figura \ref{figura1}. Si trazamos el segmento indicado, que 
    divide al polígono a la mitad, cada parte representa un espacio homeomorfo a un toro, si 
    las ``pegamos'' a lo largo del segmento se obtiene la suma conexa usual de dos toros.

    \begin{figure}[h!]
        \centering
        \includegraphics[scale=0.28]{suma_conexa_toros.png}
        \caption{Suma conexa de dos toros.}
        \label{figura1}
    \end{figure}

    \vspace{1mm}
    \noindent La identificación siempre empieza desde el polo norte y la dirección positiva es en
    sentido antihorario y la dirección negativa en sentido horario. Siempre consideramos el 
    interior del polígono.
\end{ej}

\begin{dfn}
    Sea $n>1$. Consideremos el espacio obtenido de un $2n-$polígono regular con la siguiente 
    identificación en sus aristas
    \[
        (a_{1}a_{1})(a_{2}a_{2})\cdots(a_{n}a_{n})
    \]
    Se conoce como la suma conexa de $n-$copias del plano proyectivo y se denota por 
    $\#^{n}\R\P^{2}$.
\end{dfn}

\begin{ej}
    La suma conexa de dos planos proyectivos se representa en la figura \ref{figura2}, el 
    argumento para visualizarlo es similar al toro y vemos que la definición coincide con la 
    construcción mencionada al inicio, es decir, quitar un abierto y pegar los espacios en la 
    frontera.
    
    \begin{figure}[h!]
        \centering
        \includegraphics[scale=0.15]{suma_conexa_planos_proyectivos.png}
        \caption{Suma conexa de dos planos proyectivos.}
        \label{figura2}
    \end{figure}
\end{ej}

\noindent Los ejemplos, definiciones y argumentos para la suma conexa de toros y planos 
proyectivos fueron extraidos de \cite{Munkres} (\textit{Munkres, sección 74}).

\newpage

\section{Grupo fundamental} \hspace{1mm}

\vspace{1mm}
\noindent Sea $X$ un espacio topológico y $x_{0}\in X$, recordemos que el grupo fundamental, 
$\pi_{1}(X,x_{0})$, consiste en las clases de equivalencias de lazos basados en $x_{0}$, donde la 
relación esta dada por los lazos que son homotópicos relativo a $\{0,1\}$. La operación de grupo 
es la concatenación de caminos y el neutro, que consiste en el camino constante $x_{0}$, se 
denotan por $*$ y $ct_{x_{0}}$, respectivamente. Esta construcción resulta ser un invariante 
homotópico.

\vspace{1mm}
\noindent Comenzaremos con el resultado más accesible.

\begin{teo}\label{teo:grupo fundamental}
    Sea $X$ un $H-$espacio arcoconexo, entonces $\pi_{1}(X,e)$ es abeliano.
\end{teo}

\begin{proof}
    Sean $\alpha,\beta$ lazos basados en $e$, definimos el lazo
    \[
        \alpha\beta(t):=\mu(\alpha(t),\beta(t))
    \]
    que es un lazo basado en $e$, como los mapas $x\mapsto \mu(x,e),\mu(e,x)$, son homotópicos 
    a la identidad relativo a $e$ resulta que $[\alpha]=[ct_{e}\alpha]=[\alpha ct_{e}]$.
    
    \vspace{1mm}
    \noindent Además, si $\alpha\sim\hat{\alpha}$ y $\beta\sim\hat{\beta}$ entonces 
    $\alpha\beta\sim\hat{\alpha}\hat{\beta}$. En efecto, sean $H_{\alpha}$ y $H_{\beta}$ las 
    homotopías respectivas, definimos la función continua

    \vspace{2mm}
    \centerline{
        \xymatrixcolsep{3pc}\xymatrix{
            [0,1]\times[0,1] \ar[r]^-{(H_{\alpha},H_{\beta})} \ar@/_1.5pc/[rr]_-{H}
            & X\times X \ar[r]^-{\mu} & X
        }
    }
    \noindent que corresponde a una homotopía entre los caminos $\alpha\beta$ y 
    $\hat{\alpha}\hat{\beta}$ relativa a $\{0,1\}$. Afirmamos que 
    $(\alpha*\beta)(\gamma*\delta)=\alpha\gamma*\beta\delta$. Sea $t\in[0,\frac{1}{2}]$, entonces
    \begin{align*}
        (\alpha*\beta)(\gamma*\delta)(t) &= \mu((\alpha*\beta)(t),(\gamma*\delta)(t)) \\
        &= \mu(\alpha(2t),\gamma(2t))=\alpha\gamma(2t)
    \end{align*}
    similarmente se tiene para $t\in[\frac{1}{2},1]$. Notemos lo siguiente
    \begin{equation*}
        [\alpha]*[\beta]=[ct_{e}\alpha]*[\beta ct_{e}]=[ct_{e}\alpha*\beta ct_{e}]
        =[(ct_{e}*\beta)(\alpha*ct_{e})]=[\beta\alpha]
    \end{equation*}
    y en el otro lado vemos que
    \begin{equation*}
            [\beta]*[\alpha]=[\beta ct_{e}]*[ct_{e}\alpha]=[(\beta*ct_{e})(ct_{e}*\alpha)]
            =[\beta\alpha]
    \end{equation*}
    lo que concluye la demostración.
\end{proof}

\noindent Encontramos la primera obstrucción para ser $H-$espacio, para la siguiente necesitamos 
desarrollar la teoría de cubrientes y un resultado útil es el siguiente lema

\begin{lema}
    Sean $X$ e $Y$ espacios, entonces $\pi_{1}(X\times Y,(x_{0},y_{0}))\cong\pi_{1}(X,x_{0})
    \times\pi_{1}(Y,y_{0})$.
\end{lema}

\begin{proof}
    Sean $\pi_{X}:X\times Y\to X$ y $\pi_{Y}:X\times Y\to Y$ las proyecciones, dado $\gamma$ un 
    lazo en $X\times Y$ denotamos por $\gamma_{1}:=\pi_{X}\circ\gamma$ y $\gamma_{2}
    :=\pi_{Y}\circ\gamma$, notar que $\gamma=(\gamma_{1},\gamma_{2})$. Definimos el mapa
    \begin{equation*}
        \phi([\gamma]):=([\gamma_{1}],[\gamma_{2}])
    \end{equation*}
    que es morfismo y esta bien definido porque $\pi_{X*}$ y $\pi_{Y*}$ son morfismos y están bien 
    definidos. Tiene inversa dada por $\psi([\alpha],[\beta]):=[(\alpha,\beta)]$.

    \vspace{1mm}
    \noindent El argumento para ver que la inversa esta bien definida es idéntico al usado en el 
    teorema anterior cuando probamos que $\alpha\beta\sim\hat{\alpha}\hat{\beta}$.
\end{proof}

\begin{cor}
    El producto de espacios simplemente conexos es simplemente conexo.
\end{cor}

\noindent No hay nada que probar, es consecuencia inmediata del lema.

\subsection{Espacios cubrientes} \hspace{1mm} %% Esta parte está lista

\vspace{1mm}
\noindent El espacio cubriente, al igual que el teorema de Seifert-van Kampen, permite calcular
el grupo fundamental de un espacio, sin embargo, esta no es la razón por la cual queremos estudiar
espacios cubrientes, nuestro interés radica en un resultado en particular de la teoría que nos
permitirá obtener más obstrucciones para ser $H-$espacio.

\begin{dfn}
    El par $(\hat{X},p)$ es un espacio cubriente si $p:\hat{X}\to X$ es continua tal que para todo
    $x\in X$, existe una vecindad $U\subseteq X$ de $x$ tal que
    \begin{itemize}
        \item $p^{-1}(U)=\bigsqcup_{\lambda\in A}V_{\lambda}$ con $V_{\lambda}$ abiertos 
        disjuntos.

        \item $p\big|_{V_{\lambda}}:V_{\lambda}\to U$ es homeomorfismo.
    \end{itemize}
\end{dfn}

\noindent Dos consecuencias importantes de la teoría de espacios cubrientes, que también lo son 
para nuestro propósito, son el lema del levantamiento de homotopía y el teorema del levantamiento 
y los enunciamos a continuación

\begin{lema}[Levantamiento de Homotopía]
    Sean $W$ un espacio y $p:\hat{X}\to X$ un espacio cubriente. Sea $H:W\times[0,1]\to X$ una 
    homotopía y $h:W\times\{0\}\to \hat{X}$ tal que $ph=H|_{W\times\{0\}}$. Entonces, existe una
    única homotopía $\hat{H}:W\times[0,1]\to\hat{X}$ de modo que el siguiente diagrama conmuta
    
    \vspace{1mm}
    \centerline{
        \xymatrix{
            W\times\{0\} \ar@{^{(}->}[d] \ar[r]^-{h} & \hat{X} \ar[d]^{p} \\
            W\times[0,1] \ar[r]^-{H} \ar@{-->}[ru]^{\hat{H}} & X
        }
    }
    \vspace{1mm}
    \noindent Más aún, si $H$ es una homotopía relativa, también lo es $\hat{H}$.
\end{lema}

\begin{teo}[Teorema del Levantamiento]
    Supongamos que $p:\hat{X}\to X$ es un espacio cubriente con $p(\hat{x}_{0})=x_{0}$. Dado $W$ 
    un espacio arcoconexo y localmente arcoconexo con una función $f:W\to X$ tal que 
    $f(w_{0})=x_{0}$. Entonces existe una única
    \begin{equation*}
        g:W\to \hat{X}
        \hhtext{continua de modo que }g(w_{0})=\hat{x}_{0}
        \hhtext{y}p\circ g=f
    \end{equation*}
    si y solo si $f_{*}(\pi_{1}(W,w_{0}))\subseteq p_{*}(\pi_{1}(\hat{X},\hat{x}_{0}))$.
\end{teo}

\noindent En otras palabras, la función $g:W\to\hat{X}$ es la única función que hace que el 
siguiente diagrama conmute

\centerline{
    \xymatrix{
        w_{0} \ar@{^{(}->}[d] \ar[r]^{x_{0}} & \hat{X} \ar[d]^{p} \\
        W \ar[r]^{f} \ar@{-->}[ru]^{g} & X
    }
}
\noindent Notar que si $W=\hat{X}$, $f=p$ y $w_{0}=x_{0}$, entonces $id_{\hat{X}}$ hace que el 
diagrama conmute y por teorema del levantamiento es la única que lo hace. La demostración de ambos 
resultados se encuentra en \cite{Bredon} (\textit{Bredon, Teorema 3.4}) y 
(\textit{Bredon, Teorema 4.1}), respectivamente.

\begin{dfn}
    Sea $\tilde{X}\xrightarrow[]{p}X$ un espacio cubriente arcoconexo, localmente arcoconexo y 
    simplemente conexo, decimos que $\tilde{X}$ es el cubriente universal de $X$.
\end{dfn}

\noindent El siguiente corolario nos asegura que el cubriente universal de un espacio está bien
definido.

\newpage
\begin{cor}
    El cubriente universal de un espacio es único, salvo homeomorfismo.
\end{cor}

\begin{proof}
    Sean $(Y,p)$ y $(W,q)$ espacios cubrientes de $X$ arcoconexos, localmente arcoconexos y 
    simplemente conexos, por el teorema de levantamiento, existen $\hat{p}$ y $\hat{q}$ tales 
    que los diagramas conmutan
    
    \vspace{1mm}
    \centerline{
        \xymatrix{
            y_{0} \ar[r] \ar@{^{(}->}[d] & W \ar[d]^{q} \\ Y \ar[r]_{p} \ar@{-->}[ru]_{\hat{p}} 
            & X
        }
        \hspace{2cm}
        \xymatrix{
            w_{0} \ar[r] \ar@{^{(}->}[d] & Y \ar[d]^{p} \\ W \ar[r]_{q} \ar@{-->}[ru]_{\hat{q}} 
            & X
        }
    }
    \noindent Veamos que $\hat{p}\circ\hat{q}=id_{W}$. En efecto, notemos que 
    $\hat{p}\circ\hat{q}(w_{0})=w_{0}$ y que $q\circ\hat{p}\circ\hat{q}=p\circ\hat{q}=q$ y por 
    unicidad del levantamiento se tiene la igualdad, la otra es análoga.
\end{proof}

\begin{obs}
    Como ambos espacios son simplemente conexos, por definición se tiene que $\pi_{1}(Y,y_{0})=0$ 
    y $\pi_{1}(W,w_{0})=0$, lo que permite utilizar el teorema del levantamiento. Por otro lado,
    el cubrimiento universal no siempre existe, pero en caso de hacerlo, es único.
\end{obs}

\begin{teo}
    El cubriente universal de un $H-$espacio es $H-$espacio.
\end{teo}

\begin{proof}
    Sea $X$ un $H-$espacio y $(\hat{X},p)$ su cubrimiento universal. En primer lugar, notemos 
    que el espacio $\hat{X}\times\hat{X}$ es simplemente conexo, por que el producto de espacios 
    simplemente conexos es simplemente conexo y por la misma razón, también es arcoconexo y 
    localmente arcoconexo.

    \vspace{1mm}
    \noindent Por teorema del levantamiento, existe una única función 
    $\hat{\mu}:\hat{X}\times\hat{X}\to\hat{X}$ tal que el siguiente diagrama conmuta

    \centerline{
        \xymatrixcolsep{3pc}\xymatrix{
            (\hat{e},\hat{e}) \ar[r]^{\hat{e}} \ar@{^{(}->}[d] & \hat{X} \ar[d]^-{p} \\
            \hat{X}\times\hat{X} \ar[r]^-{\mu'} \ar@{-->}[ru]^-{\hat{\mu}} 
            & X
        }
    }
    \noindent donde $p(\hat{e})=e$ y $\mu'=\mu\circ p\times p$. Veamos que $\hat{\mu}$ y $\hat{e}$ 
    le dan estructura de $H-$espacio a $\hat{X}$, para ello, debemos probar que los mapas
    $\hat{\mu}(\hat{x},\hat{e})$ y $\hat{\mu}(\hat{e},\hat{x})$ son homotópicos a la identidad
    relativo a $\hat{e}$. Demostraremos la homotopía para $\hat{\mu}(\hat{x},\hat{e})$ y la otra
    homotopía es análoga.

    \vspace{1mm}
    \noindent Sea $H$ la homotopía relativa a $e$ entre el mapa $\mu(x,e)$ y la identidad, 
    definimos
    \begin{equation*}
        H'(\hat{x},t):=H(p(\hat{x}),t)
        \htext{para todo}\hat{x}\in\hat{X}
    \end{equation*}
    que es una homotopía relativa a $\hat{e}$ entre $\mu'(\hat{x},\hat{e})$ y $p$. Como 
    $p\hat{\mu}=\mu'$, por el lema del levantamiento de homotopía, existe 
    $\hat{H}:\hat{X}\times[0,1]\to\hat{X}$ una homotopía relativa a $\hat{e}$ tal que el siguiente
    diagrama conmuta

    \vspace{1mm}
    \centerline{
        \xymatrix{
            \hat{X}\times\{0\} \ar@{^{(}->}[d] \ar[r]^-{\hat{\mu}} & \hat{X} \ar[d]^{p} \\
            \hat{X}\times[0,1] \ar[r]^-{H'} \ar@{-->}[ru]^{\hat{H}} & X
        }
    }
    \vspace{1mm}
    \noindent Veamos que es la homotopía buscada, en efecto, por conmutatividad del diagrama se 
    tiene que $\hat{H}(\hat{x},0)=\hat{\mu}(\hat{x},\hat{e})$ y por otro lado, la función 
    $\hat{H}(\hat{x},1)$ cumple que $p\hat{H}(\hat{x},1)=p(\hat{x})$ y $\hat{H}(\hat{e},1)
    =\hat{\mu}(\hat{e},\hat{e})=\hat{e}$, es decir, es un levantamiento de $p$ y por ende es la 
    identidad en $\hat{X}$.
\end{proof}

\begin{lema}
    El cubriente universal de $\R\P^{2}$ es $\S^{2}$. Luego, $\R\P^{2}$ no es $H-$espacio.
\end{lema}

\begin{proof}
    Sabemos que $\S^{2}$ es arcoconexo, localmente arcoconexo y simplemente conexo, afirmamos que 
    $(\S^{2},\pi)$ cubre a $\R\P^{2}$, donde $\pi:\S^{2}\to\R\P^{2}$ es la proyección. En $\S^{2}$
    consideramos los abiertos
    \begin{equation*}
        V_{1}=\{(x,y,z)\in\S^{3}:z>0\}
        \hhtext{y}
        V_{2}=\{(x,y,z)\in\S^{3}:z<0\}
    \end{equation*}
    Además, el abierto $V_{1}\sqcup V_{2}$ es saturado, lo que implica que 
    $U=\pi(V_{1}\sqcup V_{2})$ es abierto en $\R\P^{2}$ y además $\pi^{-1}(U)
    =\pi^{-1}(\pi(V_{1}\sqcup V_{2}))=V_{1}\sqcup V_{2}$.

    \vspace{1mm}
    \noindent Debemos probar que $\pi|_{V_{i}}:V_{i}\to U$ es homeomorfismo, en efecto, sea 
    $O\subseteq V_{i}$ un abierto, consideramos el abierto $-O=\{-x:x\in O\}$, luego, el abierto 
    $O\cup-O$ es saturado y por lo tanto $\pi(O\cup-O)=\pi(O)\cup\pi(-O)=\pi(O)$ es abierto, 
    además se tiene que $\pi(x)=[x]$ tiene inversa $i([x])=x\in V_{i}$, es decir, $\pi|_{V_{i}}$ 
    es homeomorfismo. Repitiendo el mismo argumento en las otras coordenadas se tiene el 
    resultado.
\end{proof}

\subsection{Teorema de Seifert-van Kampen} \hspace{1mm} %% Esta parte esta lista

\vspace{1mm}
\noindent Un teorema importante de esta sección es el teorema de Seifert-van Kampen, con él,
podemos calcular el grupo fundamental de un espacio topológico mediante presentaciones de grupos.
Probaremos que el producto wedge y más espacios topológicos no admiten estructura de $H-$espacio
usando este teorema para calcular su grupo fundamental y viendo que no es abeliano, lo que entra
en contradicción con el teorema inicial de esta sección.

\begin{teo}[Teorema de Seifert-van Kampen]
    Sea $X=A\cup B$ con $A$ y $B$ conjuntos abiertos tales que $A\cap B$ es arcoconexo. Sea 
    $x_{0}\in A\cap B$ un punto base, se tienen las inclusiones
    
    \vspace{1mm}
    \centerline{
        \xymatrix{
            A\cap B \ar[r]^-{i_{A}} \ar[d]^-{i_{B}} & A \ar[d]^-{j_{A}} \\
            B \ar[r]^-{j_{B}} & X
        }
    }
    \vspace{1mm}
    \noindent y el diagrama conmuta. Entonces estos mapas inducen un isomorfismo
    \begin{equation*}
        \pi_{1}(X,x_{0})\cong\pi_{1}(A,x_{0})*_{\pi_{1}(A\cap B,x_{0})}\pi_{1}(B,x_{0})
        =\frac{\pi_{1}(A,x_{0})*\pi_{1}(B,x_{0})}{N}
    \end{equation*}
    donde $N=\gen{\gen{j_{A*}i_{A*}(\gamma)j_{B*}i_{B*}(\gamma^{-1}):
    \gamma\in\pi_{1}(A\cap B,x_{0})}}$.
\end{teo}

\noindent La demostración se encuentra en \cite{Bredon} (\textit{Bredon, Teorema 9.4}). La idea es 
probar que el morfismo inducido por el diagrama conmutativo y la propiedad universal del producto 
amalgamado es isomorfismo. Veamos los ejemplos.

\begin{ej}
    El espacio $\S^{1}\vee\S^{1}$, cumple lo siguiente
    \begin{equation*}
        \pi_{1}(\S^{1}\vee\S^{1},1)=F_{2}=\gen{a,b \hspace{1mm} | \hspace{1mm}}
    \end{equation*}
    inductivamente se tiene que
    \begin{equation*}
        \pi_{1}(\vee_{i=1}^{n}\S^{1},1)=F_{n}=\gen{a_{1},\cdots,a_{n} \hspace{1mm}|\hspace{1mm}}
    \end{equation*}
    Veamos que los grupos fundamentales no son abelianos, en efecto, los elementos $a_{1}$ y 
    $a_{2}$ no conmutan, luego, el espacio $\vee_{i=1}^{n}\S^{1}$ no es $H-$espacio para todo 
    $n>1$ por [\ref{teo:grupo fundamental}]. Haremos el calculo para $X=\S^{1}\vee\S^{1}$, el paso 
    inductivo es análogo. En $\S^{1}\sqcup\S^{1}$ consideramos los abiertos
    \[
        A'=\S^{1}\sqcup\S^{1}\setminus\{-1\} \qquad y \qquad B'=\S^{1}\setminus\{-1\}\sqcup\S^{1}
    \]
    que son abiertos saturados y por lo tanto $A=\pi(A')$, $B=\pi(B')$ son abiertos en $X$, por 
    otro lado, notar que $A'\cap B'$ también es abierto saturado y por ende es abierto en $X$ y en 
    este caso se tiene que $\pi(A'\cap B')=\pi(A')\cap\pi(B')$, no es díficil ver que
    \[
        A\simeq\S^{1}, \qquad B\simeq\S^{1} \qquad y \qquad A\cap B\simeq\{*\}
    \]
    donde $\{*\}$ es el espacio que consiste en un punto. Notemos que $A\cap B$ es arcoconexo, 
    así, por Seifert-van Kampen
    \[
        \pi_{1}(X,1)\cong\pi_{1}(A,1)*_{\pi_{1}(A\cap B,1)}\pi_{1}(B,1)
    \]
    pero $\pi_{1}(A\cap B,1)=0$, por lo que $\pi_{1}(X,1)\cong\pi_{1}(A,1)*\pi_{1}(B,1)
    \cong F_{1}*F_{1}=F_{2}$, el grupo libre en dos elementos. Para $n>2$, tomamos los abiertos
    \[
        A'=\bigsqcup_{i=1}^{n-1}\S^{1}\sqcup\S^{1}\setminus\{-1\}
        \qquad y \qquad
        B'=\bigsqcup_{i=1}^{n-1}(\S^{1}\setminus\{-1\})\sqcup\S^{1}
    \]
    y usando inducción el argumento es el mismo.
\end{ej}

\begin{ej}
    Diremos que $X=\#^{n}\T^{2}$ y sea $x_{0}\in X$. Afirmamos que
    \begin{equation*}
        \pi_{1}(X,x_{0})=\gen{a_{1},b_{1},\cdots,a_{n},b_{n} \hspace{1mm} | \hspace{1mm} 
            [a_{1},b_{1}]\cdots[a_{n},b_{n}]=1
        }
    \end{equation*}
    donde $[a,b]=aba^{-1}b^{-1}$ y se le dice el conmutador entre $a$ y $b$. Nuevamente, al igual 
    que en el ejemplo anterior, el grupo fundamental no es abeliano para $n>1$, pero esto no es
    obvio, definimos el morfismo $\phi:F_{2n}\to F_{2}$ por
    \[
        \phi(a_{1})=a \qquad \phi(b_{1})=a^{-1} \qquad \phi(a_{2})=b \qquad \phi(b_{2})=b^{-1}
    \]
    y $\phi(a_{i})=\phi(b_{i})=e$ para $i>2$. Este morfismo, por la propiedad fundamental de 
    presentaciones, desciende a un morfismo $\phi:\pi_{1}(X,x_{0})\to F_{2}$ que claramente es 
    sobreyectivo, se sigue que $\pi_{1}(X,x_{0})$ es no abeliano, en caso contrario, como el 
    morfismo $\phi$ es sobreyectivo, se tendría que $F_{2}$ es abeliano. Por lo tanto 
    $\#^{n}\T^{2}$ no puede ser $H-$espacio [\ref{teo:grupo fundamental}].
    
    \vspace{1mm}
    \noindent Sea $n>1$, recordemos la definición de suma conexa de toros \ref{figura1}, podemos 
    pensar el polígono en $\R^{2}$ y cada vértice corresponde a una raíz $4n$-ésima de la unidad.

    \vspace{1mm}
    \noindent Sea $P$ el polígono en $\R^{2}$. Consideremos los abiertos $A=B_{2\varepsilon}(0)$ y 
    $B=P\setminus B_{\varepsilon}(0)$ con $\varepsilon>0$ tal que $B_{2\varepsilon}(0)
    \subseteq P$. Notemos que $A\cap B$ es arcoconexo. Además, se verifica que
    \[
        A\simeq \{*\}, \qquad B\simeq\partial P/\sim \hspace{2mm} \cong \hspace{2mm} 
        \vee_{i=1}^{n}\S^{1} \qquad y \qquad A\cap B\simeq\S^{1}
    \]
    donde $\sim$ es la realción establecida por las identificaciones. Sean $x_{0}\in A\cap B$ y 
    $c$ el lazo generador en $\pi_{1}(A\cap B,x_{0})$, este lazo es homotópico a $\partial P$ en 
    $B$, que en $\pi_{1}(B,x_{0})=F_{n}$ corresponde al elemento
    \[
        a_{1}b_{1}a_{1}^{-1}b_{1}^{-1}\cdots a_{n}b_{n}a_{n}^{-1}b_{n}^{-1}
    \]
    ver \ref{figura1}. Luego, por Seifert-van Kampen, tenemos que
    \begin{align*}
        \pi_{1}(X,x_{0}) &\cong\pi_{1}(A,x_{0})*_{\pi_{1}(A\cap B,x_{0})}\pi_{1}(B,x_{0})
        =0*_{F_{1}}F_{n} \\
        &= \gen{a_{1},b_{1},\cdots,a_{n},b_{n} \hspace{1mm}|\hspace{1mm} 
            [a_{1},b_{1}]\cdots[a_{n},b_{n}]=1
        }
    \end{align*}
\end{ej}

\noindent Haciendo un proceso análogo, podemos ver que
\begin{equation*}
    \pi_{1}(\#^{n}\R\P^{2},x_{0})=\gen{
        a_{1},\cdots,a_{n} \hspace{1mm}|\hspace{1mm} a_{1}^{2}\cdots a_{n}^{2}=1
    }
\end{equation*}
lo que implica que $\#^{n}\R\P^{2}$ no es $H-$espacio para ningún $n>1$ [\ref{teo:grupo 
fundamental}], el argumento para ver que el grupo fundamental no es abeliano es similar al ejemplo 
anterior. Cuando $n=1$ sabemos que $\S^{2}$ es el cubrimiento universal de $\R\P^{2}$, y por ende, 
si este último es $H-$espacio, se tendría que $\S^{2}$ es $H-$espacio, veremos que esto no es así.

\newpage

\section{Cohomología singular} \hspace{1mm}

\vspace{1mm}
\noindent Antes de cohomología viene homología y una de las principales razones para estudiar 
homología, es que de cierto modo podemos medir o contar los ``agujeros'' $n-$dimensionales de un
espacio topológico, dicho de manera informal, ya que después de todo ¿Qué es un agujero 
$n-$dimensional? Por ejemplo, un loop no será homotópico a un punto en $\S^{1}$.

\vspace{1mm}
\noindent Sin embargo, para esta tesis, no nos interesan los grupos de homología, más bien
son relevantes los grupos de cohomología, que se pueden pensar en cierta medida como el dual de
los grupos de homología, con ellos podemos construir el anillo de cohomología que planteará otra
restricción para ser $H-$espacio.

\subsection{Grupos de cohomología} %% Esta parte esta lista

\begin{dfn}
    Un $n-$simplice estándar es
    \begin{equation*}
        \Delta^{n}=\left\{
            (t_{0},\cdots,t_{n})\in\R^{n+1}:t_{i}\geq0,\hspace{1mm}\sum t_{i}=1
        \right\}
    \end{equation*}
\end{dfn}

\begin{obs}
    La $i-$ésima cara de $\Delta^{n}$ es $\Delta^{n}_{i}
    =\{(t_{0},\cdots,t_{n})\in\Delta^{n}:t_{i}=0\}$. Notamos que un $n-$simplice estándar tiene
    $n+1$ caras, donde cada una se parece a un $(n-1)-$simplice estándar. De hecho, mediante el 
    mapeo
    \begin{align*}
        \delta_{i}:\Delta^{n-1} &\to \Delta^{n}_{i} \\
        (t_{0},\cdots,t_{n-1}) &\to (t_{0},\cdots,t_{i-1},0,t_{i},\cdots,t_{n-1})
    \end{align*}
    resulta ser homeomorfa a $\Delta^{n-1}$.
\end{obs}

\begin{dfn}
    Sea $X$ un espacio topológico. Un $n-$simplice singular en $X$ es una función continua
    $\sigma:\Delta^{n}\to X$.
\end{dfn}

\noindent La aplicación $\sigma\circ\delta_{i}$ usualmente la denotamos como
\begin{equation*}
    \sigma\circ\delta_{i}=:\sigma\big|_{[v_{0},\cdots,\hat{v_{i}},\cdots,v_{n}]}
\end{equation*}
Podemos restringirnos a más puntos, esto correspondería a la composición de múltiples mapas 
$\delta_{i}$ y en este caso se omiten las entradas correspondientes, por ejemplo,
\begin{equation*}
    \sigma\big|_{[\hat{v_{0}},\cdots,\hat{v_{k-1}},v_{k},\cdots,v_{n}]}
    =\sigma\circ\delta_{0}^{k-1}=:\sigma\big|_{[v_{k},\cdots,v_{n}]}
    \htext{con}k\leq n
\end{equation*}
que corresponde a una función continua entre $\Delta^{n-k}$ hacia $X$, es decir, un $(n-k)-$
simplice singular.

\vspace{1mm}
\noindent Dado un espacio $X$, tenemos una colección de $n-$simplices singulares, vamos a 
considerar el grupo abeliano libre generado por este conjunto para construir un complejo de 
cadenas y así los grupos de homología.

\begin{dfn}
    Definimos los grupos
    \begin{equation*}
        C_{n}(X):=\left\{\sum n_{\sigma}\sigma\hspace{1mm}|\hspace{1mm}\sigma:\Delta^{n}\to X
        \text{ simplice singular y }n_{\sigma}\in\Z\hspace{1mm}
        \text{ nulo salvo finitos casos}\right\}
    \end{equation*}
    junto con los diferenciales $\partial_{n}:C_{n}(X)\to C_{n-1}(X)$ dados por
    \begin{equation*}
        \partial_{n}(\sigma)=\sum_{i=0}^{n}(-1)^{i}\sigma\circ\delta_{i}
    \end{equation*}
    que se extiende linealmente.
\end{dfn}

\noindent Al complejo $(C_{\sbullet}(X),\partial_{\sbullet})$ lo llamamos complejo de cadenas 
singular. Para ver que el complejo esta bien definido debemos verificar que 
$\partial_{n-1}\circ\partial_{n}=0$, lo que se muestra en el siguiente lema.

\begin{lema}
    El morfismo $\partial_{n-1}\circ \partial_{n}:C_{n}(X)\to C_{n-2}(X)$ es trivial.
\end{lema}

\begin{proof}
    Basta probar para cada elemento en la base, es decir, para $\sigma:\Delta^{n}\to X$. Tenemos 
    lo siguiente
    \begin{equation*}
        \partial_{n-1}\circ \partial_{n}(\sigma)=\sum_{i=0}^{n-1}(-1)^{i}\sum_{j=0}^{n}(-1)^{j}
        \sigma\circ\delta_{j}\circ\delta_{i}
    \end{equation*}
    Usando que $\delta_{j}\circ\delta_{i}=\delta_{i}\circ\delta_{j-1}$ si $i<j$, vemos que
    \begin{align*}
        \partial_{n-1}\circ \partial_{n}(\sigma) &= \sum_{i<j}(-1)^{i+j}\sigma\circ\delta_{j}
        \circ\delta_{i}+\sum_{i\geq j}(-1)^{i+j}\sigma\circ\delta_{j}\circ\delta_{i} \\
        &= \sum_{i<j}(-1)^{i+j}\sigma\circ\delta_{i}\circ\delta_{j-1}
        +\sum_{i\geq j}(-1)^{i+j}\sigma\circ\delta_{j}\circ\delta_{i}
    \end{align*}
    haciendo un cambio de índice en la primera expresión, resulta que
    \begin{equation*}
        \partial_{n-1}\circ \partial_{n}(\sigma)=\sum_{i\leq j}(-1)^{i+j+1}\sigma\circ\delta_{i}
        \circ\delta_{j}+\sum_{i\geq j}(-1)^{i+j}\sigma\circ\delta_{j}\circ\delta_{i}=0
    \end{equation*}
    La última igualdad se debe a que los términos se cancelan de a pares, puesto que cada término
    en la expresión de la izquierda corresponde a uno de la expresión de la derecha, pero con el
    signo opuesto.
\end{proof}

\noindent El grupo de cohomología singular, se obtiene a partir de dualizar el complejo de cadenas
$(C_{\sbullet}(X),\partial_{\sbullet})$, como se vio en el ejemplo de preliminares, esto induce
un complejo de cocadenas con sus respectivos grupos de cohomología.

\begin{dfn}
    Sea $G$ un grupo abeliano y $X$ un espacio topológico, definimos los grupos
    \begin{equation*}
        C^{i}(X;G)=Hom(C_{i}(X);G)
    \end{equation*}
    con el diferencial $\partial:C^{i}(X;G)\to C^{i+1}(X;G)$ dado por $\partial^{i}(\varphi)
    =\varphi\circ\partial_{i+1}$. Esto define un complejo de cocadenas con cohomología asociada
    \begin{equation*}
        H^{i}(X;G):=H^{i}(C^{\sbullet}(X;G))=\frac{\kr{\partial^{i}}}{\im{\partial^{i-1}}}
    \end{equation*}
    y se dice cohomología de $X$ con coeficientes en $G$.
\end{dfn}

\noindent Hemos cumplido con una parte del objetivo inicial de esta sección, ahora nos gustaría
obtener un morfismo entre los grupos de cohomología de espacios dada una función continua entre 
ellos. Para tal fin tenemos la siguiente proposición.

\begin{prop}
    Sea $f:X\to Y$ una función continua entre espacios topológicos, entonces las funciones
    \begin{equation*}
        f_{n}:C_{n}(X)\to C_{n}(Y)\hhtext{dadas en la base por}f_{n}(\sigma)=f\circ\sigma
    \end{equation*}
    forman un mapeo de cadenas $f_{\sbullet}:C_{\sbullet}(X)\to C_{\sbullet}(Y)$.
\end{prop}

\begin{proof}
    Es directo de la definición que para todo $n\in\N$ la función $f_{n}$ es morfismo de grupos 
    abelianos, por otro lado
    \begin{equation*}
        \partial_{n} f_{n}(\sigma)=\partial_{n}(f\circ\sigma)
        =\sum_{i}(-1)^{i}f\circ\sigma\circ\delta_{i}
        =\sum_{i}(-1)^{i}f_{n-1}(\sigma\circ\delta_{i})=f_{n-1}\partial_{n}(\sigma)
    \end{equation*}
\end{proof}

\noindent Esto define morfismos de cadenas, para cocadenas, definimos lo siguiente
\begin{equation*}
    f^{n}(\varphi)=\varphi\circ f_{n}
    \htext{donde} \varphi\in C^{n}(Y;G)
\end{equation*}
notemos que en este caso $f^{\sbullet}:C^{n}(Y;G)\to C^{n}(X;G)$, que resulta ser un mapa de 
cocadenas, puesto que $f_{\sbullet}$ es un mapeo de cadenas. Juntando resultados previos se tiene
el morfismo
\begin{equation*}
    f^{*}:H^{n}(Y;G)\to H^{n}(X;G)
\end{equation*}
Es directo de la definición verficar que
\begin{equation*}
    (f\circ g)^{*}=g^{*}\circ f^{*}
\end{equation*}
Además, si $f:X\to Y$ es homeomorfismo, entonces el mapa $f^{*}$ es isomorfismo, es decir, tenemos 
que $H^{*}(X;G)$ es invariante topológico.

\vspace{1mm}
\noindent Lo bueno de esta construcción, a diferencia de otras, es que las definiciones dependen 
únicamente de la topología del espacio, es decir, de los abiertos. Pero esto viene con un costo 
asociado, es prácticamente imposible calcular los grupos de cohomología usando solo la definición, 
a pesar de ello, se puede trabajar con ejemplos particulares que serán vitales para más adelante.

\begin{ej}
    Sea $X=pt=\{*\}$, el espacio que consiste de un punto. Entonces, existe un único $n-$simplice 
    singular, digamos, $\sigma_{n}:\Delta^{n}\to pt$, lo que implica que 
    $\sigma_{n}\circ\delta_{i}=\sigma_{n-1}$, luego
    \begin{equation*}
        \partial_{n}(\sigma_{n})=\sum_{i=0}^{n}(-1)^{i}\sigma_{n}\circ\delta_{i}
        =\begin{cases}
            \sigma_{n-1} &\quad\text{ si }n\text{ es par} \\
            0 &\quad\text{ si }n\text{ es impar}
        \end{cases}
    \end{equation*}
    dualizando, vemos que
    \begin{equation*}
        \partial^{n}(\varphi)(\sigma_{n+1})=\varphi(\partial_{n+1}(\sigma_{n+1}))=\begin{cases}
            0 &\quad\text{ si }n\text{ es par} \\
            \varphi(\sigma_{n}) &\quad\text{ si }n\text{ es impar}
        \end{cases}
    \end{equation*}
    para $i>1$ tenemos que
    
    \vspace{1mm}
    \centerline{
        \xymatrix{
            \cdots & C^{i+1}(pt;G) \ar[l]_-{\partial^{i+1}} 
            & C^{i}(pt;G) \ar[l]_-{\partial^{i}} & C^{i-1}(pt;G) \ar[l]_{\partial^{i-1}} 
            & \cdots \ar[l]
        }
    }
    \noindent entonces si $i$ es impar $H^{i}(pt;G)=0$ por que $\partial^{i}$ es inyectivo. Por
    otro lado, si $n$ es par, vemos que
    \begin{equation*}
        H^{i}(pt;G)=\frac{\kr{\partial^{i}}}{\im{\partial^{i-1}}}\cong\frac{G}{G}=0
    \end{equation*}
    esto por que $C^{i}(pt;G)\cong Hom(\Z,G)\cong G$ y el morfismo $\partial^{i-1}$ es 
    isomorfismo. Para $i=0$, vemos que $H^{0}(pt;G)=\kr{\partial^{0}}=C^{0}(pt;G)\cong G$. Así,
    \begin{equation*}
        H^{n}(pt;G)=\begin{cases}
            G &\quad\text{si }n=0 \\
            0 &\quad\text{si }n>0
        \end{cases}
    \end{equation*}
    con $G$ un grupo abeliano cualquiera.
\end{ej}

\begin{lema}
    Sea $X$ un espacio arcoconexo no vacío, entonces $H^{0}(X;G)=G$.
\end{lema}

\begin{proof}
    Nuevamente, por definición, tenemos que $H^{0}(X;G)=\kr{\partial^{0}}$, sea 
    $\varphi\in\kr{\partial^{0}}$, entonces para todo $\tau\in C_{1}(X)$ vemos que
    \begin{equation*}
        0=\partial^{0}(\varphi)(\tau)=\varphi(\partial_{1}(\tau))
        =\varphi(\tau\circ\delta_{0})-\varphi(\tau\circ\delta_{1})
    \end{equation*}
    Sea $x_{0}\in X$, lo vemos como un $0-$simplice singular, dado $\sigma:\Delta^{0}\to X$, como
    el espacio es arcoconexo, existe $\tau_{\sigma}:\Delta^{1}\to X$ tal que 
    $\tau_{\sigma}\circ\delta_{0}=\sigma$ y $\tau_{\sigma}\circ\delta_{1}=x_{0}$, así usando la 
    observación inicial, notamos que
    \begin{equation*}
        \varphi(\sigma)=\varphi(x_{0})
        \htext{para todo}\sigma\in C_{0}(pt)
    \end{equation*}
    entonces el morfismo $\varphi$ queda únicamente determinado por la imagen de $x_{0}$ y por lo
    tanto $H^{0}(X;G)=\kr{\partial^{0}}\cong G$.
\end{proof}

\noindent Lo anterior da pie a la siguiente proposición.

\begin{prop}
    Sea $X=\bigsqcup_{\alpha\in A}X_{\alpha}$ la unión disjunta de componentes arcoconexas,
    entonces si $A$ es finito se tiene el isomorfismo
    \begin{equation*}
        H^{n}(X;G)\cong\bigoplus_{\alpha\in A}H^{n}(X_{\alpha};G)
    \end{equation*}
\end{prop}

\begin{proof} %% Posible revisión de demostración
    La inclusión $i_{\alpha}:X_{\alpha}\to X$ es continua para todo $\alpha\in A$, denotamos por 
    $i_{\alpha,n}$ el mapa inducido a nivel de cadenas. Como $\Delta^{n}$ es arcoconexo, la imagen 
    de $\sigma$ esta en $X_{\alpha}$ para algún $\alpha\in A$, lo que prueba que el mapa
    \begin{equation*}
        i_{n}:\bigoplus_{\alpha\in A} C_{n}(X_{\alpha})\to C_{n}(X)
    \end{equation*}
    es isomorfismo, con $i_{n}=\bigoplus_{\alpha\in A} i_{\alpha,n}$, la colección de morfismos
    $(i_{n})_{n\in\N}$ es mapeo de cadenas, ya que para todo $\alpha\in A$ la colección 
    $(i_{\alpha,n})_{n\in\N}$ es mapeo de cadena, dualizando y usando que $i_{\alpha}^{\sbullet}$ 
    es mapeo de cocadena se obtiene el resultado buscado.
\end{proof}

\subsection{Herramientas y aplicaciones} \hspace{1mm} %% Esta parte esta lista

\vspace{1mm}
\noindent Veremos dos resultados importantes sobre grupos de cohomología singular de un espacio,
ellos son invarianza homotópica y la secuencia de Mayer-Vietoris, para luego aplicar estas 
herramientas a ejemplos.

\begin{teo}[Invarianza homotópica]
    Sean $f,g:X\to Y$ funciones homotópicas. Entonces, inducen el mismo morfismo en cohomología, 
    es decir,
    \begin{equation*}
        f^{*}=g^{*}:H^{*}(Y;G)\to H^{*}(X;G)
    \end{equation*}
\end{teo}

\noindent La demostración de este resultado para homología se encuentra en \cite{Hatcher} 
(\textit{Hatcher, Teorema 2.10}). La idea es construir una homotopía de cadena entre los mapas de 
cadena inducidos por $f$ y $g$ a partir de la homotopía entre ambas. Para cohomología basta 
dualizar la homotopía de cadenas y así obtener una homotopía de cocadena, lo que concluye el 
resultado.

\begin{cor}
    Sea $f:X\to Y$ una equivalencia homotópica, entonces
    \begin{equation*}
        f^{*}:H^{*}(Y;G)\to H^{*}(X;G)
    \end{equation*}
    es isomorfismo.
\end{cor}

\begin{proof}
    Sea $g:Y\to X$ una inversa homotópica, entonces
    \begin{equation*}
        g^{*}\circ f^{*}=(f\circ g)_{*}=(id_{Y})^{*}=id_{H^{*}(Y)}
    \end{equation*}
    Igualmente, tenemos que $f^{*}\circ g^{*}=id_{H^{*}(X;G)}$. Entonces $f^{*}$ es isomorfismo 
    con inversa $g^{*}$.
\end{proof}

\begin{ej}
Como $\R^{n}$ es convexo, entonces es homotópico a un punto, se sigue de un ejemplo anterior que
    \begin{equation*}
        H^{n}(\R^{n};G)=\begin{cases}
            G &\quad\text{si }n=0 \\
            0 &\quad\text{si }n>0
        \end{cases}
    \end{equation*}
\end{ej}

\begin{teo}[Mayer-Vietoris]
    Sea $X=A\cup B$ con $A$ y $B$ conjuntos abiertos. Se tienen las inclusiones
    
    \vspace{1mm}
    \centerline{
        \xymatrix{
            A\cap B \ar[r]^-{i_{A}} \ar[d]^-{i_{B}} & A \ar[d]^-{j_{A}} \\
            B \ar[r]^-{j_{B}} & X
        }
    }
    \vspace{1mm}
    \noindent Entonces, existen homomorfismos $\delta:H^{n}(A\cap B;G)\to H^{n+1}(X;G)$ tales 
    que la siguiente secuencia es exacta
    
    \vspace{1mm}
    \centerline{
        \xymatrix{
            \cdots \ar[r] & H^{n}(X;G) \ar[r]^-{j_{A}^{*}\oplus j_{B}^{*}} 
            & H^{n}(A;G)\oplus H^{n}(B;G) \ar[r]^-{i_{A}^{*}-i_{B}^{*}} & H^{n}(A\cap B;G) 
            \ar `[ld] `[l] `[llld]^{\delta} `[d] [dll] \\
            & H^{n+1}(X;G) \ar[r]^-{j_{A}^{*}\oplus j_{B}^{*}} 
            & H^{n+1}(A;G)\oplus H^{n+1}(B;G) \ar[r]^-{i_{A}^{*}-i_{B}^{*}} 
            & H^{n+1}(A\cap B;G) \ar[r] & \cdots
        }
    }
    \vspace{1mm}
\end{teo}

\noindent La demostración de este teorema se encuentra en \cite{Hatcher} (\textit{Hatcher, sección 
3.1}) y requiere de un resultado previo, el teorema de los simplices pequeños, que se encuentra en 
\cite{Hatcher} (\textit{Hatcher, Proposición 2.21}).

\vspace{1mm}
\noindent Nuestra intención ahora es utilizar las herramientas vistas para calcular los grupos de 
cohomología singular con coeficientes en $\Z$ de las $n-$esferas.

\vspace{1mm}
\noindent Denotaremos por $H^{i}(X)$ al grupo de cohomología singular de $X$ con coeficientes en 
$\Z$.

\begin{ej}
    En $\S^{1}$, consideremos los abiertos $A=\S^{1}\setminus\{N\}$ y $B=\S^{1}\setminus\{S\}$,
    donde $N$ y $S$ son el polo norte y sur respectivamente. Notemos que $A\cong\R\simeq pt$ y del
    mismo modo $B\simeq pt$, por otro lado, $A\cap B\cong\R\sqcup\R\simeq pt\sqcup pt$. Por la 
    secuencia de Mayer-Vietoris, para $i>1$, tenemos la secuencia exacta

    \vspace{1mm}
    \centerline{
        \xymatrix{
            H^{i-1}(A\cap B) \ar[r] & H^{i}(\S^{1}) \ar[r] 
            & H^{i}(A)\oplus H^{i}(B) \ar[r] & H^{i}(A\cap B)
        }
    }
    \vspace{1mm}
    \noindent por lo mencionado anteriormente, vemos que $H^{i-1}(A\cap B)=0$, $H^{i}(A)\oplus 
    H^{i}(B)=0$ y $H^{i}(A\cap B)=0$ cuando $i>1$, lo que implica que $H^{i}(\S^{1})=0$ con $i>1$.

    \vspace{1mm}
    \noindent Sabemos que $\S^{1}$ es arcoconexo, entonces $H^{0}(\S^{1})=G$. Queda el caso cuando 
    $i=1$, tenemos la secuencia exacta

    \vspace{1mm}
    \centerline{
        \xymatrix{
            H^{0}(A)\oplus H^{0}(B) \ar[r]^-{i_{A}^{*}-i_{B}^{*}} & H^{0}(A\cap B) 
            \ar[r]^-{\delta} & H^{1}(\S^{1}) \ar[r] & 0
        }
    }
    \vspace{1mm}
    \noindent Por primer teorema de isomorfismo basta encontrar $\kr{\delta}
    =\im{i_{A}^{*}-i_{B}^{*}}$. Sea $\varphi\in H^{0}(A)$, como $A$ es arcoconexo, sabemos que
    $\varphi$ queda únicamente determinado por la imagen de un único $0-$simplice, como $G=\Z$,
    el morfismo $\varphi(\sigma)=1$ genera a $H^{0}(A)$, así
    \begin{equation*}
        i^{*}_{A}(\varphi)(\sigma)=\varphi(i_{A}\circ\sigma)=1
        \htext{para todo }\sigma\in C_{0}(A\cap B)
    \end{equation*}
    Similarmente $i^{*}_{B}(\psi)(\sigma)=1$ con $\psi$ el generador de $H^{0}(B)$. Como $A\cap B$
    tiene dos componentes arcoconexas, el grupo $H^{0}(A\cap B)$ está generado por dos elementos,
    uno de ellos mapea todos los puntos de una componente a $1$ y mapea a $0$ los puntos de la 
    otra componente, mientras que el otro generador hace lo contrario. Así, el morfismo 
    $\im{i_{A}^{*}-i_{B}^{*}}$ esta representado por
    \begin{equation*}
        \begin{pmatrix}
            \hspace{1mm} 1 & -1 \hspace{1mm} \\ \hspace{1mm} 1 & -1 \hspace{1mm}
        \end{pmatrix}
    \end{equation*}
    lo que implica que $H^{1}(\S^{1})\cong\Z$. En resumen, tenemos lo siguiente
    \begin{equation*}
        H^{i}(\S^{1})=\begin{cases}
            \Z &\quad\text{ si }i=0,1 \\
            0 &\quad\text{ si }i>1
        \end{cases}
    \end{equation*}
\end{ej}

\begin{ej}
    Afirmamos que
    \begin{equation*}
        H^{i}(\S^{n})=\begin{cases}
            \Z &\quad\text{ si }i=0,n \\
            0 &\quad\text{ si }i\neq0,n
        \end{cases}
    \end{equation*}
    Procederemos por inducción, el caso base fue el ejemplo anterior, supongamos que se cumple
    para $n-1$. Consideramos los mismos abiertos que antes y se tiene que $A\cong\R^{n}\simeq pt$ 
    y $B\simeq pt$, sin embargo, para la intersección observamos que $A\cap B\cong
    \R^{n-1}\times\S^{n-1}\simeq\S^{n-1}$. Así, para $i>1$, tenemos la secuencia exacta

    \vspace{1mm}
    \centerline{
        \xymatrix{
            0 \ar[r] & H^{i-1}(A\cap B) \ar[r] & H^{i}(\S^{1}) \ar[r] & 0
        }
    }
    \vspace{1mm}
    \noindent Si $i\neq n$, entonces $H^{i}(\S^{n})=0$, de lo contrario, $H^{n}(\S^{n})\cong 
    H^{n-1}(\S^{n-1})=\Z$. Como $\S^{n}$ es arcoconexo resulta que $H^{0}(\S^{n})=\Z$, queda el 
    caso cuando $i=1$, tenemos la secuencia exacta

    \vspace{1mm}
    \centerline{
        \xymatrix{
            H^{0}(A)\oplus H^{0}(B) \ar[r]^-{i_{A}^{*}-i_{B}^{*}} & H^{0}(A\cap B) 
            \ar[r]^-{\delta} & H^{1}(\S^{n}) \ar[r] & 0
        }
    }
    \vspace{1mm}
    \noindent nuevamente, por primer teorema de isomorfismo, basta estudiar $\kr{\delta}$, para
    ello, vemos la imagen del morfismo $i_{A}^{*}-i_{B}^{*}$ que, por el mismo argumento que 
    antes, esta representado por
    \begin{equation*}
        \begin{pmatrix}
            \hspace{1mm} 1 & -1 \hspace{1mm}
        \end{pmatrix}
    \end{equation*}
    Pero esta vez hay un generador en $H^{0}(A\cap B)$, pues es arcoconexo. Por lo tanto 
    $H^{1}(\S^{n})=0$, lo que concluye la afirmación.
\end{ej}

\subsection{Anillo de cohomología} \hspace{1mm} %% Esta parte esta lista

\vspace{1mm}
\noindent Introduciremos el anillo de cohomología, que nos permitirá rescatar mayor información
sobre un espacio topológico. La idea es asignarle un anillo a un espacio, en realidad una 
$R-$álgebra graduada cuando $R$ es un anillo conmutativo, de modo que una función continua induzca
un morfismo de algebras graduadas. De ahora en adelante supondremos que $R$ es un anillo 
conmutativo.

\begin{dfn}
    Sea $X$ un espacio topológico. Sean $\phi\in C^{k}(X;R)$ y $\psi\in C^{l}(X;R)$. Definimos el
    producto cup $\phi\smile\psi\in C^{k+l}(X;R)$, por
    \begin{equation*}
        (\phi\smile\psi)(\sigma)
        =\phi(\sigma|_{[v_{0},\cdots,v_{k}]})\psi(\sigma|_{[v_{k},\cdots,v_{k+l}]})
    \end{equation*}
    donde $\sigma\in C_{k+l}(X)$ un simplice singular. Este morfismo se extiende linealmente.
\end{dfn}

\noindent Es directo de la definición, que el producto cup es bilineal y asociativo, pues la
multiplicación en $R$ lo es. Además, cumple la siguiente relación con el diferencial.

\begin{lema}
    Sean $\phi\in C^{k}(X;R)$ y $\psi\in C^{l}(X;R)$, entonces
    \begin{equation*}
        d(\phi\smile\psi)=d\phi\smile\psi+(-1)^{k}\phi\smile d\psi
    \end{equation*}
\end{lema}

\begin{proof}
    Sea $\sigma:\Delta^{k+l+1}\to X$ un simplice singular. Entonces
    \begin{align*}
        (d\phi\smile\psi)(\sigma) &= (d\phi)(\sigma\big|_{[v_{0},\cdots,v_{k+1}]})
        \cdot\psi(\sigma\big|_{[v_{k+1},\cdots,v_{k+l+1}]}) \\
        &= \phi\left(
            \sum_{i=0}^{k+1}(-1)^{i}\sigma\big|_{[v_{0},\cdots,\hat{v_{i}},\cdots,v_{k+1}]}
        \right)\cdot\psi(\sigma\big|_{[v_{k+1},\cdots,v_{k+l+1}]})
    \end{align*}
    y además
    \begin{align*}
        (\phi\smile d\psi)(\sigma) &= \phi(\sigma\big|_{[v_{0},\cdots,v_{k}]})
        \cdot(d\psi)(\sigma\big|_{[v_{k},\cdots,v_{k+l+1}]}) \\
        &= \phi(\sigma\big|_{[v_{0},\cdots,v_{k}]})\cdot\psi\left(
            \sum_{i=k}^{k+l+1}(-1)^{i-k}\sigma\big|_{[v_{k},\cdots,\hat{v_{i}},\cdots,v_{k+l+1}]}
        \right) \\
        &= (-1)^{k}\phi(\sigma\big|_{[v_{0},\cdots,v_{k}]})\cdot\psi\left(
            \sum_{i=k}^{k+l+1}(-1)^{i}\sigma\big|_{[v_{k},\cdots,\hat{v_{i}},\cdots,v_{k+l+1}]}
        \right)
    \end{align*}
    notamos que el último término de la primera expresión se cancela con el primer término de la 
    segunda expresión. Por otro lado, tenemos que
    \begin{align*}
        d(\phi\smile\psi)(\sigma) &= (\phi\smile\psi)\left(
            \sum_{i=0}^{k+l+1}(-1)^{i}\sigma\big|_{[v_{0},\cdots,\hat{v_{i}},\cdots,v_{k+l+1}]}
        \right) \\
        &= \sum_{i=0}^{k+l+1}(-1)^{i}(\phi\smile\psi)(
            \sigma\big|_{[v_{0},\cdots,\hat{v_{i}},\cdots,v_{k+l+1}]}
        ) 
    \end{align*}
    vemos que esta última expresión es igual a la primera más la segunda por $(-1)^{k}$, lo que 
    prueba la afirmación.
\end{proof}

\begin{cor}
    El producto cup induce un morfismo
    \begin{equation*}
        \smile:H^{k}(X;R)\times H^{l}(X;R)\to H^{k+l}(X;R) \\
    \end{equation*}
    dado por $[\phi]\smile[\psi]=[\phi\smile\psi]$.
\end{cor}

\begin{proof}
    Sean $\phi\in C^{k}(X;R)$ y $\psi\in C^{l}(X;R)$ cociclos y dado $\phi'\in C^{k}(X;R)$ un 
    cociclo tal que $[\phi]=[\phi']$, entonces $\phi'=\phi+d\phi''$. Luego,
    \begin{align*}
        \phi'\smile\psi &= (\phi+d\phi'')\smile\psi=\phi\smile\psi+d\phi''\smile\psi \\
        &= \phi\smile\psi+d(\phi''\smile\psi)-(-1)^{k+1}\phi''\smile d\psi \\
        &= \phi\smile\psi+d(\phi''\smile\psi)
    \end{align*}
    el otro caso es análogo.
\end{proof}

\noindent Definimos el mapa $1:C_{0}(X)\to R$ tal que $1(\sigma)=1$ para toda $0-$cadena, entonces
\begin{equation*}
    [1]\smile[\phi]=[\phi]
\end{equation*}
De este modo, definimos $H^{*}(X;R)=\bigoplus_{n\in\N}H^{n}(X;R)$ que junto con el producto cup y
el elemento $[1]$ forman un anillo con unidad. Aquí el producto cup se extiende linealmente. 
Este anillo se conoce como el anillo de cohomología de $X$ con coeficientes en $R$.

\vspace{1mm}
\noindent El producto cup es natural, lo que se expresa en la proposición que sigue.

\begin{lema}
    Sean $f:X\to Y$ una función continua entre espacios y $\alpha,\beta\in H^{*}(Y;R)$ entonces
    $f^{*}(\alpha\smile\beta)=f^{*}(\alpha)\smile f^{*}(\beta)$.
\end{lema}

\begin{proof}
    Sean $\phi\in C^{k}(Y;R)$, $\psi\in C^{l}(Y;R)$ y $\sigma\in C_{k+l}(Y)$, entonces
    \begin{align*}
        (f^{*}(\phi)\smile f^{*}(\psi))(\sigma) &= f^{*}(\phi)(\sigma\big|_{[v_{0},\cdots,v_{k}]})
        f^{*}(\psi)(\sigma\big|_{[v_{k},\cdots,v_{k+l}]}) \\
        &= \phi(f\circ\sigma\big|_{[v_{0},\cdots,v_{k}]})
        \psi(f\circ\sigma\big|_{[v_{k},\cdots,v_{k+l}]}) \\
        &= (\phi\smile\psi)(f\circ\sigma)=f^{*}(\phi\smile\psi)(\sigma)
    \end{align*}
    bajando al cociente y por linealidad, concluimos.
\end{proof}

\vspace{1mm}
\noindent El anillo de cohomología no es necesariamente conmutativo, pero el siguiente lema nos 
entrega una expresión para intercambiar los términos.

\begin{lema}
    Sean $\alpha\in H^{k}(X;R)$ y $\beta\in H^{l}(X;R)$ entonces
    \begin{equation*}
        \alpha\smile\beta=(-1)^{kl}\beta\smile\alpha
    \end{equation*}
\end{lema}

\noindent La demostración de este lema se encuentra en \cite{Hatcher} (\textit{Hatcher, Teorema 
3.11}). La discusión anterior le da estructura de anillo graduado a $H^{*}(X;R)$, más aún, el 
anillo de cohomología es un $R-$módulo graduado donde la multiplicación es bilineal, en resumen, 
diremos que es un álgebra graduada. Definimos formalmente estos conceptos a continuación.

\begin{dfn}
    Sea $A$ un anillo, decimos que es un anillo graduado si $A=\bigoplus_{n\in\N}A_{n}$ con 
    $A_{n}$ subgrupos aditivos tales que
    \begin{equation*}
        A_{n}A_{m}\subseteq A_{n+m}
        \htext{para todo }n,m\in\N
    \end{equation*}
\end{dfn}

\begin{dfn}
    Sea $A$ un $R-$módulo, decimos que es un $R-$módulo graduado si $A=\bigoplus_{n\in\N}A_{n}$ 
    con $A_{n}$ submódulos para todo $n\in\N$.
\end{dfn}

\begin{obs}
    Es directo de la definición que $R\cdot A_{n}\subseteq A_{n}$.
\end{obs}

\noindent Un morfismo de anillos graduados (resp. $R-$módulos graduados) $f:A\to B$ es un morfismo
de anillos ($R-$módulos) tal que $f(A_{n})\subseteq B_{n}$. Un morfismo de álgebras graduadas es
un morfismo de anillos y $R-$módulos graduados. En ambos casos, dado $\alpha\in A_{n}$, decimos
que su grado es $\abs{\alpha}=n$.

\vspace{1mm}
\noindent Es directo de las definiciones y resultados que un mapa continuo induce un morfismo de 
algebras graduadas en el anillo de cohomología. En conclusión, el anillo de cohomología es un 
funtor contravariante en la categoría de algebras graduadas.

\begin{ej}
    Sea $n\in\N$, entonces 
    \begin{equation*}
        H^{*}(\S^{n};\Z)=\Z[x]/(x^{2})
        \htext{con}
        \abs{x}=n
    \end{equation*}
    En efecto, como $\S^{n}$ es arcoconexo, tenemos que $H^{0}(\S^{n};\Z)=\Z$, es decir, $[1]$ es 
    generador. Por otro lado, sea $x\in H^{n}(\S^{n};\Z)$ un generador del subgrupo, que es el 
    único subgrupo de dimensión positiva no trivial. Entonces
    \begin{equation*}
        x\smile x\in H^{2n}(\S^{n};\Z)=0
    \end{equation*}
    lo que prueba la afirmación.
\end{ej}

\noindent Ahora, dados dos espacios $X$ e $Y$, nos gustaría calcular el anillo de cohomología de
$X\times Y$ wn función de los anillos de cohomología de $X$ e $Y$, para esto vamos a definir un
morfismo desde el producto tensorial de los anillos de cohomología de $X$ e $Y$ hacia el anillo de
cohomología buscado, que será un isomorfismo bajo ciertas condiciones. Primero, debemos dotar de
estructura de anillo graduado al $R-$módulo $A\otimes B$, donde $A,B$ son algebras graduadas.

\begin{dfn}
    Sean $\pi_{X}:X\times Y\to X$ y $\pi_{Y}:X\times Y\to Y$ los mapas de proyección. Se define el
    producto cruz como
    \begin{align*}
        \times:H^{k}(X;R)\times H^{l}(Y;R) &\to H^{k+l}(X\times Y;R) \\
        a\times b:=\times(a,b) &\to \pi_{X}^{*}(a)\smile\pi_{Y}^{*}(b)
    \end{align*}
\end{dfn}

\noindent Notemos que el producto cruz es bilineal, por que el producto cup es bilineal y los 
mapas $\pi_{X}^{*}$ y $\pi_{Y}^{*}$ son morfismos de algebras. Luego, por la propiedad
fundamental del producto tensorial, induce un morfismo
\begin{align*}
    \times:H^{k}(X;R)\otimes_{R} H^{l}(Y;R) &\to H^{k+l}(X\times Y;R) \\
    a\otimes_{R} b &\mapsto \pi_{X}^{*}(a)\smile\pi_{Y}^{*}(b)
\end{align*}

\vspace{1mm}
\noindent Nos gustaría que $\times$ fuese un morfismo de álgebras graduadas, para ello en primer
lugar debemos definir un produco en $A\otimes B$ con $A,B$ álgebras graduadas. Notemos que
\begin{align*}
    \times(a\otimes b)\cdot\times(c\otimes d)
    &= \pi_{X}^{*}(a)\pi_{Y}^{*}(b)\pi_{X}^{*}(c)\pi_{Y}^{*}(d) \\
    &= (-1)^{\abs{b}\abs{c}}\pi_{X}^{*}(a)\pi_{X}^{*}(c)\pi_{Y}^{*}(b)\pi_{Y}^{*}(d) \\
    &= (-1)^{\abs{b}\abs{c}}\pi_{X}^{*}(ac)\pi_{Y}^{*}(bd)
    =(-1)^{\abs{b}\abs{c}}\times(ac\otimes bd)
\end{align*}

\noindent Así, el producto en $A\otimes B$ se define por $(a\otimes b)(c\otimes d)
=(-1)^{\abs{b}\abs{c}}ac\otimes bd$ y se extiende linealmente. Con esta multiplicación, el 
producto cruz es morfismo de algebras graduadas, es más, el siguiente teorema nos dice bajo que 
hipotesis es isomorfismo.

\begin{teo}[Fórmula de Künneth]
    Sean $X$ e $Y$ espacios topológicos. Sea $R$ un dominio de ideales principales y $H^{n}(Y;R)$
    un $R-$módulo libre finitamente generado para todo $n\in\N$, entonces
    \begin{equation*}
        \times:\bigoplus_{i=0}^{n}H^{i}(X;R)\otimes_{R} H^{n-i}(Y;R)\to H^{n}(X\times Y;R)
    \end{equation*}
    es isomorfismo.
\end{teo}

\noindent La demostración de este resultado se encuentra en \cite{Hatcher} (\textit{Hatcher, 
Teorema 3.15}).

\subsection{Algebras de Hopf} \hspace{1mm} %% Esta parte esta lista

\vspace{1mm}
\noindent En esta parte vamos a probar el resultado que permite concluir el teorema principal de 
la tesis. Para ello, buscamos una estructura algebraica que restrinja la posibilidad de ser 
$H-$espacio, necesitaremos una definición previa. Sea $R$ un anillo conmutativo.

\begin{dfn}
    Sea $A$ un álgebra graduada, se dice álgebra de Hopf si cumple,
    \begin{enumerate}
        \itemsep0.5em
        
        \item Existe una identidad $1\in A_{0}$ tal que el morfismo $R\to A_{0}$ dado por 
        $r\to r\cdot1$ es isomorfismo.

        \item Existe un morfismo de álgebras graduadas $\triangle:A\to A\otimes_{R}A$ llamado 
        coproducto que satisface
        \begin{equation*}
            \triangle(\alpha)
            =\alpha\otimes_{R}1+1\otimes_{R}\alpha+\sum_{i}\alpha'_{i}\otimes_{R}\alpha''_{n-i}
        \end{equation*}
        donde $\abs{\alpha'_{i}},\abs{\alpha''_{n-i}}>0$ para todo $\alpha\in A_{n}$ con $n>0$.
    \end{enumerate}
\end{dfn}

\noindent Veremos ejemplos de álgebras de Hopf y luego pasaremos al resultado principal. Sea $A$
un álgebra graduada y $\alpha\in A$.

\begin{ej}
    Consideramos $R[\alpha]$, la primera condición para ser álgebra de Hopf se cumple 
    trivialmente. Supongamos que también se cumple la segunda, es decir, existe un coproducto, 
    luego
    \begin{equation*}
        \triangle(\alpha)
        =\alpha\otimes1+1\otimes\alpha+\sum_{i}\alpha'_{i}\otimes\alpha''_{n-i}
    \end{equation*}
    notemos que $\abs{\alpha'_{i}},\abs{\alpha''_{n-i}}<\abs{\alpha}$. Como en $R[\alpha]$ los 
    únicos elementos de menor grado que $\abs{\alpha}$ son los elementos en $R$ de grado cero, se 
    sigue que $\triangle(\alpha)=\alpha\otimes1+1\otimes\alpha$.
    
    \vspace{1mm}
    \noindent Nos gustaría determinar $\triangle$ explícitamente, tenemos dos casos:
    \begin{enumerate}
        \itemsep0.5em
        
        \item El grado de $\alpha$ es par, entonces
        \begin{align*}
            \triangle(\alpha^{2}) &= (\triangle(\alpha))^{2}
            =(\alpha\otimes1+1\otimes\alpha)^{2} \\
            &= (\alpha\otimes1)^{2}+(\alpha\otimes1)(1\otimes\alpha)
            +(1\otimes\alpha)(\alpha\otimes1)+(1\otimes\alpha)^{2} \\
            &= \alpha^{2}\otimes1+2\alpha\otimes\alpha+1\otimes\alpha^{2}
        \end{align*}
        inductivamente obtenemos que $\triangle(\alpha^{n})
        =\sum_{i}\binom{n}{i}\alpha^{i}\otimes\alpha^{n-i}$.

        \item Si $\abs{\alpha}$ es impar, del caso anterior tenemos que
        \begin{equation*}
            \triangle(\alpha^{2})=\alpha^{2}\otimes1+1\otimes\alpha^{2}
        \end{equation*}
        Buscamos reducir este caso al anterior, digamos que $\beta=\alpha^{2}$, entonces
        \begin{equation*}
            \triangle(\alpha^{2n})=\triangle(\beta^{n})
            =\sum_{i}\binom{n}{i}\beta^{i}\otimes\beta^{n-i}
            =\sum_{i}\binom{n}{i}\alpha^{2i}\otimes\alpha^{2(n-i)}
        \end{equation*}
        y por otro lado,
        \begin{equation*}
            \triangle(\alpha^{2n+1})=\triangle(\beta^{n})\triangle(\alpha)
            =\sum_{i}\binom{n}{i}\alpha\beta^{i}\otimes\beta^{n-i}
            +\sum_{i}\binom{n}{i}\beta^{i}\otimes\alpha\beta^{n-1}
        \end{equation*}
    \end{enumerate}
\end{ej}

\begin{ej}
    Tomamos el álgebra exterior $\Lambda_{R}[\alpha]=R[\alpha]/(\alpha^{2})$, al igual que antes
    la primera condición para ser álgebra de Hopf se cumple facilmente. Si $\Lambda_{R}[\alpha]$
    tiene un coproducto, entonces por la misma razón que antes
    \begin{equation*}
        \triangle(\alpha)=\alpha\otimes1+1\otimes\alpha
    \end{equation*}
    sin embargo, hay algo adicional que chequear, esto es $0=\triangle(\alpha^{2})
    =(\triangle(\alpha))^{2}$. Si $\abs{\alpha}$ es impar entonces se verifica, en cambio, si es
    par vemos que
    \begin{equation*}
        \triangle(\alpha^{2})=\alpha^{2}\otimes1+2\alpha\otimes\alpha+1\otimes\alpha^{2}
        =2\alpha\otimes\alpha
    \end{equation*}
    lo que implica que $2=0$ en $R$.
\end{ej}

\noindent Dado $X$ un espacio topológico, el anillo $H^{*}(X;R)$ tiene estructura de álgebra 
graduada sobre $R$. Todo lo anterior da pie al siguiente teorema.

\begin{teo}
    Sea $X$ un $H-$espacio arcoconexo tal que $H^{n}(X;R)$ es un $R-$módulo libre finitamente 
    generado para todo $n\in\N$. Entonces $H^{*}(X;R)$ es un álgebra de Hopf.
\end{teo}

\begin{proof}
    Como $X$ es arcoconexo, se sigue que $H^{0}(X;R)\cong R$, además, se verifican las condiciones 
    de Künneth lo que implica que
    
    \vspace{1mm}
    \centerline{
        \xymatrix{
            H^{*}(X;R)\otimes H^{*}(X;R) \ar[r]^-{\times} & H^{*}(X\times X;R)
        }
    }
    \vspace{1mm}
    \noindent es isomorfismo. La multiplicación $\mu$ induce un morfismo 
    $\mu^{*}:H^{*}(X;R)\to H^{*}(X\times X;R)$ y junto con el producto cruz, definimos el 
    coproducto

    \vspace{1mm}
    \centerline{
        \xymatrix{
            H^{*}(X;R) \ar[r]^-{\mu^{*}} \ar@/_2pc/[rr]^-{\triangle}
            & H^{*}(X\times X;R) \ar[r]^-{\times^{-1}} & H^{*}(X;R)\otimes H^{*}(X;R)
        }
    }
    \vspace{2mm}
    \noindent El mapa $\triangle$ es morfismo de algebras graduadas por que es composición de 
    morfismos de algebras graduadas. Por otro lado, debemos probar que satisface la condición para 
    ser coproducto. Sea $i:X\to X\times X$ el mapa dado por $i(x)=(x,e)$ y $j:\{e\}\to X$ el mapa 
    $j(e)=e$, notar que $i$ se puede restringir en la segunda coordenada a $\{e\}$. Afirmamos que 
    el siguiente diagrama conmuta
    
    \vspace{1mm}
    \centerline{
        \xymatrix{
            H^{*}(X;R) \ar[r]^{\mu^{*}} \ar[rd]^-{\triangle} & H^{*}(X\times X; R) \ar[r]^{i^{*}} 
            & H^{*}(X;R) \\
            & H^{*}(X;R)\otimes H^{*}(X;R) \ar[u]^{\times}_{\cong} \ar[ru]^{P} 
            \ar[r]^{id\otimes j^{*}} 
            & H^{*}(X;R)\otimes H^{*}(e;R) \ar[u]^-{i^{*}\circ\times}_{\cong}
        }
    }
    \vspace{2mm}
    \noindent y $P$ es tal que el diagrama conmuta. En efecto, por definición se tiene que 
    $\mu^{*}=\times\circ\triangle$ y como $P=i^{*}\times$ resulta que $i^{*}\mu^{*}=P\triangle$.
    Falta probar que $i^{*}\times=i^{*}\times\circ (id\otimes j^{*})$.

    \vspace{1mm}
    \noindent Sea $a\otimes b\in H^{*}(X;R)\otimes H^{*}(X;R)$, entonces
    \begin{align*}
        i^{*}\times(a\otimes b) &= i^{*}(\pi_{X}^{*}(a)\smile\pi_{X}^{*}(b))
        =i^{*}\pi_{X}^{*}(a)\smile i^{*}\pi_{X}^{*}(b) \\
        &= (\pi_{X}i)^{*}(a)\smile(\pi_{X}i)^{*}(b)=a\smile ct_{e}^{*}(b)
    \end{align*}
    y por otra parte, observamos que
    \begin{equation*}
        i^{*}\times\circ (id\otimes j^{*})(a\otimes b)=i^{*}\times(a\otimes j^{*}b)
        =(\pi_{X}i)^{*}(a)\smile(j\pi_{e}i)^{*}(b)=a\smile ct_{e}^{*}(b)
    \end{equation*}
    Sea $\alpha\in H^{n}(X;R)$, luego usando que $r\to r\cdot 1$ es isomorfismo
    \begin{equation*}
        \triangle(\alpha)=\sum_{i=0}^{n}\alpha'_{i}\otimes\alpha''_{n-i}
        =\alpha'_{n}\otimes\alpha''_{0}+\sum_{i<n}\alpha'_{i}\otimes\alpha''_{n-i}
        =r\alpha'_{n}\otimes1+\sum_{i<n}\alpha'_{i}\otimes\alpha''_{n-i}
    \end{equation*}
    Como $H^{n}(e;R)=0$ para todo $n>0$ y $\mu\circ i\sim id_{X}$, vemos que
    \begin{equation*}
        \alpha=i^{*}\mu^{*}(\alpha)=P\triangle(\alpha)=r\alpha'_{n}\smile1=r\alpha'_{n}
    \end{equation*}
    es decir, $\triangle(\alpha)=\alpha\otimes1+\sum_{i<n}\alpha'_{i}\otimes\alpha''_{n-i}$. 
    Repitiendo un argumento similar tenemos el mismo resultado para $1\otimes\alpha$.
\end{proof}

\noindent Con esto podemos demostrar finalmente el Teorema \ref{teo:esferas}, el cual enunciamos 
nuevamente a continuación.

\begin{cor}
    Sea $n\in\N$. Si $\S^{n}$ es $H-$espacio entonces $n$ es impar.
\end{cor}

\begin{proof}
    Sabemos que $H^{*}(\S^{n},\Z)=\Z[x]/(x^{2})$ con $\abs{x}=n$. Por el teorema anterior sabemos
    que $H^{*}(\S^{n},\Z)$ es álgebra de Hopf, si $n$ es par entonces por el segundo ejemplo $2=0$
    en $\Z$, lo que es absurdo. Concluimos que $n$ es impar.
\end{proof}

\newpage

\section{Conclusiones} \hspace{1mm} %% Esta parte esta lista

\vspace{1mm}
\noindent Retomando la discusión inicial y sintetizando todo lo visto, en la sección 3, revisamos
a profundidad los ejemplos de grupos topológicos, vimos que $\S^{1}$, $\T^{2}$, $\S^{3}$ y 
$\R\P^{3}$ son grupos topológicos y su construcción resulto relativamente sencilla porque vivían
en espacios con buenas propiedades, sin embargo, dar una estructura de grupo topológico a un 
espacio dado no es una tarea sencilla, en esa misma sección hablamos un poco de superficies y las 
consecuencias de las obstrucciones que se presentan para ser $H-$espacio, una noción mas general 
de grupo topológico. Luego, en la sección 4, se presentan las primeras dos obstrucciones: la 
primera de ellas es que el grupo fundamental debe ser abeliano, y la segunda es que el cubriente 
universal de un $H-$espacio es $H-$espacio. Con la restricción del grupo fundamental logramos 
descartar las superficies que son homeomorfas a sumas conexas de toros y planos proyectivos, 
también probamos que $\S^{2}$ es el cubriente universal de $\R\P^{2}$ y juntando el segundo 
resultado de la sección 4 con el resultado de la sección 5, que bajo ciertas hipotesis el anillo 
de cohomología de un $H-$espacio es un álgebra de Hopf, llegamos a la conclusión de que $\R\P^{2}$ 
no es un $H-$espacio por que $\S^{2}$ no lo es, más aún, demostramos que ninguna esfera de 
dimensión par lo es, este resultado se generalizo en la literatura a que las únicas esferas que 
son $H-$espacios corresponden a $\S^{0}$, $\S^{1}$, $\S^{3}$ y $\S^{7}$, en un futuro sería ideal 
apuntar en esta dirección. Como resultado extra, usando el teorema de clasificación de 
superficies, logramos concluir que $\T^{2}$ es la única superficie compacta y conexa que es grupo 
topológico.

\begin{thebibliography}{4}
    \bibitem{Hatcher}
        \textsc{Hatcher, A.} (2001). 
        \textit{Algebraic Topology}, 1st ed. Cambridge University Press.

    \bibitem{Bredon}
        \textsc{Bredon, Glen E.} (1993). 
        \textit{Topology and Geometry}, 1st ed. Springer.

    \bibitem{Lee}
        \textsc{Lee, John M.} (2011). 
        \textit{Introduction to Topological Manifolds}, 2nd ed. Springer.

    \bibitem{Munkres}
        \textsc{Munkres, James R.} (2000)
        \textit{Topology}, 2nd ed. Pearson
\end{thebibliography}

\end{document}