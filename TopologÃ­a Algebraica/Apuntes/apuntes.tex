\documentclass{article}
\usepackage{hyperref}
\usepackage{Style}

\nocite{*} % Comentar si quiero citar
%\addbibresource{bibliografia.bib} % Quitar el comentado si quiero usar bibliografia

\begin{document}

\begin{minipage}{2.5cm}
    \includegraphics[width=2cm]{imagen_puc.jpg}
\end{minipage}
\begin{minipage}{14cm}
    {\sc Pontificia Universidad Católica de Chile\\
    Facultad de Matemáticas\\
    Departamento de Matemática\\
    Profesor: Mauricio Bustamante -- Estudiante: Benjamín Mateluna}
\end{minipage}
\vspace{1ex}

{\centerline{\bf Topología Algebraica - MAT2850}
\centerline{\bf Apuntes}}
\centerline{\bf 05 de agosto de 2025}

\newpage
\tableofcontents

\newpage
\section*{Motivación}
\phantomsection
\addcontentsline{toc}{section}{Motivación}
\noindent Dados dos espacios topológicos $X$ e $Y$ ¿Cuando son homeomorfos?. Decimos que dos 
espacios son \textbf{homeomorfos} si existe $f:X\to Y$ continua, biyectiva y con inversa 
constinua. La topología algebraica ataca esta pregunta de la siguiente forma:
\begin{enumerate}
    \item Asigna a cada espacio topológico $X$ un objeto algebraico $G(X)$.
    \item Aigna a cada función continua $f:X\to Y$ un homomorfismo $G(f):G(X)\to G(Y)$ tal que
    \begin{enumerate}
        \item $G(f\circ g)=G(f)\circ G(g)$
        \item $G(id_{X})=id_{G(X)}$
    \end{enumerate}
\end{enumerate}
\noindent\textbf{Observación:} Ambas condiciones implican que si $f:X\to Y$ es homeomorfismo, 
entonces $G(f):G(X)\to G(Y)$ es isomorfismo. A veces los $G$ que se construyen satisfacen la 
propiedad extra que si $X$ se puede ''deformar continuamente'' en $Y$ entonces $G(X)\cong G(Y)$.

\vspace{2mm}
\noindent Decimos que $G$ es un \textbf{invariante homotópico}.(Faltan ejemplos - c1)

\vspace{2mm}
\begin{dfn}
    Una \textbf{homotopía} entre dos funciones continuas $f,g:X\to Y$ es una función continua 
    $H:X\times[0,1]\to Y$ tal que $H(x,0)=f(x)$ y $H(x,1)=g(x)$ para todo $x\in X$.
\end{dfn}
\noindent\textbf{Notación:} La función $H_{t}:X\to Y$ esta dada por $H_{t}(x):=H(x,t)$. Una 
homotopía de $f$ a $g$ se denota por $f\sim g$.

\vspace{2mm}
\begin{prop}
    Ser homotópico es una relación de equivalencia en $\mathcal{C}(X,Y)$.
\end{prop}

\vspace{2mm}
\begin{dfn}
    Decimos que $f:X\to Y$ es una \textbf{equivalencia homotópica}, si existe $g:Y\to X$ tal que 
    $g\circ f\sim id_{X}$ y $f\circ g\sim id_{Y}$
\end{dfn}
\noindent En tal caso, $X$ e $Y$ se dicen homotópicamente equivalentes o que tienen el mismo tipo 
de homotopía y se denota por $X\sim Y$.

\vspace{2mm}
\noindent\textbf{Ejemplo:}
\begin{itemize}
    \item Sea $f:X\to Y$ un homeomorfismo, en particular, tomando $g=f^{-1}$, se sigue que es 
    equivalencia homotópica.

    \item Se tiene que $\{0\}\sim\R^{n}$, consideremos la inclusión $i:\to\{0\}\to\R^{n}$, 
    afirmamos que es $i$ es equivalencia homotópica. En efecto, se verifica que $\pi:\R^{n}\to
    \{0\}$ es una inversa homotópica. Por un lado $\pi\circ i=id_{\{0\}}$ y por otro 
    $i\circ\pi=0$. Notamos que $H(x,t)=tx$ con $t\in[0,1]$ es una homotopía entre $0$ y 
    $id_{\R^{n}}$.

    \item Veamos que $\R^{n}\setminus\{0\}\sim\s^{n-1}$. Probaremos que la función 
    $i:\s^{n-1}\to\R^{n}\setminus\{0\}$ es equivalencia homotópica. En efecto,
    \begin{align*}
        \pi:\R^{n}\setminus\{0\} &\to \s^{n-1} \\
        x &\to \frac{x}{\abs{x}}
    \end{align*}
    es inversa homotópica. Es claro que $\pi\circ i=id_{s^{n-1}}$. Definimos
    \begin{equation*}
        H(x,t):=t\frac{x}{\abs{x}}+(1-t)x
    \end{equation*}
    Notamos que $H(x,0)=x$ y $H(x,1)=\frac{x}{\abs{x}}$, es decir, $H$ es una homotopia entre 
    $i\circ\pi$ e $id_{\R^{n}\setminus\{0\}}$. Además, se verifica que 
    $im(H)\subseteq\R^{n}\setminus\{0\}$.
\end{itemize}

\newpage
\section{Homología Simplicial}
\noindent Queremos asignarle a un espacio topológico $X$ arbitrario, grupos abelianos 
$H_{0}(X),H_{1}(X),\cdots$ tal que si $X\sim Y$, entonces $H_{i}(X)\cong H_{i}(Y)$ para todo $i$.
Ituitivamente, $H_{k}(X)$ estará generado por ciertos subespacios de $X$ de dimensión $k$.

\vspace{2mm}
\noindent Habrá una relación de equvalencia, $A,B\subseteq X$ de dimensión $k$ serán equivalentes
si hay un subespacio de $X$ de dimensión $k+1$ cuyo borde es $A\cup B$. (Falta ejemplo - c1)

\vspace{2mm}
\noindent Hay que restringir la clase de espacios a una con nociones de dimensión, borde, etc. 
Estos serán los complejos simpliciales. Necesitamos, adicionalmente, un objeto algebraico que 
capture esas nociones, esto corresponde a los complejos de cadenas.

\subsection{Complejos de Cadenas}
\begin{dfn}
    Un \textbf{complejo de cadenas} es una sucesión de grupos abelianos y homomorfismos
    \begin{equation*}
        \cdots\xrightarrow[]{}C_{3}\xrightarrow[]{d_{3}}C_{2}\xrightarrow[]{d_{2}}
        C_{1}\xrightarrow[]{d_{1}}C_{0}\xrightarrow[]{d_{0}}0
    \end{equation*}
    tal que $d_{i}\circ d_{i+1}=0$ para todo $i$. Se denota por $(C_{*},d_{*})$.
\end{dfn}
\noindent\textbf{Observación:} Notemos que $\im{d_{i+1}}\subseteq\kr{d_{i}}\subseteq C_{i}$. Dado 
que los grupos son abelianos, esta observación permite definir el siguiente objeto.

\vspace{2mm}
\begin{dfn}
    El \textbf{i-ésimo grupo de homología} de $(C_{*},d_{*})$ se define por
    \begin{equation*}
        H_{i}(C_{i}):=\frac{\kr{d_{i}}}{\im{d_{i+1}}}
    \end{equation*}
\end{dfn}

\vspace{2mm}
\noindent\textbf{Ejemplos:}
\begin{itemize}
    \item Si $A$ un grupo abeliano, entonces
    \begin{equation*}
        \cdots\xrightarrow[]{}0\xrightarrow[]{}0\xrightarrow[]{}A\xrightarrow[]{}
        0\xrightarrow[]{}\cdots\xrightarrow[]{}0\xrightarrow[]{}0
    \end{equation*}
    es un complejo de cadenas donde $C_{i}=A$. Entonces
    \begin{equation*}
        H_{j}(C_{*})=\begin{cases}
            0 & \quad\text{si }j\neq i \\
            A & \quad\text{si }j=i
        \end{cases}
    \end{equation*}

    \item Consideremos la cadena exacta
    \begin{equation*}
        \cdots\xrightarrow[]{}0\xrightarrow[]{}\Z\xrightarrow[]{\cdot2}
        \Z\xrightarrow[]{\pi}\Z_{2}\xrightarrow[]{}0
    \end{equation*}
    entonces $H_{j}(C_{*})=0$ para todo $i$.

    \item Veamos que
    \begin{equation*}
        \cdots\xrightarrow[]{\cdot0}\Z\xrightarrow[]{\cdot2}\Z\xrightarrow[]{\cdot0}
        \Z\xrightarrow[]{}0
    \end{equation*}
    es un complejo de cadenas. La homología asociadas son $H_{0}(C_{*})=\Z$, $H_{1}(C_{*})=\Z_{2}$
    y $H_{k}(C_{*})=0$.
\end{itemize}

\noindent Nuestro objetivo será asociar un complejo de cadenas a un espacio topológico $X$ 
arbitrario.

\subsection{Complejos Simpliciales}
\begin{dfn}
    Dados $n+1$ puntos $\{v_{0},\cdots,v_{n}\}\in\R^{\omega}$ son \textbf{afínmente 
    independientes}, si generan un $n-$plano afín, es decir, $\{v_{1}-v_{0},\cdots,v_{n}-v_{0}\}$
    es un conjunto linealmente independiente, esto es
    \begin{equation*}
        \sum_{i=0}^{n}t_{i}v_{i}=0\hhtext{y}\sum_{i=0}^{n}t_{i}=0
        \hhtext{entonces}t_{i}=0\text{ para todo }i
    \end{equation*}
\end{dfn}

\vspace{2mm}
\noindent\textbf{Ejemplo:} Dos puntos son afínmente independientes. Tres puntos son afínmente 
independientes si y solo si no son colineales. (Falta ejemplo dibujo - c1)

\vspace{2mm}
\begin{dfn}
    Si $\{v_{0},\cdots,v_{n}\}$ son afínmente independientes, ellos definen el \textbf{n-simplejo}
    \begin{equation*}
        \sigma=\gen{v_{0},\cdots,v_{n}}=\left\{x=\sum_{i=0}^{n}t_{i}v_{i},\hspace{2mm}
        \sum_{i=0}^{n}t_{i}=1\hhtext{y}t_{i}\geq0\right\}
    \end{equation*}
\end{dfn}

\noindent Decimos que $\sigma$ es el $n-$simplejo generado por $v_{0},\cdots,v_{n}$. Los puntos 
$v_{i}$ se llaman \textbf{vértices} de $\sigma$. Una \textbf{cara} de un simplejo $\sigma$ es 
un simplejo $\tau$ generado por un subconjunto de $\{v_{0},\cdots,v_{n}\}$ y lo denotamos por 
$\tau\leq\sigma$. Si el subconjunto es propio, se dice que $\tau$ es una \textbf{cara propia}.

\vspace{2mm}
\noindent La \textbf{frontera} de un $n-$simplejo $\sigma$ es la unión de todas sus caras propias, 
se denota por $\partial\sigma$, el \textbf{interior} de $\sigma$ es $int(\sigma):=
\sigma\setminus\partial\sigma$.

\vspace{2mm}
\begin{dfn}
    Un \textbf{complejo simplicial} (geométrico) $K$ es un conjunto de simplejos tales que
    \begin{enumerate}
        \item Si $\sigma\in K$ y $\tau\leq\sigma$ entonces $\tau\in K$.
        \item Si $\sigma,\tau\in K$ entonces $\sigma\cap\tau=\emptyset$ ó $\sigma\cap\tau$ es una
        cara de $\sigma$ y de $\tau$.
    \end{enumerate}
\end{dfn}
\noindent El \textbf{poliedro} asociado a un complejo simplicial $K$ es 
$\abs{K}:=\bigcup_{\sigma\in K}\sigma$. Un espacio topológico $X$ se llama un poliedro si existe
un complejo simplicial $K$ y un homeomorfismo $f:\abs{K}\to X$. Al par $(K,f)$ se le llama una 
\textbf{triangulación} de $X$.

\vspace{2mm}
\noindent\textbf{Observación:} Si $X$ es triangulable, entonces es Hausdorff por que $\abs{K}$ 
lo es. (Faltan ejemplos - c2)

%\printbibliography % Quitar el comentado si quiero usar bibliografia

\end{document}
