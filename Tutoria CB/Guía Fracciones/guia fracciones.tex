\documentclass{article}
\usepackage{hyperref}
\usepackage{Style}

\nocite{*} % Comentar si quiero citar
%\addbibresource{bibliografia.bib} % Quitar el comentado si quiero usar bibliografia

\title{Guía Fracciones}
\author{}
\date{}

\begin{document}
\maketitle

\section*{Introducción}
\noindent Las operaciones combinadas con fracciones se trabajan y respetan el mismo orden que las 
operaciones combinadas con números enteros, es decir, parentesis, multiplicación, división, suma y 
resta. Un ejemplo
\begin{equation*}
    \frac{1}{2}\cdot\left(\frac{3}{4}+\frac{1}{4}\right)-1=\frac{1}{2}\cdot1-1=\frac{1}{2}-1
    =-\frac{1}{2}
\end{equation*}
También nos gustaría resolver expresiones del tipo
\begin{equation*}
    \frac{1}{1+\frac{1}{1+\frac{1}{2}}}
\end{equation*}
Para resolverlas, debemos trabajar desde abajo hacia arriba. Primero calculamos
\begin{equation*}
    1+\frac{1}{2}=\frac{3}{2}
\end{equation*}
¿Como resolvemos una expresión que involucra una fracción de fracciones? Del siguiente modo
\begin{equation*}
    \frac{\frac{3}{4}}{\frac{6}{7}}=\frac{3\cdot7}{4\cdot6}=\frac{21}{24}=\frac{7}{8}
\end{equation*}
En palabras, el numerador de la primera fracción se multiplica con el denominador de la segunda y
el denominador de la primera con el numerador de la segunda. No olvidar simplicar la expresión. 
Regresemos a nuestro ejemplo original, ahora vamos a calcular
\begin{equation*}
    \frac{1}{1+\frac{1}{2}}=\frac{1}{\frac{3}{2}}=\frac{\frac{1}{1}}{\frac{3}{2}}=\frac{2}{3}
\end{equation*}
Notar que se tiene la igualdad $\frac{1}{1}=1$, luego
\begin{equation*}
    1+\frac{1}{1+\frac{1}{2}}=1+\frac{2}{3}=\frac{5}{3}
\end{equation*}
finalmente
\begin{equation*}
    \frac{1}{1+\frac{1}{1+\frac{1}{2}}}=\frac{1}{\frac{5}{3}}=\frac{\frac{1}{1}}{\frac{5}{3}}
    =\frac{3}{5}
\end{equation*}

\newpage
\section*{Ejercitación}
\noindent Calcular las siguientes expresiones
\begin{align*}
    & a)\hspace{2mm}\frac{1}{3}+\frac{1}{2}\left(1+\frac{5}{3}\right)\hspace{3cm}
    b)\hspace{2mm}\frac{1}{2+\frac{1}{2+\frac{2}{3}\cdot\frac{1}{2}}}\hspace{3cm}
    c)\hspace{2mm}\frac{4}{5}\div\frac{3}{4}+2-5\left(\frac{3}{2}+\frac{1}{2}\right) \\[1.5cm]
    & d)\hspace{2mm}\frac{1}{2+5}+\frac{1}{2}\left(3+5\cdot3\right)\hspace{2.3cm}
    e)\hspace{2mm}\frac{1+\frac{1}{2}}{1+\frac{1}{2+4\cdot2}}\hspace{3.1cm}
    f)\hspace{2mm}1+\frac{2}{3}\cdot\frac{1}{9+\frac{4}{3+5}} \\[1.5cm]
    & g)\hspace{2mm}\frac{7+4\cdot8-2}{2+2\cdot\frac{1\div\frac{3}{4}-1}{\frac{2}{3}+5}}
    \hspace{3.3cm}
    h)\hspace{2mm}\frac{7}{8}-\frac{1}{9}\left(2+\frac{6}{\frac{2}{3}+5}\right)\hspace{1.6cm}
    i)\hspace{2mm}\frac{1}{1+\frac{1}{1+\frac{1}{1+\frac{1}{3}}}}
\end{align*}

\vspace{2cm}
\noindent Resolver los siguientes problemas
\begin{enumerate}
    \item Jorge gana $4000\$$ al mes y destina las siguientes fracciones de su sueldo a los 
    siguientes apartados. Destina $\frac{1}{4}$ en alquiler, $\frac{1}{5}$ en comida, $\frac{1}{10}$
    en gasolina, $\frac{1}{6}$ en servicios del hogar y ropa y $\frac{1}{5}$ en gastos personales.
    Si Jorge ahorra lo restante. ¿Qué fracción ahorra Jorge? ¿A cuanto corresponde esta fracción
    en dinero?

    \item En una clase hay $22$ alumnos que nacieron en la ciudad y $8$ que nacieron fuera. ¿Qué
    fracción de la clase representan los que han nacido en la ciudad?¿Y los que nacieron fuera?

    \item Un campesino siembra todos los años su terreno de $1$ hectarea con las siguientes 
    proporciones: $\frac{3}{10}$ de trigo, $\frac{1}{12}$ de frijol, $\frac{4}{11}$ de cebada y el
    resto con maíz. ¿Qué fracción de terreno deberá ser sembrada con maíz?

    \item Luisa ha ido al mercado y le han cobrado $9,3\$$ por tres cuartos de kilo de jamón 
    serrano ¿Cuánto cuesta el kilo?

    \item Hace unos años Pedro tenía $24$ años, que representan $\frac{2}{3}$ de su edad actual.
    ¿Que edad tiene Pedro?

    \item Un trabajador de la construcción hace durante la primera hora de trabajo, $\frac{2}{7}$
    de un muro y durante la segunda hora de trabajo hace $\frac{1}{4}$. ¿Qué fracción del muro le
    queda para poder terminarlo en la tercera hora?
\end{enumerate}

%\printbibliography % Quitar el comentado si quiero usar bibliografia

\end{document}
