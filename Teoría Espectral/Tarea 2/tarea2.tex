\documentclass{article}
\usepackage{hyperref}
\usepackage{Style}

\nocite{*} % Comentar si quiero citar
%\addbibresource{bibliografia.bib} % Quitar el comentado si quiero usar bibliografia

\begin{document}

\begin{minipage}{2.5cm}
    \includegraphics[width=2cm]{imagen_puc.jpg}
\end{minipage}
\begin{minipage}{14cm}
    {\sc Pontificia Universidad Católica de Chile\\
    Facultad de Matemáticas\\
    Departamento de Matemática\\
    Profesora: Amal Taarabt -- Estudiante: Benjamín Mateluna}
\end{minipage}
\vspace{1ex}

{\centerline{\bf Teoría Espectral - MAT2820}
\centerline{\bf Tarea 2}}
\centerline{\bf 04 de noviembre de 2025}

\section*{Operadores compactos}

\begin{enumerate}
    \item La función $T_{K}$ es claramente lineal. Observemos que por lema del pegado la función 
    $K$ es continua, para ver que $T_{K}\in\B(H)$, nos damos $f\in L^{2}([0,1])$ y por la 
    desigualdad de hölder, se sigue que
    \begin{align*}
        \norm{T_{K}f}_{L^{2}}^{2} &= \int_{0}^{1}\abs{T_{K}f(x)}^{2}\hspace{1mm}dx
        =\int_{0}^{1}\abs{\int_{0}^{1}K(x,y)f(y)\hspace{1mm}dy}^{2}\hspace{1mm}dx \\
        &\leq \int_{0}^{1}\left(
            \int_{0}^{1}\abs{K(x,y)}^{2}\abs{f(y)}^{2}\hspace{1mm}dy\right)\hspace{1mm}dx
        \leq\left(\max_{(x,y)\in[0,1]^{2}}\abs{K(x,y)}\right)^{2}\norm{f}_{L^{2}}^{2}<\infty
    \end{align*}
    Por lo tanto, el operador $T_{K}$ es acotado. Para ver que es autoadjunto en primer lugar
    veamos que $K(x,y)=K(y,x)$ para todo $(x,y)\in[0,1]^{2}$ y además $K([0,1]^{2})\subseteq\R$,
    luego, por Fubini se tiene que
    \begin{align*}
        \ip{T^{*}f}{g}_{L^{2}} &= \ip{f}{Tg}_{L^{2}}
        =\int_{0}^{1}\overline{f(x)}T_{K}g(x)\hspace{1mm}dx
        =\int_{0}^{1}\overline{f(x)}\left(
            \int_{0}^{1}K(x,y)g(y)\hspace{1mm}dy\right)\hspace{1mm}dx \\
        &= \int_{0}^{1}\int_{0}^{1}\overline{f(x)}K(x,y)g(y)\hspace{1mm}dy\hspace{1mm}dx
        =\int_{0}^{1}\int_{0}^{1}\overline{K(y,x)f(x)}g(y)\hspace{1mm}dx\hspace{1mm}dy \\
        &= \int_{0}^{1}\overline{T_{K}f(y)}g(y)\hspace{1mm}dy=\ip{T_{K}f}{g}_{L^{2}}
    \end{align*}
    para toda $f,g\in L^{2}([0,1])$, lo que implica que $T_{K}$ es autoadjunto.
    
    \item Sea $f\in L^{2}([0,1])$ y $(x_{n})_{n\in\N}\subseteq[0,1]$ tal que 
    $x_{n}\xrightarrow[n\to\infty]{}x\in[0,1]$. Como $K$ es continua, para todo $y\in[0,1]$, se 
    tiene que $K(x_{n},y)f(y)\xrightarrow[n\to\infty]{} K(x,y)f(y)$, además
    \begin{equation*}
        \abs{K(x_{n},y)f(y)}\leq\norm{K}_{\infty}\abs{f(y)}=:g
    \end{equation*}
    y como $L^{2}([0,1])\subseteq L^{1}([0,1])$ la función $g$ es integrable. Así, por teorema de
    convergencia dominada resulta que
    \begin{equation*}
        \lim\limits_{n\to\infty}T_{K}f(x_{n})
        =\lim\limits_{n\to\infty}\int_{0}^{1}K(x_{n},y)f(y)\hspace{1mm}dy
        =\int_{0}^{1}\lim\limits_{n\to\infty}K(x_{n},y)f(y)\hspace{1mm}dy
        =\int_{0}^{1}K(x,y)f(y)\hspace{1mm}dy=T_{K}f(x)
    \end{equation*}
    Concluimos que $T_{K}(L^{2})\subseteq\C([0,1])$.
    
    \item \hspace{1mm}
    \begin{enumerate}
        \item Sea $(f_{n})_{n\in\N}$ una sucesión acotada, entonces existe $M>0$ tal que 
        $\norm{f_{n}}_{L^{2}}\leq M$ para todo $n\in\N$. Por lo visto en el primer ejercicio, 
        sabemos que
        \begin{equation*}
            \norm{T_{K}f_{n}}_{L^{2}}^{2}\leq\norm{K}_{\infty}^{2}\cdot\norm{f_{n}}_{L^{2}}^{2}
            \leq\norm{K}_{\infty}^{2}\cdot M^{2}
        \end{equation*}
        es decir, la sucesión $(T_{K}f_{n})_{n\in\N}$ es acotada. Veamos que es equicontinua. Dado
        que $[0,1]^{2}$ es compacto, en realidad se tiene que $K$ es uniformemente continua, sea
        $x_{0}\in[0,1]$ y $\varepsilon>0$, existe $\delta>0$ tal que si
        \begin{equation*}
            \abs{x_{0}-x}<\delta \hhtext{entonces}
            \abs{K(x_{0},y)-K(x,y)}<\frac{\varepsilon}{M} \htext{para todo}y\in[0,1]
        \end{equation*}
        Dado $n\in\N$, observemos que
        \begin{equation*}
            \abs{T_{K}f_{n}(x_{0})-T_{K}f_{n}(x)}
            \leq\int_{0}^{1}\abs{K(x_{0},y)-K(x,y)}\cdot\abs{f_{n}(y)}\hspace{1mm}dy
            <\frac{\varepsilon}{M}\cdot\norm{f_{n}}_{L^{1}}\leq\varepsilon
        \end{equation*}
        Por el teorema de Arzela Ascoli se tiene el resultado.
        
        \item Sabemos que $\B=\{\varphi_{n}=e^{2\pi inx}:n\in\Z\}$ es una base ortonormal de 
        $L^{2}([0,1])$, dado $n\in\Z$, vemos que
        \begin{equation*}
            T_{K}\varphi_{n}(x)=\int_{0}^{1}K(x,y)\varphi_{n}(y)\hspace{1mm}dy
            =\int_{0}^{x}y(1-x)e^{2\pi iny}\hspace{1mm}dy
            +\int_{x}^{1}x(1-y)e^{2\pi iny}\hspace{1mm}dy=:I_{1}+I_{2}
        \end{equation*}
        Por un lado tenemos que
        \begin{equation*}
            I_{1}=(1-x)\int_{0}^{x}ye^{2\pi iny}\hspace{1mm}dy
            =(1-x)\left(\frac{ye^{2\pi iny}}{2\pi in}\Big|_{0}^{x}
            -\frac{1}{2\pi in}\int_{0}^{x}e^{2\pi iny}\hspace{1mm}dy\right)
            =(1-x)\left(\frac{xe^{2\pi inx}}{2\pi in}-\frac{e^{2\pi inx}-1}{(2\pi in)^{2}}
            \right)
        \end{equation*}
        y por el otro lado
        \begin{equation*}
            I_{2}=x\int_{x}^{1}(1-y)e^{2\pi iny}\hspace{1mm}dy
            =x\left(\frac{(1-y)e^{2\pi iny}}{2\pi in}\Big|_{x}^{1}
            +\frac{1}{2\pi in}\int_{x}^{1}e^{2\pi iny}\hspace{1mm}dy\right)
            =x\left(-\frac{(1-x)e^{2\pi inx}}{2\pi in}
            +\frac{e^{2\pi in}-e^{2\pi inx}}{(2\pi in)^{2}}\right)
        \end{equation*}
        entonces
        \begin{equation*}
            T_{K}\varphi_{n}(x)=\frac{c_{n}(x)}{(2\pi in)^{2}}
            \htext{donde}\abs{c_{n}(x)}\leq4
            \htext{para todo}x\in[0,1]
        \end{equation*}
        de este modo obtenemos que
        \begin{equation*}
            \sum_{n\in\Z}\norm{T_{K}\varphi_{n}}_{L^{2}}
            \leq16\sum_{n\in\Z}\frac{1}{(2\pi in)^{4}}<\infty
        \end{equation*}
        Por lo tanto $T_{K}$ es un operador de Hilbert Schmidt, por ejercicio de la guía, 
        concluimos que es compacto.
    \end{enumerate}
    
    \item 
    
    \item De la parte anterior sabemos que si $\lambda\in\R^{*}$ es autovalor y $f$ su 
    correspondiente autovector, entonces satisface la EDO,
    \begin{equation*}
        f''+\frac{1}{\lambda}f=0
    \end{equation*}
    de ayudantía, sabemos que una solución de la EDO anterior se ve como 
    $f(x)=Acos(\frac{x}{\sqrt{\lambda}})+Bsen(\frac{x}{\sqrt{\lambda}})$, como $f(0)=0$ se tiene
    que $A=0$, por otro lado, si $\varphi(x)=sen(\frac{x}{\sqrt{\lambda}})$ entonces
    \begin{equation*}
        T_{K}\varphi(x)=\int_{0}^{1}K(x,y)sen\left(\frac{y}{\sqrt{\lambda}}\right)\hspace{1mm}dy
        =\int_{0}^{x}y(1-x)sen\left(\frac{y}{\sqrt{\lambda}}\right)\hspace{1mm}dy
        +\int_{x}^{1}x(1-y)sen\left(\frac{y}{\sqrt{\lambda}}\right)\hspace{1mm}dy=:I_{1}+I_{2}
    \end{equation*}
    donde,
    \begin{equation*}
        I_{1}=(1-x)\left(-\sqrt{\lambda}ycos\left(\frac{y}{\sqrt{\lambda}}\right)\Big|_{0}^{x}
        +\sqrt{\lambda}\int_{0}^{x}cos\left(\frac{y}{\sqrt{\lambda}}\right)\hspace{1mm}dy\right)
        =(1-x)\left(-\sqrt{\lambda}xcos\left(\frac{x}{\sqrt{\lambda}}\right)
        +\lambda sen\left(\frac{x}{\sqrt{\lambda}}\right)\right)
    \end{equation*}
    y
    \begin{equation*}
        I_{2}=x\left(-(1-y)\sqrt{\lambda}cos\left(\frac{y}{\sqrt{\lambda}}\right)\Big|_{x}^{1}
        +\sqrt{\lambda}\int_{x}^{1}cos\left(\frac{y}{\sqrt{\lambda}}\right)\hspace{1mm}dy\right)
        =x\left((1-x)\sqrt{\lambda}cos\left(\frac{x}{\sqrt{\lambda}}\right)
        +\lambda sen\left(\frac{x}{\sqrt{\lambda}}\right)\right)
    \end{equation*}
    
    \item 
    
    \item Usando lo anterior y el hecho de que $T_{K}$ es autoadjunto, resulta que
    \begin{equation*}
        \norm{T_{K}}_{\B(L^{2})}=r(T_{K})=\sup_{\lambda\in\sigma(T_{K})}\abs{\lambda}
        =\sup_{\lambda\in\sigma_{p}(T_{K})}\abs{\lambda}=\frac{1}{\pi^{2}}
    \end{equation*}
    
    \item 
    
    \item \textbf{Teorema de Mercer}
    \begin{enumerate}
        \item (Pendiente)
        
        \item Sabemos que
        \begin{equation*}
            \sigma_{p}(T_{K})=\left\{\lambda_{n}=\frac{1}{k^{2}\pi^{2}}:k\in\N\right\}
        \end{equation*}
        Por otro lado, tenemos que
        \begin{equation*}
            \int_{0}^{1}K(x,x)\hspace{1mm}dx=\int_{0}^{1}x(1-x)\hspace{1mm}dx
            =\left(\frac{x^{2}}{2}-\frac{x^{3}}{3}\right)\Big|_{0}^{1}=\frac{1}{6}
        \end{equation*}
        así, por el teorema de Mercer, concluimos que
        \begin{equation*}
            \sum_{n\in\N}\frac{1}{n^{2}}=\pi^{2}\sum_{n\in\N}\lambda_{n}
            =\pi^{2}\int_{0}^{1}K(x,x)\hspace{1mm}dx=\frac{\pi^{2}}{6}
        \end{equation*}
    \end{enumerate}
\end{enumerate}

%\printbibliography % Quitar el comentado si quiero usar bibliografia

\end{document}
