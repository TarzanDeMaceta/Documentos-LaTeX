\documentclass{article}
\usepackage{hyperref}
\usepackage{Style}

\nocite{*} % Comentar si quiero citar
%\addbibresource{bibliografia.bib} % Quitar el comentado si quiero usar bibliografia

\begin{document}

\begin{minipage}{2.5cm}
    \includegraphics[width=2cm]{imagen_puc.jpg}
\end{minipage}
\begin{minipage}{14cm}
    {\sc Pontificia Universidad Católica de Chile\\
    Facultad de Matemáticas\\
    Departamento de Matemática\\
    Profesora: Amal Taarabt -- Estudiante: Benjamín Mateluna}
\end{minipage}
\vspace{1ex}

{\centerline{\bf Teoría Espectral - MAT2820}
\centerline{\bf Tarea 2}}
\centerline{\bf 04 de noviembre de 2025}

\section*{Operadores compactos}

\begin{enumerate}
    \item La función $T_{K}$ es claramente lineal. Observemos que por lema del pegado la función 
    $K$ es continua, para ver que $T_{K}\in\B(H)$, nos damos $f\in L^{2}([0,1])$ y por la 
    desigualdad de Jensen, se sigue que
    \begin{align*}
        \norm{T_{K}f}_{L^{2}}^{2} &= \int_{0}^{1}\abs{T_{K}f(x)}^{2}\hspace{1mm}dx
        =\int_{0}^{1}\abs{\int_{0}^{1}K(x,y)f(y)\hspace{1mm}dy}^{2}\hspace{1mm}dx \\
        &\leq \int_{0}^{1}\left(
            \int_{0}^{1}\abs{K(x,y)}^{2}\abs{f(y)}^{2}\hspace{1mm}dy\right)\hspace{1mm}dx
        \leq\left(\max_{(x,y)\in[0,1]^{2}}\abs{K(x,y)}\right)^{2}\norm{f}_{L^{2}}^{2}<\infty
    \end{align*}
    Por lo tanto, el operador $T_{K}$ es acotado. Para ver que es autoadjunto en primer lugar
    veamos que $K(x,y)=K(y,x)$ para todo $(x,y)\in[0,1]^{2}$ y además $K([0,1]^{2})\subseteq\R$,
    luego, por Fubini se tiene que
    \begin{align*}
        \ip{T^{*}f}{g}_{L^{2}} &= \ip{f}{Tg}_{L^{2}}
        =\int_{0}^{1}\overline{f(x)}T_{K}g(x)\hspace{1mm}dx
        =\int_{0}^{1}\overline{f(x)}\left(
            \int_{0}^{1}K(x,y)g(y)\hspace{1mm}dy\right)\hspace{1mm}dx \\
        &= \int_{0}^{1}\int_{0}^{1}\overline{f(x)}K(x,y)g(y)\hspace{1mm}dy\hspace{1mm}dx
        =\int_{0}^{1}\int_{0}^{1}\overline{K(y,x)f(x)}g(y)\hspace{1mm}dx\hspace{1mm}dy \\
        &= \int_{0}^{1}\overline{T_{K}f(y)}g(y)\hspace{1mm}dy=\ip{T_{K}f}{g}_{L^{2}}
    \end{align*}
    para toda $f,g\in L^{2}([0,1])$, lo que implica que $T_{K}$ es autoadjunto.
    
    \item Sea $f\in L^{2}([0,1])$ y $(x_{n})_{n\in\N}\subseteq[0,1]$ tal que 
    $x_{n}\xrightarrow[n\to\infty]{}x\in[0,1]$. Como $K$ es continua, para todo $y\in[0,1]$, se 
    tiene que $K(x_{n},y)f(y)\xrightarrow[n\to\infty]{} K(x,y)f(y)$, además
    \begin{equation*}
        \abs{K(x_{n},y)f(y)}\leq\norm{K}_{\infty}\abs{f(y)}=:g
    \end{equation*}
    y como $L^{2}([0,1])\subseteq L^{1}([0,1])$, la función $g$ es integrable. Así, por teorema de
    convergencia dominada resulta que
    \begin{equation*}
        \lim\limits_{n\to\infty}T_{K}f(x_{n})
        =\lim\limits_{n\to\infty}\int_{0}^{1}K(x_{n},y)f(y)\hspace{1mm}dy
        =\int_{0}^{1}\lim\limits_{n\to\infty}K(x_{n},y)f(y)\hspace{1mm}dy
        =\int_{0}^{1}K(x,y)f(y)\hspace{1mm}dy=T_{K}f(x)
    \end{equation*}
    Concluimos que $T_{K}(L^{2})\subseteq\mathcal{C}([0,1])$.
    
    \item \hspace{1mm}
    \begin{enumerate}
        \item Sea $(f_{n})_{n\in\N}$ una sucesión acotada, entonces existe $M>0$ tal que 
        $\norm{f_{n}}_{L^{2}}\leq M$ para todo $n\in\N$. Por lo visto en el primer ejercicio, 
        sabemos que
        \begin{equation*}
            \abs{T_{K}f_{n}(x)}^{2}
            \leq\int_{0}^{1}\abs{K(x,y)}^{2}\abs{f_{n}(y)}^{2}\hspace{1mm}dy
            \leq\norm{K}^{2}_{\infty}\cdot\norm{f_{n}}^{2}_{L^{2}}
            \leq\norm{K}^{2}_{\infty}\cdot M^{2}<\infty
        \end{equation*}
        es decir, la sucesión $(T_{K}f_{n})_{n\in\N}$ es acotada. Veamos que es equicontinua. Dado
        que $[0,1]^{2}$ es compacto, en realidad se tiene que $K$ es uniformemente continua, sea
        $x_{0}\in[0,1]$ y $\varepsilon>0$, existe $\delta>0$ tal que si
        \begin{equation*}
            \abs{x_{0}-x}<\delta \hhtext{entonces}
            \abs{K(x_{0},y)-K(x,y)}<\frac{\varepsilon}{M} \htext{para todo}y\in[0,1]
        \end{equation*}
        Dado $n\in\N$, observemos que
        \begin{equation*}
            \abs{T_{K}f_{n}(x_{0})-T_{K}f_{n}(x)}
            \leq\int_{0}^{1}\abs{K(x_{0},y)-K(x,y)}\cdot\abs{f_{n}(y)}\hspace{1mm}dy
            <\frac{\varepsilon}{M}\cdot\norm{f_{n}}_{L^{1}}\leq\varepsilon
        \end{equation*}
        Por el teorema de Arzela Ascoli se tiene el resultado.
        
        \item Sabemos que $\B=\{\varphi_{n}=e^{2\pi inx}:n\in\Z\}$ es una base ortonormal de 
        $L^{2}([0,1])$, dado $n\in\Z$, vemos que
        \begin{equation*}
            T_{K}\varphi_{n}(x)=\int_{0}^{1}K(x,y)\varphi_{n}(y)\hspace{1mm}dy
            =\int_{0}^{x}y(1-x)e^{2\pi iny}\hspace{1mm}dy
            +\int_{x}^{1}x(1-y)e^{2\pi iny}\hspace{1mm}dy=:I_{1}+I_{2}
        \end{equation*}
        Por un lado tenemos que
        \begin{equation*}
            I_{1}=(1-x)\int_{0}^{x}ye^{2\pi iny}\hspace{1mm}dy
            =(1-x)\left(\frac{ye^{2\pi iny}}{2\pi in}\Big|_{0}^{x}
            -\frac{1}{2\pi in}\int_{0}^{x}e^{2\pi iny}\hspace{1mm}dy\right)
            =(1-x)\left(\frac{xe^{2\pi inx}}{2\pi in}-\frac{e^{2\pi inx}-1}{(2\pi in)^{2}}
            \right)
        \end{equation*}
        y por el otro lado
        \begin{equation*}
            I_{2}=x\int_{x}^{1}(1-y)e^{2\pi iny}\hspace{1mm}dy
            =x\left(\frac{(1-y)e^{2\pi iny}}{2\pi in}\Big|_{x}^{1}
            +\frac{1}{2\pi in}\int_{x}^{1}e^{2\pi iny}\hspace{1mm}dy\right)
            =x\left(-\frac{(1-x)e^{2\pi inx}}{2\pi in}
            +\frac{e^{2\pi in}-e^{2\pi inx}}{(2\pi in)^{2}}\right)
        \end{equation*}
        entonces
        \begin{equation*}
            T_{K}\varphi_{n}(x)=\frac{c_{n}(x)}{(2\pi in)^{2}}
            \htext{donde}\abs{c_{n}(x)}\leq4
            \htext{para todo}x\in[0,1]
        \end{equation*}
        de este modo obtenemos que
        \begin{equation*}
            \sum_{n\in\Z}\norm{T_{K}\varphi_{n}}_{L^{2}}^{2}
            \leq16\sum_{n\in\Z}\frac{1}{(2\pi in)^{4}}<\infty
        \end{equation*}
        Por lo tanto $T_{K}$ es un operador de Hilbert Schmidt, por ejercicio de la guía, 
        concluimos que es compacto.
    \end{enumerate}
    
    \item Sea $\lambda\in\R^{*}$ un autovalor y $f\in L^{2}([0,1])$ un autovector, entonces
    \begin{equation*}
        \int_{0}^{x}y(1-x)f(y)\hspace{1mm}dy+\int_{x}^{1}x(1-y)f(y)\hspace{1mm}dy
        =T_{K}f(x)=\lambda f(x)
    \end{equation*}
    como $f\in L^{2}([0,1])\subseteq L^{1}([0,1])$, entonces $f$ es absolutamente continua, en 
    particular, es continua y por lo tanto por teorema fundamental del cálculo resulta que $f$ es
    diferenciable, luego
    \begin{equation*}
        \lambda f'(x)=-\int_{0}^{x}yf(y)\hspace{1mm}dy+x(1-x)f(x)+\int_{x}^{1}(1-y)f(y)\hspace{1mm}dy
        -x(1-x)f(x)=-\int_{0}^{x}yf(y)\hspace{1mm}dy+\int_{x}^{1}(1-y)f(y)\hspace{1mm}dy
    \end{equation*}
    por la misma razón que antes obtenemos que $f'$ es continua y diferenciable, así
    \begin{equation*}
        \lambda f''(x)=-xf(x)-(1-x)f(x)=-f(x)
    \end{equation*}
    Notar que $f''$ es continua por que $f$ lo es. Por otro lado, veamos que 
    $\lambda f(0)=T_{K}f(0)=0$ y $\lambda f(1)=T_{K}f(1)=0$, lo anterior resulta en la siguiente 
    EDO con condiciones iniciales
    \begin{equation*}
        \begin{cases}
            f''(x)+\frac{1}{\lambda}f(x)=0 \htext{para todo}x\in[0,1] \\
            f(0)=f(1)=0
        \end{cases}
    \end{equation*}
    con $f\in\mathcal{C}^{2}([0,1])$.
    
    \item Asumiendo que $T_{K}$ es positivo, implica que si $\lambda\in\R^{*}$ es autovalor 
    entonces $\lambda\geq0$. Sea $f\in L^{2}([0,1])$ un autovector correspondiente a $\lambda$. De
    la parte anterior sabemos que $f$ satisface la EDO $f''+\frac{1}{\lambda}f=0$ con condiciones 
    iniciales $f(0)=f(1)=0$, de ayudantía, la solución viene dada por
    \begin{equation*}
        f(x)=Acos\left(\frac{x}{\sqrt{\lambda}}\right)+Bsen\left(\frac{x}{\sqrt{\lambda}}\right)
        \htext{con A,B constantes}
    \end{equation*}
    Evaluando en las condiciones iniciales vemos que $0=f(0)=A$, verifiquemos que 
    $\varphi(x)=sen(\frac{x}{\sqrt{\lambda}})$ es autovector,
    \begin{equation*}
        T_{K}\varphi(x)=\int_{0}^{1}K(x,y)sen\left(\frac{y}{\sqrt{\lambda}}\right)\hspace{1mm}dy
        =\int_{0}^{x}y(1-x)sen\left(\frac{y}{\sqrt{\lambda}}\right)\hspace{1mm}dy
        +\int_{x}^{1}x(1-y)sen\left(\frac{y}{\sqrt{\lambda}}\right)\hspace{1mm}dy=:I_{1}+I_{2}
    \end{equation*}
    donde,
    \begin{equation*}
        I_{1}=(1-x)\left(-\sqrt{\lambda}ycos\left(\frac{y}{\sqrt{\lambda}}\right)\Big|_{0}^{x}
        +\sqrt{\lambda}\int_{0}^{x}cos\left(\frac{y}{\sqrt{\lambda}}\right)\hspace{1mm}dy\right)
        =(1-x)\left(-\sqrt{\lambda}xcos\left(\frac{x}{\sqrt{\lambda}}\right)
        +\lambda sen\left(\frac{x}{\sqrt{\lambda}}\right)\right)
    \end{equation*}
    y
    \begin{equation*}
        I_{2}=x\left(-(1-y)\sqrt{\lambda}cos\left(\frac{y}{\sqrt{\lambda}}\right)\Big|_{x}^{1}
        +\sqrt{\lambda}\int_{x}^{1}cos\left(\frac{y}{\sqrt{\lambda}}\right)\hspace{1mm}dy\right)
        =x\left((1-x)\sqrt{\lambda}cos\left(\frac{x}{\sqrt{\lambda}}\right)
        +\lambda sen\left(\frac{x}{\sqrt{\lambda}}\right)\right)
    \end{equation*}

\newpage
    de este modo $T_{K}\varphi=\lambda\varphi$. Aplicando la segunda condición inicial, vemos que
    \begin{equation*}
        0=f(1)=sen\left(\frac{1}{\sqrt{\lambda}}\right)
        \hhtext{entonces}\frac{1}{\sqrt{\lambda}}=n\pi \text{ con }n\in\N,
        \htext{luego}\lambda=\frac{1}{n^{2}\pi^{2}}
    \end{equation*}
    
    \item Como $T_{K}$ es un operador autoadjunto y compacto y por lo tanto 
    $\sigma(T_{K})\setminus\{0\}$ es vacío, son finitos autovalores o corresponde a una sucesión 
    decreciente de autovalores tales que convergen a 0. De la parte anterior sabemos que
    \begin{equation*}
        \left\{\lambda_{n}=\frac{1}{n^{2}\pi^{2}}:n\in\N\right\}=\sigma_{p}(T_{K})
        \subseteq\sigma(T_{K})
    \end{equation*}
    lo que implica que $\overline{\sigma_{p}(T_{K})}=\sigma(T_{K})$.
    
    \item Usando lo anterior y el hecho de que $T_{K}$ es autoadjunto, resulta que
    \begin{equation*}
        \norm{T_{K}}_{\B(L^{2})}=r(T_{K})=\sup_{\lambda\in\sigma(T_{K})}\abs{\lambda}
        =\sup_{\lambda\in\overline{\sigma_{p}(T_{K})}}\abs{\lambda}=\frac{1}{\pi^{2}}
    \end{equation*}
    
    \item Notemos que $K(x,y)=\min\{x,y\}-xy$. Por otro lado, para $x\in[0,1]$, definimos 
    $\varphi_{t}(x)=\I_{\{t\leq x\}}(t)-x$. Veamos que
    \begin{align*}
        \int_{0}^{1}\varphi_{t}(x)\varphi_{t}(y)\hspace{1mm}dt
        &= \int_{0}^{1}(\I_{\{t\leq x\}}(t)-x)(\I_{\{t\leq y\}}(t)-y)\hspace{1mm}dt \\
        &= \int_{0}^{1}\I_{\{t\leq x,t\leq y\}}(t)\hspace{1mm}dt
        -y\int_{0}^{1}\I_{\{t\leq x\}}(t)\hspace{1mm}dt
        -x\int_{0}^{1}\I_{\{t\leq y\}}(t)\hspace{1mm}dt+\int_{0}^{1}xy\hspace{1mm}dt \\
        &= \min\{x,y\}-xy=K(x,y)
    \end{align*}
    De este modo, por tonelli y usando que $\abs{\varphi_{t}(x)}\leq2$, la función 
    $\varphi_{t}(x)\varphi_{t}(y)\overline{f(x)}f(y)$ es integrable para toda $f\in L^{2}([0,1])
    \subseteq L^{1}([0,1])$, así por Fubini vemos que
    \begin{align*}
        \ip{T_{K}f}{f} &= \int_{[0,1]^{2}}K(x,y)\overline{f(x)}f(y)\hspace{1mm}dy\hspace{1mm}dx
        =\int_{[0,1]^{3}}\varphi_{t}(x)\varphi_{t}(y)\overline{f(x)}f(y) 
        \hspace{1mm}dt\hspace{1mm}dy\hspace{1mm}dx \\
        &= \int_{0}^{1}\left(\overline{\int_{0}^{1}\varphi_{t}(x)f(x)\hspace{1mm}dx}
        \cdot\int_{0}^{1}\varphi_{t}(y)f(y)\hspace{1mm}dy\right)\hspace{1mm}dt
        =\int_{0}^{1}\abs{\int_{0}^{1}\varphi_{t}(x)f(x)\hspace{1mm}dx}^{2}\hspace{1mm}dt\geq0
    \end{align*}
    Hemos concluido que $T_{K}$ es un operador positivo.
    
    \item \textbf{Teorema de Mercer}
    
    \begin{lema}
        Sea $\{e_{\alpha}\}_{\alpha\in J}$ un conjunto ortonormal en $\mathcal{H}$. Si 
        $(\{e_{\alpha}\}_{\alpha\in J})^{\perp}=\{0\}$ entonces $\{e_{\alpha}\}_{\alpha\in J}$
        es una base ortonormal de $\mathcal{H}$. Con $\mathcal{H}$ espacio de Hilbert.
    \end{lema}
    
    \begin{proof}
        Sea $M=span(\{e_{\alpha}\}_{\alpha\in J})$, tenemos que $\overline{M}^{\perp}
        =(\{e_{\alpha}\}_{\alpha\in J})^{\perp}$, además se tiene que $\mathcal{H}
        =\overline{M}\oplus(\{e_{\alpha}\}_{\alpha\in J})^{\perp}=\overline{M}$, lo que implica 
        que $\{e_{\alpha}\}_{\alpha\in J}$ es una base ortonormal.
    \end{proof}

    \begin{lema}
        Sea $\{\psi_{n}\}_{n\in\N}$ una base ortonormal de $L^{2}([0,1])$, entonces 
        $\{\overline{\psi_{n}}\psi_{j}\}_{n,j\in\N}$ es una base ortonormal de $L^{2}([0,1])^{2}$.
    \end{lema}

    \begin{proof}
        Por Fubini, vemos que
        \begin{align*}
            \ip{\overline{\psi_{n}}\psi_{j}}{\overline{\psi_{m}}\psi_{i}}_{L^{2}}
            =\int_{0}^{1}\int_{0}^{1}\psi_{n}(x)
            \overline{\psi_{j}(y)}\overline{\psi_{m}(x)}\psi_{i}(y)\hspace{1mm}dx\hspace{1mm}dy
            =\int_{0}^{1}\overline{\psi_{m}(x)}\psi_{n}(x)\hspace{1mm}dx
            \cdot\int_{0}^{1}\overline{\psi_{j}(y)}\psi_{i}(y)\hspace{1mm}dy=\delta_{nm}\delta_{ij}
        \end{align*}
        lo que implica que el conjunto es ortonormal, para ver que es base, por el primer lema 
        basta probar que dado $f\in L^{2}([0,1]^{2})$ tal que si 
        $\ip{f}{\overline{\psi_{n}}\psi_{j}}_{L^{2}}=0$ para todo $n,j\in\N$ entonces $f=0$. Para
        $y\in[0,1]$ definimos $f^{y}(x)=f(x,y)$, que esta en $L^{2}([0,1])$ para casi todo 
        $y\in[0,1]$. Dado $n\in\N$, definimos
        \begin{equation*}
            F_{n}(y)=\ip{\overline{f^{y}}}{\psi_{n}}_{L^{2}([0,1])},
            \htext{que es medible por Fubini}
        \end{equation*}
        Por Cauchy-Schwarz y Fubini notamos que
        \begin{equation*}
            \norm{F_{n}}^{2}_{L^{2}([0,1])}
            =\int_{0}^{1}\abs{\ip{\overline{f^{y}}}{\psi_{n}}_{L^{2}([0,1])}}^{2}\hspace{1mm}dy
            \leq\int_{0}^{1}\norm{f^{y}}^{2}_{L^{2}([0,1])}\hspace{1mm}dy
            =\int_{0}^{1}\int_{0}^{1}\abs{f(x,y)}^{2}\hspace{1mm}dx\hspace{1mm}dy
            =\norm{f}_{L^{2}([0,1]^{2})}
        \end{equation*}
        Además, veamos que $\ip{F_{n}}{\psi_{j}}_{L^{2}([0,1])}
        =\ip{f}{\overline{\psi_{n}}\psi_{j}}_{L^{2}([0,1]^{2})}=0$ para todo $n,j\in\N$. Como 
        $\psi_{j}$ es base, $F_{n}(y)=0$ para casi todo $y\in[0,1]$ y todo $n\in\N$, lo que 
        implica que $f^{y}=0$ para casi todo $y\in[0,1]$, nuevamente por que $\psi_{n}$ es base,
        se sigue que $\norm{f}^{2}_{L^{2}([0,1]^{2})}=0$.
    \end{proof}
    
    \begin{enumerate}
        \item Sea $K:[0,1]^{2}\to\R$ una función continua y simétrica. Sea 
        $T_{k}:L^{2}[(0,1)]\to L^{2}([0,1])$ definido como al inicio, con kernel $K$ que también 
        es positivo. Como $K$ es continua, simétrica y positivo implica que el operador $T_{K}$ es 
        lineal, acotado, autoadjunto, compacto y positivo. Sea $\{\psi_{n}\}_{n\in\N}$ una base 
        ortonormal de autovectores de $T_{K}$ y $\{\lambda_{n}\}_{n\in\N}$ sus respectivos 
        autovalores.

        \vspace{1mm}
        Por el segundo lema se tiene que $\{\overline{\psi_{n}}\psi_{j}\}_{n,j\in\N}$ es base 
        ortonormal de $L^{2}([0,1]^{2})$, entonces
        \begin{equation*}
            K(x,y)=\sum_{n,j\in\N}\ip{K}{\overline{\psi_{n}}\psi_{j}}_{L^{2}}
            \overline{\psi_{n}(x)}\psi_{j}(y)
        \end{equation*}
        por Fubini, notamos lo siguiente
        \begin{align*}
            \ip{K}{\overline{\psi_{n}}\psi_{j}}_{L^{2}}
            &= \int_{0}^{1}\int_{0}^{1}K(x,y)\overline{\psi_{n}(x)}\psi_{j}(y)
            \hspace{1mm}dx\hspace{1mm}dy
            =\int_{0}^{1}\overline{T_{K}\psi_{n}(y)}\psi_{j}(y)\hspace{1mm}dy \\
            &= \lambda_{n}\int_{0}^{1}\overline{\psi_{n}(y)}\psi_{j}(y)\hspace{1mm}dy
            =\lambda_{n}\ip{\psi_{n}}{\psi_{j}}=\lambda_{n}\delta_{nj}
        \end{align*}
        de este modo,
        \begin{equation*}
            K(x,y)=\sum_{n,j\in\N}\lambda_{n}\delta_{nj}\overline{\psi_{n}(x)}\psi_{j}(y)
            =\sum_{n\in\N}\lambda_{n}\overline{\psi_{n}(x)}\psi_{n}(y)
            \htext{en particular, }
            K(x,x)=\sum_{n\in\N}\lambda_{n}\abs{\psi_{n}(x)}^{2}
        \end{equation*}
        Por otro lado, como el producto interno es continuo respecto a la norma en $L^{2}$, se 
        sigue que
        \begin{equation*}
            \sum_{n\in\N}\lambda_{n}
            =\sum_{n\in\N}\lambda_{n}\int_{0}^{1}\abs{\psi_{n}(x)}^{2}\hspace{1mm}dx
            =\int_{0}^{1}\sum_{n\in\N}\lambda_{n}\abs{\psi_{n}(x)}^{2}\hspace{1mm}dx
            =\int_{0}^{1}K(x,x)\hspace{1mm}dx
        \end{equation*}
        
        \item Sabemos que
        \begin{equation*}
            \sigma_{p}(T_{K})=\left\{\lambda_{n}=\frac{1}{n^{2}\pi^{2}}:n\in\N\right\}
        \end{equation*}
        Por otro lado, tenemos que
        \begin{equation*}
            \int_{0}^{1}K(x,x)\hspace{1mm}dx=\int_{0}^{1}x(1-x)\hspace{1mm}dx
            =\left(\frac{x^{2}}{2}-\frac{x^{3}}{3}\right)\Big|_{0}^{1}=\frac{1}{6}
        \end{equation*}
        así, por el teorema de Mercer, concluimos que
        \begin{equation*}
            \sum_{n\in\N}\frac{1}{n^{2}}=\pi^{2}\sum_{n\in\N}\lambda_{n}
            =\pi^{2}\int_{0}^{1}K(x,x)\hspace{1mm}dx=\frac{\pi^{2}}{6}
        \end{equation*}
    \end{enumerate}
\end{enumerate}

%\printbibliography % Quitar el comentado si quiero usar bibliografia

\end{document}
