\documentclass[aop]{imsart2}

%% Packages
\RequirePackage{amsthm,amsmath,amsfonts,amssymb}
\RequirePackage[numbers]{natbib}
\RequirePackage[colorlinks, citecolor=blue, urlcolor=blue]{hyperref}
\RequirePackage{graphicx}

\usepackage[spanish]{babel}
\usepackage[utf8]{inputenc}

%% Arregla error al intentar usar un tamaño de font no disponible
\usepackage{anyfontsize}

\usepackage{tikz} %% Se utiliza para hacer gráficos
\usepackage[all]{xy} %% Para los diagramas conmutativos
\usetikzlibrary{calc}

\usepackage{graphicx} %% Requerido para insertar imagenes

\startlocaldefs

\theoremstyle{plain}
\newtheorem{axioma}{Axioma}
\newtheorem{af}[axioma]{Afirmación}
\newtheorem{teo}{Teorema}[section]
\newtheorem{lema}[teo]{Lema}
\newtheorem{prop}[teo]{Proposición}
\newtheorem{cor}[teo]{Corolario}

\theoremstyle{remark}
\newtheorem{dfn}[teo]{Definición}
\newtheorem*{ej}{Ejemplo}
\newtheorem*{obs}{Observación}

%% Nuevos comandos específicos, necesarios para simplificar la escritura

%% Norma y valor absoluto
\newcommand{\norm}[1]{\left \lVert #1\right \rVert}
\newcommand{\abs}[1]{\left|#1 \right|}

%% Derivada y derivada parcial
\newcommand{\dv}[2]{\Crac{d#1}{d#2}}
\newcommand{\pdv}[2]{\Crac{\partial#1}{\partial#2}}

%% Producto interno y conjunto generador
\newcommand{\ip}[2]{\left\langle{#1},{#2}\right\rangle}
\newcommand{\gen}[1]{\left\langle{#1}\right\rangle}

%% Otros
\newcommand{\htext}[1]{\hspace{4mm}\text{#1 }}
\newcommand{\hhtext}[1]{\hspace{4mm}\text{#1}\hspace{4mm}}
\newcommand{\im}[1]{im\hspace{1mm}#1}
\newcommand{\kr}[1]{ker\hspace{1mm}#1}
\newcommand\sbullet[1][.5]{\mathbin{\vcenter{\hbox{\scalebox{#1}{$\bullet$}}}}}

%% Simplificar comandos

\def \ds {\displaystyle}
\def \C {\mathbb{C}}
\def \R {\mathbb{R}}
\def \Q {\mathbb{Q}}
\def \Z {\mathbb{Z}}
\def \N {\mathbb{N}}
\def \S {\mathbb{S}}

\def \A {\mathcal{A}}
\def \B {\mathcal{B}}
\def \c {\mathcal{C}}
\def \F {\mathcal{F}}
\def \H {\mathcal{H}}
\def \K {\mathcal{K}}
\def \M {\mathcal{M}}

\endlocaldefs

\begin{document}

\begin{titlepage}
    \hspace*{-3cm}
    \raisebox{-1cm}[0pt][0pt]{
        \includegraphics[width=0.4\textwidth]{logo_universidad2.png}
    }

    \vspace{2cm}
    \begin{center}
        {\Large Pontificia Universidad Católica de Chile}
        
        \vspace{2cm}
        {\Huge \bfseries Algebras de Banach de Operadores}
        
        \vspace{1.5cm}
        {\Large Benjamín Mateluna}
        
        \vspace{1.5cm}
        {\Large Docente: Amal Taarabt}
        \vfill
        
        {\large \today \par}
    \end{center}
\end{titlepage}

\tableofcontents

\newpage
\section*{Resumen} \hspace{1mm}

\vspace{1mm}
\noindent En el presente informe estudiaremos las Algebras de Banach, que resultarán ser una
generalización de $\B(\H)$, el espacio de operadores lineales acotados con dominio $\H$. Muchos de
los resultados vistos en clases se extenderán a estos espacios, adicionalmente, estudiaremos la
transformada de Gelfand y como coincide con la transformada de Fourier en $L^{1}(\R)$. Por último,
exploraremos las $\C^{*}-$algebras y el teorema de Gelfand-Naimark, que relaciona las 
$\C^{*}-$algebras con las funciones continuas de un espacio compacto.

\section{Preliminares: Topología débil} \hspace{1mm} %% Esta parte esta lista

\vspace{1mm}
\noindent Dado $X$ un espacio de Banach, recordemos que $X^{*}$ simboliza el espacio dual de $X$,
que corresponde al espacio de funcionales lineales acotados de $X$, y que también es un espacio de 
Banach. El problema de este espacio es que la bola unitaria, denotada por $\B_{X^{*}}$, no es 
compacta bajo la topología inducida por la norma, por eso, nos gustaría encontrar una topología
tal que $\B_{X^{*}}$ sea un conjunto compacto.

\vspace{1mm}
\noindent Para cada $x\in X$, definimos el funcional lineal $\varphi_{x}:X^{*}\to\C$ definido por 
$f\mapsto\varphi_{x}(f)=f(x)$, notar que estos mapas son continuos con la norma usual en $X^{*}$. 
Consideramos la colección $(\varphi_{x})_{x\in X}$ para la siguiente definición.

\begin{dfn}
    La topologíá débil$-^{*}$, denotada por $\sigma(X^{*},X)$, es la topología más pequeña tal que
    la colección $(\varphi_{x})_{x\in X}$ de funcionales lineales es continua. En otras palabras,
    es la topología generada por la sub-base
    \[
        \c=\{\varphi_{x}^{-1}(U):x\in X \hhtext{y} U\subseteq\C \text{ es abierto}\}
    \]
\end{dfn}

\noindent En un momento veremos que $\B_{X^{*}}$ es un conjunto compacto con esta topología, la
ventaja de esto es garantizar la existencia de funciones, por ejemplo, en problemas de 
minimización.

\begin{prop}
    La topología débil$-^{*}$ $\sigma(X^{*},X)$ es Hausdorff.
\end{prop}

\begin{proof}
    Sean $f_{1},f_{2}\in X^{*}$ tal que $f_{1}\neq f_{2}$, entonces existe $x\in X$ tal que 
    $f_{1}(x)\neq f_{2}(x)$. Como $\C$ es Hausdorff, existen vecindades disjuntas $U_{1},U_{2}
    \subseteq\C$ de $f_{1}(x)$ y $f_{2}(x)$ respectivamente. Luego, los abiertos
    \[
        f_{1}\in\varphi_{x}^{-1}(U_{1}) \qquad \text{y} \qquad f_{2}\in\varphi_{x}^{-1}(U_{2})
    \]
    son abiertos disjuntos, y por lo tanto, la topología débil$-^{*}$ $\sigma(X^{*},X)$ es 
    Hausdorff.
\end{proof}

\begin{prop}
    Sean $(f_{n})_{n}\subseteq X^{*}$, entonces $f_{n}\to f$ con la topología débil$-^{*}$ si y 
    solo si $f_{n}(x)\to f_{n}(x)$ para todo $x\in X$.
\end{prop}

\noindent La demostración de este resultado se encuentra en \cite{Brezis} (\textit{Brezis, 
Proposición 3.13}).

\begin{teo}[Banach-Alaoglu]
    La bola unitaria $\B_{X^{*}}$ es un conjunto compacto con la topología débil$-^{*}$ 
    $\sigma(X^{*},X)$.
\end{teo}

\noindent La demostración de este teorema se encuentra en \cite{Brezis} (\textit{Brezis, 
Teorema 3.16}).

\newpage

\section{Algebras de Banach} \hspace{1mm}

\vspace{1mm}
\noindent En esta sección hablaremos sobre las algebras de Banach, que son espacio de Banach 
equipados con una multiplicación que también es un morfismo bilineal. Además, veremos como 
resultados vistos en $\B(\H)$, el espacio de operadores lineales y acotados sobre un espacio de 
Hilbert $\H$, se extienden naturalmente a estos espacios.

\subsection{Definiciones} \hspace{1mm} %% Esta parte esta lista

\vspace{1mm}
\noindent Un álgebra sobre $\C$ es un espacio vectorial $\A$ sobre $\C$ que viene dotado de una
multiplicación de modo que $\A$ es un anillo, más aún, esta operación es un morfismo bilineal, 
notar que $\C$ puede ser considerada como un álgebra sobre $\C$. Un morfismo de algebras es un
morfismo de anillos y un morfismo lineal.

\begin{dfn}
    Un álgebra de Banach es un álgebra $\A$ sobre $\C$ que posee una norma, $\norm{\cdot}$, como
    espacio vectorial con la cual es un espacio de Banach tal que para todo $a,b\in\A$ se
    cumple
    \[
        \norm{ab}\leq\norm{a}\cdot\norm{b}
    \]
    Decimos que la norma es multiplicativa. Si $\A$ tiene una unidad, $e$, entonces $\norm{e}=1$.
\end{dfn}

\begin{obs}
    Notar que si $x_{n}\to x$ e $y_{n}\to y$ entonces $x_{n}y_{n}\to xy$, en efecto, observemos 
    que
    \[
        x_{n}y_{n}-xy=(x_{n}-x)y_{n}+x(y_{n}-y)
    \]
    Y usando desigualdad triangular y que la norma es multiplicativa se tiene la afirmación. Lo 
    anterior muestra que la multiplicación es un mapa continuo.
\end{obs}

\noindent Notemos que $\C$ puede ser visto como una subálgebra de $\A$ vía la identificación
$\alpha\mapsto\alpha e$ que satisface $\norm{\alpha e}=\abs{\alpha}$, es decir, el mapa es un
isomorfismo isométrico. A partir de este momento, denotaremos la unidad en un álgebra por $1$ y a
un elemento $\alpha e$ como $\alpha$.

\vspace{1mm}
\noindent La ausencia de identidad en un álgebra se puede ``arreglar'' y la siguiente proposición 
nos dice como.

\begin{prop} \label{adj unidad}
    Sea $\A$ un álgebra de Banach sin unidad, sea $\A_{1}=\A\times\C$. Definimos las operaciones
    algebraicas en $\A_{1}$ como sigue
    \begin{enumerate}
        \item $(a,\alpha)+(b,\beta)=(a+b,\alpha+\beta)$
        \item $\beta(a,\alpha)=(\beta a,\beta\alpha)$
        \item $(a,\alpha)(b,\beta)=(ab+\alpha b+\beta a,\alpha\beta)$
    \end{enumerate}
    Se define la norma $\norm{(a,\alpha)}:=\norm{a}+\abs{\alpha}$. Entonces $\A_{1}$ es un álgebra
    de Banach con unidad $(0,1)$ y $a\mapsto(a,0)$ un isomorfismo isométrico de $\A$ en $\A_{1}$.
\end{prop}

\begin{proof}
    Verificar las propiedades para ser álgebra es un cálculo sencillo. Veamos que la norma es 
    multiplicativa. Sean $(a,\alpha),(b,\beta)\in\A_{{1}}$, entonces
    \begin{align*}
        \norm{(a,\alpha)(b,\beta)} &= \norm{(ab+\alpha b+\beta a,\alpha\beta)}
        =\norm{ab+\alpha b+\beta a}+\abs{\alpha\beta} \\
        &\leq \norm{a}\norm{b}+\abs{\beta}\norm{a}+\abs{\alpha}\norm{b}+\abs{\alpha}\abs{\beta}
        =\norm{(a,\alpha)(b,\beta)}
    \end{align*}
    Tampoco es díficil chequear que $\A_{1}$ es un espacio de Banach. Que el mapa $a\mapsto(a,0)$
    es un isomorfismo isométrico es directo.
\end{proof}

\noindent La anterior proposición nos permite asumir, a partir de ahora, que toda álgebra de 
Banach tiene una unidad. En caso contrario, siempre existe una álgebra de Banach con unidad, que 
contiene al álgebra original como subálgebra.

\begin{ej}
    Sea $X$ un espacio compacto, entonces $\A=\c(X)$ es un álgebra de Banach con la multiplicación 
    dada por $(fg)(x)=f(x)g(x)$ para todo $f,g\in\A$ y $x\in X$. El álgebra $\A$ es abelina y 
    posee una unidad, que corresponde a la función constante $1$.
\end{ej}

\begin{ej}
    Sea $(X,\F,\mu)$ un espacio de medida $\sigma-$finita. El espacio $L^{\infty}(\mu)$ es un 
    álgebra de Banach conmutativa con unidad si la multiplicación esta definida igual que en el 
    ejemplo anterior.
\end{ej}

\begin{ej}
    Sea $\H$ un espacio de Hilbert, entonces $\A=\B(\H)$ es un álgebra de Banach, donde la 
    multiplicación corresponde a la composición de operadores y tiene una unidad que es el 
    operador identidad. Para $dim_{\C}(\H)\geq2$ el espacio no es conmutativo.

    \vspace{1mm}
    \noindent El espacio $\K(\H)\subseteq\B(\H)$, el conjunto de operadores compactos, es un 
    álgebra de Banach sin unidad si $\H$ es un espacio de dimensión infinita.
\end{ej}

\begin{ej}
    Consideremos el espacio $L^{1}(\R)$. Definimos la multiplicación en $L^{1}$ como
    \[
        (f*g)(x)=\int_{\R}f(t)g(x-t)\hspace{1mm}dt
    \]
    para $f,g\in L^{1}$. Sabemos que $L^{1}$ es un espacio de Banach y que la convolución cumple
    con las propiedades de un álgebra conmutativa, también recordemos que 
    \[
        \norm{f*g}_{L^{1}}\leq\norm{f}_{L^{1}}\norm{g}_{L^{1}}
    \]
    es decir, $L^{1}$ es un álgebra de Banach conmutativa. Más aún, por la proposición \ref{adj 
    unidad}, el espacio $L^{1}\times\C$ es un álgebra de Banach conmutativa con unidad.
\end{ej}

\subsection{Ideales y elementos invertibles} \hspace{1mm} %% Esta parte esta lista

\vspace{1mm}
\noindent Exploraremos más en profundidad la estructura algebraica de un álgebra de Banach, 
especificamente, veremos la noción de ideal, que será idéntica a la que encontramos en anillos.
Además, estudiaremos resultados relacionados a elementos invertibles.

\begin{dfn}
    Sea $\A$ un álgebra. Un ideal por la izquierda (resp. derecha) es una subálgebra $\M$ de $\A$
    tal que $ax\in\M$ (resp. $xa\in\M$) para todo $a\in\A$ y $x\in\M$. Un ideal bilateral es una
    subálgebra de $\A$ que es ideal por la izquierda y por la derecha.

    \vspace{1mm}
    \noindent Decimos que un ideal es maximal, si no existe un ideal propio que lo contenga.
\end{dfn}

\noindent Un resultado algebraico que será útil más adelante es el siguiente

\begin{prop}
    Sea $\A$ un álgebra de Banach, entonces todo ideal propio por izquierda, por derecha o 
    bilateral esta contenido en un ideal maximal del mismo tipo.
\end{prop}

\noindent Esta proposición es una aplicación del lema Zorn y es un resultado meramente algebraico
y por lo tanto no se demostrará en este infrome, ya que no es nuestro propósito.

\begin{dfn}
    Sea $a\in\A$ con $\A$ un álgebra, se dice que $a$ es invertible por la izquierda (resp. 
    derecha) si existe un elemento $x\in\A$ tal que $xa=1$ (resp. $ax=1$). Decimos que es 
    invertible si es invertible por izquierda y por derecha.
\end{dfn}

\noindent En cada caso, definimos los siguientes conjuntos
\begin{align*}
    G_{l} &:= \{a\in\A:a \text{ es invertible por izquierda}\} \\
    G_{r} &:= \{a\in\A:a \text{ es invertible por derecha}\}
\end{align*}
Y $G=G_{l}\cap G_{r}$ es el conjunto de los elementos invertibles.

\begin{obs}
    Si $a\in\A$ es invertible, entonces existe un único elemento, denotado por $a^{-1}$, tal que 
    $aa^{-1}=a^{-1}a=1$. Adicionalmente, notemos que si $\M$ es un ideal propio entonces no puede
    contener elementos invertibles.
\end{obs}

\begin{lema}
    Sea $\A$ un álgebra de Banach y $a\in\A$ tal que $\norm{a}<1$, entonces $1-a\in G$.
\end{lema}

\begin{proof}
    Inductivamente vemos que $\norm{a^{n}}\leq\norm{a}^{n}$, entonces como $\A$ en particular es 
    un espacio de Banach, la serie $\sum_{n\in\N}^{\infty}a^{n}$ es convergente en $\A$ y además
    \[
        (1-a)\sum_{n\in\N}^{\infty}a^{n}
        =\lim\limits_{m\to\infty}\left(\sum_{n=0}^{m}a^{n}-\sum_{n=0}^{m}a^{n+1}\right)
        =\lim\limits_{m\to\infty}1-a^{m+1}=1
    \]
    La otra igualdad es análoga.
\end{proof}

\begin{teo} \label{cl spec}
    Sea $\A$ un álgebra de Banach. Entonces los conjuntos $G_{l}$, $G_{r}$ y $G$ son abiertos, 
    además, el mapa $a\mapsto a^{-1}$ de $G$ en $G$ es continuo.
\end{teo}

\begin{proof}
    Sea $a\in G_{l}$, entonces existe $x\in\A$ tal que $xa=1$. Sea $u\in\A$ de modo que 
    $\norm{u-a}<\norm{x}^{-1}$, luego, $\norm{xu-1}=\norm{x(u-a)}<1$, por el lema anterior el
    elemento $xu$ es invertible, definimos $v:=(xu^{-1})x$ y es directo que $vu=1$. Esto prueba
    que $G_{l}$ es abierto y similarmente $G_{r}$ es abierto, concluimos que $G=G_{l}\cap G_{r}$ 
    es abierto.

    \vspace{1mm}
    \noindent Queda probar que $a\mapsto a^{-1}$ es continuo, para ello, basta probar que si
    $a_{n}\to1$ entonces $a_{n}^{-1}\to1$. Sea $a_{n}\to a$, por continuidad vemos que 
    $a_{n}a^{-1}\to1$, lo que implica que $aa_{n}^{-1}\to1$ y nuevamente por continuidad 
    $a_{n}^{-1}\to a^{-1}$.

    \vspace{1mm}
    \noindent Dado $\varepsilon>0$, consideramos $\delta<\min\{1,\varepsilon/(1+\varepsilon)\}$.
    Supongamos que $\norm{a_{n}-1}<\delta$, del lema anterior, $a_{n}^{-1}=(1-(1-a_{n}))^{-1}
    =\sum_{m\in\N}(1-a_{n})^{m}$, así
    \[
        \norm{a_{n}^{-1}-1}=\norm{\sum_{m\geq1}(1-a_{n})^{m}}
        \leq\sum_{m\geq1}\norm{1-a_{n}}^{m}<\frac{\delta}{1-\delta}<\varepsilon
    \]
\end{proof}

\noindent Durante la demostración utilizamos dos argumentos que vale la pena enunciar.

\begin{cor}
    Sea $\A$ un álgebra de Banach, dado $a\in\A$, se cumple lo siguiente:
    \begin{enumerate}
        \item Si $\norm{a}<1$, entonces $(1-a)^{-1}=\sum_{n\in\N}(1-a)^{n}$.
        \item Si $xa=1$ y $\norm{u-a}<\norm{x}^{-1}$, entonces $u$ es invertible por la izquierda.
    \end{enumerate}
\end{cor}

\subsection{El espectro} %% Esta parte esta lista

\begin{dfn}
    Sea $\A$ un álgebra de Banach y $a\in\A$, el espectro de $a$, denotado por $\sigma(a)$, esta 
    definido por
    \[
        \sigma(a):=\{\alpha\in\C:a-\alpha \text{ no es invertible}\}
    \]
    Análogamente se definen los espectros por izquierda y por derecha, denotados por 
    $\sigma_{l}(a)$ y $\sigma_{r}(a)$ respectivamente.

    \vspace{1mm}
    \noindent El conjunto resolvente de $a$ se define como $\rho(a)=\C\setminus\sigma(a)$. 
    Similarmente se tienen el conjunto resolvente por izquierda y por derecha, denotados por 
    $\rho_{l}(a)$ y $\rho_{r}(a)$ respectivamente.
\end{dfn}

\begin{obs}
    El mapa $\alpha\mapsto a-\alpha$ es una función continua y notemos que $\rho(a)$ es la 
    preimagen de $G$ bajo esta función, entonces $\rho(a)$ es abierto y $\sigma(a)$ es cerrado.
\end{obs}

\begin{ej}
    Sea $X$ un espacio compacto. Consideremos $f\in\c(X)$, luego $\sigma(f)=f(X)$. En efecto, si
    $\alpha=f(x_{0})$, la función $f-\alpha$ tiene un cero y por lo tanto no es invertible, de 
    este modo $f(X)\subseteq\sigma(f)$. Por otro lado, si $\alpha\not\in f(X)$, entonces 
    $f-\alpha$ no se anula y por ende $(f-\alpha)^{-1}$ esta bien definida y es continua, lo que
    prueba $\sigma(f)\subseteq f(X)$.
\end{ej}

\begin{ej}
    Sea $\H$ un espacio de Hilbert. Dado $T\in\B(\H)$, sabemos que
    \[
        \rho(T)=\{\alpha:T-\alpha \text{ es biyección}\}
    \]
    Por el teorema del mapa abierto resulta que $(T-\alpha)^{-1}\in\B(\H)$ para todo 
    $\alpha\in\rho(T)$, y por otro lado, si $T-\alpha$ es invertible claramente es biyectivo. Así, 
    ambas nociones de espectro y conjunto resolvente coinciden.
\end{ej}

\begin{teo} \label{comp spec}
    Sea $\A$ un álgebra de Banach, entonces para cada $a\in\A$, $\sigma(a)$ es un conjunto 
    compacto no vacío. Más aún, si $\abs{\alpha}>\norm{a}$ entonces $\alpha\in\rho(a)$
    y el mapa dado por $z\mapsto(z-a)^{-1}$ es una función analítica definida en $\rho(a)$.
\end{teo}

\begin{proof}
    Sea $\alpha\in\C$ tal que $\abs{\alpha}>\norm{a}$, notemos que $\alpha-a=\alpha(1-a/\alpha)$ 
    donde $\norm{a/\alpha}<1$. Lo que implica que $1-a\alpha^{-1}$ es invertible, se sigue que
    $a-\alpha$ es invertible y por lo tanto $\alpha\in\rho(a)$. De este modo
    \[
        \sigma(a)\subseteq\{\alpha\in\C:\abs{\alpha}\leq\norm{a}\}
    \]
    Como $\sigma(a)$ es cerrado, concluimos que es compacto. Para el segundo punto, definimos
    $F:\rho(a)\to\A$ por $F(z)=(z-a)^{-1}$. Notamos la identidad
    \[
        x^{-1}-y^{-1}=x^{-1}(y-x)y^{-1} \qquad \text{para todo }x,y\in G
    \]
    Reemplazamos $x=\alpha+h-a$ e $y=\alpha-a$, donde $\alpha\in\rho(a)$ y $h\in\C$ tal que
    $h\neq0$ y $\alpha+h\in\rho(a)$. Entonces
    \[
        \frac{F(\alpha+h)-F(\alpha)}{h}=\frac{(\alpha+h-a)^{-1}(-h)(\alpha-a)^{-1}}{h}
        =-(\alpha+h-a)^{-1}(\alpha-a)^{-1}
    \]
    Por continuidad tenemos que $F'(\alpha)=-(\alpha-a)^{-2}$, en otras palabras, $F$ es analítica 
    en $\rho(a)$. La función $F$ puede ser reescrita como $F(z)=z^{-1}(1-az^{-1})^{-1}$, pero 
    $(1-az^{-1})\to1$ cuando $z\to\infty$, porque $\norm{a}<\infty$. Lo anterior implica que
    $F(z)\to0$ cuando $z\to\infty$.

    \vspace{1mm}
    \noindent Supongamos, por contradicción, que $\sigma(a)=\emptyset$, entonces la función $F$ es
    entera y por Teorema de Liouville es constante, sin embargo, $F\not\equiv0$, ya que cero no es
    invertible. Lo anterior prueba que $\sigma(a)$ es no vacío. 
\end{proof}

\begin{cor}[Gelfand-Mazur]
    Sea $\A$ un álgebra de Banach tal que todo elemento no cero es invertible. Entonces $\A$ 
    consiste en múltiplos escalares de la identidad y por lo tanto $\A$ es isometricamente
    isomorfo a $\C$.
\end{cor}

\begin{proof}
    Sea $a\in\A$, existe $\lambda\in\sigma(a)$ tal que $a-\lambda\not\in G$, es decir,
    $a=\lambda$.
\end{proof}

\begin{dfn}
    Sea $\A$ un álgebra de Banach y $a\in\A$, el radio espectral de $a$, denotado por $r(a)$, se 
    define por
    \[
        r(a):=\sup\{\abs{\alpha}:\alpha\in\sigma(a)\}
    \]
\end{dfn}

\noindent Dado que $\sigma(a)$ es no vacío, el radio espectral esta bien definido, más aún, por
compacidad resulta ser finito. La siguiente proposición permite calcular el radio espectral
explícitamente,

\begin{prop}
    Sea $\A$ un álgebra de Banach y $a\in\A$, entonces el límite 
    $\lim\norm{a^{n}}^{\frac{1}{n}}$ existe y
    \[
        r(a)=\lim\limits_{n\to\infty}\norm{a^{n}}^{1/n}
    \]
\end{prop}

\noindent La demostración de este resultado se encuentra en \cite{Conway} (\textit{Conway, 
Proposición 3.8})

\subsection{Algebras de Banach conmutativas} \hspace{1mm} %% Esta parte esta lista

\vspace{1mm}
\noindent En este apartado nos enfocaremos en algebras de Banach conmutativas, el principal 
objeto de estudio serán los morfismos de algebras $h:\A\to\C$, que llamaremos morfismos, donde 
$\A$ es un álgebra de Banach.
objetivo.

\begin{prop} \label{h y ker}
    Sea $\A$ un álgebra de Banach conmutativa y $\M$ un ideal maximal, entonces existe un morfismo
    $h:\A\to\C$ tal que $\M=\kr{h}$.
\end{prop}

\noindent La demostración de este resultado se encuentra en \cite{Conway} (\textit{Conway, 
Proposición 8.2}). El siguiente resultado es fundamental para nuestro objetivo.

\begin{prop}
    Sea $\A$ es un álgebra de Banach y $h:\A\to\C$ un morfismo no trivial, entonces $\norm{h}=1$.
\end{prop}

\begin{proof}
    Sean $a\in\A$ y $\lambda=h(a)$. Supongamos que $\abs{\lambda}>\norm{a}$, entonces por \ref
    {comp spec} $1-a/\lambda$ es invertible. Sea $b=(1-a\lambda^{-1})^{-1}$, notemos que
    $1=b(1-a/\lambda)=b-ba/\lambda$ y como $h(1)=1$, vemos lo siguiente
    \[
        1=h(b-ba/\lambda)=h(b)-h(b)h(a)/\lambda=h(b)-h(b)=0
    \]
    que es una contradicción. De este modo tenemos que $\norm{a}\geq\abs{\lambda}=\abs{h(a)}$, 
    así, $\norm{h}\leq1$. Por otro lado, como $h(1)=1$, observamos que $\norm{h}=1$.
\end{proof}

\noindent La proposición anterior nos dice que todo morfismo es continuo, es más, la norma es
siempre igual a $1$, por ende, todo morfismo es un funcional lineal continuo y la anterior 
observación da pie a la siguiente definición.

\begin{dfn}
    Sea $\A$ un álgebra de Banach conmutativa. Denotamos por $\Sigma$ a la colección de morfismos
    no triviales. Dotamos a este conjunto con la topología débil$-^{*}$, como subconjunto de 
    $\A^{*}$. Este espacio se llama el espacio ideal maximal de $\A$.
\end{dfn}

\noindent El teorema de Banach-Alaoglu nos permite demostrar lo siguiente.

\begin{teo} \label{Sigma}
    Sea $\A$ un álgebra de Banach conmutativa, entonces su espacio ideal maximal $\Sigma$ es un 
    espacio compacto y Hausdorff. Más aún, si $a\in\A$, entonces
    \[
        \sigma(a)=\Sigma(a):=\{h(a):h\in\Sigma\}
    \]
\end{teo}

\begin{proof}
    Para el primer punto, basta probar que $\Sigma$ es cerrado bajo la topología débil$-^{*}$, ya
    que es subconjunto de $\B_{\A^{*}}$, que bajo esta topología es un conjunto compacto.

    \vspace{1mm}
    \noindent Sea $(h_{n})_{n}$ una sucesión en $\Sigma$ y $h\in\B_{A^{*}}$ tal que $h_{n}\to h$
    con la topología débil$-^{*}$. Sean $a,b\in\A$, entonces
    \[
        h(ab)=\lim\limits_{n\to\infty}h_{n}(ab)=\lim\limits_{n\to\infty}h_{n}(a)h_{n}(b)
        =h(a)h(b)
    \]
    Y del mismo modo $h(1)=1$. Lo que implica que $h\in\Sigma$, hemos concluido que $\Sigma$ es 
    compacto.

    \vspace{1mm}
    \noindent Para el segundo punto, dado $h\in\Sigma$ y $\lambda=h(a)$, tenemos que 
    $a-\lambda\in\kr{h}$, entonces $a-\lambda$ no es invertible, en otras palabras, $\Sigma(a)
    \subseteq\sigma(a)$. Por otro lado, si $a-\lambda$ no es invertible, el ideal $(a-\lambda)$ es 
    propio, entonces existe un ideal maximal $\M$ en $\A$ tal que $(a-\lambda)\subseteq\M$, por
    \ref{h y ker}, existe $h$ morfismo de modo que $\M=\kr{h}$, en particular
    \[
        0=h(a-\lambda)=h(a)-\lambda
    \]
    Lo que concluye la demostración.
\end{proof}

\begin{dfn}
    Sea $\A$ un álgebra de Banach con espacio ideal maximal $\Sigma$. Dado $a\in\A$, la 
    transformada de Gelfand de $a$ es la función $\hat{a}:\Sigma\to\C$ definida por 
    $\hat{a}(h)=h(a)$.
\end{dfn}

\begin{teo}
    Sea $\A$ un álgebra de Banach con espacio ideal maximal $\Sigma$, entonces 
    $\hat{a}\in\c(\Sigma)$ para todo $a\in\A$. El mapa $a\to\hat{a}$ de $\A\to\c(\Sigma)$ es un 
    morfismo de algebras continuo con norma $1$. Más aún, para cada $a\in\A$, se tiene que
    \[
        \norm{\hat{a}}_{\infty}=\lim\limits_{n\to\infty}\norm{a^{n}}^{1/n}
    \]
\end{teo}

\begin{proof}
    Sea $(h_{n})_{n}\subseteq\Sigma$ y $h\in\Sigma$ tales que $h_{n}\to h$ con la topología 
    débil$-^{*}$. Entonces, dado $a\in\A$, se tiene que
    \[
        \hat{a}(h_{n})=h_{n}(a)\to h(a)=\hat{a}(h)
    \]
    Lo que prueba la continuidad. Se define $\gamma:\A\to\c(\Sigma)$ por $\gamma(a)=\hat{a}$, sean
    $a,b\in\A$, luego
    \[
        \gamma(ab)(h)=\hat{ab}(h)=h(ab)=h(a)h(b)=\hat{a}(h)\hat{b}(h)
    \]
    Por lo tanto $\gamma(ab)=\gamma(a)\gamma(b)$. Del mismo modo se prueba que $\gamma$ es lineal,
    es decir, $\gamma$ es un morfismo de algebras. Sabemos que $\abs{\hat{a}(h)}=\abs{h(a)}
    \leq\norm{a}$, así, $\norm{\gamma(a)}_{\infty}=\norm{\hat{a}}_{\infty}\leq\norm{a}$ y por ende
    la función $\gamma$ es continua, es más, $\norm{\gamma}\leq1$ y como $\gamma(1)=1$, se tiene 
    $\norm{\gamma}=1$.

    \vspace{1mm}
    \noindent Por el teorema anterior, dado $a\in\A$, resulta que
    \[
        \norm{\hat{a}}_{\infty}=\sup_{\lambda\in\Sigma(a)}\abs{\lambda}
        =\sup\{\abs{\lambda}:\lambda\in\sigma(a)\}=r(a)=\lim\limits_{n\to\infty}\norm{a^{n}}^{1/n}
    \]
\end{proof}

\newpage

\section{\texorpdfstring{$\C^{*}$}{}-Algebras} \hspace{1mm} %% Esta parte esta lista

\vspace{1mm}
\noindent En esta sección, en primer lugar, hablaremos de algebras con una operación adicional, la
involución, que encuentra similitudes con el operador adjunto en $\B(\H)$. Además, presentaremos
el resultado principal del presente trabajo que nos permitirá trabajar aplicaciones de la teoría.

\vspace{1mm}
\noindent Sea $\A$ un álgebra de Banach, una involución es un mapa $a\mapsto a^{*}$ de $\A$ en 
$\A$ tal que las siguientes propiedades se cumplen para todo $a,b\in\A$ y $\alpha\in\C$
\begin{enumerate}
    \item $(a^{*})^{*}=a$.
    \item $(ab)^{*}=b^{*}a^{*}$.
    \item $(\alpha a+b)^{*}=\overline{\alpha}a^{*}+b^{*}$.
\end{enumerate}
Notemos que $1^{*}a=(1^{*}a)^{**}=(a^{*}1)^{*}=a$ y análogamente $a1^{*}=a$, por unicidad, 
concluimos que $1^{*}=1$.

\begin{dfn}
    Una $\C^{*}-$álgebra es un álgebra de Banach $\A$, con una involución tal que para todo 
    $a\in\A$
    \[
    \norm{a^{*}a}=\norm{a}^{2}
    \]
\end{dfn}

\begin{dfn}
    Sean $\A$ y $\B$ \hspace{2mm} $\C^{*}-$algebras y $v:\A\to\B$, se dice $^{*}-$morfismo si $v$ 
    es un morfismo de algebras tal que $v(a^{*})=v(a)^{*}$.
\end{dfn}

\begin{ej}
    Sea $X$ un espacio compacto, luego, $\c(X)$ es una $\C^{*}-$álgebra donde $f^{*}(x)
    =\overline{f(x)}$ para $f\in\c(X)$ y $x\in X$.
\end{ej}

\begin{ej}
    Sea $\H$ un espacio de Hilbert, entonces $\B(\H)$ es una $\C^{*}-$álgebra donde para cada 
    $A\in\B(\H)$, $A^{*}$ es el operador adjunto de $A$.
\end{ej}

\begin{prop}
    Sea $\A$ una $\C^{*}-$álgebra y $a\in\A$, entonces $\norm{a^{*}}=\norm{a}$.
\end{prop}

\begin{proof}
    Notemos que $\norm{a}^{2}=\norm{a^{*}a}\leq\norm{a^{*}}\norm{a}$, lo que implica que $\norm{a}
    \leq\norm{a^{*}}$. Usando que $a=a^{**}$ y reemplazando $a^{*}$ por $a$ se tiene la otra
    desigualdad.
\end{proof}

\begin{dfn}
    Sea $\A$ una $\C^{*}-$álgebra y $a\in\A$, entonces se dice hermitiano si $a=a^{*}$, normal si
    $a^{*}a=aa^{*}$ y unitario si $a^{*}a=aa^{*}=1$.
\end{dfn}

\begin{prop}
    Sea $\A$ una $\C^{*}-$álgebra y $a\in\A$,
    \begin{enumerate}
        \item Si $a$ es invertible, entonces $a^{*}$ es invertible y $(a^{*})^{-1}=(a^{-1})^{*}$.
        
        \item Entonces $a=x+iy$ con $x,y\in\A$ hermitianos.
        
        \item Si $a$ es un elemento unitario de $\A$, entonces $\norm{a}=1$.
        
        \item Sea $\B$ una $\C^{*}-$álgebra y $f:\A\to\B$ un $*-$morfismo, entonces $\norm{f(a)}
        \leq\norm{a}$.
        
        \item Si $a$ es un elemento hermitiano, entonces $\norm{a}=r(a)$.
    \end{enumerate}
\end{prop}

\begin{proof}
    Notar que $a^{*}(a^{-1})^{*}=(a^{-1}a)^{*}=1$ y la otra igualadad es idéntica. Para el segundo
    punto basta tomar
    \[
        x=\frac{a+a^{*}}{2} \qquad \text{y} \qquad y=\frac{a-a^{*}}{2i}
    \]
    Si $a$ es unitario, entonces $1=\norm{a^{*}a}=\norm{a}^{2}$. Supongamos que $a=a^{*}$, luego,
    $\norm{a^{2}}=\norm{a^{*}a}=\norm{a}^{2}$ e inductivamente $\norm{a^{2^{n}}}=\norm{a}^{2^{n}}$
    para $n\geq1$. Entonces $\norm{a^{2^{n}}}^{1/2^{n}}=\norm{a}$ y como
    \[
        r(a)=\lim \norm{a^{2^{n}}}^{1/2^{n}}=\norm{a}
    \]
    Se tiene último punto. Veamos el cuarto punto, sea $a\in\A$, afirmamos que $\sigma(f(a))
    \subseteq\sigma(a)$ y por definición se sigue que $r(f(a))\leq r(a)$. Usando que $a^{*}a$ es 
    hermitiano, vemos que
    \[
        \norm{f(a)}^{2}=\norm{f(a^{*}a)}=r(f(a^{*}a))\leq r(a^{*}a)=\norm{a^{*}a}=\norm{a}^{2}
    \]
    Probemos la afirmación, sea $\alpha\in\rho(a)$ entonces $a-\alpha$ es invertible, es sencillo
    probar que $f(x^{-1})=(f(x))^{-1}$ con $x\in\A$ un elemento invertible, entonces se verifica
    que
    \[
        (f(a)-\alpha)^{-1}=(f(a-\alpha))^{-1}=f((a-\alpha)^{-1})
    \]
    De este modo, $\sigma(f(a))\subseteq\sigma(a)$.
\end{proof}

\noindent Veremos resultados sobre morfismos de álgebra $h:\A\to\C$ que nos permitirán llegar al 
resultado final.

\begin{prop}
    Sea $\A$ una $\C^{*}-$álgebra y $h:\A\to\C$ un morfismo de álgebra no trivial, entonces:
    \begin{enumerate}
        \item $h(a)\in\R$ siempre y cuando $a=a^{*}$.
        
        \item $h(a^{*})=\overline{h(a)}$ para todo $a\in\A$.
        
        \item $h(a^{*}a)\geq0$ para todo $a\in\A$.
        
        \item Si $u$ es unitario, entonces $\abs{h(u)}=1$.
    \end{enumerate}
\end{prop}

\begin{proof}
    Sabemos que $\norm{h}=1$. Supongamos que $a=a^{*}$ y sea $t\in\R$, notemos que
    \[
        \abs{h(a+it)}^{2}\leq\norm{a+it}^{2}=\norm{(a+it)^{*}(a+it)}\leq\norm{a}^{2}+t^{2}
    \]
    y sean $\alpha,\beta\in\R$ tales que $h(a)=\alpha+i\beta$, entonces
    \[
        \norm{a}^{2}+t^{2}\geq\abs{\alpha+i(\beta+t)}^{2}=\alpha^{2}+\beta^{2}+2\beta t+t^{2}
    \]
    para todo $t\in\R$. Si $\beta\neq0$, tomando $t\to\pm\infty$, dependiendo del signo de $\beta$
    se llega a una contradicción, es decir, $h(a)\in\R$.

    \vspace{1mm}
    \noindent Para el segundo punto, existen $x,y\in\A$ hermitianos tales que $a=x+iy$, usando que
    $a^{*}=x-iy$ se tiene el resultado. Además, del punto anterior se sigue que 
    $h(a^{*}a)=h(a^{*})h(a)=\abs{h(a)}^{2}\geq0$. Finalmente, si $u$ es unitario
    \[
        \abs{h(u)}^{2}=h(u^{*})h(u)=h(u^{*}u)=h(1)=1
    \]
\end{proof}

\noindent Notemos que $\C$ puede ser visto como una $\C^{*}$-álgebra. La proposición anterior 
anterior nos permite concluir que todo morfismo de álgebra no trivial $h:\A\to\C$ es un 
$*-$morfismo. Usando lo anterior y \ref{Sigma}, concluimos el siguiente corolario

\begin{cor}
    Sea $\A$ una $\C^{*}-$álgebra conmutativa y $a$ un elemento hermitiano de $\A$, entonces
    $\sigma(a)\subseteq\R$.
\end{cor}

\noindent Con todo lo anterior podemos enunciar y demostrar el teorema principal.

\begin{teo}[Gelfand-Naimark]
    Sea $\A$ una $\C^{*}-$álgebra conmutativa y $\Sigma$ su espacio ideal maximal, entonces la 
    transformada de Gelfand $\gamma:\A\to\c(\Sigma)$ es un $^{*}-$isomorfismo isométrico.
\end{teo}

\begin{proof}
    Sea $a\in\A$ y $h\in\Sigma$, entonces $\hat{a^{*}}(h)=h(a^{*})=\overline{h(a)}
    =\overline{\hat{a}(h)}$, esto es, $\hat{a^{*}}=\overline{\hat{a}}$, por lo tanto la 
    transformada de Gelfand es un $^{*}-$morfismo. Adicionalmente,
    \[
        \norm{a}^{2}=\norm{a^{*}a}=\norm{\widehat{a^{*}a}}_{\infty}
        =\norm{\abs{\hat{a}}^{2}}_{\infty}=\norm{\hat{a}}^{2}_{\infty}
    \]
    Lo que implica que $\gamma$ es una isometría. Dado que la transformada de Gelfand es una
    isometría, su imagen es cerrada, de este modo basta probar que su conjunto imagen es denso.

    \vspace{1mm}
    \noindent Procederemos por el teorema de Stone-Weierstrass. Recordemos que $\hat{1}=1$, 
    entonces $\gamma(\A)$ es una subálgebra de $\c(\Sigma)$ que contiene las constantes. Dado que 
    $\gamma$ es un $^{*}-$morfismo, la imagen es cerrada bajo conjugación compleja. Resta ver que 
    separa puntos en $\Sigma$. Sean $h_{1},h_{2}\in\Sigma$ distintos, entonces existe $a\in\A$ tal 
    que $h_{1}(a)\neq h_{2}(a)$, dicho de otro modo, $\hat{a}(h_{1})\neq\hat{a}(h_{2})$.
\end{proof}

\section{Aplicaciones} \hspace{1mm}

\hspace{1mm}
\noindent Finalmente en esta sección veremos aplicaciones de lo estudiado, especificamente,
demostraremos que la transformada de Gelfand coincide con la transformada de Fourier en 
$L^{1}(\R)$. También aplicaremos el teorema de Gelfand-Naimark en problemas físicos.

\begin{ej}
    Sea $\A=L^{1}(\R)$ con una unidad adjunta. Definimos la transformada de Fourier en $L^{1}(\R)$
    como
    \[
        \F(f)(\xi)=\int_{\R}e^{-ix\xi}f(x)\hspace{1mm}dx=:\hat{f}(\xi)
    \]
    Que es un morfismo lineal tal que $\widehat{f*g}=\hat{f}\hat{g}$. Luego, la función 
    $h:L^{1}(\R)\to\C$ dada por
    \[
        h_{\xi}(f+\alpha)=\hat{f}(\xi)+\alpha \qquad \text{es un morfismo}
    \]
    con $\xi\in[0,\infty]$, si $\xi=\infty$ decimos que $\hat{f}(\xi)=0$.
\end{ej}

\begin{ej}
    (Gelfand-Naimark)
\end{ej}

\newpage

\begin{thebibliography}{2}
    \bibitem{Conway}
        \textsc{Conway, John B.} (2007). 
        \textit{A Course in Functional Analysis}, 2nd ed. Springer.

    \bibitem{Brezis}
        \textsc{Brezis, H.} (2011). 
        \textit{
            Functional Analysis, Sobolev Spaces and Partial Differential Equations
        }, 1st ed. Springer.
\end{thebibliography}

\end{document}