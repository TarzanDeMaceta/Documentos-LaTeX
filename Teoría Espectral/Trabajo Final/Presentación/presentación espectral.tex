\documentclass[11pt, %t,
	aspectratio=169
]{beamer}

\usepackage{booktabs} % Se usa para \toprule, \midrule y \bottomrule mejorar reglas en las tablas
\usepackage{palatino}
\usepackage[default]{opensans}
\usepackage[spanish]{babel}

\usepackage[all]{xy} %% Para los diagramas conmutativos

\usepackage{animate}
\usepackage{tikz}

%% Comandos o definiciones nuevas
\theoremstyle{plain}
\newtheorem{axioma}{Axioma}
\newtheorem{af}[axioma]{Afirmación}
\newtheorem{teo}{Teorema}[section]
\newtheorem{lema}[teo]{Lema}
\newtheorem{prop}[teo]{Proposición}
\newtheorem{cor}[teo]{Corolario}

\theoremstyle{remark}
\newtheorem{dfn}[teo]{Definición}
\newtheorem*{ej}{Ejemplo}
\newtheorem*{obs}{Observación}

%% Nuevos comandos específicos, necesarios para simplificar la escritura

%% Norma y valor absoluto
\newcommand{\norm}[1]{\left \lVert #1\right \rVert}
\newcommand{\abs}[1]{\left|#1 \right|}

%% Derivada y derivada parcial
\newcommand{\dv}[2]{\frac{d#1}{d#2}}
\newcommand{\pdv}[2]{\frac{\partial#1}{\partial#2}}

%% Producto interno y conjunto generador
\newcommand{\ip}[2]{\left\langle{#1},{#2}\right\rangle}
\newcommand{\gen}[1]{\left\langle{#1}\right\rangle}

%% Otros
\newcommand{\htext}[1]{\hspace{4mm}\text{#1 }}
\newcommand{\hhtext}[1]{\hspace{4mm}\text{#1}\hspace{4mm}}
\newcommand{\im}[1]{im\hspace{1mm}#1}
\newcommand{\kr}[1]{ker\hspace{1mm}#1}
\newcommand\sbullet[1][.5]{\mathbin{\vcenter{\hbox{\scalebox{#1}{$\bullet$}}}}}

%% Simplificar comandos

\def \ds {\displaystyle}
\def \C {\mathbb{C}}
\def \R {\mathbb{R}}
\def \Q {\mathbb{Q}}
\def \Z {\mathbb{Z}}
\def \N {\mathbb{N}}
\def \S {\mathbb{S}}

\def \A {\mathcal{A}}
\def \B {\mathcal{B}}
\def \c {\mathcal{C}}
\def \F {\mathcal{F}}
\def \H {\mathcal{H}}
\def \K {\mathcal{K}}
\def \M {\mathcal{M}}
%% Indica la carpeta donde se ubican las imágenes
\graphicspath{{Images/}{Images/homotopía/}}

%% Tema del Beamer
\usetheme{Berlin}

%% Otras configuraciones
\usefonttheme{default}
\useinnertheme{circles}
\setbeamertemplate{footline}

%% Presentación de la disertación/defensa
\title{
	Algebras de Banach de operadores
}

\subtitle{Examen Teoría Espectral (MAT2820)}
\author{Benjamín Mateluna Medina}
\institute{
	Pontificia Universidad Católica de Chile \\ \smallskip \textit{bmateluna@uc.cl}
}
\date{\today}

%% Inicio del código...
\begin{document}

%% Diapositiva --- Presentación
\begin{frame}
	\titlepage
\end{frame}

%% Resumen del trabajo/tesis
\begin{frame}
	\frametitle{Resumen}
	
	Estudiaremos las Algebras de Banach, que resultarán ser una generalización de $\B(\H)$, el 
	espacio de operadores lineales acotados con dominio $\H$. Muchos de los resultados vistos en 
	clases se extenderán a estos espacios, adicionalmente, estudiaremos la transformada de Gelfand 
	y como coincide con la transformada de Fourier en $L^{1}(\R)$. \pause

	\vspace{1mm}
	\noindent Por último, exploraremos las $\C^{*}-$algebras y el teorema de Gelfand-Naimark, que 
	relaciona las $\C^{*}-$algebras con las funciones continuas de un espacio compacto.
\end{frame}

%% Diapositiva --- Estructura de la Disertación
\begin{frame}
	\frametitle{Esquema de la Presentación}
	
	\tableofcontents
\end{frame}

\section{Algebras de Banach}

%% Se tratará la definición de álgebra de Banach y se verán ejemplos
\subsection{Definición y ejemplos} %% Esta parte esta lista

%% Definición de álgebra de Banach
\begin{frame}
	\frametitle{Algebras de Banach}

	\begin{dfn}
		Un álgebra de Banach es un álgebra $\A$ sobre $\C$ que posee una norma, $\norm{\cdot}$, 
		como espacio vectorial con la cual es un espacio de Banach tal que para todo $a,b\in\A$ se
		cumple
		\[
			\norm{ab}\leq\norm{a}\cdot\norm{b}
		\]
		Decimos que la norma es multiplicativa. Si $\A$ tiene una unidad, $e$, entonces 
		$\norm{e}=1$.
	\end{dfn}

	\noindent Que la norma sea multiplicativa nos permite concluir que la función 
	$(x,y)\mapsto xy$ es continua.
\end{frame}

%% Ejemplos de algebras de Banach, los más importantes
\begin{frame}
	\frametitle{Ejemplos}

	\begin{itemize}
		\item Sea $X$ un espacio compacto, entonces $\A=\c(X)$ es un álgebra de Banach con la 
		multiplicación dada por $(fg)(x)=f(x)g(x)$ para todo $f,g\in\A$ y $x\in X$. El álgebra 
		$\A$ es abelina y posee una unidad, que corresponde a la función constante $1$. \pause

		\item Sea $\H$ un espacio de Hilbert, entonces $\A=\B(\H)$ es un álgebra de Banach, donde 
		la multiplicación corresponde a la composición de operadores y tiene una unidad que es el 
		operador identidad. Para $dim_{\C}(\H)\geq2$ el espacio no es conmutativo.

		\vspace{1mm}
		\noindent El espacio $\K(\H)\subseteq\B(\H)$, el conjunto de operadores compactos, es un 
		álgebra de Banach sin unidad si $\H$ es un espacio de dimensión infinita.
	\end{itemize}
\end{frame}

\begin{frame}
	\begin{itemize}
		\item Consideremos el espacio $L^{1}(\R)$. Definimos la multiplicación en $L^{1}$ como
		\[
			(f*g)(x)=\int_{\R}f(t)g(x-t)\hspace{1mm}dt
		\]
		para $f,g\in L^{1}$. Sabemos que $L^{1}$ es un espacio de Banach y que la convolución 
		cumple con las propiedades de un álgebra conmutativa, también recordemos que 
		\[
			\norm{f*g}_{L^{1}}\leq\norm{f}_{L^{1}}\norm{g}_{L^{1}}
		\]
		es decir, $L^{1}$ es un álgebra de Banach conmutativa.
	\end{itemize}
\end{frame}

%% Se verá como las algebras de Banach se asemejan a \B(\H)
\subsection{Propiedades de un álgebra de Banach} %% Esta parte esta lista

%% Sobre elementos invertibles y el espectro
\begin{frame}
	\begin{exampleblock}{Lema}
		Sea $\A$ un álgebra de Banach con unidad y $a\in\A$ tal que $\norm{a}<1$, entonces $1-a$ 
		es invertible.
	\end{exampleblock}
	Donde $(1-a)^{-1}=\sum_{n\in\N}a^{n}$. El conjunto de los elementos invertibles,
	denotado por $G$ es un conjunto abierto y el mapa $a\mapsto a^{-1}$ es continuo. \pause

	\begin{dfn}
		Sea $\A$ un álgebra de Banach y $a\in\A$, el espectro de $a$, denotado por $\sigma(a)$, 
		esta definido por
		\[
			\sigma(a):=\{\alpha\in\C:a-\alpha \text{ no es invertible}\}
		\]

		\vspace{2mm}
		El conjunto resolvente de $a$ se define como $\rho(a)=\C\setminus\sigma(a)$.
	\end{dfn}

	El mapa $\alpha\mapsto a-\alpha$ es una función continua y notemos que $\rho(a)$ es la 
    preimagen de $G$ bajo esta función.
\end{frame}

%% Más resultados sobre el espectro
\begin{frame}
	\begin{itemize}
		\itemsep1em

		\item Sea $X$ un espacio compacto. Consideremos $f\in\c(X)$, luego $\sigma(f)=f(X)$.
		
		\item Sea $\H$ un espacio de Hilbert. Dado $T\in\B(\H)$, sabemos que
		\[
			\rho(T)=\{\alpha:T-\alpha \text{ es biyección}\}
		\]
		Ambas nociones de espectro y conjunto resolvente coinciden.
	\end{itemize} \pause

	El espectro de un elemento es un conjunto compacto y no vacío.
	\begin{dfn}
		Sea $\A$ un álgebra de Banach y $a\in\A$, el radio espectral de $a$, denotado por $r(a)$, 
		se define por
		\[
			r(a):=\sup\{\abs{\alpha}:\alpha\in\sigma(a)\}
			=\lim\norm{a^{n}}^{1/n}
		\]
	\end{dfn}
\end{frame}

%% Introducir la transformada de Gelfand
\subsection{Algebras de Banach conmutativas}

%% Resultados previos
\begin{frame}
	\frametitle{Algebras de Banach conmutativas}

	\begin{alertblock}{Proposición}
		Sea $\A$ es un álgebra de Banach y $h:\A\to\C$ un morfismo no trivial, entonces 
		$\norm{h}=1$.
	\end{alertblock}

	Vemos que todo morfismo es un funcional lineal continuo. Podemos definir lo siguiente. \pause

	\begin{dfn}
		Sea $\A$ un álgebra de Banach conmutativa. Denotamos por $\Sigma$ a la colección de 
		morfismos no triviales. Dotamos a este conjunto con la topología débil$-^{*}$, como 
		subconjunto de $\A^{*}$. Este espacio se llama el espacio ideal maximal de $\A$.
	\end{dfn}

	Por Banach-Alaoglu este espacio es compacto y Hausdorff.
\end{frame}

%% La transformada de Gelfand
\begin{frame}
	\frametitle{La transformada de Gelfand}

	\begin{dfn}
		Sea $\A$ un álgebra de Banach con espacio ideal maximal $\Sigma$. Dado $a\in\A$, la 
		transformada de Gelfand de $a$ es la función $\hat{a}:\Sigma\to\C$ definida por 
		$\hat{a}(h)=h(a)$.
	\end{dfn}

	Se tiene que $\hat{a}\in\c(\Sigma)$ para todo $a\in\A$ y además, la asignación $a\mapsto
	\hat{a}$ es un morfismo de algebras continuo con norma $1$.
\end{frame}

%% La transformada de Fourier como caso particular
\begin{frame}
	\frametitle{La transformada de Fourier}

	Buscamos los morfismo $h:L^{1}(\R)\to\C$. Por teorema de representación de Riesz, existe un
	único $\phi\in L^{\infty}(\R)$ tal que
	\[
		h(f)=\int_{\R}f(x)\phi(x)\hspace{1mm}dx
	\]
	Usando que $h(f*g)=h(f)h(g)$ se tiene que $\phi(x+y)=\phi(x)\phi(y)$ $\lambda-$c.t.p. Luego,
	se tiene la igualdad
	\[
		\int_{\R}f(x)\phi(x+y)\hspace{1mm}dx=\phi(y)\int_{\R}f(x)\phi(x)\hspace{1mm}dx
		=\phi(y)h(f)
	\]
	existe $f$ tal que $h(f)\neq0$ y por lo tanto
\end{frame}

\begin{frame}
	\[
		\phi(y)=\frac{1}{h(f)}\int_{\R}f(x)\phi(x+y)\hspace{1mm}dx
		=\frac{1}{h(f)}\int_{\R}f(x-y)\phi(x)\hspace{1mm}dx
	\]
	lo que implica que $\phi$ es continua. Así, $\phi(0)=1$, además, $\norm{\phi}_{\infty}
	=\norm{h}\leq1$, es decir, $\abs{\beta(x)}\leq1$. Por continuidad y $\phi(0)=1$, existe 
	$\delta>0$ tal que
	\[
		\int_{0}^{\delta}\phi(x)\hspace{1mm}dx=a\neq0
	\]
	Entonces
	\[
		a\phi(x)=\phi(x)\int_{0}^{\delta}\phi(t)\hspace{1mm}dt
		=\int_{x}^{x+\delta}\phi(t)\hspace{1mm}dt
		\qquad \text{entonces } \phi(x)=a^{-1}\int_{x}^{x+\delta}\phi(t)\hspace{1mm}dt
	\]
\end{frame}

\begin{frame}
	Por teorema fundamental del cálculo, la función $\phi$ es diferenciable, más aún,
	\[
		\frac{\phi(x+s)-\phi(x)}{s}=\phi(x)\cdot\frac{\phi(s)-1}{s}
	\]
	Tenemos una EDO $\phi'(x)=\phi'(0)\phi(x)$ con condiciones iniciales $\phi(0)=1$, cuya 
	solución es $\phi(x)=e^{-ix\xi}$. De este modo,
	\[
		\hat{f}(h)=\int_{\R}f(x)e^{-ix\xi}\hspace{1mm}dx=\F(f)(\xi)
	\]

\end{frame}

%% Cambiamos a C^{*}-algebras, para hablar del teorema de Gelfand
\section{\texorpdfstring{$\C^{*}$}{}-Algebras}

%% Definiciones previas y ejemplos
\begin{frame}
	\frametitle{$\C^{*}-$Algebras}

	\begin{dfn}
		Sea $\A$ un álgebra de Banach, una involución es un mapa $a\mapsto a^{*}$ de $\A$ en 
		$\A$ tal que las siguientes propiedades se cumplen para todo $a,b\in\A$ y $\alpha\in\C$
		\begin{enumerate}
			\item $(a^{*})^{*}=a$
			\item $(ab)^{*}=b^{*}a^{*}$
			\item $(\alpha a+b)^{*}=\overline{\alpha}a^{*}+b^{*}$
		\end{enumerate}
		Notemos que $1^{*}a=(1^{*}a)^{**}=(a^{*}1)^{*}=a$ y análogamente $a1^{*}=a$, por unicidad, 
		concluimos que $1^{*}=1$.
	\end{dfn}
\end{frame}

\begin{frame}
	\begin{dfn}
		Una $\C^{*}-$álgebra es un álgebra de Banach $\A$, con una involución tal que para todo 
		$a\in\A$
		\[
		\norm{a^{*}a}=\norm{a}^{2}
		\]
	\end{dfn} \pause

	\begin{itemize}
		\item Sea $X$ un espacio compacto, luego, $\c(X)$ es una $\C^{*}-$álgebra donde $f^{*}(x)
    	=\overline{f(x)}$ para $f\in\c(X)$ y $x\in X$.

		\item Sea $\H$ un espacio de Hilbert, entonces $\B(\H)$ es una $\C^{*}-$álgebra donde para 
		cada $A\in\B(\H)$, $A^{*}$ es el operador adjunto de $A$.
	\end{itemize} \pause

	Un morfismo de $\C^{*}-$algebras es un morfimo de algebras tal que $f(a^{*})=f(a)^{*}$. Todo
	morfismo $h:\A\to\C$ es un $^{*}-$morfismo.
\end{frame}

%% Teorema de Gelfand Naimark
\begin{frame}
	\begin{alertblock}{Teorema}
		Sea $\A$ una $\C^{*}-$álgebra conmutativa y $\Sigma$ su espacio ideal maximal, entonces la 
    	transformada de Gelfand $\gamma:\A\to\c(\Sigma)$ es un $^{*}-$isomorfismo isométrico.
	\end{alertblock}

	Es válido para $\C^{*}-$álgebras en general, es decir, no necesariamente deben ser unitarias
	o ser conmmutativas, en el primer caso el conjunto es $\c_{0}(\Sigma)$ y en el segundo caso
	el isomorfismo es a una $\C^{*}-$subálgebra de operadores acotados en un espacio de Hilbert.
\end{frame}

\end{document} 