\documentclass{article}
\usepackage{hyperref}
\usepackage{Style}

\nocite{*} % Comentar si quiero citar
%\addbibresource{bibliografia.bib} % Quitar el comentado si quiero usar bibliografia

\begin{document}

\begin{minipage}{2.5cm}
    \includegraphics[width=2cm]{imagen_puc.jpg}
\end{minipage}
\begin{minipage}{14cm}
    {\sc Pontificia Universidad Católica de Chile\\
    Facultad de Matemáticas\\
    Departamento de Matemática\\
    Profesor: Gregorio Moreno -- Estudiante: Benjamín Mateluna}
\end{minipage}
\vspace{1ex}

{\centerline{\bf Teoría de Integración - MAT2534}
\centerline{\bf Tarea 1}}
\centerline{\bf 02 de Abril de 2025}

\section*{Problema 1}
\noindent Por definición de ínfimo, dado $\varepsilon>0$, existe una partición $\Pi_{\varepsilon}$ 
de $[a,b]$ tal que $U(f,\Pi_{\varepsilon})<\mathcal{U}+\varepsilon$. Consideremos la colección
$\left(\Pi^{*}_{n}\right)_{n}$ de particiones de $[a,b]$ tales que
\begin{equation*}
    U(f,\Pi^{*}_{n})<\mathcal{U}+\frac{1}{n}
\end{equation*}
Definimos la partición
\begin{equation*}
    \Pi_{n}:=\bigcup_{i=1}^{n}\Pi^{*}_{i}
\end{equation*}
Notemos que $\Pi_{n}$ es un refinamiento de $\Pi^{*}_{n}$, luego $U(f,\Pi_{n})\leq 
U(f,\Pi^{*}_{n})$, por teorema del sandwich vemos que
\begin{equation*}
    \lim_{n\to\infty}U(f,\Pi_{n})=\mathcal{U}
\end{equation*}
Además, por construcción, $\Pi_{n+1}$ es un refinamiento de $\Pi_{n}$ y por lo tanto 
$U(f,\Pi_{n+1})\leq U(f,\Pi_{n})$.

\section*{Problema 2}
\begin{enumerate}
    \item Sea $\varepsilon>0$, como $f$ es integrable, existe una partición $\Pi$ del intervalo 
    $[a,b]$ tal que
    
    \begin{equation*}
        U(f,\Pi)-L(f,\Pi)<\varepsilon
    \end{equation*}
    Veamos que
    \begin{align*}
        U(\abs{f},\Pi)-L(\abs{f},\Pi) &= \sum \left(\sup_{x\in I_{i}}\abs{f(x)}-
        \inf_{x\in I_{i}}\abs{f(x)}\right)\abs{I_{i}}=\sum \left(\sup_{x,y\in I_{i}}
        \abs{\abs{f(x)}-\abs{f(y)}}\right)\abs{I_{i}}\\
        & \leq\sum \left(\sup_{x,y\in I_{i}}\abs{f(x)-f(y)}\right)\abs{I_{i}}
        =\sum \left(\sup_{x\in I_{i}}f(x)-\inf_{x\in I_{i}}f(x)\right)\abs{I_{i}} \\
        & =U(f,\Pi)-L(f,\Pi)<\varepsilon
    \end{align*}
    Por lo tanto $\abs{f}$ es Riemann-integrable. Por otro lado, sabemos que $f\leq\abs{f}$ y 
    $-f\leq\abs{f}$, luego, dada una partición $\Pi$ de $[a,b]$ se sigue que
    \begin{equation*}
        U(f,\Pi)\leq U(\abs{f},\Pi)\hspace{4mm}\text{y}\hspace{4mm}L(-f,\Pi)\leq L(\abs{f},\Pi)
    \end{equation*}
    Aplicando ínfimo a la izquierda, supremo a la derecha y usando que $-\inf x=\sup -x$ tenemos 
    lo siguiente
    \begin{equation*}
        \int_{a}^{b}f(x)dx\leq\int_{a}^{b}\abs{f(x)}dx\hspace{4mm}\text{y}\hspace{4mm}
        -\int_{a}^{b}f(x)dx\leq\int_{a}^{b}\abs{f(x)}dx
    \end{equation*}
    Concluimos que
    \begin{equation*}
        \abs{\int_{a}^{b}f(x)dx}\leq\int_{a}^{b}\abs{f(x)}dx
    \end{equation*}

    \item En primer lugar, demostraremos que dado un conjunto $E\subseteq[a,b]$ de medida nula, 
    entonces $E^{c}$ es denso en $[a,b]$. En efecto, sea $x\in E$ y $U$ una vecindad conexa de $x$ 
    de largo $\varepsilon>0$. Supongamos, por contradicción, que $U\subseteq E$. 
    Como $E$ es de medida nula, existe $(I_{i})_{i\in\N}$ una colección de intervalos tales que
    \begin{equation*}
        E\subseteq\bigcup_{i\in\N}I_{i}\hspace{4mm}\text{y}\hspace{4mm}
        \sum_{i\in\N}\abs{I_{i}}<\varepsilon
    \end{equation*}
    por otro lado tenemos que
    \begin{equation*}
        \varepsilon=\abs{U}=\lambda^{*}(U)\leq\lambda^{*}\left(\bigcup_{i\in\N}I_{i}\right)\leq
        \sum_{i\in\N}\lambda^{*}(I_{i})=\sum_{i\in\N}\abs{I_{i}}
    \end{equation*}
    Lo anterior es una contradicción, lo que prueba la afirmación.
    \vspace{4mm}

    \noindent Sea $\Pi$ una partición de $[a,b]$, por lo probado anteriormente, para todo $i$ se
    tiene que existe $\overline{x_{i}}\in E^{c}$ tal que $\overline{x_{i}}\in[x_{i-1},x_{i}]$. De
    este modo, para cada partición $\Pi$ de $[a,b]$, escojemos el conjunto de representantes 
    $C=\{\overline{x_{i}}\}_{i}$, entonces
    \begin{equation*}
        S(f,\Pi,C)=\sum_{i}f(\overline{x_{i}})(x_{i}-x_{i-1})=0
    \end{equation*}
    Como $f$ es Riemann integrable, se sigue que
    \begin{equation*}
        \int_{a}^{b}f(x)dx=\lim_{\abs{\Pi}\to0}S(f,\Pi,C)=\lim_{\abs{\Pi}\to0}0=0
    \end{equation*}
\end{enumerate}

\section*{Problema 3}
\noindent Denotaremos por $\overline{x_{i}}$ a los puntos en $C$ y $L:=\frac{b-a}{n}$. Por TFC 
tenemos lo siguiente
\begin{equation*}
    f(x)-f(\overline{x_{i}})=\int_{\overline{x_{i}}}^{x}f'(t)-f'(\overline{x_{i}})dt+
    f'(\overline{x_{i}})(x-\overline{x_{i}})
\end{equation*}
Por TVM existe $\alpha(t)\in(\overline{x_{i}},t)$ tal que
\begin{equation*}
    f''(\alpha(t))=\frac{f'(t)-f'(\overline{x_{i}})}{t-\overline{x_{i}}}
\end{equation*}
para $t=\overline{x_{i}}$ diremos que $f''(\alpha(\overline{x_{i}}))=f''(\overline{x_{i}})$. Luego
\begin{equation*}
    f(x)-f(\overline{x_{i}})=\int_{\overline{x_{i}}}^{x}f''(\alpha(t))(t-\overline{x_{i}})dt+
    f'(\overline{x_{i}})(x-\overline{x_{i}})
\end{equation*}
Afirmamos que la función $f''(\alpha(t))$ es continua, en efecto
\begin{equation*}
    \lim_{t\to t_{0}}f''(\alpha(t))=\lim_{t\to t_{0}}\frac{f'(t)-f'(\overline{x_{i}})}
    {t-\overline{x_{i}}}=\frac{f'(t_{0})-f'(\overline{x_{i}})}
    {t_{0}-\overline{x_{i}}}=f''(\alpha(t_{0}))
\end{equation*}
Para el caso $t_{0}=\overline{x_{i}}$ en la segunda expresión nos queda exactamente la definición 
de la segunda derivada de $f$ en el punto $\overline{x_{i}}$, por lo tanto $f(\alpha)$ es 
continua en el intervalo cerrado entre $\overline{x_{i}}$ y $x$. Como $t-\overline{x_{i}}$ no 
cambia de signo en el anterior intervalo, por TVM para integrales existe
$\alpha(t_{0})$ tal que
\begin{equation*}
    f(x)-f(\overline{x_{i}})=f''(\alpha(t_{0}))\int_{\overline{x_{i}}}^{x}t-\overline{x_{i}}
    \hspace{1mm}dt+f'(\overline{x_{i}})(x-\overline{x_{i}})=f''(\alpha(t_{0}))
    \frac{(x-\overline{x_{i}})^{2}}{2}+f'(\overline{x_{i}})(x-\overline{x_{i}})
\end{equation*}
De este modo
\begin{align*}
    \abs{\int_{a}^{b}f(x)\hspace{1mm}dx-S_{n}} &= \abs{\sum_{i=1}^{n}\int_{x_{i-1}}^{x_{i}}f(x)
    \hspace{1mm}dx-\sum_{i=1}^{n}f(\overline{x_{i}})L}=\abs{\sum_{i=1}^{n}\int_{x_{i-1}}^{x_{i}}
    f(x)-f(\overline{x_{i}})\hspace{1mm}dx} \\
    &= \abs{\sum_{i=1}^{n}\int_{x_{i-1}}^{x_{i}}f''(\alpha(t_{0}))\frac{(x-\overline{x_{i}})^{2}}
    {2}+f'(\overline{x_{i}})(x-\overline{x_{i}})\hspace{1mm}dx} \\
    &= \abs{\sum_{i=1}^{n}f''(\alpha(t_{0}))\frac{(x-\overline{x_{i}})^{3}}{6}
    \Big|_{x_{i-1}}^{x_{i}}}=\abs{\sum_{i=1}^{n}f''(\alpha(t_{0}))\frac{L^{3}}{24}}\leq
    \frac{M(b-a)^{3}}{24n^{2}}
\end{align*}

\section*{Problema 4}
\noindent En primer lugar demostraremos que si $x_{0}<a$ entonces $F(x_{0})\leq F(a^{-})$ donde 
$F(a^{-})=\lim\limits_{x\to a^{-}}f(x)$. En efecto, dado $\varepsilon>0$, existe $\delta>0$ tal que
\begin{equation*}
    F(x_{0})\leq F(a-\delta/2)<F(a^{-})+\varepsilon
\end{equation*}
esto implica que $F(x_{0})\leq F(a^{-})$.
\begin{enumerate}
    \item Sea $\varepsilon>0$, existe $\delta>0$ tal que $-F(a-\delta/2)<-F(a^{-})+\varepsilon$. 
    Consideremos el cubrimiento $(a-\delta/2,a]$ de $\{a\}$, luego
    \begin{equation*}
        \mu_{F}^{*}(\{a\})\leq F(a)-F(a-\delta/2)<F(a)-F(a^{-})+\varepsilon
    \end{equation*}
    es decir,
    \begin{equation*}
        \mu_{F}^{*}(\{a\})<F(a)-F(a^{-})+\varepsilon
    \end{equation*}
    
    como $\varepsilon$ es arbitrario, se sigue que $\mu_{F}^{*}(\{a\})\leq F(a)-F(a^{-})$. 
    \vspace{2mm} \\
    Sea $(I_{n})_{n\in\N}$ un cubrimiento de $\{a\}$, donde $I_{n}=(a_{n},b_{n}]$. Existe $n_{0}$ 
    tal que $a\in I_{n_{0}}$, así
    \begin{equation*}
        \sum_{n\in\N}\tau(I_{n})=\sum_{n\in\N,\hspace{1mm}n\neq n_{0}}\tau(I_{n})+\tau(I_{n_{0}})
        \geq\tau(I_{n_{0}})=F(b_{n_{0}})-F(a_{n_{0}})\geq F(a)-F(a^{-})
    \end{equation*}
    concluimos que $\mu_{F}^{*}(\{a\})=F(a)-F(a^{-})$.

    \item Dado $\varepsilon>0$, existe $\delta>0$ tal que $-F(a-\delta/2)<-F(a^{-})+\varepsilon$.
    Consideramos el cubrimiento $(a-\delta/2,b]$ de $[a,b]$, entonces
    \begin{equation*}
        \mu_{F}^{*}([a,b])\leq\tau((a-\delta/2,b])=F(b)-F(a-\delta/2)<F(b)-F(a^{-})+\varepsilon
    \end{equation*}

    como $\varepsilon$ es arbitrario, vemos que $\mu_{F}^{*}([a,b])\leq F(b)-F(a^{-})$. 
    \vspace{2mm}\\
    Sea $(I_{n})_{n}$ un cubrimiento de $[a,b]$, diremos que $I_{n}=(a_{n},b_{n}]$. Sea 
    $\varepsilon>0$, dado $n\in\N$ existe $\delta_{n}>0$ tal que
    \begin{equation*}
        F(b_{n}+\delta_{n}/2)-F(b_{n})<\frac{\varepsilon}{2^{n}}
    \end{equation*}
    Consideramos $(I_{n}^{*})_{n}$ un cubrimiento de $[a,b]$ donde $I_{n}^{*}:=
    (a_{n},b_{n}+\delta_{n}/2]$. Como $F$ es creciente notemos que $\tau(A)\leq\tau(B)$ para todo 
    $A\subseteq B$ con $A,B\in\mathcal{C}$, luego
    \begin{equation*}
        \sum_{n\in\N}\tau(I_{n})\leq\sum_{n\in\N}\tau(I_{n}^{*})
    \end{equation*}
    por otro lado
    \begin{equation*}
        \sum_{n\in\N}\tau(I_{n}^{*})-\tau(I_{n})=\sum_{n\in\N}F(b_{n}+{\delta_{n}/2})
        -F(b_{n})<\sum_{n\in\N}\frac{\varepsilon}{2^{n}}=\varepsilon
    \end{equation*}
    Definimos un cubrimiento abierto de $[a,b]$ dado por $(U_{n})_{n\in\N}$ donde 
    $U_{n}:=int(I_{n}^{*})$, por compacidad existe $N\in\N$ tal que $(U_{n})_{n=1}^{N}$ cubre 
    $[a,b]$, sin perdidad de generalidad podemos suponer que
    \begin{itemize}
        \item $a\in U_{1}$ y $b\in U_{N}$.
        \item $a_{n+1}<b_{n}+\delta_{n}/2$ para todo $1\leq n\leq N-1$.
    \end{itemize}
    Entonces, como $F$ es creciente y $U_{n}\subseteq I_{n}^{*}$ para todo $n\in\N$, vemos que
    \begin{align*}
        \sum_{n\in\N}\tau(I_{n})+\varepsilon &> \sum_{n\in\N}\tau(I_{n}^{*})\geq
        \sum_{n=1}^{N}\tau(I_{n}^{*})=\sum_{n=1}^{N}F(b_{n}+\delta_{n}/2)-F(a_{n}) \\
        &= \sum_{n=1}^{N-1}F(b_{n}+\delta_{n}/2)-F(a_{n+1})+F(b_{N}+\delta_{N}/2)-F(a_{1}) \\
        &\geq F(b_{N}+\delta_{N}/2)-F(a_{1})\geq F(b)-F(a^{-})
    \end{align*}
    nuevamente, como $\varepsilon$ es arbitrario, se tiene que
    \begin{equation*}
        \sum_{n\in\N}\tau(I_{n})\geq F(b)-F(a^{-})
    \end{equation*}
    es decir, $\mu_{F}^{*}([a,b])\geq F(b)-F(a^{-})$. Por lo tanto $\mu_{F}^{*}([a,b])=
    F(b)-F(a^{-})$.
    
    \item Consideremos el cubrimiento $(a,b]$ de $(a,b]$, luego
    \begin{equation*}
        \mu_{F}^{*}((a,b])\leq\tau((a,b])=F(b)-F(a)
    \end{equation*}
    Por otro lado, por subaditividad de la medida exterior vemos que
    \begin{equation*}
        \mu_{F}^{*}([a,b])\leq\mu_{F}^{*}(\{a\})+\mu_{F}^{*}((a,b])
    \end{equation*}
    entonces
    \begin{equation*}
        F(b)-F(a^{-})\leq F(a)-F(a^{-})+\mu_{F}^{*}((a,b])
    \end{equation*}
    es decir, $F(b)-F(a)\leq\mu_{F}^{*}((a,b])$. Concluimos que $\mu_{F}^{*}((a,b])=F(b)-F(a)$.
\end{enumerate}

%\printbibliography % Quitar el comentado si quiero usar bibliografia

\end{document}
