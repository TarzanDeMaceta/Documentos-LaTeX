\documentclass{article}
\usepackage{hyperref}
\usepackage{Style}

\nocite{*} % Comentar si quiero citar
%\addbibresource{bibliografia.bib} % Quitar el comentado si quiero usar bibliografia

\begin{document}

\begin{minipage}{2.5cm}
    \includegraphics[width=2cm]{imagen_puc.jpg}
\end{minipage}
\begin{minipage}{14cm}
    {\sc Pontificia Universidad Católica de Chile\\
    Facultad de Matemáticas\\
    Departamento de Matemática\\
    Profesor: Gregorio Moreno -- Estudiante: Benjamín Mateluna}
\end{minipage}
\vspace{1ex}

{\centerline{\bf Teoría de Integración - MAT2534}
\centerline{\bf Tarea 3}}
\centerline{\bf 06 de junio de 2025}

\section*{Problema 1}
\begin{enumerate}
    \item Sea $t>0$, notemos que $e^{-tx^{2}}<e^{-tx}$ para $x>1$, es positiva y continua, luego
    \begin{equation*}
        I=\int_{0}^{\infty}e^{-tx^{2}}dx=\int_{(0,\infty)}e^{-tx^{2}}d\lambda(x)
    \end{equation*}
    entonces
    \begin{align*}
        I^{2} &= \left(\int_{(0,\infty)}e^{-tx^{2}}d\lambda(x)\right)
        \left(\int_{(0,\infty)}e^{-ty^{2}}d\lambda(y)\right)
        =\int_{(0,\infty)}e^{-ty^{2}}\left(\int_{(0,\infty)}e^{-tx^{2}}d\lambda(x)\right)
        d\lambda(y) \\[2mm]
        &= \int_{(0,\infty)}\left(\int_{(0,\infty)}e^{-t(x^{2}+y^{2})}d\lambda(x)\right)d\lambda(y)
    \end{align*}
    Como la función $e^{-t(x^{2}+y^{2})}$ es continua y positiva, por tonelli, vemos que
    \begin{equation*}
        I^{2}=\int_{(0,\infty)^{2}}e^{-t(x^{2}+y^{2})}(\lambda\otimes\lambda)dxdy
    \end{equation*}
    Consideremos el cambio de variables $x=rcos(\theta)$ y $y=rsen(\theta)$ con $r\in(0,\infty)$ y 
    $\theta\in(0,\frac{\pi}{2})$ con determinante jacobiano igual a $r$, luego, por cambio de 
    variables se sigue que
    \begin{equation*}
        I^{2}=\int_{(0,\infty)^{2}}e^{-t(x^{2}+y^{2})}(\lambda\otimes\lambda)dxdy
        =\int_{\left(0,\frac{\pi}{2}\right)\times(0,\infty)}re^{-tr^{2}}(\lambda\otimes\lambda)
        d\theta dr
    \end{equation*}
    por tonelli, vemos que
    \begin{equation*}
        I^{2}=\int_{\left(0,\frac{\pi}{2}\right)}\left(\int_{(0,\infty)}re^{-tr^{2}}
        dr\right)d\theta
        =\int_{0}^{\frac{\pi}{2}}\left(\int_{0}^{\infty}re^{-tr^{2}}
        dr\right)d\theta
        =\frac{\pi}{2}\int_{0}^{\infty}re^{-tr^{2}}dr
    \end{equation*}
    donde la segunda igualdad se debe a la continuidad y positividad. Utlizando el cambio de 
    variable $u=tr^{2}$, observamos que
    \begin{equation*}
        \int_{0}^{\infty}re^{-tr^{2}}dr=\frac{1}{2t}\int_{0}^{\infty}e^{-u}du
        =\frac{1}{2t}-e^{-u}\Big|_{0}^{\infty}=\frac{1}{2t}<\infty
    \end{equation*}
    De este modo,
    \begin{equation*}
        I^{2}=\frac{\pi}{4t}
    \end{equation*}
    concluimos que $g(t)=\frac{1}{2}\sqrt{\frac{\pi}{t}}$.
    \newpage
    
    \item Consideremos la función $F(x,t)=e^{-x^{2}}cos(tx)$. Dado $t_{0}\in\R$ vemos que la 
    función $F(x,t_{0})$ es continua y por ende medible, además, $\pdv{f}{t}$ existe para todo
    $(x,t)\in\R^{2}$. Para $t=0$ notamos que
    \begin{equation*}
        f(0)=\int_{0}^{\infty}e^{-x^{2}}=\frac{\sqrt{\pi}}{2}
    \end{equation*}
    y por lo tanto $f(x,0)$ es integrable. Por otro lado,
    \begin{equation*}
        \abs{\pdv{f}{t}(x,t)}=\abs{-xe^{-x^{2}}sen(tx)}\leq \abs{x}e^{-x^{2}}=:g
    \end{equation*}
    es claro que $g$ es una función integrable, basta notar que su integral impropia es finita.
    Luego, la función $f$ es diferenciable y como 
    $\abs{F(x,t)}\leq e^{-x^{2}}$ se tiene que
    \begin{align*}
        f'(t) &= \dv{}{t}\left(\int_{0}^{\infty}e^{-x^{2}}cos(tx)dx\right)
        =\dv{}{t}\left(\int_{(0,\infty)}e^{-x^{2}}cos(tx)d\lambda(x)\right)
        =\int_{(0,\infty)}\pdv{}{t}\left(e^{-x^{2}}cos(tx)\right)d\lambda(x) \\[2mm]
        &= \int_{(0,\infty)}-xe^{2}sen(tx)d\lambda(x)=\int_{0}^{\infty}-xe^{2}sen(tx)dx \\[2mm]
        &=\frac{e^{-x^{2}}}{2}sen(tx)\Big|_{x=0}^{x=\infty}
        -\frac{t}{2}\int_{0}^{\infty}e^{-x^{2}}cos(tx)dx=\frac{t}{2}f(t)
    \end{align*}
    Tenemos una EDO de variables separables $f'=-\frac{t}{2}f$ con condición inicial
    $f(0)=\frac{\sqrt{\pi}}{2}$. Luego
    \begin{equation*}
        \int\frac{df}{f}=\int-\frac{t}{2}dt
    \end{equation*}
    se sigue que
    \begin{equation*}
        log(f)=-\frac{t^{2}}{4}+C\hspace{4mm}\text{entonces}\hspace{4mm}f=e^{-\frac{t^{2}}{4}+C}
    \end{equation*}
    evaluando en la condición inicial vemos que
    \begin{equation*}
        f(t)=\frac{\sqrt{\pi}}{2}e^{-\frac{t^{2}}{4}}
    \end{equation*}
    para todo $t\in\R$.
    
    \item Sean $0<a<b<\infty$. Notemos que
    \begin{equation*}
        \frac{e^{-ax}-e^{-bx}}{x}=\int_{a}^{b}e^{-tx}dt
    \end{equation*}
    La función $e^{-tx}$ es continua y positiva, por tonelli vemos que
    \begin{align*}
        \int_{(0,\infty)\times(a,b)}e^{-tx}dtdx &= \int_{(a,b)}\int_{(0,\infty)}e^{-tx}dxdt
        =\int_{a}^{b}\int_{0}^{\infty}e^{-tx}dxdt=\int_{a}^{b}\int_{0}^{-\infty}-\frac{e^{u}}{t}
        dudt \\[2mm]
        &= \int_{a}^{b}-\frac{1}{t}e^{u}\Big|_{0}^{-\infty}dt=\int_{a}^{b}\frac{1}{t}dt
        =log\left(\frac{b}{a}\right)
    \end{align*}
    la segunda igualdad se debe a la continuidad y que la función $e^{-tx}$ es positiva. 
    Por otro lado tenemos que
    \begin{equation*}
        \int_{(0,\infty)\times(a,b)}e^{-tx}dtdx=\int_{(0,\infty)}\int_{(a,b)}e^{-tx}dtdx
        =\int_{0}^{\infty}\int_{a}^{b}e^{-tx}dtdx=h(a,b)
    \end{equation*}
    Por lo tanto $h(a,b)=log\left(\frac{b}{a}\right)$.
\end{enumerate}

\section*{Problema 2}
\begin{enumerate}
    \item 
    \item 
\end{enumerate}

\section*{Problema 3}
\noindent En primer lugar, notemos que
\begin{equation*}
    I=\int_{[0,1]^{2}}\frac{1}{1-xy}(\lambda\otimes\lambda)(dxdy)=\int_{[0,1]^{2}}
    \sum_{n\geq1}(xy)^{n-1}(\lambda\otimes\lambda)(dxdy)
\end{equation*}
como $xy$ es positiva en $[0,1]^{2}$, por convergencia monótona y tonelli se sigue que
\begin{equation*}
    I=\sum_{n\geq1}\int_{[0,1]^{2}}(xy)^{n-1}(\lambda\otimes\lambda)(dxdy)
    =\sum_{n\geq1}\int_{[0,1]}\int_{[0,1]}x^{n-1}y^{n-1}dxdy
    =\sum_{n\geq1}\left(\int_{[0,1]}x^{n-1}dx\right)\left(\int_{[0,1]}y^{n-1}dy\right)
\end{equation*}
la última igualdad se debe a la independencia entre las variables. Por continuidad, vemos que
\begin{equation*}
    I=\sum_{n\geq1}\left(\int_{0}^{1}x^{n-1}dx\right)\left(\int_{0}^{1}y^{n-1}dy\right)
    =\sum_{n\geq1}\frac{1}{n^{2}}
\end{equation*}
Por otro lado, consideramos el cambio de variable $x=u+v$ e $y=u-v$ con determinante jacobiano 
igual a $2$, luego la nueva región de integración $\rr$ esta determinado por
\begin{equation*}
    \rr:=\left\{(u,v)\in\R^{2}:0\leq u\leq\frac{1}{2},-u\leq v\leq u\right\}\cup
    \left\{(u,v)\in\R^{2}:\frac{1}{2}\leq u\leq1,1+u\leq v\leq 1-u\right\}
\end{equation*}
Así,
\begin{equation*}
    I=\int_{\rr}\frac{2}{1-(u+v)(u-v)}(\lambda\otimes\lambda)(dxdy)
    =2\int_{\rr}\frac{1}{1-u^{2}+v^{2}}(\lambda\otimes\lambda)(dxdy)
\end{equation*}
Diremos que $\rr_{+}:=\rr\cap\{v\geq0\}$. Por simetría respecto a $v$, vemos que
\begin{equation*}
    I=4\int_{\rr_{+}}\frac{1}{1-u^{2}+v^{2}}(\lambda\otimes\lambda)(dxdy)
    =4\int_{[0,1]\times (R_{+})^{2}(u)}\frac{1}{1-u^{2}+v^{2}}(\lambda\otimes\lambda)(dxdy)
\end{equation*}
donde $(R_{+})^{2}(u)$ es la sección de $R_{+}$ respecto de $u$. Notemos que en la última igualdad 
se utilizo que la función es continua y positiva. Así, por tonelli se sigue que
\begin{equation*}
    I=4\int_{[0,1]}\int_{(R_{+})^{2}(u)}\frac{1}{1-u^{2}+v^{2}}d\lambda(v)d\lambda(u)
    =4\int_{0}^{1}\int_{0}^{f(u)}\frac{1}{1-u^{2}+v^{2}}dvdu
\end{equation*}
con $f$ función que describe el contorno superior de $\rr_{+}$ en función de u. Luego
\begin{equation*}
    I=4\left(\int_{0}^{\frac{1}{2}}\int_{0}^{u}\frac{1}{1-u^{2}+v^{2}}dvdu
    +\int_{\frac{1}{2}}^{1}\int_{0}^{1-u}\frac{1}{1-u^{2}+v^{2}}dvdu\right)
\end{equation*}
consideramos $a=\sqrt{1-u^{2}}$ que esta bien definido pues $0\leq u\leq1$, así
\begin{align*}
    I &= 4\left(\int_{0}^{\frac{1}{2}}\int_{0}^{u}\frac{1}{a^{2}+v^{2}}dvdu
    +\int_{\frac{1}{2}}^{1}\int_{0}^{1-u}\frac{1}{a^{2}+v^{2}}dvdu\right) \\[2mm]
    &= 4\left(\int_{0}^{\frac{1}{2}}\frac{1}{a}arctan\left(\frac{v}{a}\right)\Big|_{0}^{u}du
    +\int_{\frac{1}{2}}^{1}\frac{1}{a}arctan\left(\frac{v}{a}\right)\Big|_{0}^{1-u}du
    \right) \\[2mm]
    &= 4\left(\int_{0}^{\frac{1}{2}}\frac{1}{\sqrt{1-u^{2}}}arctan\left(\frac{u}{\sqrt{1-u^{2}}}
    \right)du+\int_{\frac{1}{2}}^{1}\frac{1}{\sqrt{1-u^{2}}}arctan\left(\frac{1-u}{\sqrt{1-u^{2}}}
    \right)du\right)=4(I_{1}+I_{2})
\end{align*}
Para $I_{1}$ integramos por partes, en efecto
\begin{align*}
    I_{1} &= arcsen(u)arctan\left(\frac{u}{\sqrt{1-u^{2}}}\right)\Big|_{0}^{\frac{1}{2}}
    -\int_{0}^{\frac{1}{2}}arcsen(u)\frac{1}{1+\frac{u^{2}}{1-u^{2}}}\frac{1}{\sqrt{1-u^{2}}}
    \frac{1}{1-u^{2}}du \\[2mm]
    &=\frac{\pi}{6}\cdot\frac{\pi}{6}-\int_{0}^{\frac{1}{2}}arcsen(u)\frac{1}{\sqrt{1-u^{2}}}du
    =\frac{\pi^{2}}{36}-\int_{0}^{\frac{\pi}{6}}t\hspace{1mm}dt=\frac{\pi^{2}}{36}
    -\frac{t^{2}}{2}\Big|_{0}^{\frac{\pi}{6}}=\frac{\pi^{2}}{72}
\end{align*}
donde realizamos el cambio de variable $t=arcsen(u)$. Por otra parte, para $I_{2}$ consideramos el
cambio de variable $u=cos(\theta)$ y entonces
\begin{align*}
    I_{2} &= -\int_{\frac{\pi}{3}}^{0}\frac{1}{sen(\theta)}arctan\left(
    \frac{1-cos(\theta)}{sen(\theta)}\right)sen(\theta)\hspace{1mm}d\theta
    =\int_{0}^{\frac{\pi}{3}}arctan\left(tan\left(\frac{\theta}{2}\right)\right)
    \hspace{1mm}d\theta \\[2mm]
    &= 2\int_{0}^{\frac{\pi}{6}}arctan(tan(w))\hspace{1mm}dw
    =2\int_{0}^{\frac{\pi}{6}}w\hspace{1mm}dw=w^{2}\Big|_{0}^{\frac{\pi}{6}}=\frac{\pi^{2}}{36}
\end{align*}
Luego
\begin{equation*}
    I=4(I_{1}+I_{2})=\left(\frac{\pi^{2}}{72}+\frac{\pi^{2}}{36}\right)=4\cdot\frac{3\pi^{2}}{72}
    =\frac{\pi^{2}}{6}
\end{equation*}

\section*{Problema 4}

\section*{Problema 5}
\noindent Sea $I=(c,d)\subseteq[a,b]$ un intervalo, como $g$ es absolutamente continua, por TFC 
se sigue que
\begin{equation*}
    \lambda(g(I))=\lambda((g(c),g(d)))=g(d)-g(c)=\int_{(c,d)}g'd\lambda
\end{equation*}
además utilizamos el hecho de que $g$ es creciente y $g(I)$ un intervalo.
Sea $(I_{n})_{n}$ intervalos disjuntos de a pares contenidos en $[a,b]$, entonces como $g$ es 
estrictamente creciente, en particular es inyectiva, se sigue que $g(I_{n})$ son disjuntos de a 
pares. Notemos que
\begin{equation*}
    \lambda\left(g\left(\bigcup_{n\in\N}I_{n}\right)\right)
    =\lambda\left(\bigcup_{n\in\N}g(I_{n})\right)=\sum_{n\in\N}\lambda(g(I_{n}))
    =\sum_{n\in\N}\int_{I_{n}}g'd\lambda=\int\sum_{n\in\N}\I_{n}g'd\lambda
    =\int_{\bigcup_{n\in\N}}g'd\lambda
\end{equation*}
en donde para intercambiar la suma con la integral usamos teorema de convergencia dominada. Como
todo abierto se puede escribir como unión numerable de intervalos abiertos disjuntos de a pares,
lo anterior prueba el argumento para abiertos $U\subseteq[a,b]$.

\vspace{2mm}
\noindent Sea $G\in G_{\delta}$, existen $(U_{n})_{n}$ abiertos de $[a,b]$ tales que
\begin{equation*}
    G=\bigcap_{n\in\N}U_{n}
\end{equation*}
sin perdida de generalidad podemos suponer que estos abiertos estan encajonados. Como $[a,b]$ es
compacto entonces $g([a,b])$ es compacto, así
\begin{equation*}
    \lambda(g(G))=\lambda\left(g\left(\bigcap_{n\in\N}U_{n}\right)\right)
    =\lambda\left(\bigcap_{n\in\N}g(U_{n})\right)=\lim\limits_{n\to\infty}\lambda(g(U_{n}))
    =\lim\limits_{n\to\infty}\int_{U_{n}}g'd\lambda=\int_{G}g'd\lambda
\end{equation*}
donde la última igualdad se debe al teorema de convergencia dominada, notando que $\I_{U_{n}}$ 
converge a $\I_{G}$. Sea $E\subseteq[a,b]$ medible, existe $G\in G_{\delta}$ tal que 
$\lambda(G\setminus E)=0$ lo que implica que 

%($g(E)$ es medible para $E$ medible, g manda medida nula en medida nula)

%\printbibliography % Quitar el comentado si quiero usar bibliografia

\end{document}
