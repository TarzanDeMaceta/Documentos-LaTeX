\documentclass{article}
\usepackage{hyperref}
\usepackage{Style}

\nocite{*} % Comentar si quiero citar
%\addbibresource{bibliografia.bib} % Quitar el comentado si quiero usar bibliografia

\begin{document}

\begin{minipage}{2.5cm}
    \includegraphics[width=2cm]{imagen_puc.jpg}
\end{minipage}
\begin{minipage}{14cm}
    {\sc Pontificia Universidad Católica de Chile\\
    Facultad de Matemáticas\\
    Departamento de Matemática\\
    Profesor: Gregorio Moreno -- Estudiante: Benjamín Mateluna}
\end{minipage}
\vspace{1ex}

{\centerline{\bf Teoría de Integración - MAT2534}
\centerline{\bf Tarea 2}}
\centerline{\bf 09 de Mayo de 2025}

\section*{Problema 1}
\section*{Problema 2}
\begin{enumerate}
    \item A
    \item Sea $(s_{n})_{n}$ una sucesión de funciones simples positivas tales que 
    $s_{n}\uparrow f$, por la parte anterior, tenemos que
    \begin{equation*}
        \int s_{n}\hspace{1mm}d\mu\leq\lim\limits_{n\to\infty}\int \inf_{k\geq n}
        f_{k}\hspace{1mm}d\mu\leq\liminf\limits_{n\to\infty}\int f_{n}\hspace{1mm}d\mu
    \end{equation*}
    lo que implica que
    \begin{equation*}
        \int \liminf\limits_{n\to\infty}f_{n}\hspace{1mm}d\mu=\int f\hspace{1mm}d\mu\leq
        \liminf\limits_{n\to\infty}\int f_{n}\hspace{1mm}d\mu
    \end{equation*}
    se tiene el lema de fatou.

    \item Sean $f,f_{n}:\Omega\to[0,\infty]$ funciones medibles tales que $f_{n}\leq f_{n+1}$ y
    $f_{n}\to f$ $\mu$-ctp. Por monotonía, sabemos que
    \begin{equation*}
        \lim\limits_{n\to\infty}\int f_{n}\hspace{1mm}d\mu\leq\int f\hspace{1mm}d\mu
    \end{equation*}
    por otro lado, por la parte b), tenemos que
    \begin{align*}
        \int f\hspace{1mm}d\mu &= \int \lim\limits_{n\to\infty}f_{n}\hspace{1mm}d\mu
        =\int \liminf\limits_{n\to\infty}f_{n}\hspace{1mm}d\mu
        \leq\liminf\limits_{n\to\infty}\int f_{n}\hspace{1mm}d\mu \\
        &= \lim\limits_{n\to\infty}\int f_{n}\hspace{1mm}d\mu
    \end{align*}
    lo que prueba el teorema de convergencia monótona.
\end{enumerate}

\section*{Problema 3}
\noindent Sean $f,f_{n}:\Omega\to\overline{\R}$ funciones mediles y $g\in L^{1}$ tales que $f_{n}\to f$ 
$\mu$-ctp y $\abs{f_{n}}\leq g$. Notemos que $\abs{f_{n}}\to\abs{f}$ $\mu$-ctp, luego 
$\abs{f}\leq g$ $\mu$-ctp. Dado $m\in\N$, por teorema de Egoroff, existe $\Omega_{m}^{*}\in
\mathcal{F}$ tal que
\begin{equation*}
    \mu((\Omega_{m}^{*})^{c})<\frac{1}{m}
\end{equation*}
y $f_{n}$ converge uniformemente a $f$ en $\Omega_{m}^{*}$ para todo $m\in\N$. 
Definimos la sucesión
\begin{equation*}
    \Omega_{m}:=\bigcup_{i=1}^{m}\Omega_{i}^{*}
\end{equation*}
Notemos que $\Omega_{m}^{c}\supseteq\Omega_{m+1}^{c}$, $\mu(\Omega_{m}^{c})<\frac{1}{m}$ para todo 
$m\in\N$ y $\mu(\Omega_{1}^{c})\leq\mu(\Omega)<\infty$. Además, la convergencia en $\Omega_{m}$ 
sigue siendo uniforme, basta tomar máximo. Dado $m\in\N$, se tiene lo siguiente
\begin{align*}
    \int\abs{f_{n}-f}\hspace{1mm}d\mu &= \int_{\Omega_{m}}\abs{f_{n}-f}\hspace{1mm}d\mu+
    \int_{\Omega_{m}^{c}}\abs{f_{n}-f}\hspace{1mm}d\mu \\
    &\leq \sup_{x\in\Omega_{m}}\abs{f_{n}-f}\mu(\Omega_{m})
    +\int_{\Omega_{m}^{c}}\abs{f_{n}-f}\hspace{1mm}d\mu \\
    &\leq \sup_{x\in\Omega_{m}}\abs{f_{n}-f}\mu(\Omega)
    +2\int_{\Omega_{m}^{c}}g\hspace{1mm}d\mu
\end{align*}
Como $g$ es positiva, consideramos la medida $\varphi:\Omega\to\overline{\R}$ sobre el espacio 
$(\Omega, \mathcal{F})$ dada por
\begin{equation*}
    \varphi(E)=\int_{E}g\hspace{1mm}d\mu\hspace{4mm}\text{ para }E\in\mathcal{F}
\end{equation*}
que es finita, pues $g\in L^{1}$. Luego
\begin{equation*}
    \mu\left(\bigcap_{m\in\N}\Omega_{m}^{c}\right)=\lim\limits_{n\to\infty}\mu(\Omega_{m}^{c})
    \hspace{4mm}\text{ y }\hspace{4mm}
    \varphi\left(\bigcap_{m\in\N}\Omega_{m}^{c}\right)=\lim\limits_{n\to\infty}
    \varphi(\Omega_{m}^{c})
\end{equation*}
Por lo anterior mencionado, vemos que
\begin{equation*}
    \mu\left(\bigcap_{m\in\N}\Omega_{m}^{c}\right)=\lim\limits_{m\to\infty}\mu(\Omega_{m}^{c})=0
\end{equation*}
lo que implica que
\begin{equation*}
    \varphi\left(\bigcap_{m\in\N}\Omega_{m}^{c}\right)=0
\end{equation*}
De este modo, para todo $m\in\N$ se sigue que
\begin{equation*}
    \lim\limits_{n\to\infty}\int\abs{f_{n}-f}\hspace{1mm}d\mu
    \leq \lim\limits_{n\to\infty}\sup_{x\in\Omega_{m}}\abs{f_{n}-f}\mu(\Omega)
    +2\lim\limits_{n\to\infty}\varphi(\Omega_{m}^{c})=2\varphi(\Omega_{m}^{c})
\end{equation*}
entonces
\begin{equation*}
    \lim\limits_{n\to\infty}\int\abs{f_{n}-f}\hspace{1mm}d\mu
    =\lim\limits_{m\to\infty}\lim\limits_{n\to\infty}\int\abs{f_{n}-f}\hspace{1mm}d\mu
    \leq2\lim\limits_{m\to\infty}\varphi(\Omega_{m}^{c})=0
\end{equation*}
En particular, usando desigualdad triangular, se tiene que
\begin{equation*}
    \lim\limits_{n\to\infty}\int f_{n}\hspace{1mm}d\mu=\int f\hspace{1mm}d\mu
\end{equation*}

\section*{Problema 4}
\begin{enumerate}
    \item Notemos que $\mathbbm{1}_{X}=\textbf{1}\in\V$, entonces $X\in\mathcal{G}$. Sea 
    $E\in\mathcal{G}$, luego $\I_{E}\in\mathcal{G}$, como $\V$ es espacio vectorial, vemos que
    \begin{equation*}
        \I_{E^{c}}=\textbf{1}-\I_{E}\in\V
    \end{equation*}
    se sigue que $E^{c}\in\mathcal{G}$. En primer lugar veremos que si $A,B\in\mathcal{G}$ entonces
    $A\cup B\in\mathcal{G}$. En efecto, veamos que $\I_{A\cup B}=\max\{\I_{A},\I_{B}\}\in\V$,
    inductivamente se tiene que $\mathcal{G}$ es cerrado bajo uniones finitas. Sea $(A_{n})_{n}
    \subseteq\mathcal{G}$, definimos la sucesión
    \begin{equation*}
        E_{n}:=\bigcup_{i=1}^{n}A_{i}\in\mathcal{G}
    \end{equation*}
    notemos que $E_{n}\subseteq E_{n+1}$ lo que implica que $\I_{E_{n}}\leq\I_{E_{n+1}}\leq1$ para
    todo $n\in\N$, entonces
    \begin{equation*}
        \I_{\bigcup_{n\in\N}A_{n}}=\I_{\bigcup_{n\in\N}E_{n}}=\sup_{n}\I_{E_{n}}\in\V
    \end{equation*}
    Concluimos que $\mathcal{G}$ es una $\sigma$-álgebra.

    \item Como lema, veamos que dadas $f,g\in\V$ se tiene que $\min\{f,g\}=-\max\{-f,-g\}\in\V$. 
    Sea $f\in\V$, basta ver que $A:=f^{-1}((a,\infty))\in\mathcal{G}$ para todo $a\in\R$, es decir,
    $\I_{A}\in\V$. Definimos
    \begin{equation*}
        f_{n}:=\min\{\textbf{1},\max\{\textbf{0},n(f-\textbf{a})\}\}\in\V
    \end{equation*}
    donde $\textbf{0}=\I_{\emptyset}$ y $\textbf{a}=a\textbf{1}$, ambos en $\V$. Notemos que 
    $f_{n}\leq1$ para todo $n\in\N$. Afirmamos que $f_{n}\leq f_{n+1}$ para todo $n\in\N$, tenemos
    dos casos
    \begin{itemize}
        \item $x\in A$, entonces $f(x)-\textbf{a}>0$ y luego $n(f(x)-\textbf{a})\leq(n+1)
        (f(x)-\textbf{a})$ lo que implica que $f_{n}(x)\leq f_{n+1}(x)$.

        \item $x\not\in A$, entonces $f_{n}(x)=0$.
    \end{itemize}
    por lo tanto $\sup_{n}f_{n}\in\V$. Afirmamos que $\I_{A}=\sup_{n}f_{n}$, en efecto, sea 
    $x\in A$, por propiedad arquimediana, existe $n\in\N$ tal que $1\leq n(f(x)-\textbf{a})$ y por
    ende $f_{n}(x)=1$, se sigue que $\sup_{n}f_{n}(x)=1$. Concluimos que $f$ es $\mathcal{G}$-
    medible.

    \item Definimos $\mu:X\to[0,\infty]$ sobre el espacio $(X,\mathcal{G})$, dada por 
    $\mu(E):=I(\I_{E})$ con $E\in\mathcal{G}$. Afirmamos que $\mu$ es una medida, en efecto, como
    $I(f)\geq0$ para toda $f\in\V$ con $f\geq0$, entonces dado $E\in\mathcal{G}$ vemos que 
    $\mu(E)=I(\I_{E})\geq0$. Por otro lado, $\mu(\emptyset)=I(\I_{\emptyset})=I(\textbf{0})=0$,
    ya que $I$ es lineal.
    
    \vspace{4mm}
    \noindent Sea $(A_{n})_{n}\subseteq\mathcal{G}$ disjuntos de a pares, entonces
    \begin{equation*}
        \sum_{n\in\N}\I_{A_{n}}=\I_{\bigcup_{n\in\N}A_{n}}
    \end{equation*}
    de este modo
    \begin{align*}
        \mu\left(\bigcup_{n\in\N}A_{n}\right) &= I\left(\I_{\bigcup_{n\in\N}A_{n}}\right)
        =I\left(\sum_{n\in\N}\I_{A_{n}}\right)=I\left(\sum_{i=1}^{n}\I_{A_{i}}\right)+
        I\left(\sum_{m=n+1}^{\infty}\I_{A_{m}}\right) \\
        &= \sum_{i=1}^{n}I(\I_{A_{i}})+I\left(\sum_{m=n+1}^{\infty}\I_{A_{m}}\right)
    \end{align*}
    Consideremos la sucesión $f_{n}=\sum_{m=n+1}^{\infty}\I_{A_{m}}$, luego $f_{n}\geq f_{n+1}$
    y como $\sum_{n\in\N}\I_{A_{n}}=\I_{\bigcup_{n\in\N}A_{n}}$, se sigue que 
    $\lim\limits_{n\to\infty}f_{n}=0$, por lo tanto $\lim\limits_{n\to\infty}I(f_{n})=0$, así
    \begin{equation*}
        \mu\left(\bigcup_{n\in\N}A_{n}\right)=\lim\limits_{n\to\infty}
        \sum_{i=1}^{n}I(\I_{A_{i}})+I(f_{n})=\sum_{n\in\N}I(\I_{A_{n}})=\sum_{n\in\N}\mu(A_{n})
    \end{equation*}
    Por otro lado, $\mu$ es de medida finita, en efecto, $\mu(X)=I(\textbf{1})\in\R$. En 
    particular, $\mu$ es una medida sobre el álgebra $\mathcal{G}$, por teorema de Caratheodory, 
    $\mu^{*}\big|_{\mathcal{F}}$ es la unica extensión de $\mu$ a $\sigma(G)$. Sin embargo, 
    $\sigma(G)=G$ ya que $G$ es $\sigma$-álgebra y por ende $\mu^{*}\big|_{\mathcal{F}}=\mu$.
    
    \vspace{4mm}
    \noindent Falta ver que dada $f\in\V$ se tiene que
    \begin{equation*}
        I(f)=\int f\hspace{1mm}d\mu
    \end{equation*}
    Supongamos que $f$ es simple y sea $\sum a_{i}\I_{A_{i}}$ una representación, vemos que
    \begin{equation*}
        \int f\hspace{1mm}d\mu=\sum a_{i}\mu(A_{i})=\sum a_{i}I(\I_{A_{i}})
        =I\left(\sum a_{i}\I_{A_{i}}\right)=I(f)
    \end{equation*}
    Sea $f\in\V$ tal que $f\geq0$, sea $(s_{n})_{n}$ una sucesión de funciones simples positivas
    tales que $s_{n}\uparrow f$, de este modo la sucesión $g_{n}=f-s_{n}$ es decreciente y 
    $g_{n}\to0$, entonces $I(g_{n})\to0$, se sigue que
    \begin{equation*}
        I(f)=\lim\limits_{n\to\infty}I(s_{n})=\lim\limits_{n\to\infty}\int s_{n}\hspace{1mm}d\mu
        =\int f\hspace{1mm}d\mu
    \end{equation*}
    Sea $f\in\V$, luego
    \begin{equation*}
        \int f\hspace{1mm}d\mu=\int f_{+}\hspace{1mm}d\mu-\int f_{-}\hspace{1mm}d\mu
        =I(f_{+})-I(f_{-})=I(f)
    \end{equation*}
\end{enumerate}

%\printbibliography % Quitar el comentado si quiero usar bibliografia

\end{document}
