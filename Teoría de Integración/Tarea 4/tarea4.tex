\documentclass{article}
\usepackage{hyperref}
\usepackage{Style}

\nocite{*} % Comentar si quiero citar
%\addbibresource{bibliografia.bib} % Quitar el comentado si quiero usar bibliografia

\begin{document}

\begin{minipage}{2.5cm}
    \includegraphics[width=2cm]{imagen_puc.jpg}
\end{minipage}
\begin{minipage}{14cm}
    {\sc Pontificia Universidad Católica de Chile\\
    Facultad de Matemáticas\\
    Departamento de Matemática\\
    Profesor: Gregorio Moreno -- Estudiante: Benjamín Mateluna}
\end{minipage}
\vspace{1ex}

{\centerline{\bf Teoría de Integración - MAT2534}
\centerline{\bf Tarea 4}}
\centerline{\bf 25 de junio de 2025}

\section*{Problema 1}
\begin{enumerate}
    \item Sea $f\in H$, como $V$ es un subespacio vectorial cerrado, existen únicos $f_{V}\in V$ y 
    $f_{\perp}\in V^{\perp}$ tales que $f=f_{V}+f_{\perp}$. Luego, dada $h\in V$, se tiene que
    \begin{equation*}
        \int f\overline{h}\hspace{1mm}d\mu=\int (f_{V}+f_{\perp})\overline{h}\hspace{1mm}d\mu
        =\int f_{V}\overline{h}\hspace{1mm}d\mu+\int f_{\perp}\overline{h}\hspace{1mm}d\mu
    \end{equation*}
    y como $f_{\perp}\in V^{\perp}$ vemos que
    \begin{equation*}
        \int f\overline{h}\hspace{1mm}d\mu=\int f_{V}\overline{h}\hspace{1mm}d\mu
    \end{equation*}
    tomando $g=f_{V}$ se tiene lo pedido.

    \item Afirmamos que
    \begin{equation*}
        \mathcal{G}=\sigma(E_{1},\cdots,E_{n})=\left\{\bigcup_{i\in I}E_{i}:I\subseteq
        \{1,\cdots,n\}\right\}=:\Sigma
    \end{equation*}
    es directo que $\Sigma\subseteq\mathcal{G}$. Basta probar que $\Sigma$ es $\sigma$-álgebra, 
    claramente $\emptyset\in\Sigma$. Sea $E\in\Sigma$, como los $E_{i}$ cubren $\Omega$ y son 
    disjuntos de a pares, vemos que
    \begin{equation*}
        E^{c}=\left(\bigcup_{i\in I}E_{i}\right)^{c}=\bigcup_{i\in I^{c}}E_{i}\in\Sigma
        \hspace{4mm}\text{con }I\subseteq\{1,\cdots,n\}
    \end{equation*}
    Por otro lado, como $\#\Sigma=2^{n}$, la unión numerable de elementos en $\Sigma$ sigue 
    estando en $\Sigma$ pues en realidad es una unión finita de los conjuntos $E_{i}$. 
    
    Sabemos que suma de medibles y poderación de medibles es medible y por lo tanto $M$ es un
    subespacio vectorial de $H$. Queda probar que es cerrado. Vamos a probar que dada $f\in M$ se 
    puede escribir como
    \begin{equation*}
        f=\sum_{i=1}^{n}a_{i}\I_{E_{i}}\hspace{4mm}\text{donde }a_{i}\in\C
    \end{equation*}
    Supongamos que $f\not\equiv0$, sea $a\in\C\setminus\{0\}$ tal que $f^{-1}(\{a\})
    \neq\emptyset$, en particular, se tiene que $f^{-1}(\{a\})\in\mathcal{G}$, entonces
    \begin{equation*}
        f^{-1}(\{a\})=\bigcup_{i\in I}E_{i}\hspace{4mm}\text{con }I\subseteq\{1,\cdots,n\}
    \end{equation*}
    Sea $b\neq a$, entonces si $b\in f(\Omega)$
    \begin{equation*}
        f^{-1}(\{b\})=\bigcup_{j\in J}E_{j}\hspace{4mm}\text{con }J\subseteq\{1,\cdots,n\}
    \end{equation*}
    Como $f^{-1}(\{b\})\cap f^{-1}(\{a\})=f^{-1}(\{a\}\cap\{b\})=\emptyset$, concluimos que 
    $I\cap J=\emptyset$, así, $f$ asume finitos valores, puesto que la colección 
    $(E_{i})_{i=1}^{n}$ es finita, lo que prueba la afirmación.

    Sean $f,g\in M$, observemos que
    \begin{equation*}
        f=\sum_{i=1}^{n}f_{i}\I_{E_{i}}\hspace{4mm}\text{y}\hspace{4mm}
        g=\sum_{i=1}^{n}g_{i}\I_{E_{i}}
        \hspace{4mm}\text{con }f_{i},g_{i}\in\C
    \end{equation*}
    luego, como los $E_{i}$ son disjuntos de a pares
    \begin{equation*}
        \norm{f-g}_{L^{2}}^{2}=\int \abs{f-g}^{2}\hspace{1mm}d\mu
        =\int\abs{\sum_{i=1}^{n}(f_{i}-g_{i})\I_{E_{i}}}^{2}\hspace{1mm}d\mu
        =\int\sum_{i=1}^{n}\abs{f_{i}-g_{i}}^{2}\I_{E_{i}}\hspace{1mm}d\mu
        =\sum_{i=1}^{n}\abs{f_{i}-g_{i}}^{2}\mu(E_{i})
    \end{equation*}

    Sea $(f_{m})_{m\in\N}\subseteq M$ tal que $f_{m}\to f\in L^{2}$. En particular, la sucesión 
    $f_{m}$ es de cauchy, entonces por lo anterior mencionado, la sucesión $(a_{m})_{m\in\N}
    \subseteq\C^{n}$ donde $a_{m}:=(a_{m,1},\cdots,a_{m,n})$ y
    \begin{equation*}
        f_{m}=\sum_{i=1}^{n}a_{m,i}\I_{E_{i}}\hspace{4mm}\text{es una sucesión de cauchy.}
    \end{equation*}
    Así, existe $a=(a_{1},\cdots,a_{n})\in\C^{n}$ tal que $a_{m}\to a$. Consideramos
    \begin{equation*}
        s=\sum_{i=1}^{n}a_{i}\I_{E_{i}}
    \end{equation*}
    entonces, por cauchy schwarz
    \begin{equation*}
        \norm{s-f_{m}}_{L^{2}}^{2}=\int\abs{s-f_{m}}^{2}\hspace{1mm}d\mu
        \leq\int\sum_{i=1}^{n}\abs{s_{i}-f_{m,i}}^{2}\I_{E_{i}}\hspace{1mm}d\mu
        \leq\sum_{i=1}^{n}\abs{s_{i}-f_{m,i}}^{2}\mu(\Omega)=\sum_{i=1}^{n}\abs{s_{i}-f_{m,i}}^{2}
    \end{equation*}
    por ende $f_{m}\to s$ en $L^{2}$, por unicidad del límite, concluimos que $f=s\in M$.
    
    \item Recordemos que $L^2$ viene dotado de un producto interno, que denotaremos por 
    $\ip{\cdot}{\cdot}_{L^{2}}$ que induce la norma $\norm{\cdot}_{L^{2}}$. Por la parte anterior 
    sabemos que $M$ es subespacio vectorial cerrado, así por la parte a) existe una única $g\in V$ 
    tal que
    \begin{equation*}
        \int f\overline{h}\hspace{1mm}d\mu=\int g\overline{h}\hspace{1mm}d\mu
    \end{equation*}
    y $g$ resulta ser la proyección ortogonal de $f$ en $V$. Luego
    \begin{equation*}
        g=\sum_{i=1}^{n} a_{i}\I_{E_{i}}
    \end{equation*}
    y además $\ip{f-g}{h}=0$ para todo $h\in V$. En particular, si tomamos $h=\I_{E_{i}}$, 
    resulta que
    \begin{equation*}
        \int_{E_{i}}f-g\hspace{1mm}d\mu=0
        \hspace{4mm}\text{y entonces}\hspace{4mm}
        \int_{E_{i}}f\hspace{1mm}d\mu=\int_{E_{i}}g\hspace{1mm}d\mu=c_{i}\mu(E_{i})
    \end{equation*}
    en otras palabras
    \begin{equation*}
        c_{i}=\frac{1}{\mu(E_{i})}\int_{E_{i}}f\hspace{1mm}d\mu
        =\frac{\ip{f}{\I_{E_{i}}}_{L^{2}}}{\mu(E_{i})}
    \end{equation*}
    conluimos que
    \begin{equation*}
        g=\sum_{i=1}^{n}\frac{\ip{f}{\I_{E_{i}}}_{L^{2}}}{\mu(E_{i})}\I_{E_{i}}
    \end{equation*}

\end{enumerate}

\section*{Problema 2}
\begin{enumerate}
    \item Como $u\in\cc_{0}^{\infty}[a,b]$ satisface $-u''+u=f$, entonces dada $v\in
    \cc_{0}^{\infty}[a,b]$ se tiene que $-u''v+uv=fv$, por la desigualdad de Hölder, vemos que 
    $fv$ es integrable, además $-u''v+uv\in\cc_{0}^{\infty}[a,b]$. Integrando a ambos lados y
    usando integración por partes tenemos que
    \begin{align*}
        \int_{[a,b]}fv\hspace{1mm}d\lambda &= \int_{[a,b]}-u''v+uv\hspace{1mm}d\lambda
        =-\int_{[a,b]}u''v\hspace{1mm}d\lambda+\int_{[a,b]}uv\hspace{1mm}d\lambda \\
        &= -u'v\Big|_{a}^{b}+\int_{[a,b]}u'v'\hspace{1mm}d\lambda
        +\int_{[a,b]}uv\hspace{1mm}d\lambda \\
        &= \int_{[a,b]}(uv+u'v')\hspace{1mm}d\lambda
    \end{align*}
    donde la tercera igualdad se debe a que $supp(v)\subseteq(a,b)$.

    \item Sea $[c,d]\subseteq[a,b]$, con $a<c<d<b$. Entonces, existe una sucesión acotada 
    $(v_{n})_{n}\subseteq\cc_{0}^{\infty}[a,b]$ tal que $v_{n}\xrightarrow[n\to\infty]{}
    \I_{[c,d]}$ para $x\in[a,b]$. Por otro lado, integrando por partes, observamos que
    \begin{align*}
        \int_{[a,b]}fv_{n}\hspace{1mm}d\lambda &= \int_{[a,b]}(uv_{n}+u'v_{n}')\hspace{1mm}d\lambda
        =\int_{[a,b]}u'v_{n}'\hspace{1mm}d\lambda+\int_{[a,b]}uv_{n}\hspace{1mm}d\lambda \\
        &= \int_{[a,b]}-u''v_{n}+uv_{n}\hspace{1mm}d\lambda
    \end{align*}
    Existe $M>0$ tal que $v_{n}\leq M$ para todo $n\in\N$, entonces $\abs{v_{n}f}\leq M\abs{f}$ y 
    además $v_{n}f$ converge puntualmente a $\I_{[c,d]}f$, por el mismo argumento se tiene un 
    resultado similar para $\I_{[c,d]}(-u''+u)$. De este modo, como $f,u\in\cc_{0}^{\infty}[a,b]$, 
    por teorema de convergencia dominada se sigue que
    \begin{equation*}
        \int_{[c,d]}f\hspace{1mm}d\lambda=\lim\limits_{n\to\infty}
        \int_{[a,b]}fv_{n}\hspace{1mm}d\lambda=\lim\limits_{n\to\infty}
        \int_{[a,b]}-u''v_{n}+uv_{n}\hspace{1mm}d\lambda=\int_{[c,d]}-u''+u\hspace{1mm}d\lambda
    \end{equation*}
    como esto es para $c,d$ arbitrarios, concluimos que $f=-u''+u$ $\lambda-ctp$. 
    
    Afirmamos que esta igualdad se cumple para todo $x\in(a,b)$. Supongamos, por contradicción, 
    que existe $x\in(a,b)$ tal que $f(x)\neq-u''(x)+u(x)$, como $f,u$ y $u''$ son continuas, 
    existe un intervalo abierto $I\subseteq[a,b]$ tal que $f(x)\neq-u''(x)+u(x)$ para todo 
    $x\in I$. Esto contradice que la igualdad sea en casi todas partes.

    \item Sea $u\in H$, entonces existe $(u_{n})_{n\in\N}\subseteq \cc_{0}^{\infty}[a,b]$ tal que
    $u_{n}\to u$ con la norma $\norm{\cdot}_{H}$. En particular, la sucesión $u_{n}$ es de cauchy,
    además, se tiene que
    \begin{equation*}
        \norm{u_{n}}_{H}^{2}=\int\abs{u_{n}}^{2}+\abs{u_{n}'}^{2}d\lambda
        \geq\int\abs{u_{n}'}^{2}d\lambda=\norm{u_{n}'}_{L^{2}}^{2}
    \end{equation*}
    entonces, la sucesión $v_{n}:=u_{n}'\in\cc_{0}^{\infty}[a,b]$ es de cauchy según la norma en 
    $L^{2}$. Como este espacio es un espacio métrico completo, existe $v\in L_{0}^{2}$ tal que
    \begin{equation*}
        \lim\limits_{n\to\infty}\int\abs{v_{n}-v}^{2}d\lambda=0
    \end{equation*}
    Afirmamos que $v$ es la función buscada. Sea $w\in\cc_{0}^{\infty}[a,b]$, entonces
    \begin{equation*}
        \int v_{n}w\hspace{1mm}d\lambda=\int u_{n}'w\hspace{1mm}d\lambda=u_{n}w\Big|_{a}^{b}
        -\int u_{n}w'\hspace{1mm}d\lambda=-\int u_{n}w'\hspace{1mm}d\lambda
    \end{equation*}
    Basta demostrar que $v_{n}w$ y $u_{n}w'$ convergen a $vw$ y $uw'$ en $L^{1}$ respectivamente.
    En efecto, por hölder tenemos que
    \begin{equation*}
        \int\abs{v_{n}w-vw}\hspace{1mm}d\lambda=\int\abs{v_{n}-v}\abs{w}\hspace{1mm}d\lambda
        \leq\norm{v_{n}-v}_{2}\norm{w}_{2}
    \end{equation*}
    y por otro lado
    \begin{equation*}
        \int\abs{u_{n}w'-uw'}\hspace{1mm}d\lambda=\int\abs{u_{n}-u}\abs{w'}\hspace{1mm}d\lambda
        \leq\norm{u_{n}-u}_{2}\norm{w'}_{2}\leq\norm{u_{n}-u}_{H}\norm{w'}_{2}
    \end{equation*}
    como $w\in\cc_{0}^{\infty}[a,b]$ vemos que $\norm{w}_{2},\norm{w'}_{2}<\infty$ y usando la 
    convergencia en $H$ y $L^{2}$ se tiene el resultado. Resta ver que $v$ es única. Sea 
    $\widehat{v}\in L_{0}^{2}$ tal que
    \begin{equation*}
        \int \widehat{v}w\hspace{1mm}d\lambda=-\int uw'\hspace{1mm}d\lambda
    \end{equation*}
    para toda $w\in\cc_{0}^{\infty}[a,b]$. Entonces para toda $w\in\cc_{0}^{\infty}[a,b]$, se 
    tiene que
    \begin{equation*}
        \int vw\hspace{1mm}d\lambda=\int\widehat{v}w\hspace{1mm}d\lambda
        \hspace{4mm}\text{lo que implica que}\hspace{4mm}
        \int(v-\widehat{v})w\hspace{1mm}d\lambda=0
    \end{equation*}
    por el mismo argumento que antes, para todo $c<d$ con $c,d\in(a,b)$ se tiene que
    \begin{equation*}
        \int_{[c,d]}v-\widehat{v}\hspace{1mm}d\lambda=0
    \end{equation*}
    concluimos que $v=\widehat{v}$ $\lambda-ctp$, lo que prueba la unicidad en $L^{2}$.
    
    \item Sean $u,w\in H$, existe $(w_{n})_{n\in\N}\in\cc_{0}^{\infty}[a,b]$ tal que $w_{n}\to w$ 
    según la norma $\norm{\cdot}_{H}$. Por la parte anterior, tenemos que
    \begin{equation*}
        \int u'w_{n}\hspace{1mm}d\lambda=-\int uw_{n}'\hspace{1mm}d\lambda
    \end{equation*}
    Por el mismo argumento que antes, sabemos que $\norm{u'w_{n}-u'w}_{1}\leq\norm{w_{n}-w}_{H}
    \norm{u'}_{2}$ y además $\norm{uw_{n}'-uw'}_{1}\leq\norm{u}_{2}\norm{w_{n}'-w'}_{2}$ y por lo 
    tanto
    \begin{equation*}
        \int u'w\hspace{1mm}d\lambda=\lim\limits_{n\to\infty}\int u'w_{n}\hspace{1mm}d\lambda
        =\lim\limits_{n\to\infty}-\int uw_{n}'\hspace{1mm}d\lambda=\int uw'\hspace{1mm}d\lambda
    \end{equation*}
    
    \item Como resultado previo, vamos a demostrar la linealidad de la derivada débil. Sean 
    $u,v\in H$ y $\alpha\in\R$, existen $u_{n},v_{n}\in\cc_{0}^{\infty}[a,b]$ sucesiones tales que
    $u_{n}\to u$ y $v_{n}\to v$ en $H$. Dada $w\in\cc_{0}^{\infty}[a,b]$ se sigue que
    \begin{equation*}
        -\int(u_{n}+\alpha v_{n})w'\hspace{1mm}d\lambda
        =\int(u_{n}+\alpha v_{n})'w\hspace{1mm}d\lambda
        =\int u_{n}'w\hspace{1mm}d\lambda+\alpha\int v_{n}'w\hspace{1mm}d\lambda
    \end{equation*}
    haciendo $n\to\infty$ y usando linealidad de la integral, por unicidad, concluimos que 
    $(u+\alpha v)'=u'+\alpha v'$.
    
    En $H$ definimos el siguiente producto interno
    \begin{equation*}
        \ip{u}{v}_{H}=\int uv+u'v'\hspace{1mm}d\lambda
        \hspace{4mm}\text{para }u,v\in H
    \end{equation*}
    debemos probar que en efecto, es un producto interno. En primer lugar notemos que 
    $\ip{u}{u}=\norm{u}_{H}^{2}$ y como esta última es norma, se tiene que $\ip{u}{u}_{H}\geq0$ 
    para todo $u\in H$ y la igualdad se cumple si y solo si $u=0$. Por otro lado, es claro que
    $\ip{u}{v}_{H}=\ip{v}{u}_{H}$ y además
    \begin{equation*}
        \ip{u+\alpha w}{v}=\int(u+\alpha w)v+(u+\alpha w)'v'\hspace{1mm}d\lambda
        =\int uv+u'v'\hspace{1mm}d\lambda+\alpha\int wv+w'v'\hspace{1mm}d\lambda
        =\ip{u}{v}+\alpha\ip{w}{v}
    \end{equation*}
    Donde usamos la linealidad de la derivada débil. Así, $H$ es un espacio de Hilbert, pues con 
    la norma $\norm{\cdot}_{H}$ es un espacio métrico completo. Sea $f\in L^{2}$. Definimos el 
    funcional $I:H\to\R$ dado por
    \begin{equation*}
        I(v):=\int fv\hspace{1mm}d\lambda
    \end{equation*}
    es claro que $I$ es lineal, veamos que es continuo. Por hölder notamos que
    \begin{equation*}
        I(v)=\int fv\hspace{1mm}d\lambda\leq\norm{f}_{L^{2}}\norm{v}_{L^{2}}
        \leq\norm{f}_{L^{2}}\norm{v}_{H}
    \end{equation*}
    y como $I$ es acotado, concluimos que es continuo. De este modo, por teorema de representación
    de riesz, existe un único $u\in H$ tal que
    \begin{equation*}
        \int (uv+u'v')\hspace{1mm}d\lambda=\ip{u}{v}_{H}=I(v)=\int fv\hspace{1mm}d\lambda
    \end{equation*}
\end{enumerate}

%\printbibliography % Quitar el comentado si quiero usar bibliografia

\end{document}
