\documentclass{article}
\usepackage{hyperref}
\usepackage{Style}

\nocite{*} % Comentar si quiero citar
%\addbibresource{bibliografia.bib} % Quitar el comentado si quiero usar bibliografia

\begin{document}

\begin{minipage}{2.5cm}
    \includegraphics[width=2cm]{imagen_puc.jpg}
\end{minipage}
\begin{minipage}{14cm}
    {\sc Pontificia Universidad Católica de Chile\\
    Facultad de Matemáticas\\
    Departamento de Matemática\\
    Profesor: Gregorio Moreno -- Estudiante: Benjamín Mateluna}
\end{minipage}
\vspace{1ex}

{\centerline{\bf Teoría de Integración - MAT2534}
\centerline{\bf Tarea 4}}
\centerline{\bf 25 de junio de 2025}

\section*{Problema 1}
\begin{enumerate}
    \item Sea $f\in H$, como $V$ es un subespacio cerrado, existen únicos $f_{V}\in V$ y 
    $f_{\perp}\in V^{\perp}$ tales que $f=f_{V}+f_{\perp}$. Luego, dada $g\in V$, se tiene que
    \begin{equation*}
        \int f\overline{h}\hspace{1mm}d\mu=\int (f_{V}+f_{\perp})\overline{h}\hspace{1mm}d\mu
        =\int f_{V}\overline{h}\hspace{1mm}d\mu+\int f_{\perp}\overline{h}\hspace{1mm}d\mu
    \end{equation*}
    y como $f_{\perp}\in V^{\perp}$ vemos que
    \begin{equation*}
        \int f\overline{h}\hspace{1mm}d\mu=\int f_{\perp}\overline{h}\hspace{1mm}d\mu
    \end{equation*}
    como $V$ es cerrado, es completo respecto a la norma inducida por el producto interno de $H$, 
    es decir, $V$ es un espacio de Hilbert. Consideramos el funcional lineal $I:V\to\C$,
    \begin{equation*}
        I(h)=\int f_{\perp}\overline{h}\hspace{1mm}d\mu
    \end{equation*}
    que resulta ser continuo por la desigualdad de Hölder. Por el teorema de representación de
    Riesz, existe un único $g\in V$ tal que
    \begin{equation*}
        \int f\overline{h}\hspace{1mm}d\mu=\int f_{\perp}\overline{h}\hspace{1mm}d\mu
        =\int g\overline{h}\hspace{1mm}d\mu
    \end{equation*}

    \item 
    \item 
\end{enumerate}

\section*{Problema 2}
\begin{enumerate}
    \item Como $u\in\cc_{0}^{\infty}[a,b]$ satisface $-u''+u=f$, entonces dada $v\in
    \cc_{0}^{\infty}[a,b]$ se tiene que $-u''v+uv=fv$, por la desigualdad de Hölder, vemos que 
    $fv$ es integrable, además $-u''v+uv\in\cc_{0}^{\infty}[a,b]$. Integrando a ambos lados y
    usando integración por partes tenemos que
    \begin{align*}
        \int_{[a,b]}fv\hspace{1mm}d\lambda &= \int_{[a,b]}-u''v+uv\hspace{1mm}d\lambda
        =-\int_{[a,b]}u''v\hspace{1mm}d\lambda+\int_{[a,b]}uv\hspace{1mm}d\lambda \\
        &= -u'v\Big|_{a}^{b}+\int_{[a,b]}u'v'\hspace{1mm}d\lambda
        +\int_{[a,b]}uv\hspace{1mm}d\lambda \\
        &= \int_{[a,b]}(uv+u'v')\hspace{1mm}d\lambda
    \end{align*}
    donde la tercera igualdad se debe a que $supp(v)\subseteq(a,b)$.

    \item Sea $[c,d]\subseteq[a,b]$, con $a<c<d<b$. Entonces, existe una sucesión acotada 
    $(v_{n})_{n}\subseteq\cc_{0}^{\infty}[a,b]$ tal que $v_{n}\xrightarrow[n\to\infty]{}
    \I_{[c,d]}$ para $x\in[a,b]$. Por otro lado, integrando por partes, observamos que
    \begin{align*}
        \int_{[a,b]}fv_{n}\hspace{1mm}d\lambda &= \int_{[a,b]}(uv_{n}+u'v_{n}')\hspace{1mm}d\lambda
        =\int_{[a,b]}u'v_{n}'\hspace{1mm}d\lambda+\int_{[a,b]}uv_{n}\hspace{1mm}d\lambda \\
        &= \int_{[a,b]}-u''v_{n}+uv_{n}\hspace{1mm}d\lambda
    \end{align*}
    Existe $M>0$ tal que $v_{n}\leq M$ para todo $n\in\N$, entonces $\abs{v_{n}f}\leq M\abs{f}$ y 
    además $v_{n}f$ converge puntualmente a $\I_{[c,d]}f$, por el mismo argumento se tiene para 
    $\I_{[c,d]}(-u''+u)$. De este modo, como $f,u\in\cc_{0}^{\infty}[a,b]$, por teorema de 
    convergencia dominada se sigue que
    \begin{equation*}
        \int_{[c,d]}f\hspace{1mm}d\lambda=\lim\limits_{n\to\infty}
        \int_{[a,b]}fv_{n}\hspace{1mm}d\lambda=\lim\limits_{n\to\infty}
        \int_{[a,b]}-u''v_{n}+uv_{n}\hspace{1mm}d\lambda=\int_{[c,d]}-u''+u\hspace{1mm}d\lambda
    \end{equation*}
    como esto es para $c,d$ arbitrarios, concluimos que $f=-u''+u$ $\lambda-ctp$. 
    
    Afirmamos que esta igualdad se cumple para todo $x\in(a,b)$. Supongamos, por contradicción, que 
    existe $x\in(a,b)$ tal que $f(x)\neq-u''(x)+u(x)$, como $f,u$ y $u''$ son continuas, existe 
    un intervalo abierto $I\subseteq[a,b]$ tal que $f(x)\neq-u''(x)+u(x)$ para todo $x\in I$. Esto
    contradice que la igualdad sea en casi todas partes.

    \item Sea $u\in H$, entonces existe $(u_{n})_{n\in\N}\subseteq \cc_{0}^{\infty}[a,b]$ tal que
    $u_{n}\to u$ con la norma $\norm{\cdot}_{H}$. En particular, la sucesión $u_{n}$ es de cauchy,
    además, se tiene que
    \begin{equation*}
        \norm{u_{n}}_{H}^{2}=\int\abs{u_{n}}^{2}+\abs{u_{n}'}^{2}d\lambda
        \geq\int\abs{u_{n}'}^{2}d\lambda=\norm{u_{n}'}_{L^{2}}^{2}
    \end{equation*}
    entonces, la sucesión $v_{n}:=u_{n}'\in\cc_{0}^{\infty}[a,b]$ es de cauchy según la norma en 
    $L^{2}$. Como este espacio es un espacio métrico completo, existe $v\in L_{0}^{2}$ tal que
    \begin{equation*}
        \lim\limits_{n\to\infty}\int\abs{v'_{n}-v}^{2}d\lambda=0
    \end{equation*}
    Afirmamos que $v$ es la función buscada. Sea $w\in\cc_{0}^{\infty}[a,b]$, entonces
    \begin{equation*}
        \int v_{n}w\hspace{1mm}d\lambda=\int u_{n}'w\hspace{1mm}d\lambda=u_{n}w\Big|_{a}^{b}
        -\int u_{n}w'\hspace{1mm}d\lambda=-\int u_{n}w'\hspace{1mm}d\lambda
    \end{equation*}
    Basta demostrar que $v_{n}w$ y $u_{n}w'$ convergen a $vw$ y $uw'$ en $L^{1}$ respectivamente.
    En efecto, por hölder tenemos que
    \begin{equation*}
        \int\abs{v_{n}w-vw}\hspace{1mm}d\lambda=\int\abs{v_{n}-v}\abs{w}\hspace{1mm}d\lambda
        \leq\norm{v_{n}-v}_{2}\norm{w}_{2}
    \end{equation*}
    y por otro lado
    \begin{equation*}
        \int\abs{u_{n}w'-uw'}\hspace{1mm}d\lambda=\int\abs{u_{n}-u}\abs{w'}\hspace{1mm}d\lambda
        \leq\norm{u_{n}-u}_{2}\norm{w'}_{2}\leq\norm{u_{n}-u}_{H}\norm{w'}_{2}
    \end{equation*}
    como $w\in\cc_{0}^{\infty}[a,b]$ vemos que $\norm{w}_{2},\norm{w'}_{2}<\infty$ y usando la 
    convergencia en $H$ y $L^{2}$ se tiene le resultado. Resta ver que $v$ es única. Supongamos, 
    por contradicción, que existe $\widehat{v}\in\cc_{0}^{\infty}[a,b]$ tal que
    \begin{equation*}
        \int \widehat{v}w\hspace{1mm}d\lambda=-\int uw'\hspace{1mm}d\lambda
    \end{equation*}
    para toda $w\in\cc_{0}^{\infty}[a,b]$. Entonces para toda $w\in\cc_{0}^{\infty}[a,b]$, se 
    tiene que
    \begin{equation*}
        \int vw\hspace{1mm}d\lambda=\int\widehat{v}w\hspace{1mm}d\lambda
        \hspace{4mm}\text{lo que implica que}\hspace{4mm}
        \int(v-\widehat{v})w\hspace{1mm}d\lambda=0
    \end{equation*}
    por el mismo argumento que antes, para todo $c<d$ con $c,d\in(a,b)$ se tiene que
    \begin{equation*}
        \int_{[c,d]}v-\widehat{v}\hspace{1mm}d\lambda=0
    \end{equation*}
    concluimos que $v=\widehat{v}$ $\lambda-ctp$, lo que prueba la unicidad en $L^{2}$.
    
    \item Sean $u,w\in H$, existe $(w_{n})_{n\in\N}\in\cc_{0}^{\infty}$ tal que $w_{n}\to w$ según 
    la norma $\norm{\cdot}_{H}$. Por la parte anterior, tenemos que
    \begin{equation*}
        \int u'w_{n}\hspace{1mm}d\lambda=-\int uw_{n}'\hspace{1mm}d\lambda
    \end{equation*}
    Por el mismo argumento que antes, sabemos que $\norm{u'w_{n}-u'w}_{1}\leq\norm{w_{n}-w}_{H}
    \norm{u'}_{2}$ y además $\norm{uw_{n}'-uw'}_{1}\leq\norm{u}_{2}\norm{w_{n}'-w'}_{2}$ y por lo 
    tanto
    \begin{equation*}
        \int u'w\hspace{1mm}d\lambda=\lim\limits_{n\to\infty}\int u'w_{n}\hspace{1mm}d\lambda
        =\lim\limits_{n\to\infty}-\int uw_{n}'\hspace{1mm}d\lambda=\int uw'\hspace{1mm}d\lambda
    \end{equation*}
    
    \item 
\end{enumerate}

%\printbibliography % Quitar el comentado si quiero usar bibliografia

\end{document}
