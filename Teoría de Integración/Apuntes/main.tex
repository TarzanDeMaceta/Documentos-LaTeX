\documentclass{article}
\usepackage{hyperref}
\usepackage{Style}

\nocite{*} % Comentar si quiero citar
\addbibresource{bibliografia.bib} % Quitar el comentado si quiero usar bibliografia

\begin{document}

\begin{minipage}{2.5cm}
    \includegraphics[width=2cm]{imagen_puc.jpg}
\end{minipage}
\begin{minipage}{14cm}
    {\sc Pontificia Universidad Católica de Chile\\
    Facultad de Matemáticas\\
    Departamento de Matemática\\
    Profesor: Gregorio Moreno -- Estudiante: Benjamín Mateluna}
\end{minipage}
\vspace{1ex}

{\centerline{\bf Teoría de Integración - MAT2534}
\centerline{\bf Apuntes}}
\centerline{\bf 05 de Marzo de 2025}

\newpage
\tableofcontents

\newpage
\section{Introducción}
\subsection{Evaluaciones}
\begin{itemize}
    \item Tres interrogaciones (I1, I2, I3) cada una vale un $20\%$.
    \item Cuatro tareas (T1, T2, T3, T4) cada una vale un $5\%$.
    \item Un examen (EX) que vale un $20\%$.
\end{itemize}
Posibilidad de eximición con promedio renormalizado de las tres interrogaciones y las tres 
primeras tareas sobre $5.5$ y cuarta tarea sobre $5.5$.
\vspace{4mm}

\noindent En la primera y segunda clase se realizo una introducción histórica sobre el cálculo y 
concepto de área, se dio una noción de área. Contenido disponible en el pdf de la clase 1. 

\newpage
\section{La integral de Riemann}

\newpage
\section{Espacios de Medida}

\newpage
\section{La integral de Lebesgue}

\newpage
\section{Espacio producto}

\newpage
\section{Diferenciación}

\newpage
\section{Introducción a los espacios \texorpdfstring{$L^{p}$}{}}

\newpage
\printbibliography % Quitar el comentado si quiero usar bibliografia

\end{document}
