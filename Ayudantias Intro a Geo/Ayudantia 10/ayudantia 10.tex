\documentclass{article}
\usepackage{hyperref}
\usepackage{Style}

\nocite{*} % Comentar si quiero citar
%\addbibresource{bibliografia.bib} % Quitar el comentado si quiero usar bibliografia

\begin{document}

\begin{minipage}{2.5cm}
    \includegraphics[width=2cm]{imagen_puc.jpg}
\end{minipage}
\begin{minipage}{14cm}
    {\sc Pontificia Universidad Católica de Chile\\
    Facultad de Matemáticas\\
    Departamento de Matemática\\
    Profesor: Giancarlo Urzúa -- Ayudante: Benjamín Mateluna}
\end{minipage}
\vspace{1ex}

{\centerline{\bf Introducción a la Geometría - MAT1304}
\centerline{\bf Ayudantía 10}}
\centerline{\bf 10 de septiembre de 2025}

\vspace{1cm}
\noindent\textbf{Problema 1.} Sean $\alpha,\beta\in(0,\frac{\pi}{2})$. Demuestre geometricamente
que $sen(\alpha)<sen(\beta)$. Usando lo anterior pruebe que
\begin{equation*}
    cos(\alpha)>cos(\beta)\hhtext{y}tan(\alpha)<tan(\beta)
\end{equation*}

\vspace{5mm}
\noindent\textbf{Problema 2.} Sea $x\in(0,\frac{\pi}{2})$ tal que $sec(x)-tan(x)=2$. Calcule
$sec(x)+tan(x)$.

\vspace{5mm}
\noindent\textbf{Problema 3.}

\vspace{5mm}
\noindent\textbf{Problema 4.}

\vspace{5mm}
\noindent\textbf{Problema 5.}

%\printbibliography % Quitar el comentado si quiero usar bibliografia

\end{document}
