\documentclass{article}
\usepackage{hyperref}
\usepackage{Style}

\nocite{*} % Comentar si quiero citar
%\addbibresource{bibliografia.bib} % Quitar el comentado si quiero usar bibliografia

\begin{document}

\begin{minipage}{2.5cm}
    \includegraphics[width=2cm]{imagen_puc.jpg}
\end{minipage}
\begin{minipage}{14cm}
    {\sc Pontificia Universidad Católica de Chile\\
    Facultad de Matemáticas\\
    Departamento de Matemática\\
    Profesor: Giancarlo Urzúa -- Ayudante: Benjamín Mateluna}
\end{minipage}
\vspace{1ex}

{\centerline{\bf Introducción a la Geometría - MAT1304}
\centerline{\bf Ayudantía 10}}
\centerline{\bf 10 de septiembre de 2025}

\vspace{1cm}
\noindent\textbf{Problema 1.} Sean $\alpha,\beta\in(0,\frac{\pi}{2})$ tales que $\alpha<\beta$. 
Demuestre geométricamente que $sen(\alpha)<sen(\beta)$. Usando lo anterior pruebe que
\begin{equation*}
    cos(\alpha)>cos(\beta)\hhtext{y}tan(\alpha)<tan(\beta)
\end{equation*}

\vspace{5mm}
\noindent\textbf{Problema 2.} Sea $x\in(0,\frac{\pi}{2})$ tal que $sec(x)-tan(x)=2$. Calcule
$sec(x)+tan(x)$.

\vspace{5mm}
\noindent\textbf{Problema 3.} Verifique las identidades:
\begin{equation*}
    (a)\hspace{1mm}\frac{sen(x)}{1-cos(x)}=csc(x)+cot(x)\hspace{1cm}
    (b)\hspace{1mm}tan(x)+cot(x)=sec(x)\cdot csc(x)\hspace{1cm}
    (c)\hspace{1mm}cos\left(\frac{\pi}{2}-x\right)=sen(x)
\end{equation*}

\vspace{5mm}
\noindent\textbf{Problema 4.} Exprese el área de $\triangle ABC$, sabiendo el largo del lado 
$\overline{BC}$, $\overline{AB}$ y el valor del ángulo $\angle ABC$.

\vspace{5mm}
\noindent\textbf{Problema 5.} Sea $\triangle ABC$. Sean $E,D\in\overline{BC}$ tales que 
$\angle AEB=90^{\circ}$ y $\angle ADC=120^{\circ}$. Si $\overline{AD}=4$, encuentre el área de 
$\triangle AED$.

%\printbibliography % Quitar el comentado si quiero usar bibliografia

\end{document}
