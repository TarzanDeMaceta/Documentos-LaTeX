\documentclass{article}
\usepackage{hyperref}
\usepackage{Style}

\nocite{*} % Comentar si quiero citar
%\addbibresource{bibliografia.bib} % Quitar el comentado si quiero usar bibliografia

\begin{document}

\begin{minipage}{2.5cm}
    \includegraphics[width=2cm]{imagen_puc.jpg}
\end{minipage}
\begin{minipage}{14cm}
    {\sc Pontificia Universidad Católica de Chile\\
    Facultad de Matemáticas\\
    Departamento de Matemática\\
    Profesor: Giancarlo Urzúa -- Ayudante: Benjamín Mateluna}
\end{minipage}
\vspace{1ex}

{\centerline{\bf Introdcucción a la Geometría - MAT1304}
\centerline{\bf Ayudantía 3}}
\centerline{\bf 18 de agosto de 2025}

\vspace{1cm}
\noindent\textbf{Problema 1.} Dado $\triangle ABC$, consideremos $Z$ un punto en $\overline{AB}$.
Se traza el segmento $\overline{CZ}$, trazamos la recta paralela a $\overline{CZ}$ que intersecta
a $\overleftrightarrow{BC}$ en $X$. Del mismo modo, se traza la recta paralela a $\overline{CZ}$ 
que pasa por $B$ y que intersecta a $\overleftrightarrow{AC}$ en $Y$. Muestre que
\begin{equation*}
    \frac{1}{AX}+\frac{1}{BY}=\frac{1}{CZ}
\end{equation*}

\vspace{5mm}
\noindent\textbf{Problema 2.} Sea $\triangle ABC$ un triángulo cualquiera. Consideremos los puntos 
$D,E,F$ en $\overline{BC},\overline{AC},\overline{AB}$ respectivamente, tales que 
$\overline{AD},\overline{BE}$ y $\overline{CF}$ concurren dentro del triángulo. Muestre que 
si $\overline{FE}$ y $\overline{BC}$ son paralelas, entonces $\overline{BD}=\overline{DC}$.

\vspace{5mm}
\noindent\textbf{Problema 3.} Sea $\triangle ABC$ un triángulo cualquiera. Sean $D,E$ en 
$\overline{AB}$ y $\overline{BC}$ respectivamente tales que los segmentos $\overline{AE}$ y 
$\overline{CD}$ son alturas. Muestre que $\triangle ABE\sim\triangle CBD$.\footnote{Este problema 
puede ser útil para la tarea}

\vspace{5mm}
\noindent\textbf{Problema 4.} Dados $\triangle ABC$ y $\triangle DEF$ semejantes con razón $k$, pruebe que
\begin{equation*}
    \frac{\abs{\triangle ABC}}{\abs{\triangle DEF}}=k^{2}
\end{equation*}

\vspace{5mm}
\noindent\textbf{Problema 5.} Deducir geométricamente las desigualdades
\begin{equation*}
    \frac{2ab}{a+b}\leq\sqrt{ab}\leq\frac{a+b}{2}\leq\sqrt{\frac{a^{2}+b^{2}}{2}}
\end{equation*}

%\printbibliography % Quitar el comentado si quiero usar bibliografia

\end{document}
