\documentclass{article}
\usepackage{hyperref}
\usepackage{Style}

\nocite{*} % Comentar si quiero citar
%\addbibresource{bibliografia.bib} % Quitar el comentado si quiero usar bibliografia

\begin{document}

\begin{minipage}{2.5cm}
    \includegraphics[width=2cm]{imagen_puc.jpg}
\end{minipage}
\begin{minipage}{14cm}
    {\sc Pontificia Universidad Católica de Chile\\
    Facultad de Matemáticas\\
    Departamento de Matemática\\
    Profesor: Giancarlo Urzúa -- Ayudante: Benjamín Mateluna}
\end{minipage}
\vspace{1ex}

{\centerline{\bf Introducción a la Geometría - MAT1304}
\centerline{\bf Ayudantía 28 - Repaso Examen}}
\centerline{\bf 26 de noviembre de 2025}

\vspace{1cm}
\noindent\textbf{Problema 1.} Construir $\triangle ABC$ dados lado $a$, lado $b$ y altura $h_{B}$.

\vspace{5mm}
\noindent\textbf{Problema 2.} Sea $\triangle ABC$ un triángulo cualquiera. Sea $I$ el incentro y
$D$ el punto donde la bisectriz del ángulo $\angle BAC$ corta al lado $BC$. Demuestre que
\begin{equation*}
    \frac{\overline{AI}}{\overline{ID}}=\frac{b+c}{a}
\end{equation*}

\vspace{5mm}
\noindent\textbf{Problema 3.} Pruebe que $sen(3\alpha)=3sen(\alpha)-4sen^{3}(\alpha)$ y usando lo
anterior demuestre que $sen(10^{\circ})$ es raíz de $8x^{3}-6x+1$.

\vspace{5mm}
\noindent\textbf{Problema 4.} Encontrar las rectas tangentes a $x^{2}+y^{2}=1$ tales que son 
pararelas a $y=-2x+5$.

\vspace{5mm}
\noindent\textbf{Problema 5.} Considere la ecuación paramétrica del plano
\begin{equation*}
    \{(t+s,2t+s,t):t,s\in\R\}
\end{equation*}
encuentre la ecuación general del plano.

%\printbibliography % Quitar el comentado si quiero usar bibliografia

\end{document}
