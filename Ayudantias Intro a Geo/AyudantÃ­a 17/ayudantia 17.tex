\documentclass{article}
\usepackage{hyperref}
\usepackage{Style}

\nocite{*} % Comentar si quiero citar
%\addbibresource{bibliografia.bib} % Quitar el comentado si quiero usar bibliografia

\begin{document}

\begin{minipage}{2.5cm}
    \includegraphics[width=2cm]{imagen_puc.jpg}
\end{minipage}
\begin{minipage}{14cm}
    {\sc Pontificia Universidad Católica de Chile\\
    Facultad de Matemáticas\\
    Departamento de Matemática\\
    Profesor: Giancarlo Urzúa -- Ayudante: Benjamín Mateluna}
\end{minipage}
\vspace{1ex}

{\centerline{\bf Introducción a la Geometría - MAT1304}
\centerline{\bf Ayudantía 17 - Repaso I2}}
\centerline{\bf 13 de octubre de 2025}

\vspace{1cm}

\vspace{5mm}
\noindent\textbf{Problema 1.} Calcular el valor de $sen(\frac{\pi}{5})$ y uselo para calcular 
$sen(\frac{\pi}{20})$.

\vspace{5mm}
\noindent\textbf{Problema 2.} Demuestre la siguiente identidad trigonométrica
\begin{equation*}
    tan\left(\frac{\alpha}{2}\right)=csc(\alpha)-cot(\alpha)
\end{equation*}

\vspace{5mm}
\noindent\textbf{Problema 3.} En $\triangle ABC$, se cumple que
\begin{equation*}
    \frac{a}{cos(\alpha)}=\frac{b}{cos(\beta)}=\frac{c}{cos(\gamma)}
\end{equation*}
Demuestre que el triángulo es equilátero.

\vspace{5mm}
\noindent\textbf{Problema 4.} Encontrar un polinomio $p(x)$ de grado $5$ tal que $p(x)=r(x)t(x)$
con $1$ y $-1$ raíces de $p(x)$, $r(x)$ de grado dos, y $t(x)$ sin raíces racionales.

\vspace{5mm}
\noindent\textbf{Problema 5.} Demuestre que $\sqrt{2}+\sqrt{3}$ es irracional.

%\printbibliography % Quitar el comentado si quiero usar bibliografia

\end{document}
