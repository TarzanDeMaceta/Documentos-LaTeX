\documentclass{article}
\usepackage{hyperref}
\usepackage{Style}

\nocite{*} % Comentar si quiero citar
%\addbibresource{bibliografia.bib} % Quitar el comentado si quiero usar bibliografia

\begin{document}

\begin{minipage}{2.5cm}
    \includegraphics[width=2cm]{imagen_puc.jpg}
\end{minipage}
\begin{minipage}{14cm}
    {\sc Pontificia Universidad Católica de Chile\\
    Facultad de Matemáticas\\
    Departamento de Matemática\\
    Profesor: Giancarlo Urzúa -- Ayudante: Benjamín Mateluna}
\end{minipage}
\vspace{1ex}

{\centerline{\bf Introducción a la Geometría - MAT1304}
\centerline{\bf Ayudantía 15}}
\centerline{\bf 06 de octubre de 2025}

\vspace{1cm}
\noindent\textbf{Problema 1.} Encuentre las 3 raíces del polinomio $x^{3}+3x^{2}-2x-6$.

\vspace{5mm}
\noindent\textbf{Problema 2.} Demuestre que $\sqrt[3]{2+\sqrt{5}}+\sqrt[3]{2-\sqrt{5}}=1$.

\vspace{5mm}
\noindent\textbf{Problema 3.} Mostrar que no existen $p,q\in\Q[x]$ tales que 
$x^{3}+101x^{2}-107x+2=p(x)q(x)$.

\vspace{5mm}
\noindent\textbf{Problema 4.} Demuestre que para $z\in\C$, se tiene que
\begin{enumerate}
    \item $z\in\R$ si y solo si $z=\overline{z}$.
    \item $\abs{z}=1$ y $z\neq0$ entonces $\overline{z}=\frac{1}{z}$.
\end{enumerate}

\vspace{5mm}
\noindent\textbf{Problema 5.} Sea $z\in\C$ con parte imaginaria no nula tal que $z+\frac{1}{z}$ es 
un número real. Calcule $z\overline{z}$.

%\printbibliography % Quitar el comentado si quiero usar bibliografia

\end{document}
