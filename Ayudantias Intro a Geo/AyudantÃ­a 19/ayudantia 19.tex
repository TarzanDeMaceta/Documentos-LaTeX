\documentclass{article}
\usepackage{hyperref}
\usepackage{Style}

\nocite{*} % Comentar si quiero citar
%\addbibresource{bibliografia.bib} % Quitar el comentado si quiero usar bibliografia

\begin{document}

\begin{minipage}{2.5cm}
    \includegraphics[width=2cm]{imagen_puc.jpg}
\end{minipage}
\begin{minipage}{14cm}
    {\sc Pontificia Universidad Católica de Chile\\
    Facultad de Matemáticas\\
    Departamento de Matemática\\
    Profesor: Giancarlo Urzúa -- Ayudante: Benjamín Mateluna}
\end{minipage}
\vspace{1ex}

{\centerline{\bf Introducción a la Geometría - MAT1304}
\centerline{\bf Ayudantía 19}}
\centerline{\bf 20 de octubre de 2025}

\vspace{1cm}
\noindent\textbf{Problema 1.} Halle el máximo común divisor de los siguientes pares de polinomios
y halle una combinación del lema de Bézout en cada caso

\begin{enumerate}
    \item $x^{4}+x^{3}+x^{2}+1$ y $x^{3}+1$ en $\R[x]$.
    
    \item $x^{3}+2x-i$ y $x^{2}+1$ en $\C[x]$.
\end{enumerate}

\begin{dfn}
    Sea $F\in\C[x,y]$, se dice homogéneo si para todo $\lambda\in\C$ tal que 
    $F(\lambda x,\lambda y)=\lambda^{n}F(x,y)$ para algún $n\in\N$.
\end{dfn}

\vspace{5mm}
\noindent\textbf{Problema 2.} Demuestre que todo polinomio homogéneo en $\C[x,y]$ se puede 
escribir como el producto de polinomios lineales homogéneos.

\vspace{5mm}
\noindent\textbf{Problema 3.} Demuestre que $y^{2}-x^{3}$ es irreducible en $\C[x,y]$.

\vspace{5mm}
\noindent\textbf{Problema 4.} Sean $f,g\in k[x,y]$, demuestre que
\begin{enumerate}
    \item El conjunto de ceros de $fg$ es igual a la unión del conjunto de ceros de $f$ y $g$.
    \item Si $k=\R$, el conjunto de ceros de $f^{2}+g^{2}$ es igual a la intersección del conjunto 
    de ceros de $f$ y $g$.
\end{enumerate}

\vspace{5mm}
\noindent\textbf{Problema 5.} Halle el lugar geométrico de todos los puntos que equidistan del
origen y $(-1,1)$. Después, grafiquelo en el plano.

%\printbibliography % Quitar el comentado si quiero usar bibliografia

\end{document}
