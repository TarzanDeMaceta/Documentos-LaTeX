\documentclass{article}
\usepackage{hyperref}
\usepackage{Style}

\nocite{*} % Comentar si quiero citar
%\addbibresource{bibliografia.bib} % Quitar el comentado si quiero usar bibliografia

\begin{document}

\begin{minipage}{2.5cm}
    \includegraphics[width=2cm]{imagen_puc.jpg}
\end{minipage}
\begin{minipage}{14cm}
    {\sc Pontificia Universidad Católica de Chile\\
    Facultad de Matemáticas\\
    Departamento de Matemática\\
    Profesor: Giancarlo Urzúa -- Ayudante: Benjamín Mateluna}
\end{minipage}
\vspace{1ex}

{\centerline{\bf Introducción a la Geometría - MAT1304}
\centerline{\bf Ayudantía 23 - Repaso Polinomios}}
\centerline{\bf 03 de noviembre de 2025}

\vspace{1cm}
\noindent\textbf{Problema 1.} Considere $a,b,c$ reales positivos. Muestre que no es posible que
$p(x)=ax^{2}+bx+c$, $q(x)=bx^{2}+cx+a$ y $r(x)=cx^{2}+ax+b$ tengan raíces reales de manera 
simultánea.

\vspace{5mm}
\noindent\textbf{Problema 2.} Halle el máximo común divisor del par de polinomios $x^{3}+2x-i$ y
$x^{2}+1$ en $\C[x]$ y halle una combinación del Lema de Bézout.

\vspace{5mm}
\noindent\textbf{Problema 3.} Demuestre que el polinomio $y^{2}-x^{3}$ es irreducible en 
$\C[x,y]$.

\vspace{5mm}
\noindent\textbf{Problema 4.} Encontrar $p(x)=a_{3}x^{3}+a_{2}x^{2}+a_{1}x
+a_{0}\in\R[x]$ tal que $p(-1+2i)=0$, $p(1)=0$ y $p(-1)=5$.

\vspace{5mm}
\noindent\textbf{Problema 5.} Sea $(x_{0},y_{0})$ un cero de $x^{2}-y^{2}$. Demuestre que la recta 
que pasa $(x_{0},y_{0})$ y el origen esta contenida en el conjunto de ceros de $x^{2}-y^{2}$.

%\printbibliography % Quitar el comentado si quiero usar bibliografia

\end{document}
