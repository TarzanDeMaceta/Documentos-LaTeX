\documentclass{article}
\usepackage{hyperref}
\usepackage{Style}

\nocite{*} % Comentar si quiero citar
%\addbibresource{bibliografia.bib} % Quitar el comentado si quiero usar bibliografia

\begin{document}

\begin{minipage}{2.5cm}
    \includegraphics[width=2cm]{imagen_puc.jpg}
\end{minipage}
\begin{minipage}{14cm}
    {\sc Pontificia Universidad Católica de Chile\\
    Facultad de Matemáticas\\
    Departamento de Matemática\\
    Profesor: Giancarlo Urzúa -- Ayudante: Benjamín Mateluna}
\end{minipage}
\vspace{1ex}

{\centerline{\bf Introducción a la Geometría - MAT1304}
\centerline{\bf Ayudantía 13}}
\centerline{\bf 01 de octubre de 2025}

\vspace{1cm}
\noindent\textbf{Problema 1.} Encuentre $n\in\N$ y $q(x)\in\Q[x]$ con $q(1)\neq0$ tales que
\begin{equation*}
    (x-1)^{n}q(x)=(x^{2}-1)(x^{3}-1)(x^{6}-5x+1)
\end{equation*}

\vspace{5mm}
\noindent\textbf{Problema 2.} Suponga que las raices del polinomio $x^{2}+ax+b+1$ son números 
naturales. Muestre que $a^{2}+b^{2}$ no es un número primo.

\vspace{5mm}
\noindent\textbf{Problema 3.} Sea $p(x)\in\Z[x]$ tal que toma valores $\pm1$ en $3$ enteros 
distintos. Demuestre que el polinomio no tiene raices enteras.

\vspace{5mm}
\noindent\textbf{Problema 4.} Demuestre que las funciones $sen(x),cos(x)$ y $tan(x)$ no son 
polinomios.

\vspace{5mm}
\noindent\textbf{Problema 5.} Encontrar un polinomio en $\Q[x]$ no nulo tal que 
$\sqrt{1+\sqrt{2}}$ es raíz.

%\printbibliography % Quitar el comentado si quiero usar bibliografia

\end{document}
