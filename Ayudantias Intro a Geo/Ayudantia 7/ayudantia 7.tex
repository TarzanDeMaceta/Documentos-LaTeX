\documentclass{article}
\usepackage{hyperref}
\usepackage{Style}

\nocite{*} % Comentar si quiero citar
%\addbibresource{bibliografia.bib} % Quitar el comentado si quiero usar bibliografia

\begin{document}

\begin{minipage}{2.5cm}
    \includegraphics[width=2cm]{imagen_puc.jpg}
\end{minipage}
\begin{minipage}{14cm}
    {\sc Pontificia Universidad Católica de Chile\\
    Facultad de Matemáticas\\
    Departamento de Matemática\\
    Profesor: Giancarlo Urzúa -- Ayudante: Benjamín Mateluna}
\end{minipage}
\vspace{1ex}

{\centerline{\bf Introducción a la Geometría - MAT1304}
\centerline{\bf Ayudantía 7}}
\centerline{\bf 1 de septiembre de 2025}

\vspace{1cm}
\noindent\textbf{Problema 1.} Considere una circunferencia de centro $O$. Sean $\overline{AB}$ y 
$\overline{AC}$ tangentes al círculo en $B$ y $C$ respectivamente. Sea $\overline{CE}$ 
perpendicular al diametro $\overline{BD}$ con $E$ entre $O$ y $D$. Pruebe que 
$\overline{BE}\cdot\overline{BO}=\overline{AB}\cdot\overline{CE}$.

\vspace{5mm}
\noindent\textbf{Problema 2.} Dado un segmento $\overline{AB}$ construir un cuadrado cuya diagonal
sea igual a este segmento.

\vspace{5mm}
\noindent\textbf{Problema 3.} Dada una circunferencia con centro $O$.
\begin{enumerate}
    \item Sea $P$ un punto exterior al círculo, trazar la recta tangente a la circunferencia que 
    pasa por $P$

    \item Construir un triángulo isósceles que lo tenga de incírculo, es decir, cuyos lados sean 
    tangentes a la circunferencia.
\end{enumerate}

\vspace{5mm}
\noindent\textbf{Problema 4.} Dado un segmento de $\overline{AB}$ de largo $\ell$, construya un 
rombo cuyos lados tienen largo $\ell$ que también es el largo de una de las diagonales del rombo.

%\printbibliography % Quitar el comentado si quiero usar bibliografia

\end{document}
