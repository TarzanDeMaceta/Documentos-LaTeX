\documentclass{article}
\usepackage{hyperref}
\usepackage{Style}

\nocite{*} % Comentar si quiero citar
%\addbibresource{bibliografia.bib} % Quitar el comentado si quiero usar bibliografia

\begin{document}

\begin{minipage}{2.5cm}
    \includegraphics[width=2cm]{imagen_puc.jpg}
\end{minipage}
\begin{minipage}{14cm}
    {\sc Pontificia Universidad Católica de Chile\\
    Facultad de Matemáticas\\
    Departamento de Matemática\\
    Profesor: Giancarlo Urzúa -- Ayudante: Benjamín Mateluna}
\end{minipage}
\vspace{1ex}

{\centerline{\bf Introducción a la Geometría - MAT1304}
\centerline{\bf Ayudantía 20}}
\centerline{\bf 22 de octubre de 2025}

\vspace{1cm}
\noindent\textbf{Problema 1.} Sean $f,g\in k[x,y]$, demuestre que
\begin{enumerate}
    \item El conjunto de ceros de $fg$ es igual a la unión del conjunto de ceros de $f$ y $g$.
    \item Si $k=\R$, el conjunto de ceros de $f^{2}+g^{2}$ es igual a la intersección del conjunto 
    de ceros de $f$ y $g$.
\end{enumerate}

\vspace{5mm}
\noindent\textbf{Problema 2.} Halle el lugar geométrico de todos los puntos que equidistan del
origen y $(-1,1)$. Después, grafiquelo en el plano.

\vspace{5mm}
\noindent\textbf{Problema 3.} Considere la recta que pasa por el punto $(5,3)$ y que forma un
ángulo de $-\pi/4$ con respecto al eje $x$. Calcule la distancia entre el punto $(1,2)$ y
el único punto de la recta que tiene sus dos coordenadas iguales.

\vspace{5mm}
\noindent\textbf{Problema 4.} Determine todos los posibles valores de $x_{0}$ para los que la
intersección de la recta de ecuación $2x+y-1=0$ y la recta de ecuación $y=x_{0}x-3$ tenga 
coordenadas enteras.

\vspace{5mm}
\noindent\textbf{Problema 5.} Encuentre la ecuación de la recta que pasa por los puntos de 
intersección de
\begin{equation*}
    \begin{cases}
        x^{2}+y^{2}=4 \\
        (x-1)^{2}+(y-1)^{2}=4
    \end{cases}
\end{equation*}

%\printbibliography % Quitar el comentado si quiero usar bibliografia

\end{document}
