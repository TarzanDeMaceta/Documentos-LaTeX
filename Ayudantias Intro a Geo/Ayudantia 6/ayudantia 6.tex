\documentclass{article}
\usepackage{hyperref}
\usepackage{Style}

\nocite{*} % Comentar si quiero citar
%\addbibresource{bibliografia.bib} % Quitar el comentado si quiero usar bibliografia

\begin{document}

\begin{minipage}{2.5cm}
    \includegraphics[width=2cm]{imagen_puc.jpg}
\end{minipage}
\begin{minipage}{14cm}
    {\sc Pontificia Universidad Católica de Chile\\
    Facultad de Matemáticas\\
    Departamento de Matemática\\
    Profesor: Giancarlo Urzúa -- Ayudante: Benjamín Mateluna}
\end{minipage}
\vspace{1ex}

{\centerline{\bf Introducción a la Geometría - MAT1304}
\centerline{\bf Ayudantía 6}}
\centerline{\bf 27 de agosto de 2025}

\vspace{1cm}
\noindent\textbf{Problema 1.} Sean $A$ y $B$ puntos en la circunferencia. Sean $P$ y $Q$ puntos en
el arco $\frown{BA}$ y sea $X$ el punto en $\frown{AB}$ donde la bisectriz del ángulo $\angle APB$
corta a la circunferencia. Demuestre que $\overline{QX}$ es la bisectriz del ángulo $\angle AQB$.

\vspace{2mm}
\noindent\textbf{Problema 2.} Dada una circunferencia, la cuerda $\overline{AB}$ es tal que la
recta $\overleftrightarrow{AB}$ y la tangente a la circunferencia en un punto $T$ se intersectan
en un punto $C$. Demuestre que $\overline{AC}\cdot\overline{BC}=\overline{CT}^{2}$.

\vspace{2mm}
\noindent\textbf{Problema 3.} Sea $\triangle ABC$ un triángulo equilátero. Si $P$ es un punto en
el circuncírculo de $\triangle ABC$, muestre que el mayor entre $\overline{AP},\overline{BP}$ y
$\overline{CP}$ es siempre igual a la suma de los otros dos.

\vspace{2mm}
\noindent\textbf{Problema 4.} Sea $\triangle ABC$ un triángulo escaleno, sea $D$ en 
$\overline{AC}$ tal que $\overline{BD}$ es bisectriz del ángulo en $B$ y sean $E,F,M$ las 
proyecciones de $A,C,D$ en $\overline{BD},\overline{BD}$ y $\overline{BC}$ respectivamente. 
Demuestre que $\angle DME=\angle DMF$.

\vspace{2mm}
\noindent\textbf{Problema 5.} Sea $\triangle ABC$ escaleno. Sean $P$ y $Q$ en $\overline{AB}$ y
$\overline{AC}$ respectivamente tales que $\overline{PQ}$ es paralela a $\overline{BC}$. La 
circunferencia que pasa por $P$ y es tangente a $\overline{AC}$ en $Q$ intersecta nuevamente a 
$\overline{AB}$ en $R$. Demuestre que $RQCB$ es cíclico.

%\printbibliography % Quitar el comentado si quiero usar bibliografia

\end{document}
