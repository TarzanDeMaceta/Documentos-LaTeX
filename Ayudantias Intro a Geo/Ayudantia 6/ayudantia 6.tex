\documentclass{article}
\usepackage{hyperref}
\usepackage{Style}

\nocite{*} % Comentar si quiero citar
%\addbibresource{bibliografia.bib} % Quitar el comentado si quiero usar bibliografia

\begin{document}

\begin{minipage}{2.5cm}
    \includegraphics[width=2cm]{imagen_puc.jpg}
\end{minipage}
\begin{minipage}{14cm}
    {\sc Pontificia Universidad Católica de Chile\\
    Facultad de Matemáticas\\
    Departamento de Matemática\\
    Profesor: Giancarlo Urzúa -- Ayudante: Benjamín Mateluna}
\end{minipage}
\vspace{1ex}

{\centerline{\bf Introducción a la Geometría - MAT1304}
\centerline{\bf Ayudantía 6}}
\centerline{\bf 27 de agosto de 2025}

\vspace{1cm}
\noindent\textbf{Problema 1.} Sea $\triangle ABC$ un triángulo cualquiera y puntos $D,E,F$ en los
lados $\overline{BC}, \overline{AB}$ y $\overline{AC}$ respectivamente. Muestre que las 
circunferencias circunscritas a los triángulos $\triangle AEF,\triangle BED$ y $\triangle CFD$ son
concurrentes.

\vspace{5mm}
\noindent\textbf{Problema 2.} Sea $\triangle ABC$ escaleno. Sean $P$ y $Q$ en $\overline{AB}$ y
$\overline{AC}$ respectivamente tales que $\overline{PQ}$ es paralela a $\overline{BC}$. La 
circunferencia que pasa por $P$ y es tangente a $\overline{AC}$ en $Q$ intersecta nuevamente a 
$\overline{AB}$ en $R$. Demuestre que $RQCB$ es cíclico.

\vspace{5mm}
\noindent\textbf{Problema 3.} Sean $A$ y $B$ puntos en la circunferencia. Sean $P$ y $Q$ puntos en
el arco $BA$ y sea $X$ el punto en $AB$ donde la bisectriz del ángulo 
$\angle APB$ corta a la circunferencia. Demuestre que $\overline{QX}$ es la bisectriz del ángulo 
$\angle AQB$.

\vspace{5mm}
\noindent\textbf{Problema 4.} Sea $\triangle ABC$ inscrito en una circunferencia de centro $O$.
Sea $D$ la intersección de la bisectriz de $\angle BAC$ con $\overleftrightarrow{BC}$ y $P$ la
intersección de $\overleftrightarrow{AB}$ con la perpendicular de a $\overleftrightarrow{OA}$ que
pasa por $D$. Demostrar que $\overline{AC}=\overline{AP}$.

\vspace{5mm}
\noindent\textbf{Problema 5.} Sea $\triangle ABC$ inscrito en una circunferencia, considerar $M$
y $N$ en $\overline{AC}$ tales que $\angle ABM=\angle MBN=\angle NBC$. Sea $Q$ en la intersección
de $\overrightarrow{BM}$ y la circunferencia, por útimo, sea $P\in \overrightarrow{BN}\cap
\overline{QC}$. Demuestre que
\begin{equation*}
    \frac{\overline{AM}}{\overline{AN}}+\frac{\overline{CP}}{\overline{CQ}}=1
\end{equation*}

%\printbibliography % Quitar el comentado si quiero usar bibliografia

\end{document}
