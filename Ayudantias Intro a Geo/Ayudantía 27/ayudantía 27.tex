\documentclass{article}
\usepackage{hyperref}
\usepackage{Style}

\nocite{*} % Comentar si quiero citar
%\addbibresource{bibliografia.bib} % Quitar el comentado si quiero usar bibliografia

\begin{document}

\begin{minipage}{2.5cm}
    \includegraphics[width=2cm]{imagen_puc.jpg}
\end{minipage}
\begin{minipage}{14cm}
    {\sc Pontificia Universidad Católica de Chile\\
    Facultad de Matemáticas\\
    Departamento de Matemática\\
    Profesor: Giancarlo Urzúa -- Ayudante: Benjamín Mateluna}
\end{minipage}
\vspace{1ex}

{\centerline{\bf Introducción a la Geometría - MAT1304}
\centerline{\bf Ayudantía 27}}
\centerline{\bf 19 de noviembre de 2025}

\vspace{1cm}
\noindent\textbf{Problema 1.} Determinar si las siguientes funciones $F:\R^{2}\to\R^{2}$ son 
isometrías:
\begin{enumerate}
    \item $F(x,y)=(0,y)$
    \item $F(x,y)=(-y,x)$
    \item $F(x,y)=(2x,y)$
\end{enumerate}

\vspace{5mm}
\noindent\textbf{Problema 2.} Encontrar fórmula para la simetría $F(x,y)$, que rota los puntos en
$30^{\circ}$ con respecto al origen, luego traslada en $(-2,1)$ y finalmente refleja con respecto
al eje X.  

\vspace{5mm}
\noindent\textbf{Problema 3.} Determine la veracidad de las siguiente proposiciones:
\begin{enumerate}
    \item Existe una recta tangente a $x^{2}+y^{2}=4$ la cual contiene al punto $(0,1)$.
    \item $x^{2}+y^{2}=2(x+y)$ es una circunferencia.
\end{enumerate}

\vspace{5mm}
\noindent\textbf{Problema 4.} Describa el lugar geométrico dado por
\begin{equation*}
    x^{2}-5x-6y^{2}+4=0
\end{equation*}

\vspace{5mm}
\noindent\textbf{Problema 5.} Mostrar que la siguiente cónica es vacía
\begin{equation*}
    \frac{3}{2}x^{2}-xy+\sqrt{2}x+\frac{3}{2}y^{2}-3\sqrt{2}y+4=0
\end{equation*}

%\printbibliography % Quitar el comentado si quiero usar bibliografia

\end{document}
