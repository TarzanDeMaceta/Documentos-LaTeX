\documentclass{article}
\usepackage{hyperref}
\usepackage{Style}

\nocite{*} % Comentar si quiero citar
%\addbibresource{bibliografia.bib} % Quitar el comentado si quiero usar bibliografia

\begin{document}

\begin{minipage}{2.5cm}
    \includegraphics[width=2cm]{imagen_puc.jpg}
\end{minipage}
\begin{minipage}{14cm}
    {\sc Pontificia Universidad Católica de Chile\\
    Facultad de Matemáticas\\
    Departamento de Matemática\\
    Profesor: Giancarlo Urzúa -- Ayudante: Benjamín Mateluna}
\end{minipage}
\vspace{1ex}

{\centerline{\bf Introducción a la Geometría - MAT1304}
\centerline{\bf Ayudantía 28}}
\centerline{\bf 24 de noviembre de 2025}

\vspace{1cm}
\noindent\textbf{Problema 1.} Determine si la ecuación $4x^{2}+16x+4y^{2}-4y+17=0$ corresponde a
una cónica, un punto o el conjunto vacío. Si sucede lo primero, identifique que cónica es.

\vspace{5mm}
\noindent\textbf{Problema 2.} Asuma que todas las ecuaciones a continuación representan cónicas. 
Para cada una de ellas encuentre las coordenadas del vértice, centro, focos y ejes:
\begin{enumerate}
    \item $4x^{2}+48y+12x=159$
    \item $x^{2}+4y^{2}-10x-40y+109=0$
    \item $9x^{2}-4y^{2}+54x+16y+29=0$
\end{enumerate}

\vspace{5mm} 
\noindent\textbf{Problema 3.} Demuestre que cualquier recta que pasa por $P=(-1,5)$ no puede ser 
tangente a la circunferencia de ecuación $x^{2}+y^{2}+4x-6y=-6$.

\vspace{5mm}
\noindent\textbf{Problema 4.} Demuestre que las cónicas $x^{2}+y^{2}+4x+6y-23=0$ y 
$x^{2}+y^{2}-8x-10y+25=0$ son tangentes.

\vspace{5mm}
\noindent\textbf{Problema 5.} Considerar la cónica $\frac{x^{2}}{a^{2}}-\frac{y^{2}}{b^{2}}=1$ y
$P$ un punto tal que forma un triángulo rectángulo en $P$ con los focos. Mostrar que el área del
triángulo es $b^{2}$.

%\printbibliography % Quitar el comentado si quiero usar bibliografia

\end{document}
