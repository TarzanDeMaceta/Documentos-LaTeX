\documentclass{article}
\usepackage{hyperref}
\usepackage{Style}

\nocite{*} % Comentar si quiero citar
%\addbibresource{bibliografia.bib} % Quitar el comentado si quiero usar bibliografia

\begin{document}

\begin{minipage}{2.5cm}
    \includegraphics[width=2cm]{imagen_puc.jpg}
\end{minipage}
\begin{minipage}{14cm}
    {\sc Pontificia Universidad Católica de Chile\\
    Facultad de Matemáticas\\
    Departamento de Matemática\\
    Profesor: Giancarlo Urzúa -- Ayudante: Benjamín Mateluna}
\end{minipage}
\vspace{1ex}

{\centerline{\bf Introducción a la Geometría - MAT1304}
\centerline{\bf Ayudantía 16}}
\centerline{\bf 08 de octubre de 2025}

\vspace{1cm}
\noindent\textbf{Problema 1.} Grafique en el plano complejo el conjunto
\begin{equation*}
    \mathcal{L}=\{z\in\C:Re(z)+Im(\overline{z})=Im(z)\}
\end{equation*}

\vspace{5mm}
\noindent\textbf{Problema 2.} Resuelva las siguientes ecuaciones
\begin{enumerate}
    \item $z^{4}=\sqrt{i}$.
    \item $z^{7}-2iz^{4}-iz^{3}-2=0$.
\end{enumerate}

\vspace{5mm}
\noindent\textbf{Problema 3.} Sea $z=cis(\frac{2\pi}{5})$.
\begin{enumerate}
    \item Desmuestre que $z^{4}+z^{3}+z^{2}+z^{1}+1=0$.
    
    \item Pruebe que se cumple la siguiente expresión
    \begin{equation*}
        \left(z^{2}+\frac{1}{z^{2}}\right)+\left(z+\frac{1}{z}\right)+1=0
    \end{equation*}
    
    \item Verifique que $4cos^{2}(\frac{2\pi}{5})+2cos(\frac{2\pi}{5})-1=0$.
    
    \item Deduzca que
    \begin{equation*}
        cos\left(\frac{2\pi}{5}\right)=\frac{\sqrt{5}-1}{4}
    \end{equation*}
\end{enumerate}

\vspace{5mm}
\noindent\textbf{Problema 4.} Calcule todos los valores que puede tomar la expresión 
$(3-3i)\sqrt{1+i}$.

\vspace{5mm}
\noindent\textbf{Problema 5.} Dado el número complejo $z=\sqrt{3}+i$, calcule sus raíces quintas.

%\printbibliography % Quitar el comentado si quiero usar bibliografia

\end{document}
