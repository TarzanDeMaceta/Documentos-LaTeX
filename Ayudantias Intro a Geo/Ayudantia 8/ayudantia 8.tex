\documentclass{article}
\usepackage{hyperref}
\usepackage{Style}

\nocite{*} % Comentar si quiero citar
%\addbibresource{bibliografia.bib} % Quitar el comentado si quiero usar bibliografia

\begin{document}

\begin{minipage}{2.5cm}
    \includegraphics[width=2cm]{imagen_puc.jpg}
\end{minipage}
\begin{minipage}{14cm}
    {\sc Pontificia Universidad Católica de Chile\\
    Facultad de Matemáticas\\
    Departamento de Matemática\\
    Profesor: Giancarlo Urzúa -- Estudiante: Benjamín Mateluna}
\end{minipage}
\vspace{1ex}

{\centerline{\bf Introducción a la Geometría - MAT1304}
\centerline{\bf Ayudantía 8 - Repaso I1}}
\centerline{\bf 03 de septiembre de 2025}

\vspace{1cm}
\noindent\textbf{Problema 1.} Sean $L_{1}$ y $L_{2}$ dos rectas paralelas. Se traza una recta 
$L_{3}$ secante a ambas rectas que intersecta a $L_{1}$ y $L_{2}$ en los punto $A$ y $D$ 
respectivamente. Sea $L_{4}$ una recta que intersecta a $L_{3}$ en el punto $E$ que se encuentra 
entre ambas rectas paralelas y las intersecta en los puntos $C$ y $B$ en ese orden. Muestre que
\begin{equation*}
    \angle DBE+\angle CAE=\angle BEA \hhtext{y} \angle BDE+\angle ACE=\angle DEC
\end{equation*}

\vspace{5mm}
\noindent\textbf{Problema 2.} Sea $\triangle ABC$ un triángulo tal que $\angle ABC=\angle BCA
=40^{\circ}$ y sea $D$ el punto en el lado $\overline{AC}$ de modo que $\angle ABD=\angle DBC
=20^{\circ}$. Demuestre que $\overline{AD}+\overline{BD}=\overline{BC}$.

\vspace{5mm}
\noindent\textbf{Problema 3.} Dado $\triangle ABC$, sean $D,E$ y $F$ los puntos medios de 
$\overline{AC}, \overline{AB}$ y $\overline{BC}$ respectivamente. Si $\overline{BG}$ es una altura 
de $\triangle ABC$, pruebe que $\angle EGF=\angle EDF$.

\vspace{5mm}
\noindent\textbf{Problema 4.} Sean ABCF un cuadrado y $\triangle CDE$ un triángulo equilátero,
ambos de lado 2 y tales que $B,C$ y $D$ son colineales. Sean $H\in\overline{AD}\cap\overline{EC}$ 
y $G\in\overline{AD}\cap\overline{FC}$. Calcular el área del triángulo $\triangle GCH$.

\vspace{5mm}
\noindent\textbf{Problema 5.} Sean $\overline{AC}$ y $\overline{BD}$ cuerdas perpendiculares en 
una circunferencia que se intersectan en el punto $G$. En $\triangle ADG$ la altura en $G$ 
intersecta a $\overline{AD}$ en $E$ y si la extendemos intersecta a $\overline{BC}$ en $P$. 
Muestre que $P$ es punto medio de $\overline{BC}$.

\vspace{5mm}
\noindent\textbf{Problema 6.} Dada una recta $L$ considere los puntos $M$ y $N$ al mismo lado de
$L$. Construya el punto $C$ en $L$ tal que el ángulo que forma $L$ con $\overleftrightarrow{MC}$
es congruente con el ángulo determinado por $L$ con $\overleftrightarrow{NC}$.

%\printbibliography % Quitar el comentado si quiero usar bibliografia

\end{document}
