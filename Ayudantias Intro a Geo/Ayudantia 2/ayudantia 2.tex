\documentclass{article}
\usepackage{hyperref}
\usepackage{Style}

\nocite{*} % Comentar si quiero citar
%\addbibresource{bibliografia.bib} % Quitar el comentado si quiero usar bibliografia

\begin{document}

\begin{minipage}{2.5cm}
    \includegraphics[width=2cm]{imagen_puc.jpg}
\end{minipage}
\begin{minipage}{14cm}
    {\sc Pontificia Universidad Católica de Chile\\
    Facultad de Matemáticas\\
    Departamento de Matemática\\
    Profesor: Giancarlo Urzúa -- Ayudante: Benjamín Mateluna}
\end{minipage}
\vspace{1ex}

{\centerline{\bf Introducción a la Geometría - MAT1304}
\centerline{\bf Ayudantía 2}}
\centerline{\bf 13 de agosto de 2025}

\vspace{1cm}
\noindent\textbf{Problema 1.} Considere el $\triangle ABC$ isósceles de base $\overline{BC}$, sean
$D,E$ puntos sobre $\overline{BC},\overline{AC}$ respectivamente, tales que 
$\angle BAD=30^{\circ}$ y $\overline{AD}=\overline{AE}$. Encuentre el valor del ángulo 
$\angle EDC$.

\vspace{5mm}
\noindent\textbf{Problema 2.} Sea $\triangle ABC$ rectangulo en $B$. Trazar la perpendicular por
el punto medio $M$ de $\overline{BC}$ cortando $\overline{AC}$ en el punto $N$. Mostrar que 
$2\overline{MN}=\overline{AB}$.

\vspace{5mm}
\noindent\textbf{Problema 3.} Sea $\triangle ABC$ un equilátero, sea $a$ la medida del
lado del triángulo, calcule su área.\footnote{Este resultado puede ser útil para la tarea.}

\vspace{5mm}
\noindent\textbf{Problema 4.} Dado un triángulo $\triangle ABC$ denotamos por $M$ el punto medio
de $\overline{AB}$. Demuestre que si $\angle BCA=90^{\circ}$, entonces 
$\overline{AM}=\overline{BM}=\overline{CM}$.

\vspace{5mm}
\noindent\textbf{Problema 5.} Sea $\triangle ABC$ equilátero. Sea $P$ un punto en el interior del
triángulo y suponga que $D,F,E$ son puntos de $\overline{AB},\overline{AC},\overline{BC}$ tales 
que $\overline{DP},\overline{EP}$ y $\overline{FP}$ son perpendiculares a 
$\overline{AB},\overline{AC},\overline{BC}$ respectivamente. Demuestre que 
$\overline{DP}+\overline{EP}+\overline{FP}$ es la altura del triángulo equilátero.

%\printbibliography % Quitar el comentado si quiero usar bibliografia

\end{document}
