\documentclass{article}
\usepackage{hyperref}
\usepackage{Style}

\nocite{*} % Comentar si quiero citar
%\addbibresource{bibliografia.bib} % Quitar el comentado si quiero usar bibliografia

\begin{document}

\begin{minipage}{2.5cm}
    \includegraphics[width=2cm]{imagen_puc.jpg}
\end{minipage}
\begin{minipage}{14cm}
    {\sc Pontificia Universidad Católica de Chile\\
    Facultad de Matemáticas\\
    Departamento de Matemática\\
    Profesor: Giancarlo Urzúa -- Ayudante: Benjamín Mateluna}
\end{minipage}
\vspace{1ex}

{\centerline{\bf Introducción a la Geometría - MAT1304}
\centerline{\bf Ayudantía 21}}
\centerline{\bf 27 de octubre de 2025}

\vspace{1cm}
\noindent\textbf{Problema 1.} Demuestre que $x^{2}+y^{2}+4x+6-23=0$ y $x^{2}+y^{2}-8x-10y+25=0$
son tangentes.

\vspace{5mm}
\noindent\textbf{Problema 2.} Demuestre que la expresión
\begin{equation*}
    \left(\frac{1-t^{2}}{1+t^{2}},\frac{2t}{1+t^{2}}\right)
\end{equation*}
con $t\in\R$, es una parametrización de los ceros de $x^{2}+y^{2}=1$ sin $(-1,0)$. Más aún, usando
lo anterior encuentre todas las ternas pitagóricas.

\vspace{5mm}
\noindent\textbf{Problema 3.} Dados los puntos $A=(1,1,1),B=(0,2,2),C=(1,0,-1),D=(0,1,0)$. Muestre
que $A,B,C,D$ están dentro de un mismo plano, encontrando la ecuación del plano que los contiene.

\vspace{5mm}
\noindent\textbf{Problema 4.} Sea $(x_{0},y_{0},z_{0})$ un cero de $xy+xz+yz$. Demuestre que la 
recta que pasa por $(x_{0},y_{0},z_{0})$ y el origen está contenida en el conjunto de ceros de 
$xy+xz+yz$.

\vspace{5mm}
\noindent\textbf{Problema 5.} Pruebe que los polinomios homogéneos de grado $d$ en tres variables, 
están en correspondencia con los hiperplanos $\R^{N}$ donde $N=\binom{d+2}{d}$.

%\printbibliography % Quitar el comentado si quiero usar bibliografia

\end{document}
