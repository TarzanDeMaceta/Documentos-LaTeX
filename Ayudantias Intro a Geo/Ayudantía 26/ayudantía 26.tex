\documentclass{article}
\usepackage{hyperref}
\usepackage{Style}

\nocite{*} % Comentar si quiero citar
%\addbibresource{bibliografia.bib} % Quitar el comentado si quiero usar bibliografia

\begin{document}

\begin{minipage}{2.5cm}
    \includegraphics[width=2cm]{imagen_puc.jpg}
\end{minipage}
\begin{minipage}{14cm}
    {\sc Pontificia Universidad Católica de Chile\\
    Facultad de Matemáticas\\
    Departamento de Matemática\\
    Profesor: Giancarlo Urzúa -- Ayudante: Benjamín Mateluna}
\end{minipage}
\vspace{1ex}

{\centerline{\bf Introducción a la Geometría - MAT1304}
\centerline{\bf Ayudantía 26}}
\centerline{\bf 17 de noviembre de 2025}

\vspace{1cm}
\noindent\textbf{Problema 1.} Demuestre que los puntos $A=(1,2,3)$, $B=(4,5,6)$ y $C=(7,8,9)$ son
colineales y encontrar la ecuación paramétrica de la recta que los contiene.

\vspace{5mm}
\noindent\textbf{Problema 2.} Encontrar la ecuación del plano que,
\begin{enumerate}
    \item Pasa por $P=(1,-2,3)$ y es perpendicular a $n=(2,1,-4)$.
    
    \item Contiene a los puntos $A=(1,0,1)$, $B=(2,1,3)$ y $C=(-1,2,0)$.
\end{enumerate}

\vspace{5mm}
\noindent\textbf{Problema 3.} Un plano pasa por el punto $P=(1,-1,2)$ y es perpendicular a la 
recta que esta en la intersección de los planos $x+2y-z=3$ y $2x-y+3z=1$. Encuentre su ecuación.

\vspace{5mm}
\noindent\textbf{Problema 4.} Encuentre una parametrización de la intersección entre la esfera
$x^{2}+y^{2}+z^{2}=1$ y el plano $z=x+1$.

\vspace{5mm}
\noindent\textbf{Problema 5.} Se definen las rectas en $\R^{3}$,
\begin{equation*}
    L_{1}:=\{x=z,2x-y=2\}
    \hhtext{y}
    L_{2}:=\{x+z=2,2x+y=3\}
\end{equation*}
Encontrar la recta que contiene a $(1,-2,-2)$ e intersecta a $L_{1}$ y $L_{2}$.

%\printbibliography % Quitar el comentado si quiero usar bibliografia

\end{document}
