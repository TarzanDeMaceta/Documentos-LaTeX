\documentclass{article}
\usepackage{hyperref}
\usepackage{Style}

\nocite{*} % Comentar si quiero citar
%\addbibresource{bibliografia.bib} % Quitar el comentado si quiero usar bibliografia

\begin{document}

\begin{minipage}{2.5cm}
    \includegraphics[width=2cm]{imagen_puc.jpg}
\end{minipage}
\begin{minipage}{14cm}
    {\sc Pontificia Universidad Católica de Chile\\
    Facultad de Matemáticas\\
    Departamento de Matemática\\
    Profesor: Giancarlo Urzúa -- Ayudante: Benjamín Mateluna}
\end{minipage}
\vspace{1ex}

{\centerline{\bf Introducción a la Geometría - MAT1304}
\centerline{\bf Ayudantía 1}}
\centerline{\bf 11 de agosto de 2025}

\vspace{1cm}
\noindent\textbf{Problema 1.} Dado un triángulo $\triangle ABC$, denotamos por $M$ al punto medio
de $\overline{AB}$. Demuestre, que si $\overline{AM}=\overline{BM}=\overline{CM}$, entonces 
$\angle BCA=90^{\circ}$.

\vspace{5mm}
\noindent\textbf{Problema 2.} Probar que dos rectas son paralelas si y solo si se cumple la 
propiedad de ángulos alternos internos. Esto es, dadas dos rectas paralelas y una recta secante a
ambas, entonces los ángulos alternos internos son iguales.

\vspace{5mm}
\begin{dfn}
    La bisectriz de un ángulo es el rayo con origen en el vértice del ángulo tal que que lo divide 
    en dos ángulos iguales.
\end{dfn}
\vspace{2mm}
\noindent\textbf{Problema 3.} Demuestre que la bisectriz del ángulo opuesto a la base en un 
triángulo isósceles es perpendicular a la base y la dimide (i.e. la divide en dos segmentos 
congruentes).

\vspace{5mm}
\noindent\textbf{Problema 4.} Sea $\triangle ABC$ isósceles de base $\overline{BC}$, sean $E$ y 
$D$ puntos en $\overline{AB}$ y $\overline{AC}$ respectivamente, de forma que 
$\overline{AE}=\overline{AD}$.
\begin{enumerate}
    \item Demuestre que $\overline{ED}\parallel\overline{BC}$.
    \item Sea $P\in\overline{EC}\cap\overline{DB}$, demuestre que $\overline{AP}$ es bisectriz de
    $\angle BAC$.
\end{enumerate}

\vspace{5mm}
\begin{dfn}
    Dado un segmento $\overline{AB}$ y $M$ su punto medio, se dice que la recta $L$ es simetral si 
    es perpendicular a $\overline{AB}$ y pasa por el punto $M$.
\end{dfn}
\vspace{2mm}
\noindent\textbf{Problema 5.} Sea $\overline{AB}$ un segmento y $L$ su simetral, demuestre que 
dado $P\in L$, se tiene que $\overline{AP}=\overline{BP}$.

\vspace{5mm}
\noindent\textbf{Problema 6.} Sea $P$ un polígono regular de $n$ lados con $n\geq3$. Pruebe que 
las simetrales de tres lados contiguos de $P$ son concurrentes.\footnote{Este resultado puede ser 
útil para la tarea.}

%\printbibliography % Quitar el comentado si quiero usar bibliografia

\end{document}
