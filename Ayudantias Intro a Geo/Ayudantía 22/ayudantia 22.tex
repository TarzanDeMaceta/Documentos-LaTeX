\documentclass{article}
\usepackage{hyperref}
\usepackage{Style}

\nocite{*} % Comentar si quiero citar
%\addbibresource{bibliografia.bib} % Quitar el comentado si quiero usar bibliografia

\begin{document}

\begin{minipage}{2.5cm}
    \includegraphics[width=2cm]{imagen_puc.jpg}
\end{minipage}
\begin{minipage}{14cm}
    {\sc Pontificia Universidad Católica de Chile\\
    Facultad de Matemáticas\\
    Departamento de Matemática\\
    Profesor: Giancarlo Urzúa -- Ayudante: Benjamín Mateluna}
\end{minipage}
\vspace{1ex}

{\centerline{\bf Introducción a la Geometría - MAT1304}
\centerline{\bf Ayudantía 22}}
\centerline{\bf 29 de octubre de 2025}

\vspace{1cm}
\noindent\textbf{Problema 1.} Demuestre que la pendiente de la tangente a la parábola $y=x^{2}$ 
por el punto $(x_{0},y_{0})$ es $2x_{0}$.

\vspace{5mm}
\noindent\textbf{Problema 2.} Hallar la ecuación polinomial de la curva cuyas ecuaciones 
parametricas son
\begin{equation*}
    x=2+3\cdot tan(\theta)
    \htext{y}
    y=1+4\cdot sec(\theta)
\end{equation*}

\vspace{5mm}
\noindent\textbf{Problema 3.} Hallar e identificar la ecuación del lugar geométrico de un punto 
que se mueve de tal manera que su distancia de la recta $y=-8$ es siempre igual al doble de su
distancia del punto $(0,-2)$.

\vspace{5mm}
\noindent\textbf{Problema 4.} Hallar e identificar la ecuación del lugar geométrico de los puntos 
medios de las ordenadas de los puntos de la circunferencia $x^{2}+y^{2}=9$.

\vspace{5mm}
\noindent\textbf{Problema 5.} Por el punto fijo $A=(-1,0)$ de la circunferencia 
$x^{2}+y^{2}=1$ se traza una cuerda cualquiera $AB$. Hallar la ecuación del lugar geométrico
del punto medio de $AB$.

%\printbibliography % Quitar el comentado si quiero usar bibliografia

\end{document}
