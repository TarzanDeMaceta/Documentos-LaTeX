\documentclass{article}
\usepackage{hyperref}
\usepackage{Style}

\nocite{*} % Comentar si quiero citar
%\addbibresource{bibliografia.bib} % Quitar el comentado si quiero usar bibliografia

\begin{document}

\begin{minipage}{2.5cm}
    \includegraphics[width=2cm]{imagen_puc.jpg}
\end{minipage}
\begin{minipage}{14cm}
    {\sc Pontificia Universidad Católica de Chile\\
    Facultad de Matemáticas\\
    Departamento de Matemática\\
    Profesor: Giancarlo Urzúa -- Ayudante: Benjamín Mateluna}
\end{minipage}
\vspace{1ex}

{\centerline{\bf Introducción a la Geometría - MAT1304}
\centerline{\bf Ayudantía 22}}
\centerline{\bf 29 de octubre de 2025}

\vspace{1cm}
\noindent\textbf{Problema 1.} Parametrice las soluciones del siguiente sistema de ecuaciones
\begin{equation*}
    \begin{cases}
        x^{2}+y^{2}+z^{2}=1 \\
        (x-1)^{2}+(y-1)^{2}+(z-1)^{2}=1
    \end{cases}
\end{equation*}

\vspace{5mm}
\noindent\textbf{Problema 2.} Sea $p\in\R^{3}$ un cero de $x^{2}+y^{2}+z^{2}=1$. Encuentre el 
plano tangente a la esfera que pasa por el punto $p$.

\vspace{5mm}
\noindent\textbf{Problema 3.} Encuentre la distancia de un punto a un plano.

\vspace{5mm}
\noindent\textbf{Problema 4.} 

\vspace{5mm}
\noindent\textbf{Problema 5.} 

%\printbibliography % Quitar el comentado si quiero usar bibliografia

\end{document}
