\documentclass{article}
\usepackage{hyperref}
\usepackage{Style}

\nocite{*} % Comentar si quiero citar
%\addbibresource{bibliografia.bib} % Quitar el comentado si quiero usar bibliografia

\begin{document}

\begin{minipage}{2.5cm}
    \includegraphics[width=2cm]{imagen_puc.jpg}
\end{minipage}
\begin{minipage}{14cm}
    {\sc Pontificia Universidad Católica de Chile\\
    Facultad de Matemáticas\\
    Departamento de Matemática\\
    Profesor: Giancarlo Urzúa -- Ayudante: Benjamín Mateluna}
\end{minipage}
\vspace{1ex}

{\centerline{\bf Introducción a la Geometría - MAT1304}
\centerline{\bf Ayudantía 12}}
\centerline{\bf 24 de septiembre de 2025}

\vspace{1cm}
\noindent\textbf{Problema 1.} Muestre que el decágono, esto es, un polígono regular de 10 lados,
es constructible.

\vspace{5mm}
\noindent\textbf{Problema 2.} Demuestre que $cos(3\alpha)=4cos^{3}(\alpha)-3cos(\alpha)$ para todo 
$\alpha\in\R$.

\vspace{5mm}
\noindent\textbf{Problema 3.} Resuelva la ecuación $sen^{4}(x)+cos^{4}(x)=\frac{1}{2}$ para 
$x\in[0,\frac{\pi}{2}]$.

\vspace{5mm}
\noindent\textbf{Problema 4.} Para $x,y\in\R$, demuestre las siguientes identidades de 
prostaférisis
\begin{align*}
    2\hspace{1mm}sen(x)sen(y) &= cos(x-y)-cos(x+y) \\
    2\hspace{1mm}cos(x)cos(y) &= cos(x-y)+cos(x+y)
\end{align*}

\vspace{5mm}
\noindent\textbf{Problema 5.} Sean $\alpha$ y $\beta$ ángulos tales que
\begin{equation*}
    sen(\alpha)+sen(\beta)=\frac{\sqrt{6}}{2}\hhtext{y}
    cos(\alpha)+cos(\beta)=\frac{\sqrt{2}}{2}
\end{equation*}
Calcule $tan(\alpha+\beta)$.

%\printbibliography % Quitar el comentado si quiero usar bibliografia

\end{document}
