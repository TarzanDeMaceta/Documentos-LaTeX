\documentclass{article}
\usepackage{hyperref}
\usepackage{Style}

\nocite{*} % Comentar si quiero citar
%\addbibresource{bibliografia.bib} % Quitar el comentado si quiero usar bibliografia

\begin{document}

\begin{minipage}{2.5cm}
    \includegraphics[width=2cm]{imagen_puc.jpg}
\end{minipage}
\begin{minipage}{14cm}
    {\sc Pontificia Universidad Católica de Chile\\
    Facultad de Matemáticas\\
    Departamento de Matemática\\
    Profesor: Giancarlo Urzúa -- Ayudante: Benjamín Mateluna}
\end{minipage}
\vspace{1ex}

{\centerline{\bf Introducción a la Geometría - MAT1304}
\centerline{\bf Ayudantía 5}}
\centerline{\bf 25 de agosto de 2025}

\vspace{1cm}
\noindent\textbf{Problema 1.} Demuestre que si en un triángulo dos transversales de gravedad 
tienen el mismo largo entonces el triángulo es isósceles.

\vspace{2mm}
\noindent\textbf{Problema 2.} En $\triangle ABC$, las bisectrices en $B$ y $C$ intersectan a la
mediana trazada desde $A$ en el punto $P$, es decir, son concurrentes. Demuestre que 
$\triangle ABC$ es isósceles.

\vspace{2mm}
\noindent\textbf{Problema 3.} Sea $\triangle ABC$ rectángulo en $C$. Sea $D$ es $\overline{AB}$ y
$E$ en $\overline{BC}$ tal que $\overline{DE}$ es perpendicular a $\overline{AB}$. Pruebe que 
$\overline{CD}$ es bisectriz si y solo si $\overline{AD}=\overline{DE}$.

\vspace{2mm}
\noindent\textbf{Problema 4.} Sea $\triangle ABC$. Demuestre que el producto del largo de las 
partes en que el ortocentro divide a una altura es independiente del vértice desde el cual se 
traza la altura.

%\printbibliography % Quitar el comentado si quiero usar bibliografia

\end{document}
