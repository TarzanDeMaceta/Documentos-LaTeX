\documentclass{article}
\usepackage{hyperref}
\usepackage{Style}

\nocite{*} % Comentar si quiero citar
%\addbibresource{bibliografia.bib} % Quitar el comentado si quiero usar bibliografia

\begin{document}

\begin{minipage}{2.5cm}
    \includegraphics[width=2cm]{imagen_puc.jpg}
\end{minipage}
\begin{minipage}{14cm}
    {\sc Pontificia Universidad Católica de Chile\\
    Facultad de Matemáticas\\
    Departamento de Matemática\\
    Profesor: Giancarlo Urzúa -- Ayudante: Benjamín Mateluna}
\end{minipage}
\vspace{1ex}

{\centerline{\bf Introducción a la Geometría - MAT1304}
\centerline{\bf Ayudantía 13}}
\centerline{\bf 29 de septiembre de 2025}

\vspace{1cm}
\noindent\textbf{Problema 1.} Sea $\triangle ABC$ un triángulo con $\angle ACB=\frac{\pi}{4}$ y
$\angle ABC=\frac{\pi}{3}$. Sea $D\in\overline{BC}$ tal que $3\overline{BD}=\overline{CD}$. 
Encuentre el valor de
\begin{equation*}
    \frac{sen(\angle BAD)}{sen(\angle CAD)}
\end{equation*}

\vspace{5mm}
\noindent\textbf{Problema 2.} Sea $\triangle ABC$ y $P$ en el segmento $\overline{BC}$ tal que
$\overline{AP}=p, \overline{BP}=m$ y $\overline{PC}=n$. Demostrar que $a(p^{2}+mn)=b^{2}m+c^{2}n$.

\vspace{5mm}
\noindent\textbf{Problema 3.} Usando el Teorema del Coseno, demuestre que en $\triangle ABC$,
la mediana $m_{a}$ hacia el lado $a$ satisface
\begin{equation*}
    m_{a}^{2}=\frac{2b^{2}+2c^{2}-a^{2}}{4}
\end{equation*}

\vspace{5mm}
\noindent\textbf{Problema 4.} Sea $\triangle ABC$ y $x\in\R$ positivo. Supongamos que 
$a=2x,b=x+1,c=x+2$ y que $\alpha=\frac{\pi}{3}$. Pruebe que $\beta$ es constructible con regla y 
compás.

\vspace{5mm}
\noindent\textbf{Problema 5.} Considere $\triangle ADC$ rectángulo en $C$. Sea $B\in\overline{AC}$
y $2\angle DAC=\angle DBC$. Si $\overline{BC}=\frac{3}{5}\overline{AB}$. Calcule el área de 
$\triangle ADC$ en función de $\overline{BC}$.

%\printbibliography % Quitar el comentado si quiero usar bibliografia

\end{document}
