\documentclass{article}
\usepackage{hyperref}
\usepackage{Style}

\nocite{*} % Comentar si quiero citar
%\addbibresource{bibliografia.bib} % Quitar el comentado si quiero usar bibliografia

\begin{document}

\begin{minipage}{2.5cm}
    \includegraphics[width=2cm]{imagen_puc.jpg}
\end{minipage}
\begin{minipage}{14cm}
    {\sc Pontificia Universidad Católica de Chile\\
    Facultad de Matemáticas\\
    Departamento de Matemática\\
    Profesor: Giancarlo Urzúa -- Ayudante: Benjamín Mateluna}
\end{minipage}
\vspace{1ex}

{\centerline{\bf Introducción a la Geometría - MAT1304}
\centerline{\bf Ayudantía 11}}
\centerline{\bf 22 de septiembre de 2025}

\vspace{1cm}
\noindent\textbf{Problema 1.} Demuestre la siguiente identidad trigonometrica para 
$\alpha,\beta\in\R$,
\begin{equation*}
    tan(\alpha+\beta)=\frac{tan(\alpha)+tan(\beta)}{1-tan(\alpha)tan(\beta)}
\end{equation*}
Usando lo anterior encuentre una fórmula para el ángulo doble de tangente y secante.

\vspace{5mm}
\noindent\textbf{Problema 2.} Para $x\in\R$, demuestre que
\begin{equation*}
    cos^{2}(x)=\frac{cos(2x)+1}{2}\hhtext{y}
    sen^{2}(x)=\frac{1-cos(2x)}{2}
\end{equation*}

\vspace{5mm}
\noindent\textbf{Problema 3.} Muestre que
\begin{equation*}
    cot(2x)=\frac{1}{2}\left(cot(x)-tan(x)\right)
\end{equation*}

\vspace{5mm}
\noindent\textbf{Problema 4.} Demuestre que para todo $n\in\N$ y para todo $\alpha\in\R$ tal que
$sen(\alpha)\neq0$ se tiene que
\begin{equation*}
    cos(\alpha)\cdot cos(2\alpha)\cdots cos(2^{n}\alpha)
    =\frac{sen(2^{n+1}\alpha)}{2^{n+1}sen(\alpha)}
\end{equation*}

\vspace{5mm}
\noindent\textbf{Problema 5.} Encuentre un valor explícito para $sen(\frac{\pi}{12})$ y
$cos(\frac{\pi}{12})$. Calcule
\begin{equation*}
    cos^{4}\left(\frac{\pi}{24}\right)-sen^{4}\left(\frac{\pi}{24}\right)
\end{equation*}

%\printbibliography % Quitar el comentado si quiero usar bibliografia

\end{document}
