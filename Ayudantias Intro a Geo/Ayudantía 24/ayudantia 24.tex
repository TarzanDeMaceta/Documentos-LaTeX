\documentclass{article}
\usepackage{hyperref}
\usepackage{Style}

\nocite{*} % Comentar si quiero citar
%\addbibresource{bibliografia.bib} % Quitar el comentado si quiero usar bibliografia

\begin{document}

\begin{minipage}{2.5cm}
    \includegraphics[width=2cm]{imagen_puc.jpg}
\end{minipage}
\begin{minipage}{14cm}
    {\sc Pontificia Universidad Católica de Chile\\
    Facultad de Matemáticas\\
    Departamento de Matemática\\
    Profesor: Giancarlo Urzúa -- Ayudante: Benjamín Mateluna}
\end{minipage}
\vspace{1ex}

{\centerline{\bf Introducción a la Geometría - MAT1304}
\centerline{\bf Ayudantía 24 - Repaso Geometría Cartesiana}}
\centerline{\bf 05 de noviembre de 2025}

\vspace{1cm}
\noindent\textbf{Problema 1.} Muestre que el polinomio $x^{2}-4x+y^{2}-2y-4$ describe una 
circunferencia. Encuentre su radio y centro.

\vspace{5mm}
\noindent\textbf{Problema 2.} Las parábolas descritas por los polinomios $x^{2}-x-4-y$ y 
$y^{2}-5y+1-x$ se intersectan en cuatro puntos. Muestre que tales puntos pertenecen a una 
circunferencia.

\vspace{5mm}
\noindent\textbf{Problema 3.} Considere la circunferencia $x^{2}+y^{2}=1$. Encuentre la recta 
tangente que pasa por el punto $(\frac{\sqrt{2}}{2},\frac{\sqrt{2}}{2})$ y demuestre que es 
perpendicular a la recta $y=x$.

\vspace{5mm}
\noindent\textbf{Problema 4.} Dada la circunferencia $x^{2}+y^{2}-6x-2y+6=0$, determine los 
valores de $m$ para los cuales, la recta $y=mx+3$:
\begin{enumerate}
    \item Intersecte a la circunferencia en dos puntos.
    \item Sea tangente a la circunferencia.
    \item No tenga puntos en común con la circunferencia.
\end{enumerate}

\vspace{5mm}
\noindent\textbf{Problema 5.} Halle el lugar geométrico de los centros de las circunferencias que 
son tangentes a la recta $y=1$ y la circunferencia de ecuación $x^{2}+y^{2}=9$.

%\printbibliography % Quitar el comentado si quiero usar bibliografia

\end{document}
