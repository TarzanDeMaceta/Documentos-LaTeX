\documentclass{article}
\usepackage{hyperref}
\usepackage{Style}

\nocite{*} % Comentar si quiero citar
%\addbibresource{bibliografia.bib} % Quitar el comentado si quiero usar bibliografia

\begin{document}

\begin{minipage}{2.5cm}
    \includegraphics[width=2cm]{imagen_puc.jpg}
\end{minipage}
\begin{minipage}{14cm}
    {\sc Pontificia Universidad Católica de Chile\\
    Facultad de Matemáticas\\
    Departamento de Matemática\\
    Profesor: Giancarlo Urzúa -- Ayudante: Benjamín Mateluna}
\end{minipage}
\vspace{1ex}

{\centerline{\bf Introducción a la Geometría - MAT1304}
\centerline{\bf Ayudantía 25 - Repaso I3}}
\centerline{\bf 07 de noviembre de 2025}

\vspace{1cm}
\noindent\textbf{Problema 1.} Demuestre que $x^{4}+3x^{2}+2$ y $x^{3}+x^{2}+x+1$ tienen un factor
en común en $\R[x]$.

\vspace{5mm}
\noindent\textbf{Problema 2.} Factorice completamente en irreducibles el polinomio $x^{12}-1=0$ en 
$\Q[x]$, pp$\R[x]$, $\Z[x]$.

\vspace{5mm}
\noindent\textbf{Problema 3.} Considere los puntos $A=(0,0)$, $B=(1,0)$, $C=(1,1)$ y $D=(0,1)$. 
Demuestre, usando geometría cartesiana que las diagonales del cuadrado $ABCD$ son perpendiculares.

\vspace{5mm}
\noindent\textbf{Problema 4.} Demuestre que la tangente de la elipse $\frac{x^{2}}{a^{2}}
+\frac{y^{2}}{b^{2}}=1$ por el punto $(x_{0},y_{0})$ es
\begin{equation*}
    \frac{x_{0}}{a^{2}}x+\frac{y_{0}}{b^{2}}y=1
\end{equation*}

\vspace{5mm}
\noindent\textbf{Problema 5.} Sean $A=(1,0)$ y $B=(2,3)$. Encontrar todos los puntos $C$ tales que
$\triangle ABC$ es equilátero.

%\printbibliography % Quitar el comentado si quiero usar bibliografia

\end{document}
