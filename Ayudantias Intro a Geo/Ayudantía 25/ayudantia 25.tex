\documentclass{article}
\usepackage{hyperref}
\usepackage{Style}

\nocite{*} % Comentar si quiero citar
%\addbibresource{bibliografia.bib} % Quitar el comentado si quiero usar bibliografia

\begin{document}

\begin{minipage}{2.5cm}
    \includegraphics[width=2cm]{imagen_puc.jpg}
\end{minipage}
\begin{minipage}{14cm}
    {\sc Pontificia Universidad Católica de Chile\\
    Facultad de Matemáticas\\
    Departamento de Matemática\\
    Profesor: Giancarlo Urzúa -- Ayudante: Benjamín Mateluna}
\end{minipage}
\vspace{1ex}

{\centerline{\bf Introducción a la Geometría - MAT1304}
\centerline{\bf Ayudantía 25 - Repaso I3}}
\centerline{\bf 07 de noviembre de 2025}

\vspace{1cm}
\noindent\textbf{Problema 1.} Sea $\omega\in\C$ una raíz $n-$ésima de la unidad y $p(x)\in\R[x]$ 
tal que $p(\omega)=0$. Demuestre que $p(x)$ y $q(x)=\sum_{i=0}^{n-1}x^{i}$ tienen un factor en
común en $\R[x]$.

\vspace{5mm}
\noindent\textbf{Problema 2.} Encuentre el máximo común divisor en $\Q[x]$, $\R[x]$ y $\C[x]$ de 
los polinomios
\begin{equation*}
    x^{4}-1
    \hhtext{y}
    x^{4}+x^{3}+x-1
\end{equation*}

\vspace{5mm}
\noindent\textbf{Problema 3.} Considere los puntos $A$, $B$, $C$ y $D$ que forman un cuadrado en 
el plano. Demuestre, usando geometría cartesiana que las diagonales del cuadrado $ABCD$ son 
perpendiculares.

\vspace{5mm}
\noindent\textbf{Problema 4.} Encuentre la recta tangente a la parábola $x^{2}=y$ que es paralela
a la recta $y=x+1$.

\vspace{5mm}
\noindent\textbf{Problema 5.} Sean $A=(1,0)$ y $B=(2,3)$. Encontrar todos los puntos $C$ tales que
$\triangle ABC$ es equilátero.

%\printbibliography % Quitar el comentado si quiero usar bibliografia

\end{document}
