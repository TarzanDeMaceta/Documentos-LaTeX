\documentclass{article}
\usepackage{hyperref}
\usepackage{Style}

\nocite{*} % Comentar si quiero citar
%\addbibresource{bibliografia.bib} % Quitar el comentado si quiero usar bibliografia

\begin{document}

\begin{minipage}{2.5cm}
    \includegraphics[width=2cm]{imagen_puc.jpg}
\end{minipage}
\begin{minipage}{14cm}
    {\sc Pontificia Universidad Católica de Chile\\
    Facultad de Matemáticas\\
    Departamento de Matemática\\
    Profesor: Giancarlo Urzúa -- Ayudante: Benjamín Mateluna}
\end{minipage}
\vspace{1ex}

{\centerline{\bf Introdcucción a la Geometría - MAT1304}
\centerline{\bf Ayudantía 4}}
\centerline{\bf 20 de agosto de 2025}

\vspace{1cm}
\noindent\textbf{Problema 1.} Considere un cuadrilatero convexo $ABCD$, se traza además el 
segmento $\overline{AC}$, demuestre que
\begin{enumerate}
    \item La recta determinada por los puntos medios de $\overline{AB}$ y $\overline{BC}$ es 
    paralela a $\overline{AC}$.
    
    \item Sean $M,N,O,P$ los puntos medios de cada lado del cuadrilátero, entonces $MNOP$ es un 
    paralelogramo.
\end{enumerate}

\vspace{5mm}
\noindent\textbf{Problema 2.} Sea $ABCD$ un paralelogramo y desde $D$ tracemos una recta que 
intersecta al lado $\overline{AB}$ en un punto $E$. Llamemos $F$ al punto de intersección de 
$\overline{DE}$ con $\overline{CB}$. Muestre que
\begin{equation*}
    \frac{\overline{AD}}{\overline{AE}}=\frac{\overline{FB}}{\overline{BE}}
    =\frac{\overline{FC}}{\overline{CD}}
\end{equation*}

\vspace{5mm}
\noindent\textbf{Problema 3.} Dado $\triangle ABC$ un triángulo rectángulo con ángulo recto en 
$A$, demuestre que la bisectriz en $A$ divide en dos ángulos iguales al ángulo formado por la 
altura y la transversal de gravedad que unen a $A$ con la hipotenusa.

\vspace{5mm}
\noindent\textbf{Problema 4.} Sea $O$ el centro de una circunferencia inscrita en $\triangle ABC$.
El rayo $\overrightarrow{AO}$ intersecta a $\overline{BC}$ en $D$. Demuestre que
\begin{equation*}
    \overline{AO}\cdot\overline{BC}=\overline{OD}\cdot(\overline{AB}+\overline{AC})
\end{equation*}

\vspace{5mm}
\noindent\textbf{Problema 5.} Sea $\triangle ABC$ y $D$ el punto sobre $\overleftrightarrow{AB}$ 
tal que $\overline{CD}$ es la bisectriz externa de $C$, entonces se tiene que
\begin{equation*}
    \frac{\overline{AD}}{\overline{AC}}=\frac{\overline{BD}}{\overline{BC}}
\end{equation*}

%\printbibliography % Quitar el comentado si quiero usar bibliografia

\end{document}
