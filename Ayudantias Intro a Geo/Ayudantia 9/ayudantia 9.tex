\documentclass{article}
\usepackage{hyperref}
\usepackage{Style}

\nocite{*} % Comentar si quiero citar
%\addbibresource{bibliografia.bib} % Quitar el comentado si quiero usar bibliografia

\begin{document}

\begin{minipage}{2.5cm}
    \includegraphics[width=2cm]{imagen_puc.jpg}
\end{minipage}
\begin{minipage}{14cm}
    {\sc Pontificia Universidad Católica de Chile\\
    Facultad de Matemáticas\\
    Departamento de Matemática\\
    Profesor: Giancarlo Urzúa -- Ayudante: Benjamín Mateluna}
\end{minipage}
\vspace{1ex}

{\centerline{\bf Introducción a la Geometría - MAT1304}
\centerline{\bf Ayudantía 9}}
\centerline{\bf 08 de septiembre de 2025}

\vspace{1cm}
\noindent\textbf{Problema 1.} Dada un circunferencia de centro O, construir un triángulo isósceles
que lo tenga de íncirculo, es decir, cuyos lados sean tangentes a la circunferencia.

\vspace{5mm}
\noindent\textbf{Problema 2.} Dado segmento de largo $\overline{AB}$ de largo $\ell$, construya un
rombo cuyos lados tienen el largo $\ell$ que también es el largo de una de las diagonales del 
rombo.

\vspace{5mm}
\noindent\textbf{Problema 3.} Dada una recta $L$ considere los puntos $M$ y $N$ al mismo lado de
$L$. Construya el punto $C$ en $L$ tal que el ángulo que forma $L$ con $\overleftrightarrow{MC}$
es congruente con el ángulo determinado por $L$ con $\overleftrightarrow{NC}$.

\vspace{5mm}
\noindent\textbf{Problema 4.} Dado $\overline{AB}$ construya el conjunto de todos los puntos $P$
tales que $\angle APB=45^{\circ}$.

%\printbibliography % Quitar el comentado si quiero usar bibliografia

\end{document}
